% !TEX root = ../main.tex
\documentclass[../main.tex]{subfiles}
\addbibresource{../references.bib}
\begin{document}
\chapter{Bottom Up Parser – Canonical SLR(1)}
\label{ch:slr1}

\section{Konsep SLR(1)}
SLR(1) memperkaya LR(0) dengan menggunakan himpunan FOLLOW untuk menentukan konteks reduksi yang valid. Dengan demikian, banyak konflik reduce yang hadir pada LR(0) dapat dieliminasi dengan membatasi reduksi hanya pada lookahead yang sah.\cite{aho-dragon-book-2006} Penambahan konteks ini meningkatkan akurasi tanpa perubahan besar pada automata.
Konstruksi tetap memanfaatkan automata item LR(0) sehingga biaya komputasi tidak bertambah drastis. Namun, SLR(1) masih lebih lemah daripada LR(1) penuh dan dapat gagal pada grammar dengan dependensi konteks yang kompleks.\cite{grune-parsing} Kesadaran akan batasannya membantu menentukan kapan perlu beralih ke LR(1) atau LALR(1).
Secara praktis, SLR(1) menawarkan kompromi yang baik antara kekuatan dan kompleksitas untuk berbagai bahasa domain-spesifik. Pendekatan ini juga memudahkan pengajaran karena tetap menggunakan basis automata LR(0).

\section{Penyusunan Tabel SLR(1)}
Prosedur pengisian tabel aksi mengikuti struktur LR(0), tetapi aturan reduksi ditempatkan hanya untuk terminal dalam FOLLOW dari nonterminal sisi kiri produksi. Strategi ini mengurangi konflik reduce/reduce yang tidak dapat diatasi oleh LR(0).\cite{aho-dragon-book-2006} Penempatan selektif ini memastikan langkah reduksi hanya terjadi pada konteks yang tepat.
Implementasi dalam generator seperti Bison menunjukkan bagaimana deklarasi presedensi dan asosiativitas dapat melengkapi SLR(1) untuk menyelesaikan sisa konflik.\cite{bison-manual} Pengaturan presedensi memperkenalkan kebijakan deterministik untuk kasus yang tersisa tanpa mengubah grammar.
Pengujian menyeluruh dengan suite contoh grammar penting untuk memverifikasi tidak adanya konflik sebelum parser digunakan dalam kompilasi skala besar. Laporan konflik yang bersih menjadi indikator kesiapan produksi.

\IfSubfilesClassLoaded{%
\printbibliography
}{}

\end{document}
