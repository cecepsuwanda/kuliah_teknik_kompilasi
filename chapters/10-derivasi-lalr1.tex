% !TEX root = ../main.tex
\documentclass[../main.tex]{subfiles}
\addbibresource{../references.bib}
\begin{document}
\chapter{Bottom Up Parser – Derivasi \& LALR(1)}
\label{ch:lalr1}

\section{Derivasi Bottom-Up dan LR(1)}
Derivasi bottom-up membangun parse dengan mengenali rightmost derivation in reverse, melakukan reduksi dari produksi yang cocok pada prefiks input. LR(1) memperkenalkan lookahead tunggal pada item, sehingga keputusan reduksi mempertimbangkan simbol berikutnya dan mengurangi konflik.\cite{aho-dragon-book-2006} Mekanisme ini memberikan determinisme yang kuat dalam menghadapi variasi struktur input.
Meskipun LR(1) sangat kuat, jumlah state automata dapat meledak secara kombinatorial. Oleh karena itu, teknik kompresi seperti LALR(1) diperkenalkan untuk menyatukan state yang memiliki inti identik.\cite{grune-parsing} Strategi penggabungan ini menyeimbangkan kualitas keputusan dengan ukuran tabel yang realistis.
Analisis ini menyiapkan landasan untuk memahami kompromi antara kekuatan pengenalan bahasa dan biaya komputasi pada implementasi parser industri. Pertimbangan ini memandu pemilihan metode parsing dalam desain kompilator nyata.

\section{Konstruksi LALR(1)}
LALR(1) menggabungkan state LR(1) dengan inti identik sambil menggabungkan himpunan lookahead. Hasilnya adalah tabel berukuran mendekati SLR(1) tetapi memiliki kekuatan dekat LR(1) penuh pada banyak grammar praktis.\cite{aho-dragon-book-2006} Dalam banyak kasus, ini menghilangkan konflik tanpa membebani memori secara signifikan.
Konsekuensi penggabungan dapat menimbulkan konflik yang tidak ada pada LR(1), tetapi sering kali jarang terjadi pada grammar bahasa pemrograman yang dirancang baik. Generator parser seperti Yacc/Bison dan Menhir memanfaatkan teknik ini untuk kinerja dan ukuran tabel yang wajar.\cite{bison-manual, menhir} Laporan diagnostik dari alat tersebut membantu menilai kualitas penggabungan state.
Pemilihan LALR(1) lazim dalam kompilator industri historis karena keseimbangan kekuatan dan efisiensi, meski tren modern juga mengadopsi GLR atau PEG untuk kebutuhan tertentu. Pemilihan akhir perlu mempertimbangkan kompleksitas grammar dan kebutuhan kinerja sistem.

\IfSubfilesClassLoaded{%
\printbibliography
}{}

\end{document}
