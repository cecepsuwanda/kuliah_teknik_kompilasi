% !TEX root = ../main.tex
\documentclass[../main.tex]{subfiles}
\addbibresource{../references.bib}
\begin{document}
\chapter{Analisis Semantik, Tabel Simbol \& Hash Table}
\label{ch:semantik-tabel-simbol}

\section{Analisis Semantik dan Atribut}
Analisis semantik memvalidasi properti makna program yang tidak tertangkap pada level sintaks, seperti pengecekan tipe, cakupan, dan konsistensi penggunaan identifier. Kerangka \textit{syntax-directed definition} dan sistem atribut memformalkan bagaimana informasi semantik dipropagasikan pada pohon sintaks.\cite{aho-dragon-book-2006} Pendekatan ini memungkinkan spesifikasi aturan yang sistematis dan dapat diuji.
Penerapan pada kompiler nyata mencakup pengetesan kompatibilitas tipe, pengikatan nama, serta deteksi side effect yang tidak diizinkan. Banyak sistem memanfaatkan \textit{abstract syntax tree} yang diperkaya anotasi untuk memfasilitasi optimisasi dan generasi kode berikutnya.\cite{pierce-types} Representasi yang kaya mendorong pemisahan kekhawatiran antara analisis dan transformasi.
Rancangan aturan semantik yang eksplisit menghasilkan pesan kesalahan yang lebih presisi dan dokumentatif, sehingga mendukung pengalaman pengembangan yang lebih baik. Dokumentasi yang baik atas aturan ini membantu konsistensi implementasi lintas versi bahasa.

\section{Tabel Simbol dan Struktur Data}
Tabel simbol menyimpan pemetaan dari nama ke informasi deklaratif seperti jenis, lingkup, offset, atau atribut lainnya. Struktur data hash table sering dipilih karena waktu akses rata-rata yang konstan, dengan mekanisme chaining atau open addressing untuk menangani tabrakan.\cite{aho-dragon-book-2006} Pemilihan strategi penanganan tabrakan memengaruhi jejak memori dan latensi pencarian.
Pengelolaan lingkup biasanya diimplementasikan sebagai stack dari tabel simbol, memungkinkan operasi push saat memasuki blok baru dan pop saat keluar. Praktik ini mendukung resolusi nama yang benar sesuai aturan visibilitas bahasa.\cite{muchnick} Pola ini juga memudahkan implementasi fitur seperti shadowing dan linkage antar unit kompilasi.
Desain tabel simbol yang modular memudahkan integrasi dengan fase analisis semantik serta optimisasi, dan menjadi fondasi bagi infrastruktur tooling seperti refactoring dan code completion. Antarmuka yang jelas mempercepat evolusi kompilator dan meminimalkan regresi.

\IfSubfilesClassLoaded{%
\printbibliography
}{}

\end{document}
