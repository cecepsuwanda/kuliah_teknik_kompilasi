% !TEX root = ../main.tex
\documentclass[../main.tex]{subfiles}
\addbibresource{../references.bib}
\begin{document}
\chapter{Bottom Up Parser – Canonical LR(0)}
\label{ch:lr0}

\section{Item LR(0) dan Closure}
Metode LR(0) membangun automata deterministik atas himpunan item yang merepresentasikan posisi titik pada produksi grammar. Operasi \textit{closure} menambahkan item yang diperlukan untuk mengantisipasi nonterminal setelah titik, sementara operasi \textit{goto} memindahkan titik melintasi simbol.\cite{aho-dragon-book-2006} Struktur ini menjadi dasar state machine yang memandu aksi parser.
Konstruksi kanonik menghasilkan kumpulan state yang memandu pembuatan tabel aksi dan goto untuk parser LR. Meskipun kuat, LR(0) murni sering menemui konflik shift/reduce atau reduce/reduce pada grammar praktis.\cite{grune-parsing} Konflik ini menunjukkan keterbatasan informasi konteks pada LR(0) dan memotivasi perluasan metode.
Pemahaman struktur item esensial untuk melanjutkan ke varian lebih kuat seperti SLR(1), LR(1), dan LALR(1) yang memperkaya konteks dengan lookahead. Peningkatan kekuatan ini meningkatkan cakupan grammar yang dapat diakui tanpa mengorbankan determinisme.

\section{Penyusunan Tabel LR(0)}
Tabel aksi diisi dengan operasi shift ketika transisi pada simbol terminal tersedia, sementara reduksi ditempatkan ketika titik berada pada akhir produksi. Keadaan awal yang menerima ditandai dengan produksi augmentasi S'→S.\cite{aho-dragon-book-2006} Mekanisme ini memetakan dinamika automata ke operasi parsing yang konkret.
Konflik pada LR(0) biasanya diselesaikan dengan menaikkan kekuatan metode (misal SLR(1)) atau merekayasa ulang grammar. Di lingkungan industri, generator parser seperti Yacc/Bison menyediakan laporan konflik yang memandu perbaikan.\cite{bison-manual} Analisis laporan konflik membantu mengidentifikasi produksi penyebab dan strategi mitigasi.
Keterbatasan LR(0) menegaskan kebutuhan akan informasi lookahead untuk menangkap konteks yang memadai pada grammar dunia nyata. Hal ini menjelaskan mengapa banyak compiler praktis beralih ke keluarga LR yang lebih kuat.

\IfSubfilesClassLoaded{%
\printbibliography
}{}

\end{document}
