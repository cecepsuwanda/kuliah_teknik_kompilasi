% !TEX root = ../main.tex
\documentclass[../main.tex]{subfiles}
\addbibresource{../references.bib}
\begin{document}
\chapter{TDP Non-recursive Descent Parser (LL(1))}
\label{ch:ll1}

\section{Karakteristik Grammar LL(1)}
Grammar LL(1) memungkinkan parsing deterministik dengan satu simbol lookahead, dengan pemilihan produksi berdasarkan tabel prediksi yang disusun dari himpunan FIRST dan FOLLOW. Syarat utama meliputi bebas ambiguitas lokal, tanpa left recursion, dan terfaktorkan sehingga produksi yang berbagi prefiks awal dipisahkan.\cite{aho-dragon-book-2006} Kriteria ini memastikan determinisme pilihan produksi pada setiap langkah parsing.
Kelebihan utama LL(1) adalah kesederhanaan implementasi dan kejelasan struktur kode parser yang menyerupai grammar. Keterbatasannya terletak pada daya ekspresif yang lebih sempit dibanding keluarga LR, sehingga beberapa grammar praktis memerlukan refaktorisasi.\cite{grune-parsing} Walau demikian, untuk banyak bahasa domain spesifik, LL(1) sudah memadai dan sangat mudah dirawat.
Ketaatan pada kriteria LL(1) memungkinkan deteksi kesalahan lebih awal dan pesan yang presisi karena jalur eksekusi parser unik untuk setiap nonterminal dalam konteks lookahead tertentu. Keterdugaan ini memudahkan pengajaran dan pembelajaran teknik parsing bagi mahasiswa.

\section{Konstruksi Tabel Prediksi}
Tabel prediksi dibangun dengan menempatkan produksi pada sel yang diindikasikan oleh terminal dalam FIRST dari bagian kanan produksi, dan terminal dalam FOLLOW ketika bagian kanan mampu menghasilkan epsilon. Prosedur ini memastikan bahwa setiap pasangan nonterminal-lookahead memiliki paling banyak satu produksi.\cite{aho-dragon-book-2006} Kepadatan tabel berhubungan dengan disiplin desain grammar.
Implementasi efektif menyimpan tabel sebagai peta dua tingkat dari nonterminal ke token, meminimalkan ruang untuk entri kosong dan mempercepat lookup. Pada praktiknya, generator parser seperti ANTLR mengautomasi proses ini dan memberikan diagnostik konflik yang membantu perbaikan grammar.\cite{parr-antlr} Diagnostik tersebut memberi sinyal lokasi konflik sehingga perbaikan dapat ditargetkan.
Kualitas tabel prediksi secara langsung tercermin pada kinerja dan keandalan parser; konflik mengindikasikan pelanggaran sifat LL(1) dan harus diselesaikan sebelum produksi. Pengujian regresi dengan set input representatif membantu menjaga kualitas setelah perubahan grammar.

\IfSubfilesClassLoaded{%
\printbibliography
}{}

\end{document}
