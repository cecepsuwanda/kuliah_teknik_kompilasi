% !TEX root = ../main.tex
\documentclass[../main.tex]{subfiles}
\addbibresource{../references.bib}
\begin{document}
\chapter{Struktur Kompiler}
\label{ch:struktur-kompiler}

\section{Besaran Leksikal dan Analisis Leksikal}
Analisis leksikal mengubah aliran karakter menjadi token-token bermakna seperti identifier, literal, operator, dan delimiter. Proses ini biasanya direpresentasikan oleh automata hingga (finite automata) yang diperoleh dari ekspresi regular melalui konstruksi formal, misalnya algoritma Thompson dan subset construction.\cite{aho-dragon-book-2006, hopcroft-ullman} Transformasi ini menjaga ketepatan pemisahan token dan mengurangi ambiguitas tahap awal.
Tokenisasi yang tepat membantu tahap berikutnya dengan menyaring whitespace, komentar, dan bentuk-bentuk leksikal ilegal. Implementasi praktis sering memanfaatkan generator lexer seperti Lex/Flex, atau komponen leksikal dari ANTLR, untuk menghasilkan scanner yang efisien.\cite{flex-manual, antlr-book} Generator tersebut menyederhanakan implementasi melalui spesifikasi pola deklaratif. Praktik terbaik menekankan pengujian kasus tepi leksikal secara sistematis untuk menjamin stabilitas.
Desain himpunan token dan aturan leksikal berdampak langsung pada kesederhanaan grammar sintaks. Aturan yang konsisten mengurangi ambiguitas dan backtracking pada parser, serta memudahkan deteksi kesalahan awal yang informatif bagi pengguna. Pemilihan batas token yang tepat mencegah pergeseran kompleksitas ke parser dan meningkatkan kualitas pelaporan kesalahan.

\section{Ekspresi Regular dan FSA}
Ekspresi regular merupakan formalitas untuk menyatakan himpunan string yang dikenali oleh automata hingga deterministik maupun nondeterministik. Konversi dari regex ke NFA, lalu ke DFA, dan minimisasi DFA membentuk landasan teoretis bagi pembentukan lexer.\cite{sipser} Korespondensi formal ini didasari oleh teorema Kleene dan menyediakan metode konstruksi yang sistematik.
Dalam praktik, kompiler mengadopsi variasi optimisasi seperti table-driven scanner, fallback state, dan teknik buffering ganda untuk mencapai throughput tinggi. Dokumentasi Flex menjelaskan strategi lookahead dan penanganan konflik prioritas antar pola.\cite{flex-manual} Teknik-teknik ini menyeimbangkan kecepatan dan konsumsi memori, serta mengelola prioritas aturan secara eksplisit.
Pemahaman korespondensi antara regex dan FSA membantu insinyur menulis aturan leksikal yang tidak ambigu dan terukur kinerjanya, sekaligus menurunkan kompleksitas pada tahap parsing yang sensitif terhadap bentuk token. Perancangan yang teliti menghindari jalur eksekusi yang bertele-tele di parser, sehingga keseluruhan pipeline kompilasi menjadi lebih prediktif. Dengan demikian, lexer yang dirancang dengan baik menurunkan beban disambiguasi pada parser dan meningkatkan stabilitas sistem.

\IfSubfilesClassLoaded{%
\printbibliography
}{}

\end{document}
