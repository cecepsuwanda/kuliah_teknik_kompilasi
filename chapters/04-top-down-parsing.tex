% !TEX root = ../main.tex
\documentclass[../main.tex]{subfiles}
% Ensure bibliography resource is available when compiled standalone
\addbibresource{../references.bib}
\begin{document}
\chapter{Top Down Parsing (TDP)}
\label{ch:tdp}

\section{Full Backtracking (Brute Force Method)}
Metode top-down dengan backtracking mencoba memperluas derivasi dari simbol awal menggunakan produksi yang mungkin, dan mundur ketika terjadi kebuntuan. Pendekatan ini sederhana secara konseptual namun tidak efisien karena eksplorasi ruang pencarian yang eksponensial pada grammar ambigu atau tidak terprediksi.\cite{aho-dragon-book-2006} Biaya tersebut sering tak dapat diterima untuk compiler produksi dan menghambat skalabilitas.
Analisis teoretis menunjukkan bahwa backtracking mengintroduksi nondeterminisme eksplisit pada prosedur parsing, sehingga sulit menjamin waktu respon. Karena itu, praktik modern menghindarinya kecuali untuk prototipe atau grammar kecil yang terkendali.\cite{parr-rl} Analisis kompleksitas menyoroti area yang rawan ledakan kombinatorial dan pentingnya pembatasan ruang pencarian.\cite{grune-parsing}
Sebagai alternatif, teknik pemfaktoran grammar dan penghilangan left recursion dapat mengurangi kebutuhan backtracking, menuntun ke parser yang deterministik dan lebih terukur secara kinerja. Teknik tersebut merupakan strategi inti untuk memperoleh grammar yang dapat diprediksi dalam lingkungan produksi.

\section{Tanpa Backtracking (Recursive Descent Parser)}
Recursive descent parser membangun fungsi parser per nonterminal dan memandu derivasi berdasarkan lookahead terbatas. Ketika grammar telah difaktorkan dan bebas left recursion, metode ini menjadi deterministik dan mudah diimplementasikan secara manual.\cite{parr-antlr} Struktur fungsi yang paralel dengan nonterminal meningkatkan keterbacaan implementasi.
Struktur kontrol yang eksplisit memudahkan penanganan kesalahan dengan pesan yang lebih informatif. Namun, kualitasnya sangat tergantung disiplin desain grammar dan ketersediaan set lookahead yang memadai.\cite{aho-dragon-book-2006} Desain pesan kesalahan yang baik meningkatkan pengalaman belajar dan efektivitas debugging.
Praktik industri sering mengombinasikan recursive descent dengan tabel prediksi (FIRST/FOLLOW) untuk memastikan pilihan produksi deterministik, sekaligus menjaga keterbacaan implementasi. Teknik ini juga cocok untuk bahasa domain spesifik dengan grammar sederhana.\cite{grune-parsing}

\IfSubfilesClassLoaded{%
\printbibliography
}{}

\end{document}
