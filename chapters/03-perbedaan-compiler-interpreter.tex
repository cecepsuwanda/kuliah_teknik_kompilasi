% !TEX root = ../main.tex
\documentclass[../main.tex]{subfiles}
\addbibresource{../references.bib}
\begin{document}
\chapter{Perbedaan Compiler dan Interpreter}
\label{ch:compiler-vs-interpreter}

\section{Pengantar CFG dan Derivasi}
Context-Free Grammar (CFG) menyatakan struktur sintaksis bahasa melalui himpunan produksi yang memetakan nonterminal ke rangkaian simbol. CFG memungkinkan perumusan struktur rekursif dan hirarkis, sehingga sesuai untuk menggambarkan bahasa pemrograman.\cite{aho-dragon-book-2006} Kerangka ini memudahkan analisis bentuk pohon parse dan AST yang merepresentasikan struktur program.
Derivasi dalam CFG dapat dianalisis melalui urutan penggantian simbol mulai dari simbol awal hingga menghasilkan string terminal. Dua bentuk derivasi yang umum adalah leftmost dan rightmost; keduanya mencerminkan strategi pengembangan pohon sintaks dan berkaitan erat dengan teknik parsing.\cite{sipser} Perbedaan ini penting bagi kompatibilitas teknik parsing yang berbeda dan memengaruhi bentuk pesan kesalahan.
Pemahaman dasar mengenai CFG dan derivasi menjadi prasyarat bagi analisis algoritme parsing, baik top-down maupun bottom-up, karena memengaruhi determinisme, ambiguïtas, dan kompleksitas proses penguraian. Sumber terbuka memberikan contoh formal beserta visualisasinya yang memperjelas konsep-konsep ini.\cite{grune-parsing}

\section{Spektrum Implementasi}
Pada implementasi, perbedaan compiler dan interpreter bukanlah dikotomi absolut, melainkan spektrum teknik yang mencakup pre-compilation, bytecode, virtual machine, hingga just-in-time compilation. Java dan .NET menggunakan bytecode portabel yang kemudian dioptimasi saat runtime, sementara Python tradisional mengandalkan bytecode dengan interpreter.\cite{jvm-spec, dotnet-ecma} Model tersebut memperlihatkan evolusi teknologi eksekusi program di berbagai platform.
Mesin JavaScript modern seperti V8 mengkombinasikan interpreter baseline dengan JIT optimising compiler yang beradaptasi terhadap profil eksekusi. Strategi ini memungkinkan waktu mulai yang cepat sekaligus throughput tinggi pada jalur panas.\cite{v8-design} Pendekatan adaptif ini memaksimalkan performa untuk beban kerja aktual dan mengurangi biaya cold-start.
Konsekuensi desain meliputi trade-off debugging, profiling, dan keamanan. Sistem dengan interpreter murni cenderung menyediakan introspeksi runtime yang kaya, sedangkan sistem kompilasi ahead-of-time memaksimalkan validasi statik dan kinerja native. Keamanan runtime dapat ditingkatkan dengan verifikasi bytecode dan sandboxing, sementara kebutuhan observabilitas harus dijaga agar diagnosis kesalahan efektif.

\IfSubfilesClassLoaded{%
\printbibliography
}{}

\end{document}
