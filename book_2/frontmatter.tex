\frontmatter

% ============================================================
% Halaman Sampul
% ============================================================
\begin{titlepage}
\centering
\vspace*{1.5cm}

{\fontsize{18}{22}\selectfont \textbf{BUKU AJAR}}\\[0.8cm]
{\fontsize{20}{24}\selectfont \textbf{TEKNIK KOMPILASI}}\\[0.5cm]
{\fontsize{14}{18}\selectfont Berbasis Outcome-Based Education (OBE)}\\[0.3cm]
{\fontsize{14}{18}\selectfont Praktis dengan C/C++}\\[2.5cm]

{\fontsize{12}{16}\selectfont Oleh}\\[0.5cm]
{\fontsize{14}{18}\selectfont \textbf{[Nama Dosen Pengampu]}}\\[2cm]

{\fontsize{12}{16}\selectfont \textit{Digunakan di lingkungan sendiri, sebagai buku ajar mata kuliah\\
\textbf{Teknik Kompilasi} pada Program Studi S1 Teknik Informatika}}\\[2.5cm]

{\fontsize{14}{18}\selectfont \textbf{Program Studi S1 Teknik Informatika}}\\[0.3cm]
{\fontsize{14}{18}\selectfont Fakultas Teknik}\\[0.3cm]
{\fontsize{14}{18}\selectfont Universitas}\\[2cm]

{\fontsize{12}{16}\selectfont \textbf{\the\year}}
\end{titlepage}

% ============================================================
% Halaman Informasi Buku
% ============================================================
\newpage
\thispagestyle{empty}
\vspace*{1.5cm}

\begin{center}
{\fontsize{16}{20}\selectfont \textbf{INFORMASI BUKU}}\\[1.5cm]
\end{center}

\vspace{0.5cm}

\begin{sloppypar}
\begin{center}
\begin{tabularx}{\dimexpr\textwidth-18pt\relax}{@{}>{\small\raggedright\arraybackslash}p{3cm}>{\small\raggedright\arraybackslash}X@{}}
\textbf{Judul} & : Buku Ajar Teknik Kompilasi\\
\textbf{Subjudul} & : Berbasis Outcome-Based Education (OBE),\\ & Praktis dengan C/C++\\
\textbf{Penulis} & : [Nama Dosen Pengampu]\\
\textbf{Program Studi} & : S1 Teknik Informatika\\
\textbf{Fakultas} & : Fakultas Teknik\\
\textbf{Universitas} & : Universitas\\
\textbf{Mata Kuliah} & : Teknik Kompilasi\\
\textbf{SKS} & : 3 SKS\\
\textbf{Pertemuan} & : 16 Pertemuan\\
\textbf{Tahun} & : \the\year\\
\textbf{ISBN} & : [ISBN jika ada]\\
\end{tabularx}
\end{center}\par

\vspace{1.5cm}

\begin{center}
\begin{minipage}{0.85\textwidth}
\raggedright
{\fontsize{10}{13}\selectfont \textit{Buku ajar ini disusun sebagai bahan pembelajaran untuk mata kuliah \textbf{Teknik Kompilasi} pada Program Studi S1 Teknik Informatika. Buku ini dirancang mengikuti pendekatan \textbf{Outcome-Based Education (OBE)} dengan fokus pada pembelajaran berbasis praktik.}}
\end{minipage}
\end{center}
\end{sloppypar}

\vspace*{\fill}

\cleardoublepage

% ============================================================
% Kata Pengantar (Prakata)
% ============================================================
\subfile{chapters/bab-00}
\cleardoublepage

% ============================================================
% Cara Menggunakan Buku Ini
% ============================================================
\chapter*{Cara Menggunakan Buku Ini}
\addcontentsline{toc}{chapter}{Cara Menggunakan Buku Ini}

Buku ajar ini dirancang dengan pendekatan OBE untuk memaksimalkan pencapaian pembelajaran Anda. Berikut panduan penggunaan buku ini:

\section*{Struktur Buku}

\textbf{Bab I: Pendahuluan dan Orientasi}\\
Memperkenalkan tujuan buku, keterkaitan dengan RPS, dan konteks kurikulum OBE.

\textbf{Bab II: Landasan Teori}\\
Menyajikan fondasi teoretis Teknik Kompilasi yang menjadi basis pembelajaran seluruh bab berikutnya.

\textbf{Bab III-XIX: Unit Materi Inti}\\
Setiap bab mencakup satu topik utama Teknik Kompilasi dengan struktur lengkap: Sub-CPMK, materi, aktivitas, latihan, asesmen, dan checklist.

\textbf{Bab Lainnya: Evaluasi dan Integrasi}\\
Berisi asesmen komprehensif dan panduan refleksi untuk mengukur pencapaian kompetensi secara menyeluruh.

\section*{Komponen dalam Setiap Bab}

\begin{enumerate}
  \item \textbf{Sub-CPMK}: Baca dengan seksama untuk memahami kompetensi yang harus dicapai
  \item \textbf{Materi Pokok}: Pelajari dengan cermat, jalankan semua contoh kode
  \item \textbf{Aktivitas Pembelajaran}: Lakukan secara mandiri atau berkelompok
  \item \textbf{Latihan}: Kerjakan untuk menguji pemahaman Anda
  \item \textbf{Asesmen}: Gunakan untuk mengukur pencapaian Sub-CPMK
  \item \textbf{Checklist}: Centang setelah yakin menguasai setiap indikator
\end{enumerate}

\cleardoublepage

% ============================================================
% Identitas Mata Kuliah
% ============================================================
\chapter*{Identitas Mata Kuliah}
\addcontentsline{toc}{chapter}{Identitas Mata Kuliah}

\begin{tabular}{ll}
  Nama Program Studi & : S1 Teknik Informatika \\
  Nama Mata Kuliah & : Teknik Kompilasi \\
  Kode Mata Kuliah & : [Kode MK] \\
  Semester & : [Semester] \\
  SKS / Bobot Kredit & : 3 SKS \\
  Dosen Pengampu & : [Nama Dosen Pengampu] \\
  Tanggal Penyusunan & : \today \\
\end{tabular}

\vspace{1cm}

\section*{Capaian Pembelajaran Lulusan (CPL)}

CPL yang dibebankan pada mata kuliah ini mencakup kompetensi lulusan dalam aspek pengetahuan, keterampilan, dan sikap sesuai dengan kurikulum OBE.

\section*{Capaian Pembelajaran Mata Kuliah (CPMK)}

Kemampuan atau kompetensi spesifik yang diharapkan mahasiswa kuasai setelah menyelesaikan mata kuliah:

\begin{enumerate}
  \item \textbf{CPMK-1:} Mampu menjelaskan fase-fase kompilasi dan arsitektur kompilator.
  \item \textbf{CPMK-2:} Mampu menerapkan teori bahasa formal dalam analisis leksikal dan sintaksis.
  \item \textbf{CPMK-3:} Mampu mengimplementasikan komponen kompilator sederhana menggunakan C/C++.
  \item \textbf{CPMK-4:} Mampu melakukan optimasi kode dan menangani error dalam proses kompilasi.
\end{enumerate}

\cleardoublepage

% ============================================================
% Daftar Isi
% ============================================================
\phantomsection
\addcontentsline{toc}{chapter}{Daftar Isi}
\tableofcontents
\cleardoublepage

\phantomsection
\addcontentsline{toc}{chapter}{Daftar Gambar}
\listoffigures
\cleardoublepage

\phantomsection
\addcontentsline{toc}{chapter}{Daftar Tabel}
\listoftables
\cleardoublepage
