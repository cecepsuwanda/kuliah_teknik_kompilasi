\section{Soal Ujian Akhir Semester (UAS)}

\subsection{Petunjuk}
\begin{itemize}
    \item Waktu ujian: 150 menit
    \item Materi: Bab 8 sampai Bab 15
    \item Jawablah semua soal dengan jelas dan lengkap
    \item Gunakan diagram, pseudocode, atau ilustrasi jika diperlukan
    \item Soal bersifat integratif, menghubungkan konsep dari berbagai bab
\end{itemize}

\subsection{Soal UAS}

\begin{enumerate}
    \item \textbf{[Bab 8 - 20 poin]} 
    \begin{enumerate}
        \item Buatlah spesifikasi grammar untuk Bison/Yacc yang dapat mem-parsing ekspresi aritmatika dengan operator \texttt{+}, \texttt{-}, \texttt{*}, \texttt{/}, dan tanda kurung. Grammar harus menghormati precedence dan associativity yang benar. Sertakan semantic actions untuk membangun AST.
        
        \item Jelaskan bagaimana Bison menangani shift-reduce conflict dan reduce-reduce conflict. Berikan contoh grammar yang menyebabkan masing-masing conflict dan jelaskan cara mengatasinya.
        
        \item Implementasikan kalkulator sederhana menggunakan Flex dan Bison yang dapat mengevaluasi ekspresi aritmatika. Sertakan error handling untuk kesalahan sintaks.
    \end{enumerate}
    
    \item \textbf{[Bab 9 - 20 poin]}
    \begin{enumerate}
        \item Rancang struktur data AST untuk bahasa sederhana yang mendukung:
        \begin{itemize}
            \item Deklarasi variabel: \texttt{int x;}
            \item Assignment: \texttt{x = 5;}
            \item Ekspresi aritmatika: \texttt{a + b * c}
            \item Pernyataan if-else: \texttt{if (x > 0) \{ ... \} else \{ ... \}}
            \item Perulangan while: \texttt{while (x < 10) \{ ... \}}
        \end{itemize}
        Gunakan class hierarchy dalam C++ atau struct dengan variant types.
        
        \item Implementasikan fungsi untuk melakukan post-order traversal pada AST. Jelaskan mengapa traversal ini penting untuk code generation.
        
        \item Buat fungsi visualisasi sederhana untuk mencetak AST dalam format tree. Berikan contoh output untuk ekspresi \texttt{(a + b) * c - d}.
    \end{enumerate}
    
    \item \textbf{[Bab 10 - 20 poin]}
    \begin{enumerate}
        \item Rancang implementasi symbol table yang efisien untuk bahasa dengan nested scope. Struktur data harus mendukung:
        \begin{itemize}
            \item Insert
            \item Lookup
            \item Enter scope
            \item Exit scope
        \end{itemize}
        Jelaskan pilihan struktur data dan kompleksitas operasinya.
        
        \item Implementasikan name resolution untuk program berikut dan jelaskan hasil resolusi setiap identifier:
        \begin{verbatim}
        int x = 1;
        {
            int y = 2;
            {
                int x = 3;
                y = x + y;
            }
            x = y;
        }
        \end{verbatim}
        
        \item Jelaskan perbedaan static scoping dan dynamic scoping beserta contoh program.
    \end{enumerate}
    
    \item \textbf{[Bab 11 - 20 poin]}
    \begin{enumerate}
        \item Implementasikan type checker sederhana.
        \item Jelaskan konsep type inference dan bandingkan dengan explicit type checking.
        \item Rancang algoritma semantic analysis.
    \end{enumerate}
    
    \item \textbf{[Bab 12 - 25 poin]}
    \begin{enumerate}
        \item Implementasikan generator three-address code (TAC) dari AST.
        \item Konversi ekspresi berikut ke three-address code:
        \begin{verbatim}
        x = (a + b) * (c - d) / e;
        \end{verbatim}
        \item Implementasikan representasi quadruples.
        \item Jelaskan perbedaan TAC, quadruples, dan triples.
    \end{enumerate}
    
    \item \textbf{[Bab 13 - 20 poin]}
    \begin{enumerate}
        \item Jelaskan konsep activation record.
        \item Trace eksekusi fungsi rekursif \texttt{factorial(3)}.
        \item Bandingkan stack-based dan heap-based memory management.
    \end{enumerate}
    
    \item \textbf{[Bab 14 - 25 poin]}
    \begin{enumerate}
        \item Implementasikan code generator sederhana.
        \item Konversi TAC ke assembly-like code.
        \item Jelaskan algoritma graph coloring untuk register allocation.
        \item Bandingkan one-pass dan multi-pass code generation.
    \end{enumerate}
    
    \item \textbf{[Bab 15 - 20 poin]}
    \begin{enumerate}
        \item Implementasikan optimasi dasar pada TAC.
        \item Jelaskan data flow analysis.
        \item Optimasi kode yang diberikan.
    \end{enumerate}
    
    \item \textbf{[Integratif - 30 poin]}
    Rancang dan implementasikan kompilator sederhana untuk bahasa mini.
    
    \item \textbf{[Integratif - 20 poin]}
    Analisis kompilator modern (GCC, Clang, atau lainnya) dan bandingkan dengan kompilator sederhana pada soal integratif sebelumnya.
\end{enumerate}

\vspace{1cm}
\textbf{Total: 220 poin}
