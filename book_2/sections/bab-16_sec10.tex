\section{Refleksi Pembelajaran}

Refleksi adalah bagian penting dari proses pembelajaran. Mahasiswa diharapkan melakukan refleksi terhadap:

\subsection{Technical Skills Acquired}

\begin{itemize}
    \item \textbf{Lexical Analysis}: Kemampuan mengimplementasikan lexer
    \item \textbf{Parsing}: Pemahaman tentang grammar dan parsing techniques
    \item \textbf{Semantic Analysis}: Kemampuan melakukan type checking dan scope resolution
    \item \textbf{Code Generation}: Kemampuan menghasilkan target code
    \item \textbf{Optimization}: Pemahaman tentang optimasi kompilator
    \item \textbf{Software Engineering}: Kemampuan mengintegrasikan komponen-komponen besar
\end{itemize}

\subsection{Challenges Faced}

Identifikasi tantangan yang dihadapi:
\begin{itemize}
    \item Technical challenges (implementasi, debugging)
    \item Design challenges (trade-offs, architecture decisions)
    \item Time management challenges
    \item Team collaboration challenges (jika project team-based)
\end{itemize}

\subsection{Lessons Learned}

Rangkum pembelajaran:
\begin{itemize}
    \item Apa yang bekerja dengan baik?
    \item Apa yang tidak bekerja seperti yang diharapkan?
    \item Apa yang akan dilakukan berbeda jika memulai lagi?
    \item Insight tentang compiler design dan implementation
\end{itemize}

\subsection{Areas for Improvement}

Identifikasi area untuk improvement:
\begin{itemize}
    \item Fitur yang belum diimplementasikan
    \item Optimasi yang dapat ditambahkan
    \item Error handling yang dapat ditingkatkan
    \item Dokumentasi yang dapat diperbaiki
    \item Testing yang dapat diperluas
\end{itemize}