\section{Persiapan Presentasi Project Final}

Presentasi project final adalah kesempatan untuk menunjukkan hasil kerja keras selama satu semester. Berikut adalah panduan untuk mempersiapkan presentasi yang efektif:

\subsection{Struktur Presentasi}

Presentasi sebaiknya mengikuti struktur berikut:

\textbf{1. Pendahuluan (5 menit)}
\begin{itemize}
    \item Perkenalan tim dan project
    \item Tujuan dan scope bahasa yang dikompilasi
    \item Overview arsitektur compiler
\end{itemize}

\textbf{2. Demo Compiler (10 menit)}
\begin{itemize}
    \item Live demonstration: compile dan run program contoh
    \item Menunjukkan berbagai fitur bahasa yang didukung
    \item Menampilkan error handling yang baik
\end{itemize}

\textbf{3. Arsitektur dan Implementasi (15 menit)}
\begin{itemize}
    \item Penjelasan setiap fase kompilasi
    \item Pilihan desain dan justifikasinya
    \item Tools dan teknik yang digunakan
    \item Tantangan yang dihadapi dan solusinya
\end{itemize}

\textbf{4. Evaluasi dan Analisis (10 menit)}
\begin{itemize}
    \item Perbandingan hand-written vs generator tools
    \item Trade-off analysis
    \item Benchmark hasil kompilasi
    \item Evaluasi kualitas kode yang dihasilkan
\end{itemize}

\textbf{5. Kesimpulan dan Refleksi (5 menit)}
\begin{itemize}
    \item Lesson learned
    \item Area untuk improvement
    \item Kesimpulan
\end{itemize}

\textbf{6. Q\&A (5 menit)}

\subsection{Tips Presentasi yang Efektif}

\begin{enumerate}
    \item \textbf{Persiapan Demo}: Pastikan demo berjalan lancar dengan test cases yang sudah dipersiapkan
    \item \textbf{Visual Aids}: Gunakan diagram arsitektur, flowchart, dan contoh kode yang jelas
    \item \textbf{Time Management}: Latih presentasi untuk memastikan sesuai waktu yang dialokasikan
    \item \textbf{Anticipate Questions}: Siapkan jawaban untuk pertanyaan umum tentang desain dan implementasi
    \item \textbf{Show Passion}: Tunjukkan antusiasme terhadap project yang telah dibangun
\end{enumerate}