\section{Parse Tree}

Parse tree (juga disebut derivation tree atau concrete syntax tree) adalah representasi visual dari bagaimana string diturunkan dari grammar.

\subsection{Struktur Parse Tree}

Parse tree memiliki struktur berikut:
\begin{itemize}
    \item \textbf{Root}: Labeled dengan start symbol \(S\)
    \item \textbf{Internal nodes}: Nonterminal symbols
    \item \textbf{Leaves}: Terminal symbols (dari kiri ke kanan membentuk input string)
    \item \textbf{Edges}: Menunjukkan aplikasi aturan produksi
\end{itemize}

\subsection{Contoh Parse Tree}

Untuk ekspresi \texttt{3 + 4 * 5} dengan grammar sebelumnya, parse tree-nya adalah:

\begin{figure}[!htbp]
\centering
\begin{forest}
[E
    [E
        [T
            [F [3]]
        ]
    ]
    [+]
    [T
        [T
            [F [4]]
        ]
        [*]
        [F [5]]
    ]
]
\end{forest}
\caption{Parse tree untuk ekspresi \texttt{3 + 4 * 5}}
\label{fig:parse-tree-3+4*5}
\end{figure}

\subsection{Parse Tree vs Abstract Syntax Tree (AST)}

Perbedaan penting antara parse tree dan AST:

\begin{itemize}
    \item \textbf{Parse Tree (Concrete Syntax Tree)}:
    \begin{itemize}
        \item Mencakup semua detail sintaksis, termasuk punctuation
        \item Setiap node sesuai dengan aturan produksi
        \item Lebih verbose, mencakup informasi yang tidak diperlukan untuk fase selanjutnya
    \end{itemize}
    
    \item \textbf{Abstract Syntax Tree (AST)}:
    \begin{itemize}
        \item Menghilangkan detail sintaksis yang tidak relevan (seperti parentheses grouping yang sudah jelas dari struktur)
        \item Fokus pada struktur semantik program
        \item Lebih kompak dan efisien untuk analisis semantik dan code generation
    \end{itemize}
\end{itemize}

Contoh: Untuk ekspresi \texttt{3 + 4 * 5}, AST-nya lebih sederhana:

\begin{figure}[!htbp]
\centering
\begin{forest}
for tree={
    grow'=0,
    child anchor=west,
    parent anchor=east,
    anchor=west,
    calign=first,
    edge path={
        \noexpand\path[\forestoption{edge}]
        (!u.south west) +(7.5pt,0) |- (.child anchor) \forestoption{edge label};
    },
    before typesetting nodes={
        if n=1
          {insert before={[,phantom]}}
          {}
    },
    fit=band,
    before computing xy={l=15pt},
}
[+
    [3]
    [*
        [4]
        [5]
    ]
]
\end{forest}
\caption{AST untuk ekspresi \texttt{3 + 4 * 5}}
\label{fig:ast-3+4*5}
\end{figure}

AST menghilangkan node-node intermediate seperti \(E\), \(T\), \(F\) yang tidak diperlukan untuk pemahaman semantik.