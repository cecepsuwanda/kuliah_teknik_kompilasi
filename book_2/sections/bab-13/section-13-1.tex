\section{Alokasi Register: Strategi dan Kompleksitas}

Karena jumlah register fisik dalam CPU sangat terbatas, kompilator harus memutuskan variabel mana yang layak menempati register dan mana yang harus dipindahkan (\textit{spilled}) ke RAM.

\subsection{Graph Coloring}
Masalah alokasi register dapat dimodelkan sebagai pewarnaan graf (\textit{Graph Coloring}).
\begin{itemize}
    \item Node: Variabel yang aktif (\textit{live variables}).
    \item Edge: Saling berpotongan (\textit{interfere}), artinya dua variabel hidup di waktu yang sama.
    \item Warna: Jumlah register fisik yang tersedia.
\end{itemize}

\subsection{Linear Scan}
Untuk kompilasi yang cepat (\textit{JIT compilers}), algoritma \textit{Linear Scan} digunakan karena lebih sederhana dibanding \textit{graph coloring} namun tetap memberikan hasil yang memadai.
