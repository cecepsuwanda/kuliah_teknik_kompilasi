\section{Alokasi Register: Strategi dan Kompleksitas}

Karena jumlah register fisik dalam CPU sangat terbatas, kompilator harus memutuskan variabel mana yang layak menempati register dan mana yang harus dipindahkan (\textit{spilled}) ke RAM.

\subsection{Kompleksitas Pewarnaan Graf}
Masalah alokasi register dapat dimodelkan sebagai pewarnaan graf (\textit{Graph Coloring}).
\begin{itemize}
    \item \textbf{Node}: Variabel yang aktif (\textit{live variables}).
    \item \textbf{Edge}: Saling berpotongan (\textit{interfere}), artinya dua variabel hidup di waktu yang sama.
    \item \textbf{Warna (k)}: Jumlah register fisik yang tersedia.
\end{itemize}

Secara teoretis, menentukan apakah sebuah graf dapat diwarnai dengan $k$ warna adalah masalah yang bersifat \textbf{NP-Complete}. Ini berarti untuk fungsi yang besar dengan banyak variabel, menemukan solusi optimal secara matematis akan memakan waktu yang sangat lama.

\subsection{Pendekatan Heuristik}
Kompilator praktis menggunakan algoritma heuristik seperti \cite{jhu2024compilers} yang memberikan hasil "cukup baik" dalam waktu linear atau polinomial rendah.
\begin{enumerate}
    \item \textbf{Chaitin-Briggs}: Menggunakan penyederhanaan graf secara iteratif.
    \item \textbf{Linear Scan}: Digunakan untuk kompilasi yang cepat (\textit{JIT compilers}) seperti pada V8 (JavaScript) atau JVM HotSpot, karena lebih sederhana dibanding \textit{graph coloring} namun tetap memberikan performa memadai.
\end{enumerate}

\begin{figure}[!htbp]
    \centering
    \adjustbox{max width=0.8\textwidth,center}{%
    \begin{tikzpicture}[
        node/.style={rectangle, draw=purple!50, fill=purple!10, text width=6cm, font=\tiny, align=center}
    ]
    \node[node] (comp) {NP-Complete $\rightarrow$ Tidak ada algoritma efisien untuk solusi absolut.};
    \node[node, below=0.5cm of comp] (heur) {Solusi: Heuristik (Chaitin) yang memprioritaskan variabel tersering digunakan.};
    \end{tikzpicture}%
    }
    \caption{Tantangan Teoretis dalam Alokasi Register}
\end{figure}
