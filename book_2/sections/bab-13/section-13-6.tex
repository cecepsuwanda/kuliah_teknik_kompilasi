\section{Penggabungan (Coalescing)}

\compiler{Coalescing} adalah teknik untuk menghilangkan instruksi salin (\code{MOVE}) yang tidak perlu dengan memberikan register yang sama kepada sumber dan target salinan.

\subsection{Heuristik Penggabungan yang Aman}
Penggabungan agresif dapat membuat graf yang awalnya bisa diwarnai menjadi tidak bisa diwarnai (\textit{uncolorable}). Oleh karena itu, kompilator menggunakan dua kriteria utama:
\begin{enumerate}
    \item \textbf{Kriteria Briggs}: Dua simpul dapat digabungkan jika simpul hasil penggabungan memiliki kurang dari $k$ tetangga yang memiliki derajat $\ge k$. Ini menjamin fase \textit{Simplify} masih bisa berjalan lancar.
    \item \textbf{Kriteria George}: Simpul $U$ dan $V$ dapat digabungkan jika untuk setiap tetangga $T$ dari $V$, $T$ sudah bertetangga dengan $U$ atau $T$ memiliki derajat rendah ($< k$).
\end{enumerate}

\subsection{Manfaat}
Selain mengurangi jumlah instruksi, \textit{coalescing} juga mengurangi kebutuhan register secara keseluruhan jika banyak variabel temporer yang sebenarnya hanya merupakan salinan dari variabel lain.

\begin{figure}[!htbp]
    \centering
    \adjustbox{max width=0.8\textwidth,center}{%
    \begin{tikzpicture}[
        node/.style={circle, draw, minimum size=0.6cm, font=\tiny}
    ]
    \node[node] (x) {x};
    \node[node, right=1.5cm of x] (t) {t1};
    \draw[->, double] (x) -- node[above, font=\tiny] {MOV} (t);
    \node[node, right=3.5cm of x] (combined) {x/t1};
    \node[font=\tiny] at (2.5, 0) {$\Rightarrow$};
    \end{tikzpicture}%
    }
    \caption{Coalescing: Menghapus MOVE dengan menyatukan node dalam graf interferensi}
\end{figure}
