\section{Algoritma Graph Coloring (Chaitin-Briggs)}

Algoritma ini adalah standar dalam kompilator produksi karena mampu memberikan alokasi global yang efisien melalui pendekatan terstruktur.

\subsection{Simpul Terwarna-Awal (Pre-colored Nodes)}
Dalam graf interferensi, terdapat simpul yang mewakili register fisik mesin (seperti \%rax, \%rbx pada x86). Simpul-simpul ini disebut \compiler{Pre-colored Nodes}.
\begin{itemize}
    \item \textbf{Alasan}: Beberapa instruksi mengharuskan penggunaan register tertentu (contoh: hasil fungsi selalu di \%rax).
    \item \textbf{Penanganan}: Simpul terwarna-awal tidak boleh "disederhanakan" (\textit{Simplify}) atau "dipindahkan ke memori" (\textit{Spill}). Mereka bertindak sebagai kendala tetap dalam proses pewarnaan simpul variabel lainnya.
\end{itemize}

\subsection{Fase Select dan Backtracking}
Setelah fase \textit{Simplify} mengosongkan graf (atau memindahkan beberapa simpul ke \textit{spill stack}), algoritma memasuki fase \compiler{Select}:
\begin{enumerate}
    \item Ambil simpul dari stack satu per satu.
    \item Assign warna (register) yang tidak digunakan oleh tetangganya.
    \item \textbf{Optimistic Coloring}: Briggs mengusulkan agar simpul yang awalnya ditandai \textit{spill} tetap dicoba diwarnai pada fase ini. Terkadang, setelah tetangganya diwarnai, ternyata masih ada warna tersisa yang bisa digunakan, sehingga \textit{spill} nyata bisa dihindari.
\end{enumerate}

\begin{figure}[!htbp]
    \centering
    \adjustbox{max width=0.8\textwidth,center}{%
    \begin{tikzpicture}[
        node/.style={circle, draw, minimum size=0.8cm, font=\tiny}
    ]
    \node[node, fill=red!20] (rax) {\%rax};
    \node[node, right=1.5cm of rax] (v1) {v1};
    \node[node, below=1cm of rax] (v2) {v2};
    
    \draw (rax) -- (v1);
    \draw (v1) -- (v2);
    \node[red, font=\tiny, anchor=south] at (rax.north) {Pre-colored};
    \end{tikzpicture}%
    }
    \caption{Kendala Register Fisik: v1 tidak boleh menggunakan \%rax karena interferensi}
\end{figure}
