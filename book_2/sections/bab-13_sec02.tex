\section{Pendahuluan}

Runtime environment adalah konteks di mana program yang telah dikompilasi dieksekusi. Menurut sumber dari StudyLib:

\begin{quote}
``Runtime environment: stack, heap, activation records; garbage collection intro. Managing run-time structures (activation records, memory layout, symbol tables).''\cite{studylib2024obe}
\end{quote}

Runtime environment mencakup:
\begin{itemize}
    \item \textbf{Memory Organization}: Bagaimana memory diorganisir dan dialokasikan untuk berbagai jenis data
    \item \textbf{Calling Conventions}: Mekanisme pemanggilan fungsi, passing parameter, dan return values
    \item \textbf{Scope Management}: Bagaimana variabel diakses berdasarkan scope-nya (local, non-local, global)
    \item \textbf{Memory Management}: Alokasi dan dealokasi memory untuk variabel dan data structures
\end{itemize}

Gambar \ref{fig:runtime-components} menunjukkan komponen-komponen runtime environment.

\begin{figure}[!htbp]
    \centering
    \adjustbox{max width=0.9\textwidth,center}{%
    \begin{tikzpicture}[
        comp/.style={rectangle, draw=blue!50, fill=blue!10, text width=2.5cm, text centered, minimum height=0.7cm, rounded corners, font=\footnotesize, inner sep=4pt, align=center},
        arrow/.style={->, >=stealth, thick},
        node distance=1.2cm
    ]
    
    \node[comp] (stack) {Stack};
    \node[comp, right=of stack] (heap) {Heap};
    \node[comp, below=of stack] (static) {Static\\Data};
    \node[comp, right=of static] (code) {Code\\Segment};
    
    \node[below=0.3cm of static, font=\tiny, align=center] {Runtime Environment};
    
    \end{tikzpicture}%
    }
    \caption{Komponen-komponen runtime environment}
    \label{fig:runtime-components}
\end{figure}

Runtime environment harus dirancang dengan hati-hati karena mempengaruhi:
\begin{itemize}
    \item Efisiensi eksekusi program
    \item Keamanan memory (memory safety)
    \item Kemampuan mendukung fitur bahasa (recursion, nested functions, closures, dll.)
    \item Portabilitas antar platform
\end{itemize}