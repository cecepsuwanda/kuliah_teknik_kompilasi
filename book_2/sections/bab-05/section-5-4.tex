\section{Penanganan Scope Entry dan Exit}

\subsection{Function Scope}
Saat mendeklarasikan fungsi, parser harus membuat scope baru untuk menampung parameter fungsi dan variabel lokal di dalamnya.

\subsection{Block Scope}
Setiap blok kode yang dibatasi kurung kurawal menciptakan scope sementara. Kompilator memanggil \texttt{beginScope()} saat menemukan \code{\{} dan \texttt{endScope()} saat menemukan \code{\}}.

\subsection{Contoh Kasus}
\begin{lstlisting}[language=C]
int x = 1; // Global
void f() {
    int x = 2; // Shadows global x
    {
        int x = 3; // Shadows f's x
    }
}
\end{lstlisting}
Symbol table akan mengelola tiga versi variabel \texttt{x} tersebut pada level nesting yang berbeda.
