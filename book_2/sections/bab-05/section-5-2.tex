\section{Struktur Data: Hash Table dan Scope Stack}

\subsection{Pilihan Struktur Data}
Kinerja kompilator sangat bergantung pada kecepatan Symbol Table. Mengapa? Karena setiap kali parser menemukan identifier, ia harus melakukan \textit{lookup}.
\begin{itemize}
    \item \textbf{Linear List}: $O(N)$. Sangat lambat, hanya cocok untuk bahasa mainan.
    \item \textbf{Binary Search Tree (BST)}: $O(\log N)$. Cukup cepat, tapi butuh penyeimbangan (AVL/Red-Black) agar tidak terdegradasi menjadi linked list.
    \item \textbf{Hash Table}: $O(1)$ rata-rata. Ini adalah standar industri. Dengan fungsi hash yang baik, akses ke ribuan variabel tetap instan.
\end{itemize}

\subsection{Manajemen Scope: The Cactus Stack}
Bahasa modern mendukung \textit{Nested Scopes} (blok di dalam blok). Struktur data yang paling tepat untuk ini adalah \textbf{Cactus Stack} (atau \textit{Chained Symbol Tables}).
\begin{itemize}
    \item Setiap scope memiliki Hash Table sendiri.
    \item Hash Table scope anak memiliki pointer \texttt{parent} ke Hash Table scope luar.
    \item Pencarian dimulai dari tabel saat ini. Jika tidak ketemu, lanjut ke \texttt{parent}, terus hingga \texttt{Global Scope} (yang parent-nya \texttt{NULL}).
\end{itemize}

\begin{figure}[!htbp]
    \centering
    \adjustbox{max width=0.8\textwidth,center}{%
    \begin{tikzpicture}[
        table/.style={rectangle, draw=blue!50, fill=blue!10, text width=2.5cm, minimum height=1.5cm, font=\tiny, align=center},
        arrow/.style={->, >=stealth, thick}
    ]
    \node[table] (global) {Global Scope\\ \texttt{int x}};
    \node[table, below left=1cm of global] (func) {Function Scope\\ \texttt{int y}};
    \node[table, below right=1cm of func] (block) {Block Scope\\ \texttt{int z}};
    
    \draw[arrow] (func) -- node[right, font=\tiny] {parent} (global);
    \draw[arrow] (block) -- node[right, font=\tiny] {parent} (func);
    
    \node[below=0.2cm of block, font=\itshape\footnotesize] {Lookup z: Found in Block};
    \node[right=0.2cm of block, font=\itshape\footnotesize] {Lookup x: Block $\to$ Func $\to$ Global (Found)};
    \end{tikzpicture}%
    }
    \caption{Ilustrasi Chained Symbol Tables (Cactus Stack)}
\end{figure}
