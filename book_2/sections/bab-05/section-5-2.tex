\section{Struktur Data: Hash Table dan Scope Stack}

Agar operasi pencarian (\textit{lookup}) berjalan efisien, symbol table biasanya diimplementasikan menggunakan \textbf{Hash Table}.

\subsection{Implementasi Hash Table}
Dalam C++, kita dapat menggunakan \texttt{std::unordered\_map} untuk memetakan nama identifier ke objek \texttt{Symbol}.

\begin{lstlisting}[language=C++]
struct Symbol {
    string name;
    string type;
    int line;
};

class Scope {
    unordered_map<string, Symbol*> table;
    Scope* parent;
public:
    Symbol* lookup(string name) {
        if (table.count(name)) return table[name];
        if (parent) return parent->lookup(name);
        return nullptr;
    }
};
\end{lstlisting}

\subsection{Manajemen Scope Stack}
Setiap kali kompilator menemukan blok baru (\texttt{\{}), scope baru dibuat dan ditumpuk di atas scope sebelumnya. Saat keluar dari blok (\texttt{\}}), scope teratas akan dihapus atau dinonaktifkan.
