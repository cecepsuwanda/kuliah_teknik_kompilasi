\section{Dasar Symbol Table dan Perannya}

\subsection{Definisi dan Fungsi}
\compiler{Symbol Table} adalah struktur data sentral yang berfungsi sebagai kamus pintar bagi kompilator. Ia menjembatani jurang antara nama simbolik (\code{x}, \code{count}) yang dipahami manusia dengan entitas memori (\code{0x0040}, \code{R5}) yang dipahami mesin. Tanpa Symbol Table, kompilator ''buta'' terhadap konteks variabel.

\subsection{Informasi yang Disimpan (Attributes)}
Entri dalam symbol table tidak sekadar menyimpan nama. Ia harus memuat metadata lengkap (\textit{attributes}) yang relevan untuk semua fase kompilasi:
\begin{itemize}
    \item \textbf{Basic Info}: Nama identifier (\textit{Lexeme}).
    \item \textbf{Data Type}: Tipe primitif (\code{int}, \code{float}) atau komposit (\code{struct}, \code{array}, \code{pointer}).
    \item \textbf{Storage Class}: Sifat penyimpanan (\code{static}, \code{extern}, \code{const}).
    \item \textbf{Scope Level}: Kedalaman nesting (0 untuk global, 1 untuk main, dst).
    \item \textbf{Memory Offset}: Jarak relatif dari \textit{Base Pointer} (penting untuk Code Generation).
    \item \textbf{Function Signature}: Jumlah dan tipe parameter (khusus untuk fungsi).
\end{itemize}

\subsection{Fase Penggunaan}
\begin{enumerate}
    \item \textbf{Analysis}: Parser dan Semantic Analyzer mengisi tabel saat menemukan deklarasi.
    \item \textbf{Synthesis}: Code Generator membaca tabel untuk menentukan alamat memori instruksi.
\end{enumerate}
