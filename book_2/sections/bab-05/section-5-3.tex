\section{Shadowing dan Resolusi Nama}

\subsection{Konsep Shadowing}
\textit{Shadowing} terjadi ketika deklarasi variabel di \textit{nested scope} "menutupi" variabel dengan nama yang sama di \textit{outer scope}. Kompilator perlu memberi peringatan (\textit{warning}) jika shadowing tidak disengaja, namun harus mengizinkannya secara legal.
\begin{lstlisting}[language=C]
int x = 10; // Global x
void foo() {
    int x = 20; // x ini 'melindungi' foo dari akses ke global x
    {
        int x = 30; // x ini melindung blok ini
        print(x);   // Harus 30
    }
    print(x);       // Harus 20
}
\end{lstlisting}

\subsection{Algoritma Resolusi Nama (Lookup)}
Proses pencarian identifier dilakukan secara hierarkis:
\begin{enumerate}
    \item Mulai dari \texttt{current\_scope}. Jika ditemukan, kembalikan object Symbol tersebut.
    \item Jika tidak, pindah ke \texttt{current\_scope->parent}.
    \item Ulangi langkah 2 sampai menemukan \texttt{Global Scope}.
    \item Jika sudah sampai puncak (Global) dan masih tidak ketemu, lemparkan error: \texttt{Undefined Variable 'x'}.
\end{enumerate}

\begin{figure}[!htbp]
    \centering
    \adjustbox{max width=0.8\textwidth,center}{%
    \begin{tikzpicture}[
        scope/.style={draw, rectangle, rounded corners, minimum width=4cm, minimum height=1cm, align=left, fill=white, drop shadow},
        arrow/.style={->, >=stealth, thick, dashed}
    ]
    \node[scope] (s0) {Global: \texttt{int x}};
    \node[scope, below=0.5cm of s0] (s1) {Function foo: \texttt{int x}};
    \node[scope, below=0.5cm of s1] (s2) {Inner Block: \texttt{int x}};
    
    \draw[arrow] (s2.east) to[bend right=45] node[right, font=\tiny] {Ref x (starts here)} (s2.south east);
    \draw[arrow] (s2) -- node[left, font=\tiny] {Visible?} (s1);
    \draw[arrow] (s1) -- node[left, font=\tiny] {Visible?} (s0);
    
    \node[right=2cm of s1, align=left, font=\small, text width=4cm] {
        \textbf{Lookup logic}:\\
        Search(z) $\to$ not in Block\\
        $\to$ Parent (foo)\\
        $\to$ Parent (Global)
    };
    \end{tikzpicture}%
    }
    \caption{Visualisasi Hierarki Scope untuk Resolusi Variabel}
\end{figure}
