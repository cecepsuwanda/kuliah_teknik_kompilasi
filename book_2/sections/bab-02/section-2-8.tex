\section{Grammar dan Hierarki Chomsky}

Hierarki Chomsky mengklasifikasikan bahasa formal berdasarkan kekuatan ekspresinya dan jenis automata yang dapat mengenalinya. Klasifikasi ini penting karena fase kompilasi memanfaatkan kelas bahasa yang berbeda.

\begin{itemize}
  \item \textbf{Type 0 (Unrestricted Grammar)}: Bahasa paling umum, setara dengan \textit{Turing Machine}. Semua bahasa yang dapat dihitung (recursively enumerable) berada pada level ini.
  \item \textbf{Type 1 (Context-Sensitive Grammar)}: Dikenali oleh \textit{Linear Bounded Automata}. Dapat memodelkan dependensi konteks yang tidak dapat ditangani oleh CFG.
  \item \textbf{Type 2 (Context-Free Grammar)}: Dikenali oleh \textit{Pushdown Automata}. Digunakan untuk mendefinisikan struktur sintaksis bahasa pemrograman.
  \item \textbf{Type 3 (Regular Grammar)}: Dikenali oleh \textit{Finite Automata}. Digunakan untuk mendefinisikan token pada analisis leksikal.
\end{itemize}

Dalam praktik kompilator, \textit{regular language} digunakan pada lexer dan \textit{context-free language} digunakan pada parser. Level di atasnya jarang dipakai secara langsung dalam konstruksi compiler dasar.

\begin{table}[!htbp]
\centering
\begin{tabularx}{\textwidth}{|l|l|X|}
\hline
\textbf{Tipe} & \textbf{Automata Pengenal} & \textbf{Penggunaan di Compiler} \\
\hline
Type 0 & Turing Machine & Teori komputasi umum, tidak dipakai langsung \\
\hline
Type 1 & Linear Bounded Automata & Analisis kontekstual khusus (jarang) \\
\hline
Type 2 & Pushdown Automata & Parsing dengan CFG (struktur sintaks) \\
\hline
Type 3 & Finite Automata & Tokenization di lexer (regex) \\
\hline
\end{tabularx}
\caption{Ringkasan hierarki Chomsky dan peran di kompilator}
\end{table}

\subsection{Contoh Mini Grammar}
\textbf{Regular Grammar} (Type 3):
\begin{verbatim}
R -> aR | bR | a | b
\end{verbatim}
\textbf{Context-Free Grammar} (Type 2):
\begin{verbatim}
S -> a S b | ab
\end{verbatim}
