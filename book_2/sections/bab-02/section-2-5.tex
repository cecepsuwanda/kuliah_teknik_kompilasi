\section{Contoh Praktis: Alur Kompilasi Program}

Mari kita simulasikan bagaimana satu baris kode diolah oleh berbagai fase kompilator.

\textbf{Kode Sumber:}
\begin{lstlisting}[language=C]
int sum = x + 10;
\end{lstlisting}

\begin{enumerate}
    \item \textbf{Scanner}: Menghasilkan token (\texttt{INT}, \texttt{sum}, \texttt{ASSIGN}, \texttt{x}, \texttt{PLUS}, \texttt{10}, \texttt{SEMI}).
    \item \textbf{Parser}: Membangun struktur pohon (AST) yang menunjukkan \texttt{sum} di-assign dengan hasil operasi binary plus antara \texttt{x} dan literal \texttt{10}.
    \item \textbf{Type Checker}: Memastikan \texttt{x} adalah tipe numerik (misal \texttt{int}) yang valid untuk ditambah dengan \texttt{10}.
    \item \textbf{IR Generator}: Menghasilkan \textit{Three-Address Code}:
    \begin{verbatim}
    t1 = x + 10
    sum = t1
    \end{verbatim}
    \item \textbf{Code Generator}: Menghasilkan instruksi mesin (contoh dalam assembly x86):
    \begin{verbatim}
    mov eax, [x]
    add eax, 10
    mov [sum], eax
    \end{verbatim}
\end{enumerate}
