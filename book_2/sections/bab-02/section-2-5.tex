\section{Contoh Praktis: Alur Kompilasi Program}

Untuk mengonkretkan teori fase-fase di atas, mari kita telusuri perjalanan satu baris kode C sederhana. Kita akan melihat bagaimana kode ini bermetamorfosis di setiap tahap.

\textbf{Kode Sumber:}
\begin{lstlisting}[language=C]
int sum = old_val + 10;
\end{lstlisting}

\begin{enumerate}
    \item \textbf{Scanner (Lexical Analysis)} \\
    Scanner membaca aliran karakter dan memecahnya menjadi token:
    \begin{itemize}
        \item \texttt{<KEYWORD, "int">}
        \item \texttt{<ID, "sum">}
        \item \texttt{<OP, "=">}
        \item \texttt{<ID, "old\_val">}
        \item \texttt{<OP, "+">}
        \item \texttt{<LITERAL, "10">}
        \item \texttt{<PUNCT, ";">}
    \end{itemize}

    \item \textbf{Parser (Syntax Analysis)} \\
    Parser menyusun token menjadi pohon sintaks. Ia mengenali pola deklarasi variabel:
    \begin{verbatim}
    Declaration
    |-- Type: int
    |-- Var: sum
    `-- Init: Assignment
        |-- LHS: sum
        `-- RHS: BinaryOp (+)
            |-- Left: old_val
            `-- Right: 10
    \end{verbatim}
    Jika kita lupa menulis titik-koma (\texttt{;}), parser akan melaporkan \textit{Syntax Error}.

    \item \textbf{Semantic Analyzer} \\
    Di sini kompilator memeriksa konteks.
    \begin{itemize}
        \item Apakah variabel \texttt{old\_val} sudah dideklarasikan sebelumnya?
        \item Apakah tipe data \texttt{old\_val} kompatibel untuk dijumlahkan dengan integer \texttt{10}?
    \end{itemize}
    \textit{Skenario Error}: Jika \texttt{old\_val} ternyata adalah string (\texttt{char*}), fase ini akan menghentikan proses dengan \textit{Type Mismatch Error}, meskipun secara sintaksis kalimat tersebut benar.

    \item \textbf{IR Generator \& Optimizer} \\
    Pohon di atas diterjemahkan menjadi kode sementara (misal: Quadruples):
    \begin{verbatim}
    LOAD  t1, old_val  ; Muat nilai variabel ke temp
    ADD   t2, t1, 10   ; Lakukan penjumlahan
    STORE sum, t2      ; Simpan hasil ke sum
    \end{verbatim}
    \textit{Optimasi}: Jika \texttt{old\_val} diketahui bernilai konstan 5, optimizer dapat langsung mengubahnya menjadi \texttt{STORE sum, 15}.

    \item \textbf{Code Generator (x86-64)} \\
    IR akhirnya dipetakan ke register dan instruksi mesin nyata:
    \begin{verbatim}
    mov eax, [rbp-4]           ; Ambil old_val dari stack
    add eax, 10                 ; Tambahkan 10
    mov DWORD PTR [rbp-8], eax  ; Simpan ke lokasi sum
    \end{verbatim}
\end{enumerate}
