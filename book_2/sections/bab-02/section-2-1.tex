\section{Konsep dan Definisi Kunci}

\subsection{Definisi Kompilator}

Secara tradisional, kompilator dipandang sebagai "kotak hitam" yang mengubah kode sumber menjadi kode target executable. Menurut \cite{aho2006compilers}, proses ini sebenarnya terdiri dari serangkaian fase yang saling terkait \cite{diznr2024phases}.
Secara formal, \compiler{compiler} adalah program yang melakukan translasi dari bahasa sumber (source language) ke bahasa target (target language), dengan mempertahankan makna semantik dari program sumber.

Menurut Aho, Lam, Sethi, dan Ullman dalam buku klasik "Compilers: Principles, Techniques, dan Tools"\cite{aho2006compilers}:

\begin{quote}
``A compiler is a program that can read a program in one language (the source language) and translate it into an equivalent program in another language (the target language).''
\end{quote}

\subsection{Karakteristik Kompilator}

Kompilator memiliki beberapa karakteristik penting yang membedakannya dari pemroses bahasa lainnya:

\begin{itemize}
    \item \textbf{Translasi Lengkap}: Kompilator membaca dan menganalisis seluruh program sumber sebelum menghasilkan output.
    \item \textbf{Analisis Mendalam}: Melakukan pengecekan struktur (\textit{syntax}) dan makna (\textit{semantic}) secara menyeluruh.
    \item \textbf{Output Terpisah}: Menghasilkan file terpisah (seperti \textit{object file} atau \textit{executable}) yang dapat berjalan tanpa kode sumber asli.
    \item \textbf{Optimasi}: Menggunakan teknik matematis dan heuristik untuk meningkatkan efisiensi kode hasil translasi.
\end{itemize}

\subsection{Interpreter vs Compiler}

Perbedaan fundamental antara \compiler{interpreter} dan \compiler{compiler}:

\begin{table}[h]
\centering
\begin{tabular}{|l|l|l|}
\hline
\textbf{Aspek} & \textbf{Compiler} & \textbf{Interpreter} \\
\hline
Eksekusi & Compile lalu run & Run langsung \\
Kecepatan & Lebih cepat & Lebih lambat \\
Debugging & Lebih sulit & Lebih mudah \\
Platform & Platform-dependent & Platform-independent \\
Memory usage & Lebih besar & Lebih kecil \\
\hline
\end{tabular}
\caption{Perbandingan Compiler dan Interpreter}
\end{table}

\subsection{Arsitektur Kompilator}

Kompilator modern memiliki arsitektur berlapis yang terdiri dari beberapa fase:

\begin{enumerate}
  \item \textbf{Analysis Phase}
  \begin{itemize}
    \item Lexical Analysis (Scanner)
    \item Syntax Analysis (Parser)
    \item Semantic Analysis
  \end{itemize}
  \item \textbf{Synthesis Phase}
  \begin{itemize}
    \item Intermediate Code Generation
    \item Code Optimization
    \item Code Generation
  \end{itemize}
\end{enumerate}
