\section{Teori Utama Compiler}

\subsection{Formal Language Theory}

Teori bahasa formal menjadi dasar bagi compiler design:

\begin{itemize}
  \item \textbf{Regular Expressions}: Untuk lexical analysis
  \item \textbf{Context-Free Grammars}: Untuk syntax analysis
  \item \textbf{Finite Automata}: Model recognizer untuk tokens
  \item \textbf{Pushdown Automata}: Model recognizer untuk parsing
\end{itemize}

\subsection{One-Pass vs Multi-Pass Compiler}

\subsubsection{One-Pass Compiler}

Kompilator \textit{single-pass} mencoba menyelesaikan semua fase dalam satu kali pembacaan kode (\textit{pass}).
\begin{itemize}
  \item \textbf{Kelebihan}: Proses kompilasi sangat cepat dan hemat memori.
  \item \textbf{Kekurangan}: Lebih sulit dikembangkan, optimasi sangat terbatas, dan tidak fleksibel terhadap struktur bahasa yang kompleks.
  \item \textbf{Contoh}: Pascal, implementasi awal bahasa C.
\end{itemize}

\subsubsection{Multi-Pass Compiler}

Kompilator modern umumnya menggunakan pendekatan \textit{multi-pass}, di mana setiap fase (atau kelompok fase) dijalankan dalam \textit{pass} terpisah.
\begin{itemize}
  \item \textbf{Kelebihan}: Modularitas tinggi, pemisahan perhatian tiap fase, dan memungkinkan optimasi global yang mendalam.
  \item \textbf{Kekurangan}: Membutuhkan lebih banyak memori dan waktu kompilasi dibandingkan \textit{single-pass}.
  \item \textbf{Contoh}: GCC, LLVM/Clang, modern C++, Java compiler.
\end{itemize}

\begin{figure}[!htbp]
    \centering
    \adjustbox{max width=0.85\textwidth,center}{%
    \begin{tikzpicture}[
        box/.style={rectangle, draw=blue!50, fill=blue!10, text width=2.5cm, text centered, minimum height=0.7cm, rounded corners, font=\footnotesize, inner sep=4pt},
        bigbox/.style={rectangle, draw=blue!50, fill=blue!10, text width=2.8cm, text centered, minimum height=3.2cm, rounded corners, font=\footnotesize, inner sep=6pt},
        arrow/.style={->, >=stealth, thick},
        title/.style={font=\bfseries\small},
        node distance=0.4cm and 3.0cm
    ]
    
    \node[title] (mp-title) {MULTI-PASS};
    \node[box, below=of mp-title] (mp1) {Pass 1: Lexical};
    \node[box, below=of mp1] (mp2) {Pass 2: Syntax};
    \node[box, below=of mp2] (mp3) {Pass 3: Semantic};
    \node[box, below=of mp3] (mpn) {... Code Gen};
    \draw[arrow] (mp1) -- (mp2);
    \draw[arrow] (mp2) -- (mp3);
    \draw[arrow] (mp3) -- (mpn);
    
    \node[title, right=of mp-title] (sp-title) {SINGLE-PASS};
    \node[bigbox, below=of sp-title] (sp-all) {Semua fase\\dalam satu pass};
    
    \end{tikzpicture}%
    }
    \caption{Perbandingan arsitektur Multi-Pass dan Single-Pass Compiler}
    \label{fig:multipass-vs-singlepass}
\end{figure}
