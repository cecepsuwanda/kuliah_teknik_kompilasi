\section{Alur Kerja dan Arsitektur Kompilator}

\subsection{Alur Kerja Kompilator: Dari Source ke Executable}

Sebelum membahas detail arsitektur, mari kita lihat gambaran umum alur kerja kompilator secara utuh. Gambar \ref{fig:compiler-flow} menunjukkan transformasi kode dari teks mentah hingga menjadi file yang dapat dieksekusi oleh mesin, sesuai dengan model sistem kompilasi standar \cite{uw2024compiler}.

\begin{figure}[!htbp]
\centering
\adjustbox{max width=0.85\textwidth,center}{%
\begin{tikzpicture}[
    process/.style={rectangle, draw=blue!50, fill=blue!10, text width=2.5cm, text centered, minimum height=0.6cm, rounded corners, font=\footnotesize},
    output/.style={rectangle, draw=green!50, fill=green!10, text width=2cm, text centered, minimum height=0.5cm, rounded corners, font=\tiny},
    arrow/.style={->, >=stealth, thick}
]
    \node[process] (source) {\textbf{Source Code}\\\footnotesize(C, C++, dll.)};
    \node[process, below=0.6cm of source] (preproc) {\textbf{Preprocessing}};
    \node[process, below=0.6cm of preproc] (compiler) {\textbf{Compiler}};
    \node[output, right=1cm of compiler] (asm) {Assembly Code};
    \node[process, below=0.6cm of compiler] (assemble) {\textbf{Assembler}};
    \node[output, right=1cm of assemble] (obj) {Object Code};
    \node[process, below=0.6cm of assemble] (link) {\textbf{Linker}};
    \node[output, right=1cm of link] (exe) {\textbf{Executable}};
    
    \draw[arrow] (source) -- (preproc);
    \draw[arrow] (preproc) -- (compiler);
    \draw[arrow] (compiler) -- (asm);
    \draw[arrow] (compiler) -- (assemble);
    \draw[arrow] (assemble) -- (obj);
    \draw[arrow] (assemble) -- (link);
    \draw[arrow] (link) -- (exe);
\end{tikzpicture}%
}
\caption{Alur kerja sistem kompilasi secara keseluruhan}
\label{fig:compiler-flow}
\end{figure}

\subsection{Dua Sisi Kompilator: Front-End dan Back-End}

Kompilator modern umumnya dibagi menjadi dua bagian utama: \textbf{front-end} (analisis) dan \textbf{back-end} (sintesis). Pemisahan ini memungkinkan satu front-end (misal untuk bahasa C) digunakan untuk berbagai back-end (misal untuk mesin Intel x86 dan ARM).

\begin{figure}[!htbp]
    \centering
    \adjustbox{max width=0.9\textwidth,center}{%
    \begin{tikzpicture}[
        box/.style={rectangle, draw=blue!50, fill=blue!10, text width=2.2cm, text centered, minimum height=0.9cm, rounded corners, font=\footnotesize, inner sep=4pt},
        arrow/.style={->, >=stealth, thick},
        section/.style={rectangle, draw=black!50, fill=gray!20, text width=3.0cm, text centered, minimum height=1.0cm, rounded corners, font=\bfseries\small, inner sep=5pt},
        irbox/.style={rectangle, draw=purple!50, fill=purple!10, text width=2.4cm, text centered, minimum height=1.0cm, rounded corners, font=\footnotesize, inner sep=4pt},
        node distance=0.8cm and 1.5cm
    ]
    
    \node[box] (source) {Source Code};
    \node[section, right=of source] (frontend) {FRONT-END\\(Analisis)};
    \node[irbox, right=of frontend] (ir) {Representasi\\Intermediate (IR)};
    \node[section, right=of ir] (backend) {BACK-END\\(Sintesis)};
    \node[box, right=of backend] (target) {Target Code};
    
    \draw[arrow] (source) -- (frontend);
    \draw[arrow] (frontend) -- (ir);
    \draw[arrow] (ir) -- (backend);
    \draw[arrow] (backend) -- (target);
    \end{tikzpicture}%
    }
    \caption{Struktur logis kompilator: Pemisahan Analisis dan Sintesis}
    \label{fig:compiler-architecture-simple}
\end{figure}
