\section{Fase-Fase Kompilasi Secara Detail}

Di balik pembagian besar front-end dan back-end, terdapat enam fase utama yang bekerja secara sekuensial untuk melakukan transformasi kode.

\subsection{Analisis Leksikal (Scanner)}
Fase ini memecah karakter-karakter dalam \textit{source code} menjadi unit-unit atomik bermakna yang disebut \textbf{token}.
\begin{itemize}
    \item \textbf{Input}: String karakter kode sumber.
    \item \textbf{Output}: Stream token (\texttt{keyword}, \texttt{id}, \texttt{literal}, dll.).
\end{itemize}

\subsection{Analisis Sintaksis (Parser)}
Mengambil token dari scanner dan memeriksa apakah urutannya membentuk struktur yang valid sesuai \textit{grammar} bahasa.
\begin{itemize}
    \item \textbf{Output}: \textit{Abstract Syntax Tree} (AST).
\end{itemize}

\subsection{Analisis Semantik}
Memeriksa makna dari kode, seperti kecocokan tipe data (\textit{type checking}) dan keberadaan deklarasi variabel (\textit{scope resolution}).

\subsection{Generasi Intermediate Code (IR)}
Translasi AST ke bentuk yang lebih dekat dengan instruksi mesin tetapi tetap independen terhadap jenis prosesor tertentu.

\subsection{Optimasi Kode}
Transformasi IR untuk menghasilkan kode yang lebih efisien (cepat dijalankan atau hemat memori) tanpa mengubah maksud program aslinya.

\subsection{Generasi Kode Target}
Tahap akhir yang mengubah IR menjadi instruksi spesifik untuk arsitektur mesin target (misalnya assembly x86 atau ARM).

\begin{figure}[!htbp]
    \centering
    \adjustbox{max width=0.88\textwidth,center}{%
    \begin{tikzpicture}[
        node distance=0.8cm,
        astnode/.style={circle, draw=blue!50, fill=blue!10, minimum size=0.6cm, font=\footnotesize},
        tacnode/.style={rectangle, draw=green!50, fill=green!10, text width=1.8cm, text centered, minimum height=0.6cm, rounded corners, font=\footnotesize},
        arrow/.style={->, >=stealth, thick}
    ]
        \node[astnode] (plus) at (0,1.5) {+};
        \node[astnode, below left=0.6cm of plus] (a) {a};
        \node[astnode, below right=0.6cm of plus] (b) {b};
        \draw[arrow] (plus) -- (a);
        \draw[arrow] (plus) -- (b);
        \node[left=1.5cm of plus] {\textbf{AST}};
        
        \node[right=2.2cm of plus] (arrow) {\Large $\Rightarrow$};
        
        \node[tacnode, right=1.6cm of arrow] (tac) {\texttt{t1 = a + b}};
        \node[above=0.2cm of tac, font=\small] {\textbf{Intermediate Code}};
    \end{tikzpicture}%
    }
    \caption{Visualisasi transformasi dari AST ke Intermediate Code (TAC)}
\end{figure}
