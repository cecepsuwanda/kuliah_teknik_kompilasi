\section{Arsitektur Kompilator}

Kompilator modern umumnya dibagi menjadi dua bagian utama: \textbf{front-end} dan \textbf{back-end}. Gambar \ref{fig:compiler-architecture} menunjukkan struktur arsitektur kompilator secara keseluruhan.

\begin{figure}[!htbp]
    \centering
    \adjustbox{max width=0.88\textwidth,center}{%
    \begin{tikzpicture}[
        box/.style={rectangle, draw=blue!50, fill=blue!10, text width=2.2cm, text centered, minimum height=0.9cm, rounded corners, font=\footnotesize, inner sep=4pt},
        arrow/.style={->, >=stealth, thick},
        section/.style={rectangle, draw=black!50, fill=gray!20, text width=3.0cm, text centered, minimum height=1.0cm, rounded corners, font=\bfseries\small, inner sep=5pt},
        irbox/.style={rectangle, draw=purple!50, fill=purple!10, text width=2.4cm, text centered, minimum height=1.0cm, rounded corners, font=\footnotesize, inner sep=4pt},
        node distance=1.1cm and 1.8cm
    ]
    
    % Source
    \node[box] (source) {Source Code\\(C/C++, Java, dll.)};
    
    % Front-end label
    \node[section, right=of source] (frontend) {FRONT-END\\(Analisis)};
    
    % Front-end phases (vertical)
    \node[box, below=of frontend] (lexical) {Lexical Analysis};
    \node[box, below=of lexical] (syntax) {Syntax Analysis};
    \node[box, below=of syntax] (semantic) {Semantic Analysis};
    
    % IR
    \node[irbox, right=of syntax] (ir) {Intermediate\\Representation (IR)};
    
    % Back-end label
    \node[section, right=of ir] (backend) {BACK-END\\(Sintesis)};
    
    % Back-end phases (vertical)
    \node[box, below=of backend] (optimize) {Code Optimization};
    \node[box, below=of optimize] (codegen) {Code Generation};
    
    % Target
    \node[box, right=of backend] (target) {Target Code\\(Assembly / Machine)};
    
    % Arrows (alur utama)
    \draw[arrow] (source) -- (frontend);
    \draw[arrow] (frontend) -- (lexical);
    \draw[arrow] (lexical) -- (syntax);
    \draw[arrow] (syntax) -- (semantic);
    \draw[arrow] (semantic) -- (ir);
    \draw[arrow] (ir) -- (backend);
    \draw[arrow] (backend) -- (optimize);
    \draw[arrow] (optimize) -- (codegen);
    \draw[arrow] (codegen) -- (target);
    
    \end{tikzpicture}%
    }
    \caption{Arsitektur kompilator: Front-end (analisis) dan Back-end (sintesis)}
    \label{fig:compiler-architecture}
    \end{figure}
    

\subsection{Front-End (Analisis)}

Front-end bertanggung jawab untuk menganalisis source code dan membangun representasi internal. Menurut sumber terbuka:

\begin{quote}
``A typical compiler front end comprises several sequential phases: lexical analysis (scanning), syntax analysis (parsing), and semantic analysis. Lexical analysis breaks input text into lexemes which correspond to tokens, eliminating comments and whitespace. Syntax analysis checks grammar validity and builds a structural representation (parse tree or AST). Semantic analysis performs scope resolution, type checking, name resolution, and checks language-specific semantic rules.''\cite{diznr2024phases}
\end{quote}

Fase-fase dalam front-end meliputi:

\begin{enumerate}
    \item \textbf{Lexical Analysis (Tokenization)}: Memecah source code menjadi token-token (identifiers, keywords, operators, literals, dll.)
    \item \textbf{Syntax Analysis (Parsing)}: Menganalisis struktur grammar dan membangun parse tree atau Abstract Syntax Tree (AST)
    \item \textbf{Semantic Analysis}: Memeriksa aturan semantik bahasa, seperti type checking, scope resolution, dan name resolution
\end{enumerate}

\subsection{Back-End (Sintesis)}

Setelah front-end selesai menganalisis source code dan menghasilkan representasi intermediate, back-end bertanggung jawab untuk menghasilkan target code dari representasi tersebut. Fase-fase dalam back-end meliputi:

\begin{enumerate}
    \item \textbf{Intermediate Code Generation}: Mengubah AST menjadi intermediate representation (misalnya three-address code, quadruples, atau bytecode)
    \item \textbf{Code Optimization}: Mengoptimasi kode intermediate untuk meningkatkan efisiensi tanpa mengubah semantik
    \item \textbf{Code Generation}: Menghasilkan target code (assembly atau machine code) dari kode intermediate yang sudah dioptimasi
\end{enumerate}
