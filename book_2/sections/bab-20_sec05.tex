\section{Assembling dan Linking menjadi Executable}

Setelah code generator menghasilkan assembly code, langkah berikutnya adalah meng-assemble dan meng-link menjadi executable file (.exe).

\subsection{Proses Assembling}

Assembling adalah proses mengubah assembly code menjadi object file (.obj). Kita akan menggunakan NASM (Netwide Assembler).

\subsubsection{Command NASM}

\begin{verbatim}
nasm -f win64 hello.asm -o hello.obj
\end{verbatim}

Penjelasan parameter:
\begin{itemize}
    \item \texttt{-f win64}: Format output untuk Windows 64-bit
    \item \texttt{hello.asm}: File input (assembly code)
    \item \texttt{-o hello.obj}: File output (object file)
\end{itemize}

\subsection{Proses Linking}

Linking adalah proses mengubah object file menjadi executable. Kita akan menggunakan Microsoft Linker atau MinGW linker.

\subsubsection{Menggunakan Microsoft Linker (link.exe)}

\begin{verbatim}
link hello.obj /subsystem:console /entry:_start /out:hello.exe kernel32.lib
\end{verbatim}

Penjelasan parameter:
\begin{itemize}
    \item \texttt{hello.obj}: Object file yang akan di-link
    \item \texttt{/subsystem:console}: Subsystem untuk console application
    \item \texttt{/entry:\_start}: Entry point program
    \item \texttt{/out:hello.exe}: Nama output executable
    \item \texttt{kernel32.lib}: Library yang berisi Windows API functions (GetStdHandle, WriteFile, ExitProcess)
\end{itemize}

\subsubsection{Menggunakan MinGW Linker (ld.exe)}

Jika menggunakan MinGW, command-nya sedikit berbeda:

\begin{verbatim}
ld hello.obj -o hello.exe -e _start -subsystem:console kernel32.lib
\end{verbatim}

Atau dengan gcc (lebih mudah):

\begin{verbatim}
gcc hello.obj -o hello.exe -nostdlib -e _start kernel32.lib
\end{verbatim}

\subsection{Batch File untuk Automasi}

Untuk memudahkan proses build, kita dapat membuat batch file:

\begin{lstlisting}[language=bash, caption={build.bat - Build kompilator}]
@echo off
echo Building kompilator...

REM Compile kompilator
gcc -o compiler.exe main.c lexer.c parser.c codegen.c

if errorlevel 1 (
    echo Compilation failed!
    exit /b 1
)

echo Kompilator built successfully: compiler.exe
\end{lstlisting}

\begin{lstlisting}[language=bash, caption={compile.bat - Compile hello.c menjadi hello.exe}]
@echo off
echo Compiling hello.c...

REM Step 1: Run compiler to generate assembly
compiler.exe hello.c hello.asm

if errorlevel 1 (
    echo Compilation failed at code generation!
    exit /b 1
)

REM Step 2: Assemble with NASM
nasm -f win64 hello.asm -o hello.obj

if errorlevel 1 (
    echo Assembly failed!
    exit /b 1
)

REM Step 3: Link with Microsoft Linker
link hello.obj /subsystem:console /entry:_start /out:hello.exe kernel32.lib

if errorlevel 1 (
    echo Linking failed!
    exit /b 1
)

REM Step 4: Cleanup temporary files
del hello.obj hello.asm

echo Compilation successful: hello.exe
\end{lstlisting}

\subsection{Main Program - Driver}

Berikut adalah main program yang mengintegrasikan semua komponen:

\begin{lstlisting}[language=C, caption={main.c - Driver program}]
#include <stdio.h>
#include <stdlib.h>
#include "lexer.h"
#include "parser.h"
#include "codegen.h"

int main(int argc, char* argv[]) {
    if (argc != 3) {
        fprintf(stderr, "Usage: %s <input.c> <output.asm>\n", argv[0]);
        return 1;
    }
    
    // Read source file
    FILE* input = fopen(argv[1], "r");
    if (input == NULL) {
        fprintf(stderr, "Cannot open input file: %s\n", argv[1]);
        return 1;
    }
    
    // Get file size
    fseek(input, 0, SEEK_END);
    long size = ftell(input);
    fseek(input, 0, SEEK_SET);
    
    // Read source code
    char* source = (char*)malloc(size + 1);
    fread(source, 1, size, input);
    source[size] = '\0';
    fclose(input);
    
    // Initialize lexer
    initLexer(source);
    
    // Parse
    ASTNode* ast = parse();
    if (ast == NULL) {
        fprintf(stderr, "Parse error\n");
        free(source);
        freeLexer();
        return 1;
    }
    
    // Generate code
    FILE* output = fopen(argv[2], "w");
    if (output == NULL) {
        fprintf(stderr, "Cannot open output file: %s\n", argv[2]);
        freeAST(ast);
        free(source);
        freeLexer();
        return 1;
    }
    
    generateCode(ast, output);
    fclose(output);
    
    // Cleanup
    freeAST(ast);
    free(source);
    freeLexer();
    
    printf("Compilation successful: %s\n", argv[2]);
    return 0;
}
\end{lstlisting}

\subsection{Testing Lengkap}

Langkah-langkah untuk testing kompilator lengkap:

\subsubsection{1. Build Kompilator}

\begin{verbatim}
build.bat
\end{verbatim}

Ini akan menghasilkan \texttt{compiler.exe}.

\subsubsection{2. Buat File Test}

Buat file \texttt{hello.c}:
\begin{verbatim}
print("hello world !!!");
\end{verbatim}

\subsubsection{3. Kompilasi dengan Kompilator}

\begin{verbatim}
compiler.exe hello.c hello.asm
\end{verbatim}

Ini akan menghasilkan \texttt{hello.asm}.

\subsubsection{4. Assemble dengan NASM}

\begin{verbatim}
nasm -f win64 hello.asm -o hello.obj
\end{verbatim}

Ini akan menghasilkan \texttt{hello.obj}.

\subsubsection{5. Link dengan Linker}

\begin{verbatim}
link hello.obj /subsystem:console /entry:_start /out:hello.exe kernel32.lib
\end{verbatim}

Ini akan menghasilkan \texttt{hello.exe}.

\subsubsection{6. Jalankan Executable}

\begin{verbatim}
hello.exe
\end{verbatim}

Output yang diharapkan:
\begin{verbatim}
hello world !!!
\end{verbatim}

\subsection{Script Lengkap untuk Testing}

Berikut adalah script lengkap yang mengotomasi semua proses:

\begin{lstlisting}[language=bash, caption={test.bat - Script testing lengkap}]
@echo off
echo ========================================
echo   Testing Simple C Compiler
echo ========================================
echo.

REM Step 1: Build kompilator
echo [1/5] Building kompilator...
call build.bat
if errorlevel 1 (
    echo ERROR: Failed to build kompilator
    exit /b 1
)
echo.

REM Step 2: Create test file
echo [2/5] Creating test file...
echo print("hello world !!!"); > hello.c
echo Test file created: hello.c
echo.

REM Step 3: Compile with our kompilator
echo [3/5] Compiling hello.c to hello.asm...
compiler.exe hello.c hello.asm
if errorlevel 1 (
    echo ERROR: Compilation failed
    exit /b 1
)
echo Assembly generated: hello.asm
echo.

REM Step 4: Assemble
echo [4/5] Assembling hello.asm...
nasm -f win64 hello.asm -o hello.obj
if errorlevel 1 (
    echo ERROR: Assembly failed
    echo Make sure NASM is installed and in PATH
    exit /b 1
)
echo Object file created: hello.obj
echo.

REM Step 5: Link
echo [5/5] Linking hello.obj...
link hello.obj /subsystem:console /entry:_start /out:hello.exe kernel32.lib
if errorlevel 1 (
    echo ERROR: Linking failed
    echo Make sure Microsoft Linker is available
    exit /b 1
)
echo Executable created: hello.exe
echo.

REM Step 6: Run
echo ========================================
echo   Running hello.exe...
echo ========================================
hello.exe
echo.

REM Step 7: Cleanup
echo Cleaning up temporary files...
del hello.obj hello.asm 2>nul
echo.
echo ========================================
echo   Test completed successfully!
echo ========================================
\end{lstlisting}

\subsection{Troubleshooting}

\subsubsection{Error: NASM not found}
\begin{itemize}
    \item Pastikan NASM sudah di-install
    \item Pastikan NASM ada di PATH environment variable
    \item Atau gunakan full path ke nasm.exe
\end{itemize}

\subsubsection{Error: Linker not found}
\begin{itemize}
    \item Untuk Microsoft Linker: Pastikan Visual Studio atau Windows SDK terinstall
    \item Untuk MinGW: Pastikan MinGW ada di PATH
    \item Atau gunakan gcc untuk linking: \texttt{gcc hello.obj -o hello.exe -nostdlib -e \_start kernel32.lib}
\end{itemize}

\subsubsection{Error: kernel32.lib not found}
\begin{itemize}
    \item Pastikan Windows SDK terinstall
    \item Atau gunakan full path ke kernel32.lib
    \item Untuk MinGW, biasanya tidak perlu specify kernel32.lib secara eksplisit
\end{itemize}

\subsection{Kesimpulan}

Dengan langkah-langkah di atas, kita telah berhasil membuat compiler sederhana yang dapat:
\begin{enumerate}
    \item Membaca source code (hello.c)
    \item Melakukan lexical analysis
    \item Melakukan syntax analysis
    \item Menghasilkan assembly code
    \item Meng-assemble menjadi object file
    \item Meng-link menjadi executable
    \item Menghasilkan program yang dapat dijalankan
\end{enumerate}

Compiler sederhana ini menunjukkan semua fase kompilasi secara lengkap dan menghasilkan executable yang benar-benar dapat dijalankan di Windows.
