\section{Referensi dan Bahan Bacaan Lanjutan}

Untuk memperdalam pemahaman tentang implementasi lexer, mahasiswa disarankan membaca:

\begin{itemize}
    \item \textbf{Dragon Book}: Aho, Lam, Sethi, \& Ullman (2006). \textit{Compilers: Principles, Techniques, and Tools} \cite{aho2006compilers} - Bab 3: Lexical Analysis
    
    \item \textbf{Engineering a Compiler}: Cooper \& Torczon (2011) \cite{cooper2011engineering} - Bab 2: Scanning
    
    \item \textbf{OpenGenus - Build Lexer}: Tutorial tentang hand-written lexer \cite{opengenus2024lexer}
    
    \item \textbf{Aoyama Gakuin University}: Lecture notes tentang lexical analysis \cite{aoyama2024lexical}
    
    \item \textbf{GeeksforGeeks}: Contoh implementasi lexical analyzer dalam C++ \footnote{\url{https://www.geeksforgeeks.org/cpp/lexical-analyzer-in-cpp/}}
    
    \item \textbf{Programming Notes}: Tutorial tentang simple lexer menggunakan finite state machine \footnote{\url{https://www.programmingnotes.org/4699/cpp-simple-lexer-using-a-finite-state-machine/}}
\end{itemize}