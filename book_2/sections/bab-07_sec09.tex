\section{Perbandingan Top-Down vs Bottom-Up Parsing}

\subsection{Perbandingan Karakteristik}

\begin{table}[H]
\centering
\begin{tabular}{|l|c|c|}
\hline
\textbf{Aspek} & \textbf{Top-Down} & \textbf{Bottom-Up} \\
\hline
Parse Tree Direction & Root -> Leaves & Leaves -> Root \\
\hline
Derivation & Leftmost & Rightmost (reverse) \\
\hline
Implementation & Recursive descent & Table-driven \\
\hline
Lookahead & Usually 1 token & Usually 1 token \\
\hline
Left Recursion & Problem & No problem \\
\hline
Right Recursion & No problem & Less efficient \\
\hline
Parsing Power & LL(1) grammars & LR(1) grammars \\
\hline
Error Detection & Early & Later \\
\hline
Error Messages & More intuitive & Less intuitive \\
\hline
Table Size & Small & Larger \\
\hline
\end{tabular}
\caption{Perbandingan top-down dan bottom-up parsing}
\label{tab:top-down-vs-bottom-up}
\end{table}

\subsection{Kapan Menggunakan Masing-masing}

\textbf{Gunakan Top-Down Parsing jika:}
\begin{itemize}
    \item Grammar sudah dalam bentuk yang sesuai (tidak ada left recursion)
    \item Error messages yang intuitif penting
    \item Implementasi manual diperlukan
    \item Grammar relatif sederhana
\end{itemize}

\textbf{Gunakan Bottom-Up Parsing jika:}
\begin{itemize}
    \item Grammar memiliki left recursion
    \item Parsing power yang lebih besar diperlukan
    \item Menggunakan parser generator (Bison/Yacc)
    \item Grammar kompleks dengan banyak precedence levels
\end{itemize}