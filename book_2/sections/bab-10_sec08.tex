\section{Visualisasi Symbol Table}

Visualisasi symbol table sangat berguna untuk debugging dan pembelajaran. Berikut adalah fungsi untuk mencetak isi symbol table:

\begin{lstlisting}[language=C++, caption={Fungsi visualisasi symbol table}]
// Fungsi helper untuk visualisasi (perlu menambahkan getCurrentScope() 
// ke class SymbolTable atau menggunakan pendekatan lain)
void printSymbolTableHelper(Scope* scope) {
    if (scope == nullptr) return;
    
    // Rekursif ke parent dulu (agar print dari global ke current)
    printSymbolTableHelper(scope->getParent());
    
    // Print scope ini
    std::cout << "\nLevel " << scope->getLevel() << ":" << std::endl;
    std::cout << "  Symbols:" << std::endl;
    
    // Print semua symbols di scope ini
    for (const auto& name : scope->getDeclaredNames()) {
        Symbol* sym = scope->lookupLocal(name);
        if (sym != nullptr) {
            std::cout << "    " << sym->name 
                      << " : " << sym->type
                      << " (line " << sym->line_number << ")" 
                      << std::endl;
        }
    }
}

void printSymbolTable(SymbolTable& st) {
    std::cout << "\n=== Symbol Table ===" << std::endl;
    
    // Asumsikan SymbolTable memiliki method getCurrentScope()
    // Atau kita bisa mengakses current_scope jika public/protected
    // Untuk contoh ini, kita asumsikan ada method helper
    Scope* current = st.getCurrentScope();  // Perlu ditambahkan ke class
    
    printSymbolTableHelper(current);
    
    std::cout << "===================\n" << std::endl;
}
\end{lstlisting}

Contoh output visualisasi:

\begin{verbatim}
=== Symbol Table ===

Level 0:
  Symbols:
    x : int (line 1)

Level 1:
  Symbols:
    y : int (line 4)
    x : int (line 5)

Level 2:
  Symbols:
    z : int (line 8)
    y : int (line 9)
===================
\end{verbatim}