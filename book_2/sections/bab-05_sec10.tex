\section{Manual Parsing: Latihan Derivation}

Mari kita lakukan manual parsing untuk ekspresi \texttt{2 + 3 * 4} menggunakan grammar:

\begin{verbatim}
E → E + T | E - T | T
T → T * F | T / F | F
F → ( E ) | number
\end{verbatim}

\textbf{Leftmost Derivation}:

\begin{enumerate}
    \item \(E \Rightarrow E + T\)
    \item \(E + T \Rightarrow T + T\)
    \item \(T + T \Rightarrow F + T\)
    \item \(F + T \Rightarrow \texttt{2} + T\)
    \item \(\texttt{2} + T \Rightarrow \texttt{2} + T * F\)
    \item \(\texttt{2} + T * F \Rightarrow \texttt{2} + F * F\)
    \item \(\texttt{2} + F * F \Rightarrow \texttt{2} + \texttt{3} * F\)
    \item \(\texttt{2} + \texttt{3} * F \Rightarrow \texttt{2} + \texttt{3} * \texttt{4}\)
\end{enumerate}

\textbf{Parse Tree}:

\begin{figure}[H]
\centering
\begin{forest}
for tree={
    grow'=0,
    child anchor=west,
    parent anchor=east,
    anchor=west,
    calign=first,
    edge path={
        \noexpand\path[\forestoption{edge}]
        (!u.south west) +(7.5pt,0) |- (.child anchor) \forestoption{edge label};
    },
    before typesetting nodes={
        if n=1
          {insert before={[,phantom]}}
          {}
    },
    fit=band,
    before computing xy={l=15pt},
}
[E
    [E
        [T
            [F [2]]
        ]
    ]
    [+]
    [T
        [T
            [F [3]]
        ]
        [*]
        [F [4]]
    ]
]
\end{forest}
\caption{Parse tree untuk ekspresi \texttt{2 + 3 * 4}}
\label{fig:parse-tree-2+3*4}
\end{figure}