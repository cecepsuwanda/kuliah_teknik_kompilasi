\section{Benchmarking dan Evaluasi Kinerja}

Evaluasi compiler tidak hanya tentang correctness, tetapi juga tentang performa. Berikut adalah metrik yang dapat digunakan:

\subsection{Metrik Compiler Performance}

\textbf{1. Compilation Speed}
\begin{itemize}
    \item Waktu kompilasi untuk berbagai ukuran program
    \item Throughput (lines/second atau tokens/second)
    \item Memory usage selama kompilasi
\end{itemize}

\textbf{2. Generated Code Quality}
\begin{itemize}
    \item Execution time dari program yang dikompilasi
    \item Code size (executable size)
    \item Memory footprint
    \item Instruction count
\end{itemize}

\textbf{3. Compiler Correctness}
\begin{itemize}
    \item Test coverage
    \item Number of bugs found
    \item Error detection rate
\end{itemize}

\subsection{Contoh Benchmark Results}

Berikut adalah contoh format untuk melaporkan hasil benchmark:

\begin{table}[!htbp]
\centering
\begin{tabular}{|l|c|c|c|}
\hline
\textbf{Metric} & \textbf{Baseline} & \textbf{Optimized} & \textbf{Improvement} \\
\hline
Compilation Time (s) & 2.5 & 3.1 & -24\% (slower) \\
\hline
Generated Code Size (KB) & 64 & 48 & 25\% reduction \\
\hline
Execution Time (ms) & 100 & 75 & 25\% faster \\
\hline
Memory Usage (MB) & 8 & 6 & 25\% reduction \\
\hline
\end{tabular}
\caption{Contoh hasil benchmark compiler}
\label{tab:benchmark-results}
\end{table}

\subsection{Test Suite}

Untuk evaluasi yang komprehensif, siapkan test suite yang mencakup:

\begin{enumerate}
    \item \textbf{Unit Tests}: Test setiap fase secara terpisah
    \begin{itemize}
        \item Lexer tests: Valid/invalid tokens
        \item Parser tests: Valid/invalid syntax
        \item Semantic tests: Type checking, scope resolution
        \item Code generation tests: Correctness of generated code
    \end{itemize}
    
    \item \textbf{Integration Tests}: Test seluruh pipeline
    \begin{itemize}
        \item End-to-end compilation
        \item Error propagation through phases
        \item Optimization correctness
    \end{itemize}
    
    \item \textbf{Performance Tests}: Test dengan program besar
    \begin{itemize}
        \item Compilation time
        \item Memory usage
        \item Generated code performance
    \end{itemize}
    
    \item \textbf{Regression Tests}: Test untuk memastikan tidak ada regresi
\end{enumerate}