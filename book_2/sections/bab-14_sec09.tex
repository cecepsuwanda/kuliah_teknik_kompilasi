\section{Testing dan Validasi}

Setelah code generator menghasilkan assembly, kita perlu:
\begin{enumerate}
    \item \textbf{Assemble}: Mengubah assembly menjadi object file
    \item \textbf{Link}: Menyatukan object files menjadi executable
    \item \textbf{Run}: Mengeksekusi program dan memverifikasi hasilnya
\end{enumerate}

\subsection{Workflow Lengkap}

\begin{verbatim}
Source Code (C/C++)
    ↓
[Compiler Front-end]
    ↓
TAC / IR
    ↓
[Code Generator] → Assembly Code (.s)
    ↓
[Assembler] → Object File (.o)
    ↓
[Linker] → Executable
    ↓
[Run] → Verify Output
\end{verbatim}

\subsection{Contoh Testing}

Misalkan kita memiliki program sederhana:
\begin{verbatim}
int main() {
    int a = 10;
    int b = 20;
    int c = a + b;
    return c;
}
\end{verbatim}

TAC yang dihasilkan:
\begin{verbatim}
t1 = 10
a = t1
t2 = 20
b = t2
t3 = a + b
c = t3
return c
\end{verbatim}

Assembly yang dihasilkan (RISC-V):
\begin{verbatim}
main:
    ADDI sp, sp, -16
    SW ra, 12(sp)
    SW fp, 8(sp)
    ADDI fp, sp, 16
    
    # a = 10
    LI t0, 10
    SW t0, -4(fp)    # Store a di stack
    
    # b = 20
    LI t0, 20
    SW t0, -8(fp)    # Store b di stack
    
    # c = a + b
    LW t1, -4(fp)    # Load a
    LW t2, -8(fp)    # Load b
    ADD t0, t1, t2
    SW t0, -12(fp)   # Store c
    
    # return c
    LW a0, -12(fp)   # Return value
    
    LW fp, 8(sp)
    LW ra, 12(sp)
    ADDI sp, sp, 16
    RET
\end{verbatim}