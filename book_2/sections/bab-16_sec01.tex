\section{Tujuan Pembelajaran}

Bab ini memfokuskan presentasi dan review \textbf{compiler subset C} yang telah dibangun secara bertahap dari Bab 2 hingga Bab 15: spesifikasi token dan grammar (Bab 1, 2, 5), lexer (Bab 3, 4), parser (Bab 6, 8), AST (Bab 9), symbol table (Bab 10), type checking (Bab 11), IR (Bab 12), runtime (Bab 13), code generation (Bab 14), optimasi (Bab 15). Setelah mempelajari bab ini, mahasiswa diharapkan mampu:
\begin{enumerate}
    \item Mempresentasikan project final (compiler subset C) dengan baik
    \item Mendemonstrasikan kemampuan compiler subset C yang telah dibangun
    \item Mengevaluasi dan membandingkan tools kompilator (parser generators vs hand-written parsers)
    \item Menganalisis trade-off antara waktu kompilasi, kualitas kode, dan efisiensi runtime
    \item Melakukan refleksi pembelajaran dan evaluasi diri terhadap seluruh materi
    \item Menyusun dokumentasi proyek yang komprehensif
\end{enumerate}