\section{Pendahuluan}

Dalam proyek compiler subset C, parser (Bab 8, \texttt{simplec.y}) membangun AST sebagai representasi internal program. Definisinya untuk subset C disajikan di Bagian~\ref{sec:ast-proyek-subset-c}; contoh berikut mengacu ke node types tersebut.

Setelah parser berhasil memverifikasi bahwa token-token membentuk struktur yang valid menurut grammar, parser perlu membangun representasi internal program. Representasi ini dapat berupa parse tree (concrete syntax tree) atau Abstract Syntax Tree (AST).

Menurut sumber terbuka:

\begin{quote}
``Abstract Syntax Trees represent the nested structure of language constructs (expressions, statements, declarations). AST representation: whether nodes represent every symbol from the grammar or a simplified form; whether annotations (types, scopes) are added later.''\cite{diznr2024phases}
\end{quote}

Gambar \ref{fig:ast-overview} menunjukkan posisi AST dalam pipeline kompilator.

\begin{figure}[!htbp]
    \centering
    \adjustbox{max width=0.9\textwidth,center}{%
    \begin{tikzpicture}[
        box/.style={rectangle, draw=blue!50, fill=blue!10, text width=2.5cm, text centered, minimum height=0.7cm, rounded corners, font=\footnotesize, inner sep=4pt, align=center},
        arrow/.style={->, >=stealth, thick},
        node distance=1.2cm
    ]
    
    \node[box] (parse) {Parse Tree};
    \node[box, right=of parse] (ast) {AST};
    \node[box, right=of ast] (semantic) {Semantic\\Analysis};
    \node[box, below=of ast] (codegen) {Code\\Generation};
    
    \draw[arrow] (parse) -- node[above, font=\tiny, align=center] {Simplify} (ast);
    \draw[arrow] (ast) -- (semantic);
    \draw[arrow] (ast) -- (codegen);
    
    \end{tikzpicture}%
    }
    \caption{Posisi AST dalam pipeline kompilator}
    \label{fig:ast-overview}
\end{figure}

\subsection{Perbedaan Parse Tree dan AST}

Perbedaan utama antara parse tree dan AST:

\begin{itemize}
    \item \textbf{Parse Tree (Concrete Syntax Tree)}:
    \begin{itemize}
        \item Mencakup semua detail sintaksis, termasuk semua non-terminal dari grammar
        \item Setiap node sesuai dengan aturan produksi grammar
        \item Lebih verbose dan mencakup informasi yang tidak diperlukan untuk fase selanjutnya
        \item Contoh: Node untuk operator precedence yang sudah jelas dari struktur
    \end{itemize}
    
    \item \textbf{Abstract Syntax Tree (AST)}:
    \begin{itemize}
        \item Menghilangkan detail sintaksis yang tidak relevan
        \item Fokus pada struktur semantik program
        \item Lebih kompak dan efisien untuk analisis semantik dan code generation
        \item Hanya menyertakan informasi yang diperlukan untuk fase-fase selanjutnya
    \end{itemize}
\end{itemize}

\subsection{Contoh Perbandingan}

Untuk ekspresi \texttt{3 + 4 * 5}, parse tree-nya akan mencakup semua non-terminal:

\begin{figure}[!htbp]
\centering
\begin{forest}
for tree={
    grow'=0,
    child anchor=west,
    parent anchor=east,
    anchor=west,
    calign=first,
    edge path={
        \noexpand\path[\forestoption{edge}]
        (!u.south west) +(7.5pt,0) |- (.child anchor) \forestoption{edge label};
    },
    before typesetting nodes={
        if n=1
          {insert before={[,phantom]}}
          {}
    },
    fit=band,
    before computing xy={l=15pt},
}
[E
    [E
        [T
            [F [3]]
        ]
    ]
    [+]
    [T
        [T
            [F [4]]
        ]
        [*]
        [F [5]]
    ]
]
\end{forest}
\caption{Parse tree untuk ekspresi \texttt{3 + 4 * 5}}
\label{fig:parse-tree-example}
\end{figure}

Sedangkan AST-nya lebih sederhana dan langsung mencerminkan semantik:

\begin{figure}[!htbp]
\centering
\begin{forest}
for tree={
    grow'=0,
    child anchor=west,
    parent anchor=east,
    anchor=west,
    calign=first,
    edge path={
        \noexpand\path[\forestoption{edge}]
        (!u.south west) +(7.5pt,0) |- (.child anchor) \forestoption{edge label};
    },
    before typesetting nodes={
        if n=1
          {insert before={[,phantom]}}
          {}
    },
    fit=band,
    before computing xy={l=15pt},
}
[+
    [3]
    [*
        [4]
        [5]
    ]
]
\end{forest}
\caption{AST untuk ekspresi \texttt{3 + 4 * 5}}
\label{fig:ast-example}
\end{figure}

AST menghilangkan node-node intermediate seperti \(E\), \(T\), \(F\) yang tidak diperlukan untuk pemahaman semantik.

Gambar \ref{fig:ast-node-types} menunjukkan berbagai jenis node dalam AST.

\begin{figure}[!htbp]
    \centering
    \adjustbox{max width=\textwidth,center}{%
    \begin{tikzpicture}[
        node/.style={
            rectangle, draw=blue!50, fill=blue!10,
            text width=3.2cm, minimum height=0.8cm,
            font=\footnotesize, align=center,
            rounded corners=4pt, inner sep=6pt
        },
        arrow/.style={->, >=stealth, thick},
        node distance=2.5cm and 2.8cm
    ]
    
    % =========================
    % Expression Nodes (Top)
    % =========================
    \node[node] (expr) {Expression Nodes};
    
    \node[node, below left=of expr] (int) {IntLiteral};
    \node[node, below=of expr] (bin) {BinaryExpr};
    \node[node, below right=of expr] (id) {Identifier};
    
    \draw[arrow] (expr) -- (int);
    \draw[arrow] (expr) -- (bin);
    \draw[arrow] (expr) -- (id);
    
    % =========================
    % Statement Nodes (Bottom)
    % =========================
    \node[node, below=4cm of bin] (stmt) {Statement Nodes};
    
    \node[node, below left=of stmt] (assign) {Assignment};
    \node[node, below=of stmt] (ifn) {IfStmt};
    \node[node, below right=of stmt] (while) {WhileStmt};
    
    \draw[arrow] (stmt) -- (assign);
    \draw[arrow] (stmt) -- (ifn);
    \draw[arrow] (stmt) -- (while);
    
    \end{tikzpicture}%
    }
    \caption{Klasifikasi jenis node dalam Abstract Syntax Tree (AST)}
    \label{fig:ast-node-types}
    \end{figure}    
    
     Precedence operator sudah tercermin dalam struktur tree.