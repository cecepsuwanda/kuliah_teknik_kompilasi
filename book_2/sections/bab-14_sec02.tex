\section{Pendahuluan}

Dalam proyek compiler subset C, code generator mengambil IR dari Bab 12 (TAC/quadruples dari AST proyek) dan asumsi runtime dari Bab 13, lalu menghasilkan assembly untuk target yang dipilih (mis.\ x86 atau RISC-V subset). Symbol table proyek (Bab 10) menyediakan alamat/offset variabel. Output code generation proyek menjadi masukan assembler/linker dalam pipeline lengkap (Bab 16).

Code generation adalah fase terakhir dalam back-end kompilator yang bertanggung jawab untuk menghasilkan target code dari intermediate representation (IR) yang telah dioptimasi. Menurut sumber dari StudyLib:

\begin{quote}
``Code generation: instruction selection, machine model. Implement code generation for a target architecture, mapping intermediate code into efficient target code, managing run-time structures.''\cite{studylib2024obe}
\end{quote}

Code generator mengambil IR (biasanya dalam bentuk three-address code atau format serupa) dan menghasilkan kode assembly atau machine code yang dapat dieksekusi pada target architecture tertentu.

Gambar \ref{fig:code-generation-overview} menunjukkan proses code generation.

\begin{figure}[!htbp]
    \centering
    \adjustbox{max width=0.9\textwidth,center}{%
    \begin{tikzpicture}[
        box/.style={rectangle, draw=blue!50, fill=blue!10, text width=2.5cm, text centered, minimum height=0.7cm, rounded corners, font=\footnotesize, inner sep=4pt, align=center},
        arrow/.style={->, >=stealth, thick},
        node distance=1.2cm
    ]
    
    \node[box] (ir) {IR\\(TAC)};
    \node[box, right=of ir] (select) {Instruction\\Selection};
    \node[box, right=of select] (alloc) {Register\\Allocation};
    \node[box, below=of alloc] (asm) {Assembly\\Code};
    
    \draw[arrow] (ir) -- (select);
    \draw[arrow] (select) -- (alloc);
    \draw[arrow] (alloc) -- (asm);
    
    \end{tikzpicture}%
    }
    \caption{Proses code generation}
    \label{fig:code-generation-overview}
\end{figure}

\subsection{Tugas Code Generator}

Code generator memiliki beberapa tugas utama:

\begin{enumerate}
    \item \textbf{Instruction Selection}: Memilih instruksi machine yang tepat untuk setiap operasi IR
    \item \textbf{Register Allocation}: Mengalokasikan register untuk variabel dan temporary values
    \item \textbf{Instruction Scheduling}: Mengatur urutan instruksi untuk optimasi pipeline
    \item \textbf{Address Assignment}: Mengalokasikan memory untuk variabel dan data structures
    \item \textbf{Code Emission}: Menghasilkan assembly atau machine code dalam format yang sesuai
\end{enumerate}

\subsection{Input dan Output Code Generator}

\textbf{Input Code Generator:}
\begin{itemize}
    \item Optimized Intermediate Representation (TAC, quadruples, atau format IR lainnya)
    \item Symbol table dengan informasi tipe dan alamat variabel
    \item Target architecture specification (instruction set, register set, addressing modes)
\end{itemize}

\textbf{Output Code Generator:}
\begin{itemize}
    \item Assembly code (untuk assembler) atau machine code langsung
    \item Relocation information (jika diperlukan)
    \item Debug information (optional)
\end{itemize}