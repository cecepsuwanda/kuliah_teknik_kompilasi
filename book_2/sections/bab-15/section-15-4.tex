\section{Compiler Frameworks}

\subsection{LLVM (Low Level Virtual Machine)}
LLVM adalah kumpulan teknologi kompilator modular yang menyediakan \textit{intermediate representation} (IR) yang sangat dioptimalkan.

\subsection{MLIR (Multi-Level Intermediate Representation)}
\compiler{MLIR} adalah sub-proyek LLVM yang dirancang untuk mengatasi fragmentasi dalam representasi kode \cite{jhu2024compilers}.
\begin{itemize}
    \item \textbf{Abstraksi Bertingkat}: MLIR memungkinkan kompilator merepresentasikan kode pada berbagai tingkat (dari tingkat tinggi seperti operasi matriks hingga tingkat instruksi mesin).
    \item \textbf{Dialects}: MLIR menggunakan sistem "dialek" yang memungkinkan bahasa baru mendefinisikan operasi kustom mereka sendiri (contoh: dialek \textit{Linalg} untuk aljabar linear atau dialek \textit{TensorFlow}).
    \item \textbf{Efisiensi}: Dengan optimasi yang terjadi pada level dialek yang tepat, informasi semantik tingkat tinggi tidak hilang saat proses penurunan ke kode mesin.
\end{itemize}

\subsection{GCC (GNU Compiler Collection)}
Meskipun LLVM sangat populer untuk proyek riset, GCC tetap menjadi standar emas untuk sistem Linux tradisional karena kestabilannya dan dukungan arsitektur perangkat keras yang sangat luas \cite{levine2009flex}.

\begin{figure}[!htbp]
    \centering
    \adjustbox{max width=0.8\textwidth,center}{%
    \begin{tikzpicture}[
        node/.style={rectangle, draw=purple!50, fill=purple!10, text width=6cm, font=\tiny, align=center}
    ]
    \node[node] (lang) {Bahasa Target (C, Rust, Python)};
    \node[node, below=0.2cm of lang] (mlir) {MLIR (High-level Dialects)};
    \node[node, below=0.2cm of mlir] (lvm) {LLVM IR (Low-level IR)};
    \node[node, below=0.2cm of lvm] (bin) {Machine Code (x86, ARM)};
    
    \draw[->] (lang) -- (mlir);
    \draw[->] (mlir) -- (lvm);
    \draw[->] (lvm) -- (bin);
    \end{tikzpicture}%
    }
    \caption{Alur Kompilasi Modern: Dari Dialek MLIR ke Kode Mesin}
\end{figure}
