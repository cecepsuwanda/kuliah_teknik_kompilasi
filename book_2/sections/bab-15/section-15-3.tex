\section{Parser Generators}

\subsection{Bison (GNU Yacc)}
Bison adalah standar industri untuk pembuatan parser berbasis LALR(1). Ia sangat efisien untuk pengolahan tumpukan (\textit{batch processing}) kode sumber yang lengkap.

\subsection{Tree-sitter: Parsing Inkremental}
Baru-baru ini, \compiler{Tree-sitter} telah menjadi pilihan utama untuk alat bantu pengembang (\textit{editor, linter}). Berbeda dengan Bison, Tree-sitter dirancang untuk:
\begin{itemize}
    \item \textbf{Incremental Parsing}: Hanya memperbarui bagian pohon sintaksis yang berubah saat pengguna mengetik, bukan membedah ulang seluruh berkas.
    \item \textbf{Error Tolerance}: Mampu memberikan struktur pohon yang berguna meskipun kode sedang dalam keadaan tidak lengkap atau memiliki kesalahan sintaksis.
    \item \textbf{CST vs AST}: Tree-sitter menghasilkan \textit{Concrete Syntax Tree} (CST) yang menyimpan setiap detail termasuk spasi dan komentar.
\end{itemize}

\subsection{ANTLR (Another Tool for Language Recognition)}
ANTLR adalah parser generator modern yang sangat populer di ekosistem Java dan C\# karena kemudahannya dalam membangun pohon sintaksis abstrak (AST) secara otomatis menggunakan algoritma ALL(*).

\begin{table}[h]
\centering
\begin{tabular}{|l|l|l|}
\hline
\textbf{Fitur} & \textbf{Bison} & \textbf{Tree-sitter} \\
\hline
Tujuan & Kompilator (Batch) & Editor/IDE (Real-time) \\
Kecepatan & Sangat Cepat & Cepat (Incremental) \\
Penanganan Error & Terhenti/Terbatas & Robust (Tetap jalan) \\
\hline
\end{tabular}
\caption{Bison vs Tree-sitter}
\end{table}
