\section{Pengenalan Compiler Tools}

\compiler{Compiler Tools} adalah perangkat lunak pendukung yang mengotomatisasi berbagai tahapan dalam pembuatan kompilator. Penggunaan alat bantu ini memungkinkan pengembang fokus pada logika bahasa daripada rincian implementasi mesin yang repetitif.

\subsection{Kategori Alat Bantu}
Ekosistem pengembangan kompilator modern melibatkan berbagai standar industri:
\begin{itemize}
  \item \textbf{Lexer Generators (Flex, re2c)}: Mengubah spesifikasi ekspresi reguler menjadi kode C/C++ untuk pengenalan token.
  \item \textbf{Parser Generators (Bison, ANTLR)}: Mengubah tata bahasa bebas konteks (CFG) menjadi penganalisis sintaksis.
  \item \textbf{Frameworks (LLVM, GCC)}: Menyediakan infrastruktur untuk optimasi dan pembuatan kode mesin (\textit{back-end}).
  \item \textbf{Build Systems (Make, CMake)}: Mengelola proses kompilasi kode sumber yang kompleks.
\end{itemize}

\subsection{Language Server Protocol (LSP)}
Salah satu evolusi paling signifikan dalam sepuluh tahun terakhir adalah munculnya \compiler{Language Server Protocol (LSP)} \cite{oxford2024compilers}.
\begin{itemize}
    \item \textbf{Masalah M x N}: Dahulu, setiap editor (M) harus mengimplementasikan dukungan untuk setiap bahasa (N) secara manual.
    \item \textbf{Solusi LSP}: Kompilator bertindak sebagai "Server" yang menyediakan informasi cerdas (\textit{autocompletion, diagnostics, go to definition}) melalui protokol standar berbasis JSON-RPC.
    \item \textbf{Dampak}: Satu \textit{Language Server} dapat digunakan oleh editor mana pun yang mendukung LSP (VS Code, Vim, Emacs), memisahkan kecerdasan bahasa dari antarmuka pengguna.
\end{itemize}

\begin{figure}[!htbp]
    \centering
    \adjustbox{max width=0.8\textwidth,center}{%
    \begin{tikzpicture}[
        rect/.style={rectangle, draw=blue!50, fill=blue!10, text width=3cm, font=\tiny, align=center}
    ]
    \node[rect] (cl1) {Editor A};
    \node[rect, below=0.2cm of cl1] (cl2) {Editor B};
    \node[rect, right=2cm of cl1, minimum height=1.6cm, fill=orange!10] (lsp) {LSP Standard};
    \node[rect, right=2cm of lsp] (srv) {Compiler / Language Server};
    
    \draw[<->] (cl1) -- (lsp);
    \draw[<->] (cl2) -- (lsp);
    \draw[<->] (lsp) -- (srv);
    \end{tikzpicture}%
    }
    \caption{Efisiensi LSP: Menghubungkan berbagai editor ke satu otak kompilator}
\end{figure}
