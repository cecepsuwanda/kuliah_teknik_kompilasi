\section{Performance Analysis}

\subsection{FlameGraphs: Visualisasi Bottleneck}
Kompilator adalah aplikasi yang sangat haus sumber daya. Untuk mengidentifikasi fungsi mana yang paling lambat, pengembang menggunakan \compiler{FlameGraphs}.
\begin{itemize}
    \item \textbf{Visual}: Memberikan representasi visual dari pohon panggilan (\textit{call stacks}). Semakin lebar sebuah kotak, semakin banyak waktu CPU yang dihabiskan dalam fungsi tersebut.
    \item \textbf{Manfaat}: Memungkinkan identifikasi instan terhadap bagian optimasi kompilator yang terlalu membebani performa.
\end{itemize}

\subsection{Infrastruktur: Reproducible Builds}
Dalam proyek kompilator skala besar, konsistensi lingkungan sangat penting.
\begin{itemize}
    \item \textbf{Docker}: Digunakan untuk membungkus seluruh \textit{toolchain} kompilator agar setiap pengembang dan server CI/CD menggunakan versi GCC/LLVM, Flex, dan Bison yang identik. Hal ini menjamin biner yang dihasilkan selalu konsisten (\textit{reproducible}).
\end{itemize}

\begin{figure}[!htbp]
    \centering
    \adjustbox{max width=0.8\textwidth,center}{%
    \begin{tikzpicture}[
        rect/.style={rectangle, draw=orange!50, fill=orange!10, text width=5cm, align=center, font=\tiny}
    ]
    \node[rect] (p) {Profiling (Perf/Gprof)};
    \node[rect, below=0.2cm of p] (f) {Visualization (FlameGraphs)};
    \node[rect, below=0.2cm of f] (o) {Optimization (Refactor hot spots)};
    \draw[->] (p) -- (f);
    \draw[->] (f) -- (o);
    \end{tikzpicture}%
    }
    \caption{Siklus Optimasi Performa Kompilator}
\end{figure}
