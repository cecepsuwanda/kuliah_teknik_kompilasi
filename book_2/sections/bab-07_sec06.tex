\section{Konstruksi LR Parsing Table}

\subsection{Augmented Grammar}

Langkah pertama dalam konstruksi LR parsing table adalah membuat \textbf{augmented grammar}. Kita menambahkan production baru:
\begin{verbatim}
S' -> S
\end{verbatim}
di mana S adalah start symbol asli. Ini memungkinkan state accept yang unambiguous.

\subsection{LR Items}

LR item adalah production dengan dot (•) yang menandai posisi parsing saat ini. Format: \texttt{A -> $\alpha$ • $\beta$}

Contoh:
\begin{itemize}
    \item \texttt{E -> • E + T}: Belum membaca apapun dari production ini
    \item \texttt{E -> E • + T}: Sudah membaca E, menunggu +
    \item \texttt{E -> E + • T}: Sudah membaca E dan +, menunggu T
    \item \texttt{E -> E + T •}: Sudah membaca seluruh RHS, siap untuk reduce
\end{itemize}

\subsubsection{LR(0) Items}

LR(0) item hanya berisi production dengan dot, tanpa informasi lookahead.

\subsubsection{LR(1) Items}

LR(1) item adalah LR(0) item yang ditambahkan dengan lookahead token. Format: \texttt{[A -> $\alpha$ • $\beta$, a]} di mana \texttt{a} adalah lookahead token.

\subsection{Closure Operation}

Closure operation menambahkan semua production yang relevan ke set items. Jika kita memiliki item \texttt{[A -> $\alpha$ • B $\beta$]} dalam set, kita menambahkan semua items \texttt{[B -> • $\gamma$]} untuk setiap production \texttt{B -> $\gamma$}.

Algoritma closure:
\begin{enumerate}
    \item Mulai dengan set items awal
    \item Untuk setiap item \texttt{[A -> $\alpha$ • B $\beta$]}:
    \begin{itemize}
        \item Tambahkan semua items \texttt{[B -> • $\gamma$]} untuk setiap production \texttt{B -> $\gamma$}
        \item Jika LR(1), hitung lookahead untuk items baru
    \end{itemize}
    \item Ulangi sampai tidak ada item baru yang ditambahkan
\end{enumerate}

\subsection{GOTO Operation}

GOTO operation memindahkan dot melewati simbol grammar X. Jika kita memiliki item \texttt{[A -> $\alpha$ • X $\beta$]} dan membaca X, kita mendapatkan item \texttt{[A -> $\alpha$ X • $\beta$]}.

Algoritma GOTO:
\begin{enumerate}
    \item Mulai dengan set items I dan simbol grammar X
    \item Untuk setiap item \texttt{[A -> $\alpha$ • X $\beta$]} dalam I:
    \begin{itemize}
        \item Tambahkan \texttt{[A -> $\alpha$ X • $\beta$]} ke set baru
    \end{itemize}
    \item Ambil closure dari set baru
\end{enumerate}

\subsection{Canonical Collection of Item Sets}

Canonical collection adalah kumpulan semua state yang mungkin dalam LR automaton. Konstruksinya:

\begin{enumerate}
    \item Mulai dengan \texttt{I\_0 = closure(\{S' -> • S, \$\})}
    \item Untuk setiap state I dan setiap simbol grammar X:
    \begin{itemize}
        \item Hitung \texttt{GOTO(I, X)}
        \item Jika hasilnya non-empty dan belum ada, tambahkan sebagai state baru
    \end{itemize}
    \item Ulangi sampai tidak ada state baru
\end{enumerate}

\subsection{Konstruksi Action dan GOTO Tables}

Setelah canonical collection dibuat, kita konstruksi dua tabel:

\subsubsection{Action Table}

Action table menentukan aksi berdasarkan state dan lookahead token:
\begin{itemize}
    \item \textbf{Shift}: Jika \texttt{GOTO(I, a) = J} untuk terminal \texttt{a}, maka \texttt{action[I, a] = shift J}
    \item \textbf{Reduce}: Jika item \texttt{[A -> $\alpha$ •, a]} ada di state I, maka \texttt{action[I, a] = reduce A -> $\alpha$}
    \item \textbf{Accept}: Jika item \texttt{[S' -> S •, \$]} ada di state I, maka \texttt{action[I, \$] = accept}
    \item \textbf{Error}: Jika tidak ada aksi yang valid
\end{itemize}

\subsubsection{GOTO Table}

GOTO table menentukan state berikutnya setelah reduce:
\begin{itemize}
    \item Jika \texttt{GOTO(I, A) = J} untuk non-terminal \texttt{A}, maka \texttt{goto[I, A] = J}
\end{itemize}

\subsection{Contoh Konstruksi Parsing Table (Simplified)}

Mari kita lihat contoh sederhana untuk grammar:
\begin{verbatim}
S -> A A
A -> a A | b
\end{verbatim}

Augmented grammar:
\begin{verbatim}
S' -> S
S -> A A
A -> a A | b
\end{verbatim}

Langkah-langkah konstruksi (disederhanakan):
\begin{enumerate}
    \item Buat I\_0 dengan closure dari \texttt{S' -> • S}
    \item Hitung GOTO untuk setiap simbol
    \item Lanjutkan sampai semua state ditemukan
    \item Konstruksi action dan goto tables
\end{enumerate}