\section{Referensi dan Bahan Bacaan Lanjutan}

Untuk memperdalam pemahaman tentang top-down parsing dan recursive descent, mahasiswa disarankan membaca:

\begin{itemize}
    \item \textbf{Dragon Book}: Aho, Lam, Sethi, \& Ullman (2006). \textit{Compilers: Principles, Techniques, and Tools} \cite{aho2006compilers} - Bab 4: Syntax Analysis, Section 4.4: Top-Down Parsing
    
    \item \textbf{Engineering a Compiler}: Cooper \& Torczon (2011) \cite{cooper2011engineering} - Bab 3: Scanners, Section 3.4: Top-Down Parsing
    
    \item \textbf{OpenGenus Tutorial}: Build Lexer \cite{opengenus2024lexer} - Bagian tentang recursive descent parsing
    
    \item \textbf{USNA Course Notes}: Top-Down Parsing \footnote{\url{https://www.usna.edu/Users/cs/wcbrown/courses/F20SI413/lec/l09/lec.html}}
    
    \item \textbf{Ernest Chu Course Notes}: Syntax Analysis - Top-Down Parsing \footnote{\url{https://ernestchu.github.io/course-notes/courses/cse360-design-and-implementation-of-compiler/syntax-analysis/top-down-parsing.html}}
    
    \item \textbf{TutorialsPoint}: Compiler Design - Top Down Parser \footnote{\url{https://www.tutorialspoint.com/compiler_design/compiler_design_top_down_parser.htm}}
\end{itemize}