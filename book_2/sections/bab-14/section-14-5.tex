\section{Penanganan Eksepsi (Exception Handling)}

Saat terjadi kesalahan (\textit{error/exception}), program harus dapat "melompat" keluar dari banyak fungsi ke \textit{handler} yang sesuai tanpa merusak struktur data stack.

\subsection{Stack Unwinding}
Proses membersihkan \textit{activation records} satu per satu dari atas ke bawah untuk mencari penangkap (\textit{catch block}) yang cocok disebut \compiler{Stack Unwinding}.

\subsection{Metadata DWARF dan CFI}
Kompilator modern seringkali tidak menggunakan \textit{Frame Pointer} (\code{EBP}) untuk menghemat satu register. Tanpa \code{EBP}, bagaimana kita bisa menelusuri stack?
\begin{itemize}
    \item \textbf{CFI (Call Frame Information)}: Kompilator menyisipkan tabel metadata di bagian khusus berkas biner (\code{.eh\_frame} atau \code{.debug\_frame}).
    \item \textbf{Aturan Unwinding}: Metadata ini memberikan instruksi kepada \textit{runtime library} (seperti \code{libgcc}) tentang bagaimana cara menghitung alamat kembalian dan mengembalikan register hanya dengan menggunakan informasi lokasi kode (\code{Program Counter}).
\end{itemize}

\begin{figure}[!htbp]
    \centering
    \adjustbox{max width=0.8\textwidth,center}{%
    \begin{tikzpicture}[
        node/.style={rectangle, draw=purple!50, fill=purple!10, text width=6cm, font=\tiny, align=center}
    ]
    \node[node] (bin) {Berkas Biner (.exe / .elf)};
    \node[node, below=0.2cm of bin] (eh) {Seksi .eh\_frame (Tabel Aturan DWARF)};
    \node[node, below=0.2cm of eh] (unw) {Unwinder: Membaca tabel saat EXCEPTION terjadi.};
    \draw[->] (bin) -- (eh);
    \draw[->] (eh) -- (unw);
    \end{tikzpicture}%
    }
    \caption{Penanganan Eksepsi Tanpa Frame Pointer menggunakan DWARF}
\end{figure}
