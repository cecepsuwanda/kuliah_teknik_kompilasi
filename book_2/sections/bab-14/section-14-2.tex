\section{Procedure Call Mechanism}

\subsection{Calling Convention}

Prosedur call untuk function \code{func(a, b, c)}:

\begin{lstlisting}[language=C]
// Caller side:
push c              // Push parameters (right-to-left)
push b
push a
call func           // Push return address and jump

// Callee side (func entry):
push ebp            // Save old frame pointer
mov ebp, esp         // Set new frame pointer
sub esp, local_size // Allocate space for locals

// Function body here...

// Callee side (func exit):
mov esp, ebp         // Deallocate locals
pop ebp             // Restore old frame pointer
ret                 // Pop return address and jump
\end{lstlisting}

\subsection{Parameter Passing}

\begin{table}[h]
\centering
\begin{tabular}{|l|l|l|}
\hline
\textbf{Method} & \textbf{Pros} & \textbf{Cons} \\
\hline
Stack & Simple, unlimited parameters & Slow, memory access \\
Register & Fast, no memory access & Limited registers \\
Mixed & Balance of speed & Complex \\
\hline
\end{tabular}
\caption{Parameter Passing Methods}
\end{table}
