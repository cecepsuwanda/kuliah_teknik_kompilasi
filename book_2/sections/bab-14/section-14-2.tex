\section{Mekanisme Panggilan Prosedur (Procedure Call)}

Kompilator harus mengikuti standar \textit{Application Binary Interface} (ABI) agar kode yang dihasilkannya dapat berinteraksi dengan perpustakaan (\textit{libraries}) lain.

\subsection{Konvensi System V x86-64 ABI}
Pada sistem Linux x86-64, argumen pertama dikirimkan melalui register untuk performa maksimal \cite{jhu2024compilers}:
\begin{itemize}
    \item \code{RDI, RSI, RDX, RCX, R8, R9} (untuk argumen bilangan bulat/pointer).
    \item \code{XMM0 - XMM7} (untuk argumen \textit{floating-point}).
    \item Argumen selanjutnya baru diletakkan di dalam \textit{stack}.
\end{itemize}

\subsection{Perataan Stack (16-byte Alignment)}
ABI mengharuskan \textit{Stack Pointer} (\code{RSP}) selaras pada kelipatan 16 byte sebelum instruksi \code{CALL}.
\begin{itemize}
    \item \textbf{Alasan}: Mendukung instruksi vektor modern (SSE/AVX) yang akan mengalami \textit{crash} jika data di stack tidak selaras 16-byte.
    \item Kompilator menyisipkan \textit{padding} di dalam \textit{activation record} untuk menjamin keselarasan ini.
\end{itemize}

\subsection{Fungsi Variadik (va\_list)}
Fungsi seperti \code{printf(...)} yang menerima jumlah argumen fleksibel memerlukan penanganan khusus.
\begin{enumerate}
    \item \textbf{Register Save Area}: Di awal fungsi, semua register argumen (\code{RDI-R9}) disimpan ke area khusus di stack.
    \item \textbf{va\_list}: Adalah struktur data yang menyimpan \textit{offset} ke area penyimpanan register dan pointer ke area stack untuk argumen tambahan.
\end{enumerate}

\begin{figure}[!htbp]
    \centering
    \adjustbox{max width=0.8\textwidth,center}{%
    \begin{tikzpicture}[
        node/.style={rectangle, draw=blue!50, fill=blue!10, text width=5cm, font=\tiny, align=center}
    ]
    \node[node] (reg) {Register Area (Fast): RDI, RSI, RDX...};
    \node[node, below=0.2cm of reg] (stack) {Stack Area (Overflow): Arg \#7, \#8...};
    \draw[->] (reg) -- (stack) node[midway, right, font=\tiny] {va\_arg traversal};
    \end{tikzpicture}%
    }
    \caption{Layout Argumen untuk Fungsi Variadik}
\end{figure}
