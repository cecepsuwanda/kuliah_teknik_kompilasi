\section{Pengenalan Activation Records}

\compiler{Activation Record} (disebut juga \textit{stack frame}) adalah struktur data yang dibuat setiap kali sebuah fungsi dipanggil. Struktur ini menyimpan semua informasi yang diperlukan untuk eksekusi fungsi tersebut dan cara kembali ke fungsi pemanggil (\textit{caller}).

\subsection{Komponen Utama}
Secara umum, sebuah \textit{activation record} menyimpan:
\begin{itemize}
  \item \textbf{Local Variables}: Variabel yang dideklarasikan di dalam fungsi.
  \item \textbf{Parameters}: Argumen yang dikirimkan oleh pemanggil.
  \item \textbf{Return Address}: Alamat instruksi selanjutnya yang harus dijalankan setelah fungsi selesai.
  \item \textbf{Saved Registers}: Nilai register yang harus dijaga agar tidak rusak oleh fungsi ini.
  \item \textbf{Dynamic Link}: Penunjuk ke basis \textit{stack frame} fungsi pemanggil (\textit{previous FP}).
\end{itemize}

\subsection{Keamanan: Stack Canaries}
Untuk mencegah serangan \compiler{Buffer Overflow} yang dapat menimpa alamat kembalian (\textit{return address}), kompilator modern menyisipkan variabel penjaga yang disebut \textbf{Stack Canary} \cite{jhu2024compilers}.
\begin{enumerate}
    \item \textbf{Placement}: Kenari (nilai acak) diletakkan di antara variabel lokal dan data kendali (\textit{return address}).
    \item \textbf{Verification}: Sebelum kembali (\textit{ret}), kompilator memeriksa apakah nilai kenari masih sama.
    \item \textbf{Action}: Jika nilai berubah (indikasi \textit{stack smashing}), program akan segera dihentikan (\textit{aborted}).
\end{enumerate}

\begin{figure}[!htbp]
    \centering
    \adjustbox{max width=0.8\textwidth,center}{%
    \begin{tikzpicture}[
        rect/.style={rectangle, draw, minimum width=4cm, minimum height=0.6cm, font=\tiny, align=center}
    ]
    \node[rect, fill=blue!10] (ret) {Return Address};
    \node[rect, below=0pt of ret, fill=blue!10] (fp) {Saved EBP/RBP};
    \node[rect, below=0pt of fp, fill=red!20] (can) {\textbf{STACK CANARY (GUARD)}};
    \node[rect, below=0pt of can, fill=green!10] (vars) {Local Variables};
    \node[rect, below=0pt of vars, fill=green!10] (buf) {Buffer[N]};
    
    \draw[->, >=stealth, red, thick] (buf.west) .. controls (-3,-2) and (-3,-1) .. (ret.west) node[midway, left] {Overflow Path};
    \end{tikzpicture}%
    }
    \caption{Penempatan Stack Canary untuk melindungi Return Address}
\end{figure}
