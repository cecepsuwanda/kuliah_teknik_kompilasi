\section{Teknik Optimasi Stack}

Optimasi pada tingkat manajemen stack sangat krusial karena setiap panggilan fungsi memiliki biaya overhead (\textit{prologue} dan \textit{epilogue}).

\subsection{Optimasi Fungsi Daun (Leaf Function)}
Fungsi daun (\textit{leaf function}) adalah fungsi yang tidak memanggil fungsi lain.
\begin{itemize}
    \item \textbf{Optimasi}: Kompilator seringkali tidak membuat \textit{activation record} sama sekali jika variabel lokal dapat ditampung sepenuhnya dalam register. Instruksi \code{push} dan \code{pop} dihilangkan sepenuhnya.
\end{itemize}

\subsection{Tail Call vs Tail Recursion}
Sebuah \textit{panggilan ekor} (\cite{jhu2024compilers}) terjadi jika sebuah fungsi memanggil fungsi lain sebagai tindakan terakhirnya.
\begin{itemize}
    \item \textbf{Tail Recursion}: Fungsi memanggil dirinya sendiri di akhir. Kompilator mengubah rekursi menjadi \textit{loop} biasa (melompat ke awal fungsi).
    \item \textbf{Tail Call}: Fungsi memanggil fungsi BERBEDA di akhir. Kompilator dapat menghapus \textit{frame} saat ini sebelum melakukan \code{JUMP} (bukan \code{CALL}) ke fungsi target. Hal ini mencegah pertumbuhan stack yang tidak perlu.
\end{itemize}

\begin{figure}[!htbp]
    \centering
    \adjustbox{max width=0.8\textwidth,center}{%
    \begin{tikzpicture}[
        node/.style={rectangle, draw=red!50, fill=red!10, text width=6cm, font=\tiny, align=center}
    ]
    \node[node] (std) {Standard Call: PUSH Args $\rightarrow$ CALL f $\rightarrow$ RET};
    \node[node, below=0.2cm of std] (opt) {Tail Call Opt: POP Frame $\rightarrow$ JUMP f};
    \end{tikzpicture}%
    }
    \caption{Efisiensi Tail Call Optimization}
\end{figure}
