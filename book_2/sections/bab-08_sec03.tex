\section{Struktur File Grammar Bison}

File grammar Bison memiliki ekstensi \texttt{.y} dan terdiri dari tiga bagian utama yang dipisahkan oleh \texttt{\%\%}:

\subsection{Bagian 1: Definitions (Prologue)}

Bagian ini berisi:
\begin{itemize}
    \item Deklarasi token (\texttt{\%token})
    \item Deklarasi tipe semantic value (\texttt{\%union}, \texttt{\%type})
    \item Precedence dan associativity (\texttt{\%left}, \texttt{\%right}, \texttt{\%nonassoc})
    \item Kode C yang akan disisipkan (\texttt{\%\{ ... \%\}})
    \item Deklarasi start symbol (\texttt{\%start})
\end{itemize}

\subsection{Bagian 2: Grammar Rules}

Bagian ini berisi aturan-aturan grammar dengan semantic actions. Formatnya:

\begin{verbatim}
nonterminal: production1 { action1 }
           | production2 { action2 }
           | ...
           ;
\end{verbatim}

Gambar \ref{fig:bison-structure} menunjukkan struktur file grammar Bison.

\begin{figure}[!htbp]
    \centering
    \adjustbox{max width=0.85\textwidth,center}{%
    \begin{tikzpicture}[
        section/.style={rectangle, draw=blue!50, fill=blue!10, text width=6cm, minimum height=1cm, font=\footnotesize, align=left, inner sep=8pt, rounded corners},
        separator/.style={rectangle, draw=red!50, fill=red!10, text width=6cm, minimum height=0.3cm, font=\small\bfseries, align=center},
        arrow/.style={->, >=stealth, thick},
        node distance=0.4cm
    ]
    
    \node[section] (def) {Definitions Section\\\texttt{\%token}, \texttt{\%union}\\\texttt{\%left}, \texttt{\%right}\\\texttt{\%\{ \%\}} C code};
    \node[separator, below=of def] (sep1) {\%\%};
    \node[section, below=of sep1] (rules) {Grammar Rules Section\\Production rules\\Semantic actions};
    \node[separator, below=of rules] (sep2) {\%\%};
    \node[section, below=of sep2] (user) {Auxiliary Code Section\\\texttt{main()}, \texttt{yyerror()}\\Helper functions};
    
    \draw[arrow] (def) -- (sep1);
    \draw[arrow] (sep1) -- (rules);
    \draw[arrow] (rules) -- (sep2);
    \draw[arrow] (sep2) -- (user);
    
    \end{tikzpicture}%
    }
    \caption{Struktur file grammar Bison}
    \label{fig:bison-structure}
\end{figure}

\subsection{Bagian 3: Auxiliary Code}

Bagian ini berisi kode C tambahan seperti:
\begin{itemize}
    \item Fungsi \texttt{main()}
    \item Fungsi \texttt{yyerror()} untuk error handling
    \item Fungsi pendukung lainnya
\end{itemize}

Gambar \ref{fig:bison-workflow-complete} menunjukkan workflow lengkap dari specification hingga executable.

\begin{figure}[!htbp]
    \centering
    \adjustbox{max width=0.9\textwidth,center}{%
    \begin{tikzpicture}[
        file/.style={rectangle, draw=blue!50, fill=blue!10, text width=2.5cm, text centered, minimum height=0.7cm, rounded corners, font=\footnotesize, inner sep=4pt, align=center},
        process/.style={rectangle, draw=green!50, fill=green!10, text width=2.5cm, text centered, minimum height=0.7cm, rounded corners, font=\footnotesize, inner sep=4pt, align=center},
        arrow/.style={->, >=stealth, thick},
        node distance=1cm
    ]
    
    \node[file] (y) {parser.y};
    \node[file, below=of y] (l) {lexer.l};
    \node[process, right=of y] (bison) {bison};
    \node[process, right=of l] (flex) {flex};
    \node[file, right=of bison] (tab-c) {parser.tab.c};
    \node[file, right=of flex] (yy-c) {lex.yy.c};
    \node[process, below=of tab-c] (gcc) {gcc};
    \node[file, below=of gcc] (exe) {program};
    
    \draw[arrow] (y) -- (bison);
    \draw[arrow] (l) -- (flex);
    \draw[arrow] (bison) -- (tab-c);
    \draw[arrow] (flex) -- (yy-c);
    \draw[arrow] (tab-c) -- (gcc.north west);
    \draw[arrow] (yy-c) -- (gcc.north east);
    \draw[arrow] (gcc) -- (exe);
    
    \end{tikzpicture}%
    }
    \caption{Workflow lengkap Flex + Bison}
    \label{fig:bison-workflow-complete}
\end{figure}