\section{Kesimpulan}

Dalam bab ini, kita telah mempelajari:

\begin{enumerate}
    \item Keuntungan menggunakan lexer generator dibanding hand-written lexer
    \item Cara menggunakan Flex untuk membuat specification file dan generate lexer
    \item Cara menggunakan re2c dengan embedded specification
    \item Perbandingan antara hand-written dan generator-based lexer
    \item Integrasi lexer dengan parser menggunakan token definitions dan semantic values
\end{enumerate}

Generator-based lexer adalah pilihan yang tepat untuk sebagian besar kasus karena memberikan keseimbangan yang baik antara produktivitas, maintainability, dan performa. Namun, pemahaman tentang hand-written lexer (seperti yang dipelajari di bab sebelumnya) tetap penting untuk memahami proses tokenization secara mendalam.

Gambar \ref{fig:lexer-generator-ecosystem} menunjukkan ekosistem tools yang terkait dengan lexer generator.

\begin{figure}[H]
    \centering
    \adjustbox{max width=0.9\textwidth,center}{%
    \begin{tikzpicture}[
        tool/.style={rectangle, draw=blue!50, fill=blue!10, text width=2.5cm, minimum height=0.7cm, font=\footnotesize, align=center, rounded corners, inner sep=4pt},
        arrow/.style={->, >=stealth, thick},
        node distance=1.2cm
    ]
    
    \node[tool] (flex) {Flex};
    \node[tool, right=of flex] (re2c) {re2c};
    \node[tool, below=of flex] (bison) {Bison};
    \node[tool, right=of bison] (yacc) {Yacc};
    \node[tool, below=of bison] (antlr) {ANTLR};
    \node[tool, right=of antlr] (ragel) {Ragel};
    
    \draw[arrow] (flex) -- (bison);
    \draw[arrow] (re2c) -- (bison);
    \draw[arrow] (flex) to[out=225, in=45] (antlr);
    \draw[arrow] (re2c) to[out=315, in=135] (ragel);
    
    \end{tikzpicture}%
    }
    \caption{Ekosistem tools lexer dan parser generator}
    \label{fig:lexer-generator-ecosystem}
\end{figure}