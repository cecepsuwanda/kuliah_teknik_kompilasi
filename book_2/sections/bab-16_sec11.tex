\section{Best Practices untuk Project Final}

Berdasarkan pengalaman dan best practices dari berbagai compiler projects:

\subsection{Code Quality}

\begin{itemize}
    \item \textbf{Clean Code}: Kode yang readable dan well-structured
    \item \textbf{Modularity}: Komponen yang terpisah dengan jelas
    \item \textbf{Error Handling}: Comprehensive error handling dan reporting
    \item \textbf{Comments}: Dokumentasi yang adequate dalam kode
    \item \textbf{Consistency}: Konsistensi dalam coding style
\end{itemize}

\subsection{Testing}

\begin{itemize}
    \item \textbf{Test Coverage}: Test coverage yang komprehensif
    \item \textbf{Edge Cases}: Test untuk edge cases dan error conditions
    \item \textbf{Automated Testing}: Automated test suite
    \item \textbf{Regression Testing}: Test untuk mencegah regresi
\end{itemize}

\subsection{Documentation}

\begin{itemize}
    \item \textbf{Completeness}: Dokumentasi yang lengkap
    \item \textbf{Clarity}: Dokumentasi yang jelas dan mudah dipahami
    \item \textbf{Examples}: Contoh yang membantu
    \item \textbf{Maintenance}: Dokumentasi yang up-to-date
\end{itemize}

\subsection{Presentation}

\begin{itemize}
    \item \textbf{Preparation}: Persiapan yang matang
    \item \textbf{Clarity}: Presentasi yang jelas dan terstruktur
    \item \textbf{Demo}: Demo yang berjalan lancar
    \item \textbf{Time Management}: Manajemen waktu yang baik
\end{itemize}