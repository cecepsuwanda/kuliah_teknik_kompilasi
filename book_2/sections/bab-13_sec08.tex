\section{Kesimpulan}

Dalam bab ini, kita telah mempelajari:

\begin{enumerate}
    \item Runtime environment adalah konteks eksekusi program yang mencakup memory organization, calling conventions, dan memory management
    
    \item Memory layout terdiri dari code segment, static/global data, stack, dan heap, masing-masing dengan karakteristik dan tujuan penggunaan yang berbeda
    
    \item Activation records (stack frames) menyimpan informasi tentang eksekusi fungsi, termasuk parameters, local variables, return address, dan links
    
    \item Stack-based allocation cocok untuk local variables dengan automatic management, sementara heap allocation diperlukan untuk dynamic data dengan flexible lifetime
    
    \item Garbage collection adalah teknik automatic memory management yang membebaskan unreachable objects, dengan berbagai strategi seperti mark-and-sweep, reference counting, dan generational GC
\end{enumerate}

Pemahaman tentang runtime environment dan memory management sangat penting untuk:
\begin{itemize}
    \item Merancang compiler yang efisien
    \item Memahami bagaimana program dieksekusi
    \item Mengoptimalkan penggunaan memory
    \item Mengimplementasikan fitur bahasa seperti recursion, closures, dan dynamic allocation
\end{itemize}