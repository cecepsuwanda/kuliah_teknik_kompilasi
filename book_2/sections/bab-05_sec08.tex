\section{Left Recursion dan Left Factoring}

Gambar \ref{fig:left-recursion} menunjukkan konsep left recursion.

\begin{figure}[H]
    \centering
    \adjustbox{max width=0.85\textwidth,center}{%
    \begin{tikzpicture}[
        grammar/.style={rectangle, draw=blue!50, fill=blue!10, text width=4cm, minimum height=0.6cm, font=\tiny\ttfamily, align=left, inner sep=4pt, rounded corners},
        problem/.style={rectangle, draw=red!50, fill=red!10, text width=4cm, minimum height=0.6cm, font=\tiny, align=center, inner sep=4pt, rounded corners},
        solution/.style={rectangle, draw=green!50, fill=green!10, text width=4cm, minimum height=0.6cm, font=\tiny\ttfamily, align=left, inner sep=4pt, rounded corners},
        arrow/.style={->, >=stealth, thick},
        node distance=0.4cm
    ]
    
    \node[grammar] (g1) {E → E + T | T};
    \node[problem, below=of g1] (p1) {Problem: Infinite loop\\in top-down parser};
    \node[solution, below=of p1] (s1) {E → T E'\\E' → + T E' | ε};
    
    \draw[arrow] (g1) -- node[right, font=\tiny] {Eliminate} (p1);
    \draw[arrow] (p1) -- node[right, font=\tiny] {Solution} (s1);
    
    \end{tikzpicture}%
    }
    \caption{Left recursion dan eliminasi}
    \label{fig:left-recursion}
\end{figure}

\subsection{Left Recursion}

Left recursion terjadi ketika nonterminal muncul di posisi paling kiri dari produksinya sendiri. Contoh:

\begin{verbatim}
E → E + T | T
\end{verbatim}

Left recursion dapat menyebabkan masalah pada top-down parser (khususnya recursive descent) karena dapat menyebabkan infinite loop. Parser akan terus mencoba mem-parse \(E\) tanpa pernah maju.

\textbf{Eliminasi Left Recursion}:

Grammar dengan left recursion:
\begin{verbatim}
A → A a | b
\end{verbatim}

Dapat diubah menjadi:
\begin{verbatim}
A  → b A'
A' → a A' | epsilon
\end{verbatim}

Contoh: \texttt{E → E + T | T} menjadi:
\begin{verbatim}
E  → T E'
E' → + T E' | epsilon
\end{verbatim}

Gambar \ref{fig:left-factoring} menunjukkan konsep left factoring.

\begin{figure}[H]
    \centering
    \adjustbox{max width=0.85\textwidth,center}{%
    \begin{tikzpicture}[
        before/.style={rectangle, draw=blue!50, fill=blue!10, text width=4cm, minimum height=0.6cm, font=\tiny\ttfamily, align=left, inner sep=4pt, rounded corners},
        after/.style={rectangle, draw=green!50, fill=green!10, text width=4cm, minimum height=0.6cm, font=\tiny\ttfamily, align=left, inner sep=4pt, rounded corners},
        arrow/.style={->, >=stealth, thick},
        node distance=0.4cm
    ]
    
    \node[before] (b1) {S → if E then S else S};
    \node[before, below=of b1] (b2) {| if E then S};
    \node[after, below=1cm of b2] (a1) {S → if E then S S'};
    \node[after, below=of a1] (a2) {S' → else S | ε};
    
    \draw[arrow] (b2) to[out=-45, in=135] node[right, font=\tiny] {Factor} (a1);
    
    \end{tikzpicture}%
    }
    \caption{Left factoring: sebelum dan sesudah}
    \label{fig:left-factoring}
\end{figure}

\subsection{Left Factoring}

Left factoring diperlukan ketika beberapa produksi dimulai dengan simbol yang sama, membuat parser tidak dapat memutuskan produksi mana yang harus digunakan tanpa lookahead lebih lanjut.

Contoh grammar yang membutuhkan left factoring:
\begin{verbatim}
S → if E then S else S
  | if E then S
\end{verbatim}

Setelah left factoring:
\begin{verbatim}
S → if E then S S'
S' → else S | epsilon
\end{verbatim}