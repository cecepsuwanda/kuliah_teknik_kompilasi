\section{Tujuan Buku Ajar}

Pada era rekayasa perangkat lunak modern, pemahaman mendalam tentang mekanisme internal kompilator bukan sekadar tentang menerjemahkan kode, melainkan tentang penguasaan seni perancangan sistem yang kompleks. Buku ajar ini dirancang sebagai panduan komprehensif untuk menguasai \compiler{Teknik Kompilasi} secara sistematis dan terukur, selaras dengan standar \textit{Outcome-Based Education} (OBE) \cite{studylib2024obe}. Fokus utama buku ini adalah menjembatani kesenjangan antara teori ilmu komputer yang abstrak dengan praktik konstruksi perangkat lunak yang nyata, memastikan mahasiswa mampu membangun fondasi teoretis yang kuat sekaligus keterampilan praktis yang relevan dengan industri.

Kompilator adalah fondasi dari seluruh ekosistem komputasi, bertugas mentransformasi logika tingkat tinggi yang dipahami manusia menjadi instruksi mesin yang dapat dieksekusi oleh perangkat keras. Sebagaimana dijelaskan oleh \cite{aho2006compilers, uw2024compiler}, proses kompilasi ini melibatkan serangkaian fase yang rumit namun terstruktur, mulai dari analisis leksikal hingga optimasi kode target. Dengan memahami setiap fase ini, mahasiswa akan memperoleh wawasan mendalam tentang bagaimana perangkat lunak berinteraksi dengan sumber daya perangkat keras, sebuah kompetensi krusial untuk membangun aplikasi yang efisien, aman, dan berkinerja tinggi.

Pendekatan OBE yang diterapkan dalam buku ini memastikan bahwa setiap proses pembelajaran didorong oleh tujuan yang jelas dan hasil yang terukur (\textit{Outcome}). Buku ini tidak hanya bertujuan untuk menyampaikan materi, tetapi untuk memastikan bahwa setiap bab berkontribusi langsung pada pembentukan kompetensi spesifik mahasiswa. Mahasiswa tidak hanya dituntut untuk mengetahui "apa" itu \textit{parser} atau \textit{lexer}, tetapi juga "bagaimana" merancang dan mengimplementasikannya dalam sebuah proyek kompilator yang memenuhi standar kualitas industri.

Tujuan spesifik dari buku ajar ini meliputi:
\begin{enumerate}
  \item Memberikan pemahaman mendalam tentang setiap fase kompilasi, mulai dari analisis leksikal hingga generasi kode target.
  \item Mengembangkan kemampuan merancang dan mengimplementasikan komponen-komponen utama kompilator seperti \textit{lexer}, \textit{parser}, dan \textit{semantic analyzer}.
  \item Membangun keterampilan dalam optimasi kode dan manajemen memori pada \textit{runtime}.
  \item Memfasilitasi pencapaian \textit{Capaian Pembelajaran Lulusan} (CPL) dan \textit{Capaian Pembelajaran Mata Kuliah} (CPMK) yang telah ditetapkan dalam kurikulum.
\end{enumerate}

Setelah mempelajari buku ini secara menyeluruh, mahasiswa diharapkan mampu:
\begin{itemize}
  \item Menjelaskan arsitektur kompilator dan fungsi setiap fasenya.
  \item Membangun pemroses bahasa (\textit{language processor}) menggunakan teknik manual maupun generator (\textit{Flex/Bison}).
  \item Mengelola struktur data kompleks seperti \textit{Symbol Table} dan \textit{Abstract Syntax Tree} (AST).
  \item Menghasilkan kode target yang efisien untuk arsitektur mesin tertentu.
  \item Melakukan evaluasi performa dan optimasi pada tingkat \textit{intermediate code} dan \textit{target code}.
\end{itemize}
