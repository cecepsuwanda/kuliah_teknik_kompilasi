\section{Tujuan Buku Ajar}

Buku ajar ini dirancang sebagai panduan komprehensif untuk menguasai \compiler{Teknik Kompilasi} secara sistematis dan terukur, selaras dengan standar \textit{Outcome-Based Education} (OBE) \cite{studylib2024obe}. Fokus utama buku ini adalah pada pembangunan fondasi teoretis dan keterampilan praktis dalam membangun arsitektur kompilator modern. Menurut \cite{aho2006compilers}, kompilasi adalah proses transformasi dari bahasa sumber ke bahasa sasaran.
\begin{enumerate}
  \item Memberikan pemahaman mendalam tentang setiap fase kompilasi, mulai dari analisis leksikal hingga generasi kode target.
  \item Mengembangkan kemampuan merancang dan mengimplementasikan komponen-komponen utama kompilator seperti \textit{lexer}, \textit{parser}, dan \textit{semantic analyzer}.
  \item Membangun keterampilan dalam optimasi kode dan manajemen memori pada \textit{runtime}.
  \item Memfasilitasi pencapaian \textit{Capaian Pembelajaran Lulusan} (CPL) dan \textit{Capaian Pembelajaran Mata Kuliah} (CPMK) yang telah ditetapkan dalam kurikulum.
\end{enumerate}

Setelah mempelajari buku ini secara menyeluruh, mahasiswa diharapkan mampu:
\begin{itemize}
  \item Menjelaskan arsitektur kompilator dan fungsi setiap fasenya.
  \item Membangun pemroses bahasa (\textit{language processor}) menggunakan teknik manual maupun generator (\textit{Flex/Bison}).
  \item Mengelola struktur data kompleks seperti \textit{Symbol Table} dan \textit{Abstract Syntax Tree} (AST).
  \item Menghasilkan kode target yang efisien untuk arsitektur mesin tertentu.
  \item Melakukan evaluasi performa dan optimasi pada tingkat \textit{intermediate code} dan \textit{target code}.
\end{itemize}
