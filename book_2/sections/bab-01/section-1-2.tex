\section[Keterkaitan Buku Ajar dengan RPS Berbasis OBE]{Keterkaitan Buku Ajar dengan RPS \protect\\ Berbasis OBE}

Buku ajar ini disusun dengan penyelarasan yang ketat terhadap Rencana Pembelajaran Semester (RPS) mata kuliah Teknik Kompilasi yang menerapkan kerangka kerja \textit{Outcome-Based Education} (OBE). Keterkaitan ini menjamin bahwa setiap aktivitas kognitif yang dilakukan mahasiswa—mulai dari memahami definisi hingga merancang sistem—memiliki kontribusi langsung terhadap pencapaian Capaian Pembelajaran Lulusan (CPL). Integrasi yang kuat antara konten buku dan CPL memastikan bahwa keterampilan yang dikuasai mahasiswa relevan dengan kebutuhan dan standar kompetensi di industri teknologi informasi modern \cite{neu2024compiler}.

Secara spesifik, buku ini dirancang untuk memenuhi CPL-1 (Pengetahuan Rekayasa) dengan menyajikan teori otoritatif mengenai bahasa formal, automata, dan tata bahasa bebas konteks. Di sisi lain, buku ini juga secara intensif menargetkan CPL-3 (Perancangan dan Pengembangan Solusi) dengan membimbing mahasiswa melalui proses iteratif pembangunan kompilator yang fungsional. Keseimbangan ini memastikan mahasiswa tidak hanya tumbuh sebagai teoretisi yang memahami konsep abstrak, tetapi juga sebagai insinyur perangkat lunak yang mampu menghasilkan solusi konkret untuk masalah yang kompleks.

Aktivitas pembelajaran dalam buku ini disusun secara bertingkat mengikuti hierarki Taksonomi Bloom untuk memandu perkembangan kognitif mahasiswa. Dimulai dari level dasar seperti memahami sintaksis dan semantik bahasa, mahasiswa dibimbing menuju level analisis untuk memecahkan konflik \textit{parsing}, dan akhirnya mencapai level tertinggi yaitu menciptakan (\textit{creating}) sebuah pemroses bahasa yang utuh. Progresi bertahap ini sangat krusial dalam pendidikan teknik untuk membantu mahasiswa menguasai permasalahan rekayasa yang rumit secara terstruktur dan percaya diri.

\section{Arsitektur Kompilator Modern}

Arsitektur kompilator modern umumnya terbagi menjadi front-end dan back-end \cite{cooper2011engineering}.

\subsection{Alignment dengan CPL dan CPMK}

Struktur buku ini disusun untuk mendukung pencapaian indikator-indikator berikut:
\begin{itemize}
  \item \textbf{CPL-1 (Pengetahuan)}: Menguasai konsep teoretis analisis leksikal, sintaksis, semantik, dan generasi kode secara mendalam.
  \item \textbf{CPL-3 (Keterampilan Khusus)}: Mampu merancang, mengimplementasikan, dan mengevaluasi sistem kompilator lengkap.
  \item \textbf{CPMK-1 s.d CPMK-6}: Meliputi seluruh spektrum pengembangan kompilator dari arsitektur awal hingga evaluasi performa akhir.
\end{itemize}

Setiap bab dalam buku ini memuat daftar \textbf{Sub-CPMK} di bagian awal untuk memberikan fokus yang jelas bagi mahasiswa mengenai kompetensi spesifik yang akan dikuasai.

\subsection{Integrasi Metode Pembelajaran Aktif}

Sesuai dengan RPS berbasis OBE, buku ini mendukung berbagai metode pembelajaran:
\begin{itemize}
  \item \textbf{Problem-Based Learning}: Melalui studi kasus penanganan \textit{semantic errors} dan optimasi lokal.
  \item \textbf{Project-Based Learning}: Pengembangan kompilator secara bertahap dalam setiap bab.
  \item \textbf{Praktikum Terbimbing}: Implementasi komponen menggunakan \textit{tooling} industri seperti LLVM atau Clang.
\end{itemize}

\subsection{Sistem Evaluasi Berbasis Kompetensi}

Komponen asesmen yang disediakan di setiap akhir bab (latihan, asesmen, dan \textit{checklist}) dirancang untuk mengukur pencapaian \textit{Sub-CPMK} secara objektif, yang nantinya akan menjadi bobot penilaian utama dalam UTS dan UAS (total 35\% sesuai RPS).
