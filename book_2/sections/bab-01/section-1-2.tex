\section[Keterkaitan Buku Ajar dengan RPS Berbasis OBE]{Keterkaitan Buku Ajar dengan RPS \protect\\ Berbasis OBE}

Buku ajar ini dirancang selaras dengan Rencana Pembelajaran Semester (RPS) mata kuliah Teknik Kompilasi yang berbasis \textit{Outcome-Based Education} (OBE). Keterkaitan ini memastikan bahwa setiap materi yang disajikan memiliki kontribusi langsung terhadap kompetensi lulusan, sebagaimana diterapkan dalam standar kurikulum internasional \cite{neu2024compiler}.

\section{Arsitektur Kompilator Modern}

Arsitektur kompilator modern umumnya terbagi menjadi front-end dan back-end \cite{cooper2011engineering}.

\subsection{Alignment dengan CPL dan CPMK}

Struktur buku ini disusun untuk mendukung pencapaian indikator-indikator berikut:
\begin{itemize}
  \item \textbf{CPL-1 (Pengetahuan)}: Menguasai konsep teoretis analisis leksikal, sintaksis, semantik, dan generasi kode secara mendalam.
  \item \textbf{CPL-3 (Keterampilan Khusus)}: Mampu merancang, mengimplementasikan, dan mengevaluasi sistem kompilator lengkap.
  \item \textbf{CPMK-1 s.d CPMK-6}: Meliputi seluruh spektrum pengembangan kompilator dari arsitektur awal hingga evaluasi performa akhir.
\end{itemize}

Setiap bab dalam buku ini memuat daftar \textbf{Sub-CPMK} di bagian awal untuk memberikan fokus yang jelas bagi mahasiswa mengenai kompetensi spesifik yang akan dikuasai.

\subsection{Integrasi Metode Pembelajaran Aktif}

Sesuai dengan RPS berbasis OBE, buku ini mendukung berbagai metode pembelajaran:
\begin{itemize}
  \item \textbf{Problem-Based Learning}: Melalui studi kasus penanganan \textit{semantic errors} dan optimasi lokal.
  \item \textbf{Project-Based Learning}: Pengembangan kompilator secara bertahap dalam setiap bab.
  \item \textbf{Praktikum Terbimbing}: Implementasi komponen menggunakan \textit{tooling} industri seperti LLVM atau Clang.
\end{itemize}

\subsection{Sistem Evaluasi Berbasis Kompetensi}

Komponen asesmen yang disediakan di setiap akhir bab (latihan, asesmen, dan \textit{checklist}) dirancang untuk mengukur pencapaian \textit{Sub-CPMK} secara objektif, yang nantinya akan menjadi bobot penilaian utama dalam UTS dan UAS (total 35\% sesuai RPS).
