\section{Peta Konsep Teknik Kompilasi}

Buku ini disusun dalam 19 bab yang mencakup seluruh spektrum pengembangan kompilator, yang dapat dipandang sebagai sebuah ``pipa transformasi'' data. Proses dimulai dari aliran karakter mentah (\textit{source code}), yang kemudian diubah menjadi token linear, disusun menjadi struktur pohon hierarkis (\textit{Syntax Tree}), diperkaya dengan informasi semantik, diterjemahkan menjadi kode antara (\textit{Intermediate Representation}) yang agnostik terhadap mesin, dan akhirnya diekspansi menjadi instruksi mesin spesifik (\textit{Assembly}) yang optimal. Setiap bab dalam buku ini membedah satu ruas dari pipa tersebut secara mendalam.

Secara holistik, peta konsep ini tidak hanya mengajarkan teknik isolasi komponen, tetapi juga integrasi sistem. Mahasiswa diajak untuk melihat bagaimana keputusan desain di satu fase (misalnya, desain IR) berdampak signifikan pada fase berikutnya (optimasi dan generasi kode). Pemahaman lintas-fase ini adalah esensi dari pemikiran sistem (\textit{systems thinking}) yang ingin dibangun melalui mata kuliah ini.

Berikut adalah rincian materi per bab:
\begin{enumerate}
  \item \textbf{Bab I}: Pengenalan dan Konteks OBE
  \item \textbf{Bab II}: Arsitektur Kompilator - Gambaran umum sistem
  \item \textbf{Bab III-IV}: \textit{Front-end} - Analisis leksikal dan representasi regular
  \item \textbf{Bab V-VI}: \textit{Syntax Analysis} - \textit{Parsing} dan \textit{grammar} formal
  \item \textbf{Bab VII-X}: \textit{Middle-end} - \textit{Intermediate code}, tabel simbol, analisis semantik, dan penanganan kesalahan
  \item \textbf{Bab XI-XIV}: \textit{Back-end} - Tata letak memori, \textit{code generation}, alokasi register, dan manajemen \textit{stack}
  \item \textbf{Bab XV-XVI}: \textit{Analysis \& Evaluation} - \textit{Compiler tools} dan evaluasi performa
  \item \textbf{Bab XVII-XIX}: \textit{Assessment \& Resources} - Evaluasi kompetensi, lampiran, dan daftar referensi
\end{enumerate}

\textbf{Alur Pembelajaran:}
\begin{itemize}
  \item \textbf{Fase Analisis (Bab II-VI)}: Memahami bagaimana bahasa manusia diterjemahkan menjadi token dan pohon hirarki.
  \item \textbf{Fase Transformasi (Bab VII-X)}: Memastikan kebenaran makna dan mengubahnya menjadi representasi antara.
  \item \textbf{Fase Sintesis (Bab XI-XIV)}: Membangun instruksi mesin yang optimal sesuai arsitektur target.
  \item \textbf{Fase Profesional (Bab XV-XIX)}: Menggunakan alat bantu modern dan melakukan standarisasi kualitas.
\end{itemize}
