\section{Konteks Kurikulum OBE}

\subsection{Filosofi OBE dalam Teknik Kompilasi}

\compiler{Outcome-Based Education (OBE)} adalah pendekatan pendidikan yang berpusat pada mahasiswa (\textit{Student-Centered Learning}), di mana kurikulum dirancang berdasarkan apa yang seharusnya mampu dilakukan mahasiswa setelah lulus. Dalam konteks Teknik Kompilasi, filosofi ini mengubah fokus dari sekadar penyampaian materi algoritma menjadi pembentukan kemampuan rekayasa nyata. Tujuannya bukan hanya agar mahasiswa mengetahui teori \textit{parsing}, tetapi agar mereka mampu membangun parser yang efisien, tangguh, dan sesuai standar industri.

Pendekatan ini juga menekankan pentingnya penyelarasan konstruktif (\textit{Constructive Alignment}) antara aktivitas belajar, metode asesmen, dan capaian pembelajaran. Setiap tugas pemrograman dan ujian teori dalam buku ini dirancang secara spesifik untuk memvalidasi pencapaian kompetensi tertentu. Hal ini menjamin transparansi dalam proses penilaian, sehingga mahasiswa memahami dengan jelas ekspektasi dan standar kualitas yang harus mereka penuhi.

\textbf{Empat Prinsip Utama OBE dalam Buku Ini:}
\begin{enumerate}
  \item \textbf{Clarity of Focus}: Fokus pada hasil akhir berupa kompilator yang berfungsi.
  \item \textbf{Designing Down}: Materi disusun mundur dari kebutuhan akhir sebuah sistem \textit{backend} kompilator.
  \item \textbf{High Expectations}: Mahasiswa didorong untuk mengimplementasikan optimasi kode yang efisien.
  \item \textbf{Expanded Opportunity}: Menyediakan berbagai aktivitas belajar mulai dari teori hingga proyek tim.
\end{enumerate}

\subsection{Implementasi Tahapan OBE}

Implementasi OBE dalam buku ini juga mencakup siklus perbaikan kualitas berkelanjutan (\textit{Continuous Quality Improvement}). Melalui mekanisme \textit{feedback} yang terintegrasi dalam setiap bab, mahasiswa diajak untuk terus mengevaluasi dan memperbaiki hasil kerja mereka. Proses iteratif ini meniru siklus hidup pengembangan perangkat lunak profesional, di mana kode selalu ditinjau, diuji, dan dioptimalkan secara berkala.

Buku ini membagi proses pencapaian kompetensi dalam empat pilar utama:

\begin{table}[!htbp]
\centering
\begin{tabular}{|l|p{10.5cm}|}
\hline
\textbf{Komponen OBE} & \textbf{Implementasi dalam Teknik Kompilasi} \\
\hline
\textit{Defined Outcomes} & Sub-CPMK eksplisit untuk setiap fase (Leksikal, Sintaksis, dst). \\
\hline
\textit{Designing Down} & Kurikulum dimulai dari pengenalan bahasa ke deteksi kesalahan hingga emisi kode. \\
\hline
\textit{Student Activity} & Implementasi manual mesin \textit{state} dan penggunaan alat otomatisasi generator. \\
\hline
\textit{Continuous Assessment} & \textit{Weekly reflection} dan audit kualitas kode secara berkala. \\
\hline
\end{tabular}
\caption{Penerapan OBE dalam Pengembangan Kompilator}
\end{table}

\subsection{Hierarki Capaian Pembelajaran}

\begin{konsep}
Pemahaman mahasiswa terhadap Teknik Kompilasi divalidasi melalui hierarki capaian:
\begin{itemize}
    \item \textbf{CPL}: Mahasiswa menguasai teori kompilasi secara utuh sebagai sarjana teknik.
    \item \textbf{CPMK}: Mahasiswa mampu mengintegrasikan fase-fase kompilasi menjadi sistem yang koheren.
    \item \textbf{Sub-CPMK}: Mahasiswa mahir dalam satu spesialisasi fase (misalnya: \textit{Register Allocation}).
\end{itemize}
\end{konsep}
