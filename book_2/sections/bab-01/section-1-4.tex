\section{Konteks Kurikulum OBE}

\subsection{Filosofi OBE dalam Teknik Kompilasi}

\compiler{Outcome-Based Education (OBE)} adalah pendekatan yang menekankan pada apa yang bisa dilakukan oleh mahasiswa di akhir masa studi, bukan sekadar apa yang diajarkan. Dalam Teknik Kompilasi, hal ini berarti mahasiswa tidak hanya menghafal algoritma \textit{parsing}, tetapi mampu membangun sebuah program yang secara nyata dapat menerjemahkan sebuah bahasa ke bahasa lain.

\textbf{Empat Prinsip Utama OBE dalam Buku Ini:}
\begin{enumerate}
  \item \textbf{Clarity of Focus}: Fokus pada hasil akhir berupa kompilator yang berfungsi.
  \item \textbf{Designing Down}: Materi disusun mundur dari kebutuhan akhir sebuah sistem \textit{backend} kompilator.
  \item \textbf{High Expectations}: Mahasiswa didorong untuk mengimplementasikan optimasi kode yang efisien.
  \item \textbf{Expanded Opportunity}: Menyediakan berbagai aktivitas belajar mulai dari teori hingga proyek tim.
\end{enumerate}

\subsection{Implementasi Tahapan OBE}

Buku ini membagi proses pencapaian kompetensi dalam empat pilar utama:

\begin{table}[!htbp]
\centering
\begin{tabular}{|l|p{10.5cm}|}
\hline
\textbf{Komponen OBE} & \textbf{Implementasi dalam Teknik Kompilasi} \\
\hline
\textit{Defined Outcomes} & Sub-CPMK eksplisit untuk setiap fase (Leksikal, Sintaksis, dst). \\
\hline
\textit{Designing Down} & Kurikulum dimulai dari pengenalan bahasa ke deteksi kesalahan hingga emisi kode. \\
\hline
\textit{Student Activity} & Implementasi manual mesin \textit{state} dan penggunaan alat otomatisasi generator. \\
\hline
\textit{Continuous Assessment} & \textit{Weekly reflection} dan audit kualitas kode secara berkala. \\
\hline
\end{tabular}
\caption{Penerapan OBE dalam Pengembangan Kompilator}
\end{table}

\subsection{Hierarki Capaian Pembelajaran}

\begin{konsep}
Pemahaman mahasiswa terhadap Teknik Kompilasi divalidasi melalui hierarki capaian:
\begin{itemize}
    \item \textbf{CPL}: Mahasiswa menguasai teori kompilasi secara utuh sebagai sarjana teknik.
    \item \textbf{CPMK}: Mahasiswa mampu mengintegrasikan fase-fase kompilasi menjadi sistem yang koheren.
    \item \textbf{Sub-CPMK}: Mahasiswa mahir dalam satu spesialisasi fase (misalnya: \textit{Register Allocation}).
\end{itemize}
\end{konsep}
