\section{Petunjuk Penggunaan Buku Ajar}

\subsection{Untuk Mahasiswa}

Mengingat kompleksitas pengembangan kompilator, mahasiswa disarankan untuk menggunakan buku ini dengan langkah-langkah berikut:

\textbf{Tahap Persiapan:}
\begin{enumerate}
  \item Pahami target \textit{Sub-CPMK} di awal bab agar fokus belajar tetap terjaga.
  \item Tinjau kembali materi prasyarat (Struktur Data dan Algoritma) jika diperlukan.
\end{enumerate}

\textbf{Tahap Implementasi (Eksperimental):}
\begin{enumerate}
  \item Pelajari kode contoh yang disediakan dan jalankan menggunakan \textit{compiler} atau \textit{interpreter} yang sesuai.
  \item Lakukan modifikasi pada parameter \textit{lexer} atau aturan \textit{grammar} untuk melihat dampaknya terhadap proses \textit{parsing}.
  \item Gunakan perangkat lunak pendukung seperti \textit{Flex}, \textit{Bison}, atau \textit{Graphviz} untuk memvisualisasikan AST.
\end{enumerate}

\textbf{Tahap Evaluasi:}
\begin{enumerate}
  \item Kerjakan latihan refleksi untuk memperdalam pemahaman teoretis.
  \item Lakukan penilaian mandiri menggunakan \textit{checklist} kompetensi di akhir bab.
  \item Gabungkan komponen yang telah dibuat di setiap bab menjadi satu proyek kompilator utuh.
\end{enumerate}

\subsection{Untuk Dosen}

Dosen dapat memanfaatkan buku ini sebagai instrumen pembelajaran utama:
\begin{itemize}
  \item \textbf{Modul Praktikum}: Gunakan aktivitas pembelajaran di setiap bab sebagai panduan tugas mingguan.
  \item \textbf{Bank Soal}: Manfaatkan bagian asesmen sebagai referensi dalam menyusun soal UTS (CPMK-1, 2) dan UAS (CPMK 1 s.d 6).
  \item \textbf{Alat Ukur Capaian}: Gunakan rubrik penilaian dan indikator dalam buku untuk mengukur ketercapaian outcomes mahasiswa.
\end{itemize}
