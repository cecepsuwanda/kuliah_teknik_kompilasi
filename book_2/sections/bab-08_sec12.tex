\section{Contoh Praktis: Parser untuk Subset C}

Mari kita buat parser yang lebih kompleks untuk subset bahasa C yang mendukung:
\begin{itemize}
    \item Deklarasi variabel (int, float)
    \item Assignment statements
    \item Ekspresi aritmatika
    \item Print statements
\end{itemize}

\subsection{File Lexer (simplec.l)}

\begin{lstlisting}[language=C, caption={Lexer untuk subset C}]
%{
#include "simplec.tab.h"
#include <string.h>
%}

%%
"int"       { return INT; }
"float"     { return FLOAT; }
"print"     { return PRINT; }
[a-zA-Z_][a-zA-Z0-9_]* { 
    yylval.strval = strdup(yytext); 
    return IDENTIFIER; 
}
[0-9]+      { yylval.intval = atoi(yytext); return NUMBER; }
[0-9]+\.[0-9]+ { yylval.dval = atof(yytext); return FLOAT_LITERAL; }
[ \t\n]     { /* skip */ }
"="         { return ASSIGN; }
";"         { return SEMICOLON; }
"+"         { return PLUS; }
"-"         { return MINUS; }
"*"         { return MULTIPLY; }
"/"         { return DIVIDE; }
"("         { return LPAREN; }
")"         { return RPAREN; }
.           { return yytext[0]; }
%%

int yywrap() { return 1; }
\end{lstlisting}

\subsection{File Parser (simplec.y)}

\begin{lstlisting}[language=C, caption={Parser untuk subset C}]
%{
#include <stdio.h>
#include <stdlib.h>
#include <string.h>

extern int yylex();
void yyerror(const char *s);

// Symbol table sederhana
typedef struct {
    char *name;
    int type;  // 0 = int, 1 = float
    union {
        int intval;
        double dval;
    } value;
} Symbol;

Symbol symbols[100];
int sym_count = 0;
%}

%union {
    int intval;
    double dval;
    char *strval;
}

%token <intval> NUMBER
%token <dval> FLOAT_LITERAL
%token <strval> IDENTIFIER
%token INT FLOAT PRINT ASSIGN SEMICOLON
%token PLUS MINUS MULTIPLY DIVIDE LPAREN RPAREN

%left PLUS MINUS
%left MULTIPLY DIVIDE

%type <dval> expr

%%
program:
    /* empty */
  | program statement
  ;

statement:
    declaration SEMICOLON
  | assignment SEMICOLON
  | print_stmt SEMICOLON
  ;

declaration:
    INT IDENTIFIER { 
        // Add to symbol table
        symbols[sym_count].name = strdup($2);
        symbols[sym_count].type = 0;
        sym_count++;
        printf("Declared int variable: %s\n", $2);
    }
  | FLOAT IDENTIFIER {
        symbols[sym_count].name = strdup($2);
        symbols[sym_count].type = 1;
        sym_count++;
        printf("Declared float variable: %s\n", $2);
    }
  ;

assignment:
    IDENTIFIER ASSIGN expr {
        // Find variable in symbol table and assign
        printf("Assigning %g to %s\n", $3, $1);
    }
  ;

print_stmt:
    PRINT expr {
        printf("Print: %g\n", $2);
    }
  ;

expr:
    NUMBER { $$ = (double)$1; }
  | FLOAT_LITERAL { $$ = $1; }
  | IDENTIFIER { 
        // Lookup in symbol table
        $$ = 0.0; // Simplified
    }
  | expr PLUS expr { $$ = $1 + $3; }
  | expr MINUS expr { $$ = $1 - $3; }
  | expr MULTIPLY expr { $$ = $1 * $3; }
  | expr DIVIDE expr { 
        if ($3 == 0.0) {
            yyerror("Division by zero");
            $$ = 0.0;
        } else {
            $$ = $1 / $3;
        }
    }
  | LPAREN expr RPAREN { $$ = $2; }
  ;

%%

void yyerror(const char *s) {
    fprintf(stderr, "Syntax error: %s\n", s);
}

int main() {
    printf("Simple C Parser - Enter statements (Ctrl+D to exit)\n");
    yyparse();
    return 0;
}
\end{lstlisting}