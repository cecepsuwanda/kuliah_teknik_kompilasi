\section{Recursive Descent Parsing}

\subsection{Konsep Recursive Descent}

Recursive descent parsing adalah teknik implementasi top-down parsing di mana setiap non-terminal dalam grammar direpresentasikan sebagai sebuah fungsi. Fungsi-fungsi ini saling memanggil secara recursive sesuai dengan struktur grammar.

Menurut sumber dari Ernest Chu:

\begin{quote}
``Recursive-descent parsing is a hand-written parser (one function per non-terminal), possibly with backtracking. When you eliminate left recursion and factor grammar properly, you can build deterministic predictive parsers (LL(1))—recursive descent without backtracking.''\footnote{\url{https://ernestchu.github.io/course-notes/courses/cse360-design-and-implementation-of-compiler/syntax-analysis/top-down-parsing.html}}
\end{quote}

Struktur dasar recursive descent parser:
\begin{enumerate}
    \item Setiap non-terminal memiliki fungsi sendiri
    \item Fungsi membaca token dari input stream
    \item Fungsi memanggil fungsi lain sesuai dengan production rules
    \item Terminal dicocokkan langsung dengan token saat ini
\end{enumerate}

Gambar \ref{fig:recursive-descent-structure} menunjukkan struktur recursive descent parser.

\begin{figure}[H]
    \centering
    \adjustbox{max width=0.85\textwidth,center}{%
    \begin{tikzpicture}[
        func/.style={rectangle, draw=blue!50, fill=blue!10, text width=2.5cm, minimum height=0.6cm, font=\footnotesize, align=center, rounded corners, inner sep=4pt},
        arrow/.style={->, >=stealth, thick},
        node distance=0.6cm
    ]
    
    \node[func] (e) {E()};
    \node[func, below left=of e] (t) {T()};
    \node[func, below right=of e] (ep) {E'()};
    \node[func, below=of t] (f) {F()};
    
    \draw[arrow] (e) -- (t);
    \draw[arrow] (e) -- (ep);
    \draw[arrow] (t) -- (f);
    
    \node[below=0.3cm of f, font=\tiny, align=center] {One function\\per nonterminal};
    
    \end{tikzpicture}%
    }
    \caption{Struktur recursive descent parser}
    \label{fig:recursive-descent-structure}
\end{figure}

\subsection{Implementasi Dasar}

Mari kita lihat contoh implementasi recursive descent parser untuk grammar ekspresi aritmatika sederhana. Grammar yang akan kita gunakan:

\begin{verbatim}
E  -> T E'
E' -> + T E' | epsilon
T  -> F T'
T' -> * F T' | epsilon
F  -> ( E ) | id | num
\end{verbatim}

Grammar ini sudah dalam bentuk yang sesuai untuk LL(1) parsing karena:
\begin{itemize}
    \item Tidak ada left recursion
    \item Sudah di-factoring (E' dan T' menangani associativity)
    \item Setiap production dapat diputuskan dengan satu token lookahead
\end{itemize}

Gambar \ref{fig:recursive-descent-call} menunjukkan contoh pemanggilan fungsi dalam recursive descent parser.

\begin{figure}[H]
    \centering
    \adjustbox{max width=0.85\textwidth,center}{%
    \begin{tikzpicture}[
        func/.style={rectangle, draw=blue!50, fill=blue!10, text width=2.5cm, minimum height=0.6cm, font=\footnotesize\ttfamily, align=center, rounded corners, inner sep=4pt},
        call/.style={rectangle, draw=green!50, fill=green!10, text width=2cm, minimum height=0.5cm, font=\tiny\ttfamily, align=center, inner sep=4pt},
        arrow/.style={->, >=stealth, thick},
        node distance=0.6cm
    ]
    
    \node[func] (e) {parseE()};
    \node[call, below left=of e] (t) {parseT()};
    \node[call, below right=of e] (ep) {parseE'()};
    \node[call, below=of t] (f) {parseF()};
    
    \draw[arrow] (e) -- node[left, font=\tiny] {calls} (t);
    \draw[arrow] (e) -- node[right, font=\tiny] {calls} (ep);
    \draw[arrow] (t) -- node[left, font=\tiny] {calls} (f);
    
    \node[below=0.3cm of f, font=\tiny, align=center] {Recursive calls};
    
    \end{tikzpicture}%
    }
    \caption{Contoh pemanggilan fungsi recursive descent}
    \label{fig:recursive-descent-call}
\end{figure}

Implementasi dalam C++:

\begin{lstlisting}[language=C++, caption={Struktur dasar recursive descent parser}]
#include <iostream>
#include <string>
#include <vector>

enum TokenType {
    TOK_ID, TOK_NUM, TOK_PLUS, TOK_MUL,
    TOK_LPAREN, TOK_RPAREN, TOK_END, TOK_ERROR
};

struct Token {
    TokenType type;
    std::string lexeme;
    int line, col;
};

class RecursiveDescentParser {
private:
    std::vector<Token> tokens;
    size_t current;
    Token lookahead;
    
    void nextToken() {
        if (current < tokens.size()) {
            lookahead = tokens[current++];
        } else {
            lookahead = {TOK_END, "", 0, 0};
        }
    }
    
    void match(TokenType expected) {
        if (lookahead.type == expected) {
            nextToken();
        } else {
            error("Expected " + tokenToString(expected) + 
                  " but got " + lookahead.lexeme);
        }
    }
    
    void error(const std::string& msg) {
        std::cerr << "Syntax error at line " << lookahead.line 
                  << ", col " << lookahead.col << ": " << msg << std::endl;
        exit(1);
    }
    
public:
    RecursiveDescentParser(const std::vector<Token>& t) 
        : tokens(t), current(0) {
        nextToken();
    }
    
    // Grammar: E -> T E'
    void parseE() {
        parseT();
        parseEPrime();
    }
    
    // Grammar: E' -> + T E' | epsilon
    void parseEPrime() {
        if (lookahead.type == TOK_PLUS) {
            match(TOK_PLUS);
            parseT();
            parseEPrime();
        }
        // else: epsilon production, do nothing
    }
    
    // Grammar: T -> F T'
    void parseT() {
        parseF();
        parseTPrime();
    }
    
    // Grammar: T' -> * F T' | epsilon
    void parseTPrime() {
        if (lookahead.type == TOK_MUL) {
            match(TOK_MUL);
            parseF();
            parseTPrime();
        }
        // else: epsilon production, do nothing
    }
    
    // Grammar: F -> ( E ) | id | num
    void parseF() {
        if (lookahead.type == TOK_LPAREN) {
            match(TOK_LPAREN);
            parseE();
            match(TOK_RPAREN);
        } else if (lookahead.type == TOK_ID) {
            match(TOK_ID);
        } else if (lookahead.type == TOK_NUM) {
            match(TOK_NUM);
        } else {
            error("Expected identifier, number, or '('");
        }
    }
    
    void parse() {
        parseE();
        if (lookahead.type != TOK_END) {
            error("Extra input after expression");
        }
        std::cout << "Parse successful!" << std::endl;
    }
};
\end{lstlisting}