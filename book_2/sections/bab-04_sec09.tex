\section{Referensi dan Bahan Bacaan Lanjutan}

Untuk memperdalam pemahaman tentang lexer generator, mahasiswa disarankan membaca:

\begin{itemize}
    \item \textbf{Flex Manual}: Dokumentasi resmi Flex \footnote{\url{https://www.gnu.org/software/flex/manual/}}
    
    \item \textbf{re2c Documentation}: Dokumentasi dan tutorial re2c \footnote{\url{https://re2c.org/}}
    
    \item \textbf{flex \& bison}: Levine, J. R. (2009). \textit{flex \& bison: Text Processing Tools} \cite{levine2009flex} - Bab 2: Using Flex
    
    \item \textbf{Engineering a Compiler}: Cooper \& Torczon (2011) \cite{cooper2011engineering} - Bab 2: Lexical Analysis (bagian tentang lexer generators)
    
    \item \textbf{IT Trip - C Parser Flex Bison}: Tutorial tentang integrasi Flex dan Bison \cite{ittrip2024bison}
    
    \item \textbf{Wikipedia - re2c}: Artikel tentang re2c \cite{wikipedia2024re2c}
    
    \item \textbf{Wikipedia - RE/flex}: Artikel tentang RE/flex (modern C++ lexer generator) \footnote{\url{https://en.wikipedia.org/wiki/Draft:RE/flex}}
\end{itemize}