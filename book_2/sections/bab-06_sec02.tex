\section{Pendahuluan}

Setelah mempelajari lexical analysis dan context-free grammar pada bab-bab sebelumnya, kita sekarang akan mempelajari bagaimana mengimplementasikan parser yang menganalisis struktur sintaksis dari stream token yang dihasilkan oleh lexer. Top-down parsing adalah salah satu pendekatan yang paling intuitif dan mudah diimplementasikan secara manual.

Menurut sumber terbuka:

\begin{quote}
``Top-down parsers (recursive descent) – easy to hand-write; better for LL(1) grammars, when unambiguous. Works by writing functions for grammar nonterminals (e.g. expression(), term(), factor()) that consume tokens one at a time.''\cite{opengenus2024lexer}
\end{quote}

Pendekatan top-down parsing dimulai dari start symbol grammar dan mencoba menurunkan (derive) input dengan membangun parse tree dari root ke leaves. Ini berbeda dengan bottom-up parsing yang membangun parse tree dari leaves ke root.

Gambar \ref{fig:top-down-vs-bottom-up} menunjukkan perbandingan top-down dan bottom-up parsing.

\begin{figure}[H]
    \centering
    \adjustbox{max width=0.9\textwidth,center}{%
    \begin{tikzpicture}[
        box/.style={rectangle, draw=blue!50, fill=blue!10, text width=3cm, text centered, minimum height=0.7cm, rounded corners, font=\footnotesize, inner sep=4pt, align=center},
        arrow/.style={->, >=stealth, thick},
        node distance=1.5cm
    ]
    
    % Top-down
    \node[box] (td-title) {TOP-DOWN};
    \node[box, below=of td-title] (td1) {Start Symbol};
    \node[box, below=of td1] (td2) {Apply Rules};
    \node[box, below=of td2] (td3) {Match Input};
    \node[box, below=of td3] (td4) {Parse Tree};
    
    \draw[arrow] (td1) -- (td2);
    \draw[arrow] (td2) -- (td3);
    \draw[arrow] (td3) -- (td4);
    
    % Bottom-up
    \node[box, right=4cm of td-title] (bu-title) {BOTTOM-UP};
    \node[box, below=of bu-title] (bu1) {Input Tokens};
    \node[box, below=of bu1] (bu2) {Reduce Rules};
    \node[box, below=of bu2] (bu3) {Build Up};
    \node[box, below=of bu3] (bu4) {Start Symbol};
    
    \draw[arrow] (bu1) -- (bu2);
    \draw[arrow] (bu2) -- (bu3);
    \draw[arrow] (bu3) -- (bu4);
    
    \end{tikzpicture}%
    }
    \caption{Perbandingan top-down dan bottom-up parsing}
    \label{fig:top-down-vs-bottom-up}
\end{figure}