\section{Semantic Values dan Semantic Actions}

\subsection{Konsep Semantic Values}

Setiap token (terminal) dan nonterminal dalam grammar dapat membawa \textbf{semantic value}—sebuah nilai data yang diasosiasikan dengan simbol tersebut. Semantic value dapat berupa:
\begin{itemize}
    \item Integer (untuk literal angka)
    \item String (untuk identifier)
    \item Pointer ke AST node
    \item Tipe data kompleks lainnya
\end{itemize}

Dalam Bison:
\begin{itemize}
    \item \texttt{\$\$} merujuk pada semantic value dari \textbf{left-hand side (LHS)} (hasil produksi)
    \item \texttt{\$1}, \texttt{\$2}, ..., \texttt{\$n} merujuk pada semantic value dari simbol ke-1, ke-2, ..., ke-n di \textbf{right-hand side (RHS)}
\end{itemize}

\subsection{Semantic Actions}

\textbf{Semantic actions} adalah blok kode C yang dieksekusi ketika sebuah produksi di-reduce. Actions dapat digunakan untuk:
\begin{itemize}
    \item Membangun AST
    \item Mengevaluasi ekspresi
    \item Memvalidasi semantik
    \item Menghasilkan output
\end{itemize}

Contoh sederhana:

\begin{lstlisting}[language=C, caption={Contoh semantic action sederhana}]
expr: expr '+' expr { $$ = $1 + $3; }
    | expr '-' expr { $$ = $1 - $3; }
    | NUMBER        { $$ = $1; }
    ;
\end{lstlisting}

Pada contoh di atas, ketika parser menemukan \texttt{expr '+' expr}, action \texttt{\$\$ = \$1 + \$3} akan dieksekusi untuk menghitung jumlah dari kedua ekspresi.

Gambar \ref{fig:semantic-actions} menunjukkan bagaimana semantic actions bekerja dalam Bison.

\begin{figure}[H]
    \centering
    \adjustbox{max width=0.85\textwidth,center}{%
    \begin{tikzpicture}[
        rule/.style={rectangle, draw=blue!50, fill=blue!10, text width=4cm, minimum height=0.6cm, font=\tiny\ttfamily, align=left, inner sep=4pt, rounded corners},
        action/.style={rectangle, draw=green!50, fill=green!10, text width=4cm, minimum height=0.6cm, font=\tiny\ttfamily, align=left, inner sep=4pt, rounded corners},
        arrow/.style={->, >=stealth, thick},
        node distance=0.4cm
    ]
    
    \node[rule] (r1) {expr: expr '+' expr};
    \node[action, below=of r1] (a1) {\{ \$\$ = \$1 + \$3; \}};
    
    \draw[arrow] (r1) -- node[right, font=\tiny] {Action} (a1);
    
    \node[below=0.3cm of a1, font=\tiny, align=center] {\$\$ = LHS, \$1, \$2, \$3 = RHS};
    
    \end{tikzpicture}%
    }
    \caption{Semantic actions dalam Bison}
    \label{fig:semantic-actions}
\end{figure}