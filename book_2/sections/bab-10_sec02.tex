\section{Pendahuluan}

Symbol table adalah struktur data fundamental dalam kompilator yang menyimpan informasi tentang identifier yang digunakan dalam program. Menurut sumber terbuka:

\begin{quote}
``Symbol tables: data structures mapping names to declarations, with nested scopes. Semantic analysis includes name resolution: every use of a variable, function, type must refer to a declaration.''\cite{nguyen2024semantic}
\end{quote}

Symbol table berfungsi sebagai \textit{database} yang menghubungkan setiap penggunaan identifier dengan deklarasinya. Tanpa symbol table, kompilator tidak dapat memverifikasi apakah variabel yang digunakan sudah dideklarasikan, apakah tipe data sesuai, atau apakah fungsi dipanggil dengan parameter yang benar.

Gambar \ref{fig:symbol-table-overview} menunjukkan peran symbol table dalam kompilator.

\begin{figure}[!htbp]
    \centering
    \adjustbox{max width=0.9\textwidth,center}{%
    \begin{tikzpicture}[
        box/.style={rectangle, draw=blue!50, fill=blue!10, text width=2.5cm, text centered, minimum height=0.7cm, rounded corners, font=\footnotesize, inner sep=4pt, align=center},
        table/.style={rectangle, draw=green!50, fill=green!10, text width=3cm, minimum height=0.7cm, font=\footnotesize, align=center, rounded corners, inner sep=4pt},
        arrow/.style={->, >=stealth, thick},
        node distance=1.2cm
    ]
    
    \node[box] (parser) {Parser};
    \node[table, right=of parser] (st) {Symbol\\Table};
    \node[box, right=of st] (checker) {Type\\Checker};
    \node[box, below=of st] (codegen) {Code\\Generator};
    
    \draw[arrow] (parser) -- node[above, font=\tiny, align=center] {Insert} (st);
    \draw[arrow] (st) -- node[above, font=\tiny, align=center] {Lookup} (checker);
    \draw[arrow] (st) -- node[right, font=\tiny, align=center] {Info} (codegen);
    
    \end{tikzpicture}%
    }
    \caption{Peran symbol table dalam kompilator}
    \label{fig:symbol-table-overview}
\end{figure}

\subsection{Informasi yang Disimpan dalam Symbol Table}

Setiap entry dalam symbol table menyimpan berbagai informasi tentang identifier:

\begin{itemize}
    \item \textbf{Nama Identifier}: String yang merepresentasikan nama variabel, fungsi, atau tipe
    \item \textbf{Tipe Data}: Tipe dari identifier (int, float, function, struct, dll.)
    \item \textbf{Scope Level}: Level nesting scope di mana identifier dideklarasikan
    \item \textbf{Memory Location}: Alamat atau offset memory untuk variabel (digunakan dalam code generation)
    \item \textbf{Line Number}: Posisi deklarasi dalam source code (untuk error reporting)
    \item \textbf{Attributes Tambahan}: 
    \begin{itemize}
        \item Untuk fungsi: parameter list, return type, calling convention
        \item Untuk variabel: storage class (static, auto, register), initial value
        \item Untuk array: dimensi dan ukuran
        \item Untuk struct: field list
    \end{itemize}
\end{itemize}

\subsection{Operasi Dasar pada Symbol Table}

Symbol table harus mendukung operasi-operasi berikut:

\begin{enumerate}
    \item \textbf{Insert}: Menambahkan entry baru untuk identifier yang dideklarasikan
    \item \textbf{Lookup}: Mencari identifier dalam symbol table untuk name resolution
    \item \textbf{Delete}: Menghapus entry ketika scope berakhir (untuk nested scopes)
    \item \textbf{Update}: Memperbarui informasi identifier (misalnya setelah type inference)
\end{enumerate}