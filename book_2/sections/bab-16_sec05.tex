\section{Review Materi: Fase-Fase Kompilasi}

Sebelum melakukan evaluasi tools dan teknik, penting untuk mereview kembali semua fase kompilasi yang telah dipelajari:

\subsection{Front-End Phases}

\textbf{1. Lexical Analysis}
\begin{itemize}
    \item Memecah source code menjadi tokens
    \item Implementasi: hand-written atau menggunakan Flex/re2c
    \item Output: stream of tokens
\end{itemize}

\textbf{2. Syntax Analysis}
\begin{itemize}
    \item Memverifikasi struktur grammar
    \item Implementasi: recursive descent, LR parser, atau Bison
    \item Output: Abstract Syntax Tree (AST)
\end{itemize}

\textbf{3. Semantic Analysis}
\begin{itemize}
    \item Type checking, scope resolution, name resolution
    \item Implementasi: tree traversal dengan symbol table
    \item Output: Annotated AST dengan type information
\end{itemize}

\subsection{Back-End Phases}

\textbf{4. Intermediate Code Generation}
\begin{itemize}
    \item Mengkonversi AST menjadi IR (three-address code)
    \item Output: Intermediate Representation
\end{itemize}

\textbf{5. Code Optimization}
\begin{itemize}
    \item Optimasi lokal dan global
    \item Output: Optimized IR
\end{itemize}

\textbf{6. Code Generation}
\begin{itemize}
    \item Mengkonversi IR menjadi target code (assembly)
    \item Output: Assembly code atau machine code
\end{itemize}

\subsection{Integration Points}

Setiap fase harus terintegrasi dengan baik:
\begin{itemize}
    \item Lexer → Parser: Token stream
    \item Parser → Semantic Analyzer: AST
    \item Semantic Analyzer → IR Generator: Annotated AST
    \item IR Generator → Optimizer: IR
    \item Optimizer → Code Generator: Optimized IR
    \item Code Generator → Assembler: Assembly code
\end{itemize}