\section{Pendahuluan}

Setelah fase syntax analysis menghasilkan Abstract Syntax Tree (AST), kompilator perlu memastikan bahwa program tidak hanya valid secara sintaksis, tetapi juga secara semantik. Semantic analysis adalah fase yang memverifikasi bahwa program memenuhi aturan semantik bahasa pemrograman.

Menurut sumber dari Nguyen Thanh Vu:

\begin{quote}
``Type checking: operator operands must be type-compatible. Return types match declared types. Implicit/explicit conversions. Semantic analysis ensures that the parse tree makes sense under language rules.''\cite{nguyen2024semantic}
\end{quote}

Semantic analysis bertanggung jawab untuk memeriksa berbagai aspek semantik program:

\begin{itemize}
    \item \textbf{Type Checking}: Memastikan operasi dilakukan pada tipe yang kompatibel
    \item \textbf{Scope Resolution}: Memastikan setiap identifier merujuk ke deklarasi yang valid
    \item \textbf{Name Resolution}: Menyelesaikan referensi variabel, fungsi, dan tipe
    \item \textbf{Contextual Checks}: Memeriksa aturan spesifik bahasa (misalnya: break hanya dalam loop, return type match, dll.)
\end{itemize}

Gambar \ref{fig:semantic-analysis-overview} menunjukkan proses semantic analysis.

\begin{figure}[H]
    \centering
    \adjustbox{max width=0.9\textwidth,center}{%
    \begin{tikzpicture}[
        box/.style={rectangle, draw=blue!50, fill=blue!10, text width=2.5cm, text centered, minimum height=0.7cm, rounded corners, font=\footnotesize, inner sep=4pt, align=center},
        arrow/.style={->, >=stealth, thick},
        node distance=1.2cm
    ]
    
    \node[box] (ast) {AST};
    \node[box, right=of ast] (type) {Type\\Checking};
    \node[box, right=of type] (scope) {Scope\\Resolution};
    \node[box, below=of type] (annotated) {Annotated\\AST};
    
    \draw[arrow] (ast) -- (type);
    \draw[arrow] (type) -- (scope);
    \draw[arrow] (scope) -- (annotated);
    
    \end{tikzpicture}%
    }
    \caption{Proses semantic analysis}
    \label{fig:semantic-analysis-overview}
\end{figure}

\subsection{Input dan Output Semantic Analysis}

Semantic analyzer bekerja dengan:

\textbf{Input:}
\begin{itemize}
    \item Abstract Syntax Tree (AST) dari syntax analyzer
    \item Symbol table yang sudah dibangun (dari bab sebelumnya)
    \item Type information dari deklarasi
\end{itemize}

\textbf{Output:}
\begin{itemize}
    \item Annotated AST dengan informasi tipe pada setiap node
    \item Symbol table yang dilengkapi dengan informasi tipe
    \item Daftar semantic errors (jika ada)
    \item Type-checked program yang siap untuk code generation
\end{itemize}