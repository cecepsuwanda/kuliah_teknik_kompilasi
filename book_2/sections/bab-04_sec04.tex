\section{re2c (Regular Expressions to Code)}

re2c adalah lexer generator modern yang menghasilkan kode C/C++ dengan performa tinggi. Berbeda dengan Flex yang menggunakan file terpisah, re2c menggunakan \textbf{embedded specification} dalam kode C/C++.

Menurut dokumentasi resmi re2c:

\begin{quote}
``re2c is a tool that generates fast lexers for C, C++ and Go. It compiles regular expressions to deterministic finite automata and encodes them as conditional jumps and comparisons. The generated code is highly optimized and does not use tables.''\footnote{\url{https://re2c.org/}}
\end{quote}

\subsection{Struktur re2c Specification}

re2c specification ditulis dalam komentar khusus `/*!re2c ... */` yang disisipkan dalam kode C/C++:

\begin{lstlisting}[language=C, caption={Struktur dasar re2c}]
#include <stdio.h>

static int lex(const char *YYCURSOR) {
    const char *YYMARKER;
    /*!re2c
        re2c:define:YYCTYPE = "char";
        re2c:yyfill:enable = 0;
        
        // Named patterns
        digit    = [0-9];
        letter   = [a-zA-Z_];
        id       = letter (letter | digit)*;
        number   = digit+;
        
        // Rules
        *        { return 0; }  // error
        number   { printf("Number: %.*s\n", (int)(YYCURSOR - YYMARKER), YYMARKER); return 1; }
        id       { printf("ID: %.*s\n", (int)(YYCURSOR - YYMARKER), YYMARKER); return 1; }
        [ \t\n]+ { continue; }  // skip whitespace
    */
}

int main(int argc, char *argv[]) {
    for (int i = 1; i < argc; i++) {
        lex(argv[i]);
    }
    return 0;
}
\end{lstlisting}


Gambar \ref{fig:re2c-state-machine} menunjukkan bagaimana re2c menghasilkan state machine untuk pattern matching.

\begin{figure}[!htbp]
    \centering
    \adjustbox{max width=0.85\textwidth,center}{%
    \begin{tikzpicture}[
        state/.style={circle, draw=blue!50, fill=blue!10, minimum size=0.8cm, font=\tiny, align=center},
        accept/.style={circle, draw=green!50, fill=green!10, minimum size=0.8cm, font=\tiny, align=center, double},
        arrow/.style={->, >=stealth, thick},
        label/.style={font=\tiny, above},
        node distance=1.5cm
    ]
    
    \node[state] (s0) {S0};
    \node[state, right=of s0] (s1) {S1};
    \node[accept, right=of s1] (s2) {S2\\Accept};
    
    \draw[arrow] (s0) -- node[label] {digit} (s1);
    \draw[arrow] (s1) -- node[label] {digit} (s2);
    \draw[arrow] (s2) to[out=45, in=135, looseness=1.5] node[label] {digit} (s2);
    
    \node[below=0.3cm of s0, font=\tiny] {Start};
    \node[below=0.3cm of s2, font=\tiny] {Pattern: digit+};
    
    \end{tikzpicture}%
    }
    \caption{State machine yang dihasilkan re2c untuk pattern \texttt{digit+}}
    \label{fig:re2c-state-machine}
\end{figure}

\subsection{Key Concepts dalam re2c}

Gambar \ref{fig:re2c-variables} menunjukkan penggunaan variabel khusus dalam re2c.

\begin{figure}[!htbp]
    \centering
    \adjustbox{max width=\textwidth,center}{%
    \begin{tikzpicture}[
        char/.style={rectangle, draw=gray!40, fill=gray!5, minimum width=0.6cm, minimum height=0.6cm,
                     font=\footnotesize\ttfamily, align=center},
        pointer/.style={rectangle, draw=red!60, fill=red!10, minimum width=0.25cm, minimum height=0.25cm},
        label/.style={font=\scriptsize},
        arrow/.style={->, >=stealth, thick},
        node distance=0.35cm,
    ]
    
    % =========================
    % Input Buffer as Characters
    % =========================
    \matrix (m) [matrix of nodes, column sep=0.5cm] {
    |[char]| i & |[char]| n & |[char]| t & |[char]| ~ & |[char]| x & |[char]| ~ & |[char]| = & |[char]| ~ & |[char]| 4 & |[char]| 2 & |[char]| ; \\
    };
    
    % =========================
    % Pointers
    % =========================
    \node[pointer, above=0.5cm of m-1-1] (cursor) {};
    \node[pointer, above=0.5cm of m-1-5, draw=blue!60, fill=blue!10] (marker) {};
    \node[pointer, above=0.5cm of m-1-11, draw=green!60, fill=green!10] (limit) {};
    
    \node[label, above=0.15cm of cursor] {YYCURSOR};
    \node[label, above=0.15cm of marker] {YYMARKER};
    \node[label, above=0.15cm of limit] {YYLIMIT};
    
    % =========================
    % Arrows
    % =========================
    \draw[arrow, red!70] (cursor) -- (m-1-1);
    \draw[arrow, blue!70] (marker) -- (m-1-5);
    \draw[arrow, green!70] (limit) -- (m-1-11);
    
    \end{tikzpicture}%
    }
    \caption{Ilustrasi pointer internal \texttt{re2c} dalam buffer input}
    \label{fig:re2c-variables}
    \end{figure}
    

\subsubsection{Variables Khusus}

re2c menggunakan variabel khusus untuk tracking posisi input:
\begin{itemize}
    \item \texttt{YYCURSOR}: Pointer ke posisi saat ini dalam input
    \item \texttt{YYMARKER}: Pointer untuk backtracking
    \item \texttt{YYLIMIT}: Pointer ke akhir buffer
    \item \texttt{YYCTYPE}: Tipe data untuk karakter (default: \texttt{unsigned char})
\end{itemize}

\subsubsection{Directives}

Directives mengkonfigurasi behavior re2c:
\begin{itemize}
    \item \texttt{re2c:define:YYCTYPE}: Mendefinisikan tipe karakter
    \item \texttt{re2c:yyfill:enable}: Enable/disable buffer filling
    \item \texttt{re2c:input}: Mendefinisikan cara membaca input
    \item \texttt{re2c:conditions}: Enable start conditions (seperti Flex)
\end{itemize}

Gambar \ref{fig:re2c-pattern-examples} menunjukkan berbagai pattern yang dapat digunakan dalam re2c.

\begin{figure}[!htbp]
    \centering
    \adjustbox{max width=0.85\textwidth,center}{%
    \begin{tikzpicture}[
        pattern/.style={rectangle, draw=blue!50, fill=blue!10, text width=3cm, minimum height=0.6cm, font=\tiny\ttfamily, align=left, inner sep=4pt, rounded corners},
        example/.style={rectangle, draw=green!50, fill=green!10, text width=3cm, minimum height=0.5cm, font=\tiny\ttfamily, align=left, inner sep=4pt},
        arrow/.style={->, >=stealth, thick},
        node distance=0.3cm
    ]
    
    \node[pattern] (p1) {Pattern: \texttt{[0-9]+}};
    \node[example, below=of p1] (e1) {Matches: 42, 123};
    
    \node[pattern, right=of p1] (p2) {Pattern: \texttt{[a-z]+}};
    \node[example, below=of p2] (e2) {Matches: hello, world};
    
    \node[pattern, right=of p2] (p3) {Pattern: \texttt{"if"}};
    \node[example, below=of p3] (e3) {Matches: if};
    
    \draw[arrow] (p1) -- (e1);
    \draw[arrow] (p2) -- (e2);
    \draw[arrow] (p3) -- (e3);
    
    \end{tikzpicture}%
    }
    \caption{Contoh pattern dalam re2c}
    \label{fig:re2c-pattern-examples}
\end{figure}

\subsection{Contoh Lengkap: re2c Lexer untuk Identifier dan Number}

\begin{lstlisting}[language=C, caption={Contoh re2c lexer (lexer.re.c)}]
#include <stdio.h>
#include <stdlib.h>
#include <string.h>

typedef enum {
    TOKEN_EOF,
    TOKEN_IDENTIFIER,
    TOKEN_NUMBER,
    TOKEN_PLUS,
    TOKEN_MINUS,
    TOKEN_MULTIPLY,
    TOKEN_DIVIDE,
    TOKEN_ERROR
} TokenType;

typedef struct {
    TokenType type;
    char *value;
    int line;
} Token;

static Token tokenize(const char *input, int *line) {
    const char *YYCURSOR = input;
    const char *YYMARKER;
    const char *start;
    Token token = {TOKEN_EOF, NULL, *line};
    
    /*!re2c
        re2c:define:YYCTYPE = "char";
        re2c:yyfill:enable = 0;
        re2c:define:YYCURSOR = "YYCURSOR";
        
        digit    = [0-9];
        letter   = [a-zA-Z_];
        id       = letter (letter | digit)*;
        number   = digit+;
        ws       = [ \t]+;
        newline  = "\n";
        
        * {
            token.type = TOKEN_ERROR;
            return token;
        }
        
        "\x00" {
            token.type = TOKEN_EOF;
            return token;
        }
        
        ws {
            continue;
        }
        
        newline {
            (*line)++;
            continue;
        }
        
        number {
            start = YYMARKER;
            int len = YYCURSOR - start;
            token.value = (char*)malloc(len + 1);
            strncpy(token.value, start, len);
            token.value[len] = '\0';
            token.type = TOKEN_NUMBER;
            token.line = *line;
            return token;
        }
        
        id {
            start = YYMARKER;
            int len = YYCURSOR - start;
            token.value = (char*)malloc(len + 1);
            strncpy(token.value, start, len);
            token.value[len] = '\0';
            token.type = TOKEN_IDENTIFIER;
            token.line = *line;
            return token;
        }
        
        "+" {
            token.type = TOKEN_PLUS;
            token.line = *line;
            return token;
        }
        
        "-" {
            token.type = TOKEN_MINUS;
            token.line = *line;
            return token;
        }
        
        "*" {
            token.type = TOKEN_MULTIPLY;
            token.line = *line;
            return token;
        }
        
        "/" {
            token.type = TOKEN_DIVIDE;
            token.line = *line;
            return token;
        }
    */
}

int main(int argc, char *argv[]) {
    if (argc < 2) {
        fprintf(stderr, "Usage: %s <input>\n", argv[0]);
        return 1;
    }
    
    int line = 1;
    Token token;
    
    do {
        token = tokenize(argv[1], &line);
        printf("Token: %d, Value: %s, Line: %d\n", 
               token.type, token.value ? token.value : "NULL", token.line);
        if (token.value) free(token.value);
    } while (token.type != TOKEN_EOF && token.type != TOKEN_ERROR);
    
    return 0;
}
\end{lstlisting}

Untuk mengkompilasi:
\begin{verbatim}
re2c -o lexer.c lexer.re.c
gcc lexer.c -o lexer
\end{verbatim}


Gambar \ref{fig:comparison-table} menunjukkan perbandingan visual antara hand-written, Flex, dan re2c.

\begin{figure}[!htbp]
    \centering
    \adjustbox{max width=0.95\textwidth,center}{%
    \begin{tikzpicture}[
        box/.style={rectangle, draw=blue!50, fill=blue!10, text width=2.8cm, text centered, minimum height=0.6cm, rounded corners, font=\footnotesize, inner sep=4pt, align=center},
        good/.style={rectangle, draw=green!50, fill=green!10, text width=2.8cm, text centered, minimum height=0.6cm, rounded corners, font=\tiny, inner sep=4pt, align=center},
        medium/.style={rectangle, draw=orange!50, fill=orange!10, text width=2.8cm, text centered, minimum height=0.6cm, rounded corners, font=\tiny, inner sep=4pt, align=center},
        node distance=0.3cm and 0.2cm
    ]
    
    % Headers
    \node[box, font=\bfseries] (h1) {Aspek};
    \node[box, right=of h1, font=\bfseries] (h2) {Hand-Written};
    \node[box, right=of h2, font=\bfseries] (h3) {Flex};
    \node[box, right=of h3, font=\bfseries] (h4) {re2c};
    
    % Rows
    \node[box, below=of h1] (r1) {Produktivitas};
    \node[medium, below=of h2] (r1-hw) {Rendah};
    \node[good, below=of h3] (r1-flex) {Tinggi};
    \node[good, below=of h4] (r1-re2c) {Tinggi};
    
    \node[box, below=of r1] (r2) {Maintainability};
    \node[medium, below=of r1-hw] (r2-hw) {Sedang};
    \node[good, below=of r1-flex] (r2-flex) {Tinggi};
    \node[good, below=of r1-re2c] (r2-re2c) {Tinggi};
    
    \node[box, below=of r2] (r3) {Performa};
    \node[good, below=of r2-hw] (r3-hw) {Tinggi};
    \node[medium, below=of r2-flex] (r3-flex) {Sedang};
    \node[good, below=of r2-re2c] (r3-re2c) {Sangat Tinggi};
    
    \node[box, below=of r3] (r4) {Fleksibilitas};
    \node[good, below=of r3-hw] {Sangat Tinggi};
    \node[medium, below=of r3-flex] {Sedang};
    \node[medium, below=of r3-re2c] {Sedang};
    
    \end{tikzpicture}%
    }
    \caption{Perbandingan hand-written, Flex, dan re2c}
    \label{fig:comparison-table}
\end{figure}

Gambar \ref{fig:development-time} menunjukkan perbandingan waktu development antara berbagai pendekatan.

\begin{figure}[!htbp]
    \centering
    \adjustbox{max width=0.85\textwidth,center}{%
    \begin{tikzpicture}[
        bar/.style={rectangle, draw=blue!50, fill=blue!10, minimum height=0.6cm, font=\tiny, align=center, inner sep=2pt},
        label/.style={font=\footnotesize, align=center},
        node distance=0.3cm
    ]
    
    % Development time bars
    \node[label] (l1) {Hand-Written};
    \node[bar, right=0.2cm of l1, minimum width=3cm] (b1) {100\%};
    
    \node[label, below=0.3cm of l1] (l2) {Flex};
    \node[bar, right=0.2cm of l2, minimum width=1.5cm, draw=green!50, fill=green!10] (b2) {50\%};
    
    \node[label, below=0.3cm of l2] (l3) {re2c};
    \node[bar, right=0.2cm of l3, minimum width=1.2cm, draw=green!50, fill=green!10] (b3) {40\%};
    
    \node[right=3.5cm of l1, font=\tiny] {Development Time};
    
    \end{tikzpicture}%
    }
    \caption{Perbandingan waktu development (hand-written = baseline)}
    \label{fig:development-time}
\end{figure}