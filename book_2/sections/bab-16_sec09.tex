\section{Dokumentasi Proyek}

Dokumentasi yang baik adalah bagian penting dari project final. Dokumentasi harus mencakup:

\subsection{README.md}

README harus berisi:
\begin{itemize}
    \item \textbf{Overview}: Deskripsi singkat tentang compiler
    \item \textbf{Features}: Fitur-fitur yang didukung
    \item \textbf{Build Instructions}: Cara mengkompilasi compiler
    \item \textbf{Usage}: Cara menggunakan compiler
    \item \textbf{Examples}: Contoh program dan cara mengkompilasinya
    \item \textbf{Architecture}: Overview arsitektur compiler
    \item \textbf{Testing}: Cara menjalankan test suite
\end{itemize}

\subsection{Design Document}

Design document mencakup:
\begin{itemize}
    \item \textbf{Language Specification}: Grammar, syntax, semantics
    \item \textbf{Architecture Overview}: Diagram arsitektur compiler
    \item \textbf{Component Design}: Desain setiap fase kompilasi
    \item \textbf{Data Structures}: AST nodes, symbol table, IR format
    \item \textbf{Algorithm Choices}: Justifikasi pilihan algoritma
    \item \textbf{Trade-offs}: Diskusi tentang trade-off yang dibuat
\end{itemize}

\subsection{API Documentation}

Jika compiler menyediakan library atau API:
\begin{itemize}
    \item Function signatures
    \item Parameter descriptions
    \item Return values
    \item Usage examples
\end{itemize}