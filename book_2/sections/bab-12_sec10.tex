\section{Kesimpulan}

Dalam bab ini, kita telah mempelajari:

\begin{enumerate}
    \item Intermediate code generation adalah fase yang mengubah AST menjadi IR yang lebih dekat ke machine code
    \item Three-address code (TAC) dan quadruples adalah format IR yang populer
    \item Generator TAC bekerja dengan recursive traversal pada AST
    \item Control flow statements memerlukan label dan jump instructions
    \item Common subexpression elimination adalah optimasi dasar yang penting
    \item IR memungkinkan portabilitas dan optimasi yang lebih baik
\end{enumerate}

Pemahaman tentang intermediate code generation menjadi dasar penting untuk fase code generation dan optimasi yang akan dipelajari dalam bab-bab selanjutnya. IR proyek subset C (TAC/quadruples dari AST proyek) menjadi input untuk code generation (Bab 14) dan optimasi (Bab 15) dalam pipeline compiler proyek.