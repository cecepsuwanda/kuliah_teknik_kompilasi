\section{Kesimpulan}

Dalam bab ini, kita telah mempelajari:

\begin{enumerate}
    \item Semantic analysis memverifikasi bahwa program memenuhi aturan semantik bahasa
    \item Type checking memastikan operasi dilakukan pada tipe yang kompatibel
    \item Type inference memungkinkan kompilator menentukan tipe secara otomatis
    \item Type compatibility menentukan apakah satu tipe dapat digunakan di tempat tipe lain
    \item Semantic error detection dan reporting memberikan feedback yang jelas kepada programmer
    \item Type checking terintegrasi dengan symbol table untuk mengakses informasi deklarasi
    \item Annotated AST menyimpan informasi tipe untuk digunakan pada fase selanjutnya
\end{enumerate}

Pemahaman tentang type checking dan semantic analysis ini penting karena memastikan bahwa program yang dikompilasi tidak hanya valid secara sintaksis, tetapi juga benar secara semantik sebelum masuk ke fase code generation. Type checking proyek subset C memakai AST proyek (Bab 9) dan symbol table (Bab 10); annotated AST menjadi input untuk IR generation (Bab 12) dan code generation (Bab 14).