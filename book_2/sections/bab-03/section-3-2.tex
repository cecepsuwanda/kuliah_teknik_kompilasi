\section{Regular Expression}

\subsection{Definisi dan Operasi Dasar}

Regular expression (regex) adalah notasi formal untuk mendeskripsikan pola string dalam suatu \textbf{regular language}. Operasi dasar meliputi:

\begin{table}[h]
\centering
\begin{tabular}{|l|l|l|}
\hline
\textbf{Operasi} & \textbf{Simbol} & \textbf{Contoh} \\
\hline
Karakter & $a$ & string "a" \\
Alternasi & $|$ & $a|b$ (a atau b) \\
Konkatenasi & $ab$ & string "ab" \\
Kleene Star & $*$ & $a^*$ (0 atau lebih) \\
Positive Plus & $+$ & $a^+$ (1 atau lebih) \\
Optional & $?$ & $a?$ (0 atau 1) \\
\hline
\end{tabular}
\caption{Operator dasar Regular Expression}
\end{table}

\subsection{Implementasi Leksikal dengan Regex}

Dalam pembangunan kompilator, setiap elemen leksikal (token) didefinisikan dengan regex guna menjamin kepastian pola:
\begin{itemize}
  \item \textbf{Identifier}: \texttt{[a-zA-Z\_][a-zA-Z0-9\_]*}
  \item \textbf{Integer}: \texttt{[0-9]+}
  \item \textbf{Float}: \texttt{[0-9]+\textbackslash.[0-9]+([eE][+-]?[0-9]+)?}
  \item \textbf{String Literal}: \texttt{"([\textasciicircum"\\]|\textbackslash\textbackslash.)*"}
\end{itemize}

\subsection{Sifat Regular Language}
Bahasa reguler dapat dikenali secara efisien menggunakan finite automata dan tertutup terhadap operasi gabungan, pengulangan, serta penyambungan, namun tidak dapat mendeskripsikan struktur \textit{nested} mendalam.
