\section{Regular Expression}

\subsection{Definisi dan Keterbatasan}

Regular expression (Regex) adalah notasi aljabar untuk mendeskripsikan himpunan string (\textit{language}). Regex dipilih untuk analisis leksikal karena sifatnya yang deklaratif: kita cukup mendeskripsikan "seperti apa bentuk tokennya", bukan "bagaimana cara mencarinya".

Namun, Regex memiliki keterbatasan fundamental: ia \textbf{tidak memiliki memori}. Regex tidak bisa digunakan untuk mengenali struktur yang bersarang (\textit{nested}) atau saling berpasangan, seperti:
\begin{itemize}
    \item Menyeimbangkan kurung buka dan tutup: \texttt{((...))}.
    \item Mengenali string palindrome: \texttt{w w\^{}R}.
\end{itemize}
Karena itulah Regex hanya digunakan untuk token linear sederhana, sedangkan struktur program yang kompleks ditangani oleh \textit{Context-Free Grammar} di fase parsing.

\subsection{Notasi dan Ekstensi Standar}

Selain operator dasar (union, concat, kleene star), implementasi modern menyediakan "gula sintaksis" (\textit{syntactic sugar}) untuk memudahkan penulisan:

\begin{table}[!htbp]
\centering
\begin{tabular}{|l|l|l|}
\hline
\textbf{Notasi} & \textbf{Arti} & \textbf{Ekuivalen Dasar} \\
\hline
\texttt{[a-z]} & Character Class & $a|b|c|...|z$ \\
\texttt{[a-zA-Z]} & Huruf apa saja & $a|...|z|A|...|Z$ \\
\texttt{\textbackslash d} & Digit & \texttt{[0-9]} \\
\texttt{\textbackslash w} & Word Character & \texttt{[a-zA-Z0-9\_]} \\
\texttt{\textbackslash s} & Whitespace & \texttt{[ \textbackslash t\textbackslash n\textbackslash r]} \\
\texttt{.} & Any Character & Semua kecuali newline \\
\texttt{[\^{}abc]} & Negasi & Karakter selain a, b, c \\
\hline
\end{tabular}
\caption{Ekstensi notasi Regular Expression umum}
\end{table}

\subsection{Implementasi Leksikal dengan Regex} \cite{opengenus2024lexer}

Dalam pembangunan kompilator, setiap elemen leksikal didefinisikan secara presisi:
\begin{enumerate}
  \item \textbf{Identifier}: \texttt{[a-zA-Z\_][a-zA-Z0-9\_]*} --- Harus dimulai huruf/\_, boleh ada angka setelahnya.
  \item \textbf{Integer}: \texttt{[0-9]+} --- Satu digit atau lebih.
  \item \textbf{Float}: \texttt{\textbackslash d+\textbackslash .\textbackslash d+([eE][+-]?\textbackslash d+)?} --- Mendukung desimal dan notasi ilmiah.
  \item \textbf{String}: \texttt{"([\textasciicircum"\textbackslash\textbackslash]|\textbackslash\textbackslash.)*"} --- String dalam kutip, mendukung escape character.
\end{enumerate}
