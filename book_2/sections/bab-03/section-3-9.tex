\section{Lexer Generator: Flex dan re2c}

Lexer generator adalah alat yang secara otomatis membangun \textit{finite automata} dari spesifikasi ekspresi reguler.

\subsection{Flex (Fast Lexical Analyzer)}
Flex adalah generator standar yang paling luas digunakan. Spesifikasinya ditulis dalam file \texttt{.l} dengan tiga bagian utama: \textit{Definitions}, \textit{Rules}, dan \textit{User Code}.

\begin{figure}[!htbp]
    \centering
    \adjustbox{max width=0.8\textwidth,center}{%
    \begin{tikzpicture}[
        file/.style={rectangle, draw=blue!50, fill=blue!10, text width=2cm, font=\tiny, align=center},
        proc/.style={rectangle, draw=green!50, fill=green!10, text width=2cm, font=\tiny, align=center},
        arrow/.style={->, >=stealth, thick}
    ]
    \node[file] (spec) {lexer.l};
    \node[proc, right=1cm of spec] (flex) {Flex Generator};
    \node[file, right=1cm of flex] (code) {lex.yy.c};
    \draw[arrow] (spec) -- (flex);
    \draw[arrow] (flex) -- (code);
    \end{tikzpicture}%
    }
    \caption{Alur kerja Flex}
\end{figure}

\subsection{re2c}
re2c adalah generator modern yang menghasilkan kode C/C++ berperforma sangat tinggi. Berbeda dengan Flex, re2c menyisipkan spesifikasinya langsung di dalam komentar khusus pada file sumber C/C++.

\subsection{Perbandingan Pendekatan}
\begin{table}[!htbp]
\centering
\begin{tabular}{|l|c|c|c|}
\hline
\textbf{Fitur} & \textbf{Hand-written} & \textbf{Flex} & \textbf{re2c} \\
\hline
Produktivitas & Rendah & Tinggi & Tinggi \\
Performa & Tinggi & Sedang & Sangat Tinggi \\
Maintenance & Sulit & Mudah & Mudah \\
\hline
\end{tabular}
\caption{Perbandingan metode konstruksi lexer}
\end{table}
