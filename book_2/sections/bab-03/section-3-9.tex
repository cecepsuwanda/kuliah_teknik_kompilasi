\section{Lexer Generator: Flex dan re2c}

\subsection{Flex: The Classic Table-Driven Generator}
Flex (Fast Lexical Analyzer) adalah standar industri yang menghasilkan \textit{State Machine} berbasis tabel.
\begin{itemize}
    \item \textbf{Cara Kerja}: Flex mengonversi regex menjadi DFA, lalu menyimpannya sebagai array 2D integer besar (baris=state, kolom=input char).
    \item \textbf{Runtime}: Loop utama hanya berupa: \texttt{next\_state = table[current\_state][input\_char]}.
    \item \textbf{Kelebihan}: Ukuran kode executable kecil, waktu kompilasi cepat.
    \item \textbf{Kekurangan}: Akses memori ke tabel bisa menyebabkan \textit{cache miss} pada prosesor modern.
\end{itemize}

\subsection{re2c: The High-Performance Code Generator}
re2c mengambil pendekatan berbeda. Alih-alih tabel, ia menghasilkan kode \textit{hard-coded goto} untuk setiap transisi.
\begin{itemize}
    \item \textbf{Cara Kerja}: Mengubah DFA langsung menjadi sarang \texttt{if-else} dan \texttt{goto} dalam C++.
    \item \textbf{Runtime}: CPU dapat melakukan \textit{branch prediction} dan kode instruksinya muat di \textit{instruction cache}.
    \item \textbf{Performa}: Seringkali 2-3x lebih cepat dari Flex, digunakan oleh proyek performa tinggi seperti PHP dan Ninja Build.
\end{itemize}

\begin{table}[!htbp]
\centering
\begin{tabular}{|l|c|c|c|}
\hline
\textbf{Fitur} & \textbf{Hand-written} & \textbf{Table-Driven (Flex)} & \textbf{Direct-Coded (re2c)} \\
\hline
Produktivitas & Rendah & Tinggi & Tinggi \\
Kecepatan Eksekusi & Tinggi & Sedang & Sangat Tinggi \\
Ukuran Kode Biner & Sedang & Kecil (tabel) & Besar (kode) \\
Fleksibilitas & Sangat Tinggi & Terbatas & Terbatas \\
\hline
\end{tabular}
\caption{Perbandingan metode konstruksi lexer}
\end{table}
