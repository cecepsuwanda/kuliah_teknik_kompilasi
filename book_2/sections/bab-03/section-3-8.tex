\section{Integrasi dengan Parser}

Tujuan akhir lexer adalah melayani parser. Oleh karena itu, antarmuka komunikasi keduanya harus disepakati.

\subsection{Protokol Komunikasi Standard}
Dalam ekosistem UNIX (Yacc/Bison), lexer dan parser berkomunikasi melalui variabel global:
\begin{enumerate}
    \item \textbf{yylex()}: Fungsi utama yang dipanggil parser. Mengembalikan \texttt{int} yang merupakan ID Token (misal: 257 untuk \texttt{ID}, 258 untuk \texttt{INT}).
    \item \textbf{yylval}: Variabel union global untuk menyimpan \textit{atribut} token. Jika tokennya \texttt{INT}, \texttt{yylval.ival} diisi nilai integer-nya. Jika \texttt{ID}, \texttt{yylval.sval} diisi string namanya.
    \item \textbf{yylloc}: Struktur untuk menyimpan lokasi (baris, kolom) token saat ini.
\end{enumerate}

\subsection{Best Practices Implementasi}
\begin{itemize}
    \item \textbf{Lookahead Minimal}: Usahakan lexer hanya butuh mengintip 1 karakter (\texttt{LL(1)}). Logika yang butuh lookahead jauh (misal membedakan deklarasi pointer vs perkalian di C++) sebaiknya ditunda hingga fase parsing.
    \item \textbf{Hindari State Kompleks}: Jangan menaruh logika sintaksis di lexer. Lexer tidak perlu tahu apakah \texttt{\{} adalah awal fungsi atau awal blok. Ia hanya perlu melapor: "Ini kurung kurawal buka".
    \item \textbf{String Interning}: Untuk penghematan memori, semua identifier yang sama bisa menunjuk ke alamat string yang sama di memori (\textit{string pool}).
\end{itemize}
