\section{Implementasi Lexer Hand-written}

Meskipun alat bantu (\textit{lexer generator}) seperti \textit{re2c} \cite{wikipedia2024re2c} atau \textit{RE/flex} \cite{wikipedia2024reflex} tersedia, menulis lexer secara manual memberikan kontrol penuh atas kinerja dan penanganan kesalahan.

\subsection{Teknik Input Buffering}
Membaca file satu karakter setiap kali (\texttt{fgetc}) merupakan operasi yang sangat lambat karena melibatkan \textit{system call} ke OS. Solusi standar adalah menggunakan \textbf{Dual Buffer} dengan teknik \textbf{Sentinel}.
\begin{itemize}
    \item \textbf{Buffer}: Kita memuat blok besar (misal 4KB) dari disk ke memori sekaligus.
    \item \textbf{Sentinel}: Alih-alih memeriksa apakah kita mencapai akhir buffer setiap kali maju satu langkah (yang membutuhkan dua perbandingan: \texttt{isEOF} dan \texttt{isEndBuf}), kita menaruh karakter spesial (\texttt{EOF sentinel}) di akhir buffer. Loop pembacaan bisa berjalan sangat cepat tanpa pengecekan batas buffer, sampai ia menabrak sentinel. Ini bisa meningkatkan kecepatan scanning hingga 30\%.
\end{itemize}

\subsection{Struktur Data Token}
Token yang dihasilkan haruslah ringkas namun informatif:
\begin{lstlisting}[language=C++]
enum class TokenType {
    ID, KW_IF, KW_ELSE, INT_LIT, OP_PLUS, EOF_TOKEN, ERROR
};

struct Token {
    TokenType type;
    std::string lexeme; // Teks asli ("count", "123")
    std::any value;     // Nilai semantik (123 untuk INT_LIT)
    int line, col;      // Lokasi untuk error reporting
};
\end{lstlisting}

\subsection{Arsitektur Kelas Lexer}
Kelas Lexer mempertahankan state posisi saat ini (\texttt{cursor}) dan menyediakan metode utama \texttt{nextToken()} yang dipanggil oleh parser.
\begin{figure}[!htbp]
    \centering
    \adjustbox{max width=0.8\textwidth,center}{%
    \begin{tikzpicture}[
        class/.style={rectangle, draw=blue!50, fill=blue!10, text width=3cm, minimum height=1cm, font=\footnotesize, align=center, rounded corners},
        method/.style={rectangle, draw=green!50, fill=green!10, text width=2.5cm, minimum height=0.5cm, font=\tiny, align=center},
        arrow/.style={->, >=stealth, thick}
    ]
    \node[class] (lexer) {Lexer Interface};
    \node[method, below=0.6cm of lexer] (p1) {advance() / peek()};
    \node[method, right=0.3cm of p1] (p2) {nextToken()};
    \draw[arrow] (lexer) -- (p1);
    \draw[arrow] (lexer) -- (p2);
    \end{tikzpicture}%
    }
    \caption{Komponen utama kelas Lexer}
\end{figure}
