\section{Implementasi Lexer Hand-written}

Pendekatan \textit{hand-written} memberikan kontrol penuh dan sangat berguna untuk memahami detail proses tokenisasi.

\subsection{Struktur Data Token}
Token minimal harus menyimpan kategori (\textit{type}), string asli (\textit{lexeme}), serta informasi posisi (baris dan kolom) untuk keperluan pelaporan kesalahan.

\begin{lstlisting}[language=C++]
enum class TokenType {
    IDENTIFIER, KEYWORD, INTEGER_LITERAL, FLOAT_LITERAL,
    OP_PLUS, OP_ASSIGN, SEMICOLON, END_OF_FILE, INVALID
};

struct Token {
    TokenType type;
    std::string lexeme;
    int line;
    int column;
};
\end{lstlisting}

\subsection{Arsitektur Kelas Lexer}
Kelas Lexer mengelola input string dan memprosesnya karakter demi karakter menggunakan pointer posisi.

\begin{figure}[!htbp]
    \centering
    \adjustbox{max width=0.8\textwidth,center}{%
    \begin{tikzpicture}[
        class/.style={rectangle, draw=blue!50, fill=blue!10, text width=3cm, minimum height=1cm, font=\footnotesize, align=center, rounded corners},
        method/.style={rectangle, draw=green!50, fill=green!10, text width=2.5cm, minimum height=0.5cm, font=\tiny, align=center},
        arrow/.style={->, >=stealth, thick}
    ]
    \node[class] (lexer) {Lexer Interface};
    \node[method, below=0.6cm of lexer] (p1) {peek() / get()};
    \node[method, right=0.3cm of p1] (p2) {nextToken()};
    \draw[arrow] (lexer) -- (p1);
    \draw[arrow] (lexer) -- (p2);
    \end{tikzpicture}%
    }
    \caption{Komponen utama kelas Lexer}
\end{figure}
