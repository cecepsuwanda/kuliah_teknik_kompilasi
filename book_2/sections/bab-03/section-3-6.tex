\section{Logika State Machine Token}

Dalam implementasi manual, kita menggunakan logika \textit{Finite State Machine} (FSM) untuk mengenali berbagai jenis token.

\subsection{State Machine untuk Identifier dan Keywords}
Transisi dimulai dari karakter alfabet atau \texttt{\_}, kemudian diikuti oleh karakter alfanumerik. Setelah lexeme terkumpul, dilakukan pengecekan apakah ia termasuk kata kunci (\textit{keyword}). Selain Flex, terdapat berbagai generator lexer modern lainnya seperti re2c yang berorientasi pada kecepatan \cite{wikipedia2024re2c} dan RE/flex yang mendukung C++ dengan lebih baik \cite{wikipedia2024reflex}.

\begin{figure}[!htbp]
    \centering
    \adjustbox{max width=0.8\textwidth,center}{%
    \begin{tikzpicture}[
        state/.style={circle, draw=blue!50, fill=blue!10, minimum size=0.8cm, font=\tiny, align=center},
        start/.style={circle, draw=red!50, fill=red!10, minimum size=0.8cm, font=\tiny, align=center},
        accept/.style={circle, draw=green!50, fill=green!10, minimum size=0.8cm, font=\tiny, double},
        arrow/.style={->, >=stealth, thick}
    ]
    \node[start] (q0) {START};
    \node[state, right=2cm of q0] (q1) {IN\_ID};
    \node[accept, right=2cm of q1] (q2) {DONE};
    \draw[arrow] (q0) -- node[above, font=\tiny] {letter/\_} (q1);
    \draw[arrow] (q1) to[loop above] node[above, font=\tiny] {alnum/\_} (q1);
    \draw[arrow] (q1) -- node[above, font=\tiny] {other} (q2);
    \end{tikzpicture}%
    }
    \caption{State machine untuk identifikasi identifier}
\end{figure}

\subsection{State Machine untuk Angka (Literals)}
\begin{itemize}
    \item \textbf{Integer}: Kumpulan digit [0-9].
    \item \textbf{Float}: Digit diikuti oleh titik (.), kemudian digit lagi.
\end{itemize}

\begin{lstlisting}[language=C++]
Token Lexer::scanNumber() {
    bool isFloat = false;
    std::string lexeme;
    while (isdigit(peek())) lexeme += get();
    if (peek() == '.') {
        isFloat = true;
        lexeme += get();
        while (isdigit(peek())) lexeme += get();
    }
    return Token(isFloat ? FLOAT : INT, lexeme);
}
\end{lstlisting}
