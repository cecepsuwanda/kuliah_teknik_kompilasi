\section{Pengenalan Lexical Analysis}

\subsection{Peran Lexical Analyzer}

\compiler{Lexical Analysis} (atau \compiler{Lexer/Scanner}) adalah tahap pertama dalam kompilasi yang bertugas membaca aliran karakter dari program sumber dan mengelompokkannya ke dalam unit-unit bermakna yang disebut \textit{token} \cite{aoyama2024lexical, opengenus2024lexer}.

\begin{itemize}
  \item Menghapus whitespace dan komentar.
  \item Melaporkan lexical error.
\end{itemize}

\subsection{Alur Teoretis Lexical Analysis}

Sebagai landasan untuk memahami lexical analysis, kita perlu mempelajari teori formal yang mendasarinya. Alur ini menunjukkan bahwa lexical analysis dalam kompilator modern menggunakan teori \textit{formal language}, khususnya \textit{regular languages}.

\begin{figure}[!htbp]
    \centering
    \adjustbox{max width=0.85\textwidth,center}{%
    \begin{tikzpicture}[
        box/.style={rectangle, draw=blue!50, fill=blue!10, text width=1.8cm, text centered, minimum height=0.8cm, rounded corners, font=\footnotesize, inner sep=4pt, align=center},
        arrow/.style={->, >=stealth, thick},
        label/.style={font=\tiny, above, align=center},
        node distance=1.8cm
    ]
    \node[box] (regex) {Regular\\Expression};
    \node[box, right=of regex] (nfa) {$\epsilon$-NFA};
    \node[box, right=of nfa] (dfa) {DFA};
    \node[box, right=of dfa] (impl) {Scanner};
    
    \draw[arrow] (regex) -- node[label] {Thompson} (nfa);
    \draw[arrow] (nfa) -- node[label] {Subset\\Const.} (dfa);
    \draw[arrow] (dfa) -- node[label] {Code\\Gen.} (impl);
    \end{tikzpicture}%
    }
    \caption{Alur konversi dari regular expression ke implementasi scanner}
    \label{fig:lexical-theory-overview}
\end{figure}

\subsection{Token dan Lexeme}

\begin{itemize}
  \item \textbf{Token}: Kategori unit leksikal (misal: \keyword{IDENTIFIER}, \keyword{KEYWORD}).
  \item \textbf{Lexeme}: String aktual yang mewakili token (misal: \code{x}, \code{if}).
  \item \textbf{Attribute Value}: Informasi tambahan (misal: nilai literal, entri tabel simbol).
\end{itemize}
