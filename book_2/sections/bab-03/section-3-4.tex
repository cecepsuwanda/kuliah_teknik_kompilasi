\section{Metode Konstruksi NFA: Algoritma Thompson}

Algoritma Thompson menyediakan cara standar untuk mengubah pola Regular Expression menjadi NFA secara sistematis melalui template-template kecil yang digabungkan. Setiap konstruksi memiliki satu state awal dan satu state akhir, dan jumlah state hanya bertambah secara linear mengikuti panjang regex.

\subsection{Template Dasar Thompson}
\begin{enumerate}
    \item \textbf{Literal ("a")}: Transisi langsung dari state $s$ ke $f$ dengan input 'a'.
    \item \textbf{Concatenation ($RS$)}: Menghubungkan state akhir NFA $R$ ke state awal NFA $S$ dengan transisi $\epsilon$.
    \item \textbf{Alternation ($R|S$)}: Membuat state awal baru yang bercabang dengan transisi $\epsilon$ ke NFA $R$ dan $S$, lalu state akhir keduanya digabung ke state akhir baru via $\epsilon$.
    \item \textbf{Kleene Closure ($R^*$)}: Menambahkan loop $\epsilon$ dari akhir $R$ ke awal $R$ (pengulangan), serta jalur pintas $\epsilon$ dari awal $R$ ke akhir $R$ (skip).
\end{enumerate}

Contoh Jejak Konstruksi untuk \texttt{(a|b)*} :
\begin{itemize}
    \item \textbf{Langkah 1}: Buat NFA kecil untuk \texttt{a} dan \texttt{b}.
    \item \textbf{Langkah 2}: Gabungkan keduanya dengan template \textit{Alternation} menjadi \texttt{(a|b)}. Ini melibatkan penambahan state awal baru yang membelah ke $a$ dan $b$.
    \item \textbf{Langkah 3}: Bungkus hasilnya dengan template \textit{Kleene Star}. Tambahkan transisi $\epsilon$ yang memungkinkan kita "melompat kembali" ke awal setelah selesai mengenali \texttt{(a|b)}, atau langsung "keluar" (match empty string).
\end{itemize}

\begin{figure}[!htbp]
    \centering
    \adjustbox{max width=0.8\textwidth,center}{%
    \begin{tikzpicture}[
        state/.style={circle, draw=blue!50, fill=blue!10, minimum size=0.6cm, font=\tiny, align=center},
        arrow/.style={->, >=stealth, thick},
        eps/.style={->, >=stealth, thick, dashed, blue!70}
    ]
    % Simple a|b
    \node[state] (q0) at (0,0) {0};
    \node[state] (q1) at (1,0.5) {1}; \node[state] (q2) at (2,0.5) {2}; % a branch
    \node[state] (q3) at (1,-0.5) {3}; \node[state] (q4) at (2,-0.5) {4}; % b branch
    \node[state] (q5) at (3,0) {5};
    
    \draw[eps] (q0) -- (q1); \draw[arrow] (q1) -- node[above, font=\tiny]{a} (q2); \draw[eps] (q2) -- (q5);
    \draw[eps] (q0) -- (q3); \draw[arrow] (q3) -- node[below, font=\tiny]{b} (q4); \draw[eps] (q4) -- (q5);
    
    % Kleene wrap
    \node[state, left=0.6cm of q0] (qs) {Start};
    \node[state, right=0.6cm of q5] (qf) {Final};
    
    \draw[eps] (qs) -- (q0);      % Enter loop
    \draw[eps] (q5) to[bend left=60] (q0); % Loop back
    \draw[eps] (q5) -- (qf);      % Exit loop
    \draw[eps] (qs) to[bend right=60] (qf); % Skip loop (epsilon)
    
    \end{tikzpicture}%
    }
    \caption{Visualisasi NFA untuk (a|b)*}
\end{figure}
