\section{Finite Automata}

Finite automata adalah model matematika yang digunakan untuk mengenali string dalam suatu bahasa. 

\subsection{NFA (Nondeterministic Finite Automata)}
NFA dapat memiliki beberapa kemungkinan transisi untuk simbol input yang sama dan mengizinkan $\epsilon$-transitions (pindah state tanpa membaca input). Hal ini memudahkan konstruksi dari regex tetapi memerlukan simulasi yang lebih kompleks.

\subsection{DFA (Deterministic Finite Automata)}
DFA bersifat deterministik (tepat satu transisi untuk setiap simbol) dan tidak mengizinkan $\epsilon$-transitions. DFA sangat efisien untuk dijalankan di mesin karena hanya membutuhkan satu kali pembacaan karakter untuk setiap transisi state.

\begin{figure}[!htbp]
    \centering
    \adjustbox{max width=1.0\textwidth,center}{%
    \begin{tikzpicture}[
        state/.style={circle, draw=blue!50, fill=blue!10, minimum size=0.6cm, font=\tiny},
        accept/.style={circle, draw=green!50, fill=green!10, minimum size=0.6cm, font=\tiny, double},
        start/.style={circle, draw=red!50, fill=red!10, minimum size=0.6cm, font=\tiny},
        arrow/.style={->, >=stealth, thick},
        eps/.style={->, >=stealth, thick, dashed, blue!70},
        node distance=1.2cm
    ]
    \node[start] (nq0) at (0,0) {$q_0$};
    \node[accept] (nq1) at (2,0) {$q_1$};
    \draw[arrow] (nq0) -- node[above, font=\tiny] {a} (nq1);
    \draw[arrow] (nq1) to[loop above] node[above, font=\tiny] {a} (nq1);
    \node[below=0.2cm of nq0] {DFA untuk $a^+$};
    \end{tikzpicture}%
    }
    \caption{Contoh DFA sederhana}
\end{figure}

\subsection{Konversi NFA ke DFA: Subset Construction}

Algoritma subset construction digunakan untuk menghasilkan DFA yang ekuivalen dari sebuah NFA.
\begin{enumerate}
    \item \textbf{Start state DFA} adalah $\epsilon$-closure dari start state NFA.
    \item Untuk setiap state DFA, hitung transisi ke set NFA states berikutnya berdasarkan simbol input.
    \item Ambil $\epsilon$-closure dari setiap set tersebut untuk menjadi state DFA baru.
    \item State DFA menjadi \textbf{accept state} jika mengandung setidaknya satu accept state NFA.
\end{enumerate}
