\section{Instrumen Penilaian OBE}

\subsection{Rubrik Proyek Kompilator (Berdasarkan Sub-CPMK)}
Penilaian proyek akhir menggunakan standar \textit{Outcome-Based Education} (OBE) untuk mengukur kompetensi teknis mahasiswa secara spesifik \cite{jhu2024compilers}.

\begin{table}[!htbp]
\centering
\resizebox{\textwidth}{!}{%
\begin{tabular}{|l|l|l|l|l|}
\hline
\textbf{Kriteria} & \textbf{Sangat Baik (A)} & \textbf{Baik (B)} & \textbf{Cukup (C)} & \textbf{Kurang (D/E)} \\
\hline
\textbf{Lexer/Parser} & Grammar kompleks, & Grammar benar, & Grammar sederhana, & Banyak error, \\
(Sub-CPMK 2.1-2.4) & \textit{error recovery} sempurna & \textit{error reporting} jelas & \textit{recovery} minim & tidak bisa parsing \\
\hline
\textbf{Analisis Semantik} & \textit{Type checking} & \textit{Scoping} dan & Hanya \textit{symbol table} & Tidak ada \textit{contextual} \\
(Sub-CPMK 3.1-3.3) & dan \textit{Inference} lengkap & \textit{Type checking} benar & dasar & \textit{validation} \\
\hline
\textbf{Optimasi \& IR} & Optimasi lokal \& & Menggunakan TAC & IR dasar tanpa & Langsung ke target \\
(Sub-CPMK 4.1-4.3) & global (DCE, Folding) & standar & optimasi & tanpa IR \\
\hline
\textbf{Kualitas Kode} & Modular, terdokumentasi, & Bersih, terdokumentasi & Sedikit berantakan, & Kode sulit dibaca, \\
& \textit{pipeline} CI/CD & manual & dokumentasi minim & tanpa komentar \\
\hline
\end{tabular}%
}
\caption{Rubrik Penilaian Proyek Berbasis OBE}
\end{table}

\subsection{Lembar Penilaian Rekan (\textit{Peer Review})}
Mahasiswa diharapkan memberikan umpan balik konstruktif terhadap portofolio rekan sekelas.

\begin{checklist}
  \item \textbf{Alur Logika}: Apakah desain parser mudah diikuti?
  \item \textbf{Penanganan Kesalahan}: Apakah sistem memberikan pesan error yang bermakna bagi pengguna?
  \item \textbf{Efisiensi}: Apakah penggunaan memori dan register dipertimbangkan?
  \item \textbf{Dokumentasi}: Apakah \textit{README} cukup jelas untuk menjalankan proyek dari nol?
\end{checklist}

\begin{figure}[!htbp]
    \centering
    \adjustbox{max width=0.8\textwidth,center}{%
    \begin{tikzpicture}[
        node/.style={rectangle, draw=blue!50, fill=blue!10, text width=5cm, font=\tiny, align=center}
    ]
    \node[node] (theory) {Teori \& Kuis (20\%)};
    \node[node, below=0.2cm of theory] (practice) {Praktikum Mingguan (30\%)};
    \node[node, below=0.2cm of practice] (project) {Proyek Akhir + Portfolio (50\%)};
    \draw[->] (theory) -- (practice);
    \draw[->] (practice) -- (project);
    \end{tikzpicture}%
    }
    \caption{Distribusi Bobot Penilaian Kompetensi}
\end{figure}
