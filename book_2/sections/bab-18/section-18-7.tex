\section{Panduan Praktis: Debugging dengan Sanitizers}

Mencari kesalahan memori atau perilaku tidak terdifinisi (\textit{Undefined Behavior}) secara manual sangatlah sulit. Kompilator modern menyediakan fitur \compiler{Sanitizers} untuk mendeteksi masalah ini saat program dijalankan.

\subsection{Mengaktifkan Sanitizer}
Tambahkan bendera berikut saat mengompilasi kode C Anda menggunakan GCC atau Clang:
\begin{lstlisting}[language=sh]
# ASan (AddressSanitizer) untuk kesalahan memori
# UBSan (UndefinedBehaviorSanitizer) untuk kesalahan logika standar
gcc -fsanitize=address,undefined -g source.c -o my_compiler
\end{lstlisting}

\subsection{Kesalahan yang Terdeteksi}
\begin{itemize}
    \item \textbf{Buffer Overflow}: Mengakses array di luar batas yang ditentukan.
    \item \textbf{Use-After-Free}: Mengakses memori yang telah dibebaskan (\textit{free}).
    \item \textbf{Memory Leaks}: Alokasi memori (\textit{malloc}) yang tidak pernah dibebaskan.
    \item \textbf{Integer Overflow}: Hasil operasi aritmatika yang melebihi kapasitas tipe data.
\end{itemize}

\subsection{Interpretasi Laporan}
Jika terjadi kesalahan, program akan berhenti dan mencetak \textit{stack trace} yang menunjukkan baris kode sumber yang bermasalah.
\begin{lstlisting}[language=sh]
==1234==ERROR: AddressSanitizer: heap-buffer-overflow on address ...
READ of size 4 at 0x603000000010 thread T0
    #0 0x400c47 in main (source.c:15)
\end{lstlisting}
Laporan ini memberi tahu Anda bahwa baris 15 pada \code{source.c} mencoba membaca di luar batas memori yang dialokasikan.
