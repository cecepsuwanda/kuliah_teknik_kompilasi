\section{Kesimpulan}

Dalam bab ini, kita telah mempelajari:

\begin{enumerate}
    \item Bottom-up parsing membangun parse tree dari leaves ke root menggunakan rightmost derivation dalam reverse
    \item Shift-reduce parsing menggunakan empat operasi: shift, reduce, accept, dan error
    \item LR parser adalah kelas bottom-up parser yang powerful dengan berbagai varian (LR(0), SLR(1), CLR(1), LALR(1))
    \item Konstruksi LR parsing table melibatkan augmented grammar, LR items, closure, GOTO, dan canonical collection
    \item Parser generator seperti Bison/Yacc secara otomatis menghasilkan parser dari grammar specification
    \item Integrasi Flex dan Bison memungkinkan pembangunan lexer dan parser yang terintegrasi
    \item Semantic actions memungkinkan pembangunan AST selama parsing
    \item Bottom-up parsing lebih powerful tetapi top-down parsing lebih mudah diimplementasikan secara manual
\end{enumerate}

Pemahaman tentang bottom-up parsing dan parser generator ini penting untuk mengimplementasikan parser yang robust dan efisien untuk bahasa pemrograman yang kompleks.