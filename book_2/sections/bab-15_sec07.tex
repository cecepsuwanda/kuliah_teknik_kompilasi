\section{Data-Flow Analysis Dasar}

Data-flow analysis adalah teknik untuk menghitung informasi tentang kemungkinan perilaku program. Menurut GeeksforGeeks\footnote{\url{https://www.geeksforgeeks.org/data-flow-analysis-compiler/}}:

\begin{quote}
``Data-flow analysis is a technique to compute information about possible program behaviors (how definitions, uses of variables, expressions, etc. propagate through the code). Usually done on a CFG.''
\end{quote}

Data-flow analysis adalah fondasi untuk optimasi global yang lebih advanced.

\subsection{Konsep Dasar Data-Flow Analysis}

Data-flow analysis bekerja dengan:

\begin{enumerate}
    \item \textbf{Domain}: Himpunan informasi yang ingin dihitung
    \begin{itemize}
        \item Live variables: set variabel yang live
        \item Reaching definitions: set definisi yang mencapai suatu titik
        \item Available expressions: set ekspresi yang sudah dihitung
    \end{itemize}
    
    \item \textbf{Transfer Function}: Bagaimana informasi berubah setelah eksekusi instruksi
    
    \item \textbf{Meet/Join Operation}: Bagaimana menggabungkan informasi dari multiple paths
    
    \item \textbf{Fixpoint Iteration}: Iterasi hingga mencapai fixpoint (tidak ada perubahan)
\end{enumerate}

\subsection{Live Variable Analysis}

Live variable analysis menentukan variabel mana yang "live" (nilainya mungkin digunakan) di setiap titik program.

\textbf{Definisi:}
\begin{itemize}
    \item Variabel \texttt{v} dikatakan \textbf{live} di titik \texttt{p} jika ada path dari \texttt{p} ke penggunaan \texttt{v} tanpa assignment ke \texttt{v} di antara keduanya
    \item Variabel \texttt{v} dikatakan \textbf{dead} jika tidak live
\end{itemize}

\textbf{Algoritma (Backward Analysis):}
\begin{enumerate}
    \item Inisialisasi: \texttt{LIVE[exit] = \{\}}
    \item Untuk setiap basic block \texttt{B} (dari exit ke entry):
    \begin{itemize}
        \item \texttt{LIVE[B] = UNION(LIVE[successors])}
        \item \texttt{LIVE[B] = LIVE[B] - DEF[B] + USE[B]}
        \item \texttt{DEF[B]}: variabel yang didefinisikan di B
        \item \texttt{USE[B]}: variabel yang digunakan di B
    \end{itemize}
    \item Ulangi hingga fixpoint
\end{enumerate}

\subsection{Reaching Definitions}

Reaching definitions analysis menentukan definisi variabel mana yang "mencapai" suatu titik program.

\textbf{Definisi:}
Definisi \texttt{d} dikatakan \textbf{reach} titik \texttt{p} jika ada path dari \texttt{d} ke \texttt{p} tanpa definisi lain untuk variabel yang sama.

\textbf{Kegunaan:}
\begin{itemize}
    \item Constant propagation (global)
    \item Deteksi penggunaan variabel sebelum inisialisasi
    \item Optimasi lainnya
\end{itemize}

\subsection{Available Expressions}

Available expressions analysis menentukan ekspresi mana yang sudah dihitung dan masih valid (operand-nya belum berubah).

\textbf{Kegunaan:}
\begin{itemize}
    \item Common subexpression elimination
    \item Optimasi lainnya
\end{itemize}