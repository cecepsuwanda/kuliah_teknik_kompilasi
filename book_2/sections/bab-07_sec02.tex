\section{Pendahuluan}

Setelah mempelajari top-down parsing pada bab sebelumnya, kita sekarang akan mempelajari pendekatan alternatif yang lebih powerful: bottom-up parsing. Menurut sumber terbuka:

\begin{quote}
``Bottom-up parsers (LR, LALR, GLR) – more powerful; often generated by tools like Bison/Yacc. The choice affects ease of specification and parsing power.''\cite{diznr2024phases}
\end{quote}

Bottom-up parsing membangun parse tree dari leaves (token) ke root (start symbol), yang merupakan kebalikan dari top-down parsing. Pendekatan ini lebih powerful karena dapat menangani lebih banyak jenis grammar, termasuk grammar dengan left recursion yang tidak dapat ditangani langsung oleh top-down parser.

Gambar \ref{fig:bottom-up-overview} menunjukkan alur bottom-up parsing.

\begin{figure}[!htbp]
    \centering
    \adjustbox{max width=0.9\textwidth,center}{%
    \begin{tikzpicture}[
        box/.style={rectangle, draw=blue!50, fill=blue!10, text width=2.5cm, text centered, minimum height=0.7cm, rounded corners, font=\footnotesize, inner sep=4pt, align=center},
        stack/.style={rectangle, draw=green!50, fill=green!10, text width=2.5cm, minimum height=0.7cm, font=\footnotesize, align=center, rounded corners, inner sep=4pt},
        arrow/.style={->, >=stealth, thick},
        node distance=1.2cm
    ]
    
    \node[box] (input) {Input\\Tokens};
    \node[stack, right=of input] (stack) {Stack};
    \node[box, right=of stack] (reduce) {Reduce};
    \node[box, below=of reduce] (start) {Start\\Symbol};
    
    \draw[arrow] (input) -- (stack);
    \draw[arrow] (stack) -- (reduce);
    \draw[arrow] (reduce) -- (start);
    
    \end{tikzpicture}%
    }
    \caption{Alur bottom-up parsing}
    \label{fig:bottom-up-overview}
\end{figure}