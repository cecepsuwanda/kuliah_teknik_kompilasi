\section{Alat Ukur Kinerja (Measurement Tools)}

Untuk mendapatkan analisis yang mendalam, kita memerlukan alat yang dapat melihat ke dalam mesin (perangkat keras) maupun perangkat lunak.

\subsection{Pengukuran Waktu dan Memori}
Kompilator diukur berdasarkan seberapa cepat ia memproses kode (\textit{compilation speed}) dan seberapa cepat kode yang dihasilkannya berjalan (\textit{runtime speed}). Penggunaan memori puncak (\textit{Peak RSS}) sangat penting untuk memastikan kompilasi tidak menyebabkan sistem kehabisan RAM.

\subsection{Hardware Performance Counters (HPC)}
Alat profil perangkat lunak standar memberi tahu kita fungsi mana yang lambat, tetapi \compiler{Hardware Performance Counters (HPC)} memberi tahu kita \textbf{mengapa} fungsi tersebut lambat \cite{jhu2024compilers}.
Alat seperti \code{perf} pada Linux memungkinkan akses ke register perangkat keras di dalam CPU untuk mengukur:
\begin{itemize}
    \item \textbf{Cache Misses}: Seberapa sering CPU harus menunggu data dari RAM karena tidak ada di cache L1/L2.
    \item \textbf{Branch Mispredictions}: Seberapa sering unit prediksi cabang CPU salah menebak alur instruksi (sangat berguna untuk mengoptimalkan \textit{switch-case} pada parser).
    \item \textbf{IPC (Instructions Per Cycle)}: Menunjukkan efisiensi penggunaan instruksi per siklus detak CPU.
\end{itemize}

\begin{figure}[!htbp]
    \centering
    \adjustbox{max width=0.8\textwidth,center}{%
    \begin{tikzpicture}[
        node/.style={rectangle, draw=gray, fill=gray!10, text width=6cm, font=\tiny, align=center}
    ]
    \node[node] (sw) {Software Profiling: "Function \code{parse()} takes 60\% of time"};
    \node[node, below=0.2cm of sw, fill=green!10] (hw) {Hardware Counters: "\code{parse()} is slow due to 80\% L2 Cache Misses"};
    \draw[->] (sw) -- (hw);
    \end{tikzpicture}%
    }
    \caption{Keunggulan Analisis Perangkat Keras (HPC)}
\end{figure}
