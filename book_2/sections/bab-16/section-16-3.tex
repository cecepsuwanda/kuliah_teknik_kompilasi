\section{Measurement Tools}

\subsection{Time Measurement}

\begin{lstlisting}[language=C]
#include <time.h>
#include <sys/time.h>

// High-resolution timer
double get_time_ms() {
    struct timespec ts;
    clock_gettime(CLOCK_MONOTONIC, &ts);
    return ts.tv_sec * 1000.0 + ts.tv_nsec / 1000000.0;
}

// Measure compilation time
double measure_compilation_time(char *command) {
    double start = get_time_ms();
    
    int result = system(command);
    
    double end = get_time_ms();
    return end - start;
}
\end{lstlisting}

\subsection{Memory Measurement}

\begin{lstlisting}[language=C]
#include <sys/resource.h>
#include <unistd.h>

// Measure peak memory usage
size_t measure_peak_memory() {
    struct rusage usage;
    getrusage(RUSAGE_CHILDREN, &usage);
    return usage.ru_maxrss;  // Peak resident set size
}

// Measure current memory usage
size_t measure_current_memory() {
    FILE *status = fopen("/proc/self/status", "r");
    if (!status) return 0;
    
    char line[256];
    size_t memory = 0;
    
    while (fgets(line, sizeof(line), status)) {
        if (strncmp(line, "VmRSS:", 6) == 0) {
            sscanf(line + 7, "%zu", &memory);
            memory *= 1024;  // Convert KB to bytes
            break;
        }
    }
    
    fclose(status);
    return memory;
}
\end{lstlisting}
