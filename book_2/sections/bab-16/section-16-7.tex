\section{Otomasi Benchmark}

Dalam pengembangan kompilator modern, performa harus dipantau secara otomatis (CI/CD) untuk mencegah degradasi kinerja secara tidak sengaja.

\subsection{Pelacakan Regresi Performa}
Setiap kali ada perubahan kode sumber (\textit{commit}), sistem otomatis harus menjalankan \textit{benchmark suite}. Jika performa turun di bawah ambang batas (\textit{threshold}) tertentu, sistem akan menandainya sebagai \compiler{Performance Regression}.

\subsection{Teknik Bisection untuk Performa}
Jika terjadi penurunan kinerja yang signifikan, kita menggunakan teknik \textbf{Bisection} \cite{ashermancinelli2024bisecting}.
\begin{itemize}
    \item \textbf{Prinsip}: Melakukan pencarian biner (\textit{binary search}) pada riwayat perubahan kode untuk menemukan \textit{commit} tunggal mana yang menyebabkan perlambatan.
    \item \textbf{Automated Bisect}: Kompilator seperti LLVM menggunakan alat otomatis yang membangun versi kompilator di tengah rentang waktu tertentu, menjalankan benchmark, dan mengulangi proses hingga penyebab utama ditemukan.
\end{itemize}

\begin{figure}[!htbp]
    \centering
    \adjustbox{max width=0.8\textwidth,center}{%
    \begin{tikzpicture}[
        compilation_step/.style={circle, draw, fill=gray!20, minimum size=0.6cm, font=\tiny}
    ]
    \node[compilation_step] (s1) {Good};
    \node[compilation_step, right=0.5cm of s1] (s2) {Good};
    \node[compilation_step, right=0.5cm of s2, fill=red!30] (s3) {Bad?};
    \node[compilation_step, right=0.5cm of s3, fill=red!50] (s4) {Slow};
    \node[compilation_step, right=0.5cm of s4, fill=red!50] (s5) {Slow};
    
    \draw[->] (s1) -- (s2);
    \draw[->] (s2) -- (s3);
    \draw[->] (s3) -- (s4);
    \node[below=0.1cm of s3, font=\tiny] {\textbf{Regresi ditemukan di sini}};
    \end{tikzpicture}%
    }
    \caption{Proses Bisection untuk Mencari Penurunan Performa}
\end{figure}
