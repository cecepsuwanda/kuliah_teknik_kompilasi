\section{Metrik Kinerja Dunia Nyata}

\subsection{Benchmark Industri: SPEC CPU2017}
Untuk mengevaluasi kompilator secara profesional, industri menggunakan \compiler{SPEC CPU2017}. Ini adalah kumpulan aplikasi dunia nyata (ditulis dalam C, C++, dan Fortran) yang dirancang untuk menguras kemampuan komputasi CPU dan memori \cite{spec2024cpu2017}.
\begin{itemize}
  \item \textbf{SPECspeed}: Mengukur seberapa cepat sistem menyelesaikan satu tugas instruksi intensif.
  \item \textbf{SPECrate}: Mengukur \textit{throughput} (seberapa banyak tugas yang dapat diselesaikan secara paralel).
  \item \textbf{Integer vs Floating Point}: Memisahkan pengujian berdasarkan jenis operasi matematika untuk akurasi profil.
\end{itemize}

\subsection{Interpretasi Data: Geometric Mean}
Saat membandingkan performa di banyak kasus uji (seperti pada SPEC), skor akhir tidak dihitung dengan rata-rata aritmatika biasa, melainkan dengan \textbf{Geometric Mean}.
\begin{itemize}
    \item \textbf{Alasan}: Rata-rata geometris tidak mudah dipengaruhi oleh satu hasil yang sangat ekstrem (\textit{outliers}).
    \item \textbf{Keadilan}: Menjamin bahwa peningkatan performa 10\% pada program kecil memiliki bobot yang sama dengan peningkatan 10\% pada program besar, sehingga data perbandingan menjadi lebih adil dan representatif.
\end{itemize}

\begin{figure}[!htbp]
    \centering
    \adjustbox{max width=0.8\textwidth,center}{%
    \begin{tikzpicture}[
        rect/.style={rectangle, draw=orange!50, fill=orange!10, text width=6cm, font=\tiny, align=center}
    ]
    \node[rect] (spec) {SPEC CPU2017: 43+ Benchmarks (C, C++, Fortran)};
    \node[rect, below=0.2cm of spec] (res) {Final Score = Geometric Mean of all results};
    \draw[->] (spec) -- (res);
    \end{tikzpicture}%
    }
    \caption{Standar Evaluasi Kinerja Kompilator Profesional}
\end{figure}
