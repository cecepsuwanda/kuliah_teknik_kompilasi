\section{Metodologi Benchmarking}

\subsection{Desain Pengujian (Benchmark Design)}

\textit{Benchmarking} bukan sekadar menjalankan program dan mencatat waktu. Hasil yang akurat memerlukan metodologi yang ketat untuk meminimalkan anomali data \cite{jhu2024compilers}.

Prinsip-prinsip benchmark yang baik:
\begin{itemize}
  \item \textbf{Reproducibility}: Jika dijalankan berkali-kali di mesin yang sama, hasilnya harus konsisten.
  \item \textbf{Fairness}: Semua kandidat harus diuji dengan beban kerja dan lingkungan yang identik.
  \item \textbf{Representative}: Kasus uji harus mencerminkan beban kerja yang akan dihadapi pengguna di dunia nyata.
  \item \textbf{Statistical Significance}: Menggunakan rata-rata dari banyak iterasi daripada satu kali percobaan.
\end{itemize}

\subsection{Siklus Pemanasan (Warm-up Cycle)}
Salah satu kesalahan pemula adalah langsung mengukur eksekusi pertama.
\begin{itemize}
    \item \textbf{Masalah}: Eksekusi pertama seringkali lebih lambat karena \textit{cold caches} (data belum ada di L1/L2), \textit{branch predictor} yang belum terlatih, dan \textit{disk I/O} yang lambat.
    \item \textbf{Solusi}: Jalankan program beberapa kali sebelum mulai mengambil data (\textit{warm-up phase}) agar performa mencapai kondisi stabil (\textit{steady state}).
\end{itemize}

\subsection{Derau Sistem (System Noise)}
Gangguan dari latar belakang dapat merusak data benchmark.
\begin{itemize}
    \item \textbf{Penyebab}: Proses latar belakang (OS \textit{updates}, peramban web), \textit{frequency scaling} (CPU yang menurunkan kecepatan karena panas), dan interupsi jaringan.
    \item \textbf{Mitigasi}: Matikan aplikasi yang tidak perlu, gunakan \textit{performance governor} yang statis pada Linux, dan jalankan benchmark di lingkungan terisolasi.
\end{itemize}

\begin{figure}[!htbp]
    \centering
    \adjustbox{max width=0.8\textwidth,center}{%
    \begin{tikzpicture}[
        node/.style={rectangle, draw=blue!50, fill=blue!10, text width=5cm, font=\tiny, align=center}
    ]
    \node[node] (warm) {Warm-up Iterations (Cold State) $\rightarrow$ discard data};
    \node[node, below=0.2cm of warm] (bench) {Measurement Iterations (Steady State) $\rightarrow$ collect data};
    \draw[->] (warm) -- (bench);
    \end{tikzpicture}%
    }
    \caption{Alur Pengukuran yang Akurat}
\end{figure}
