\section{Benchmarking Methodology}

\subsection{Benchmark Design}

Prinsip benchmark yang baik:

\begin{itemize}
  \item \textbf{Reproducibility}: Hasil dapat direproduksi
  \item \textbf{Fairness}: Perbandingan yang adil
  \item \textbf{Representative}: Mewakili kasus nyata
  \item \textbf{Statistical significance}: Hasil statistik valid
\end{itemize}

\subsection{Test Suite}

\begin{lstlisting}[language=C]
// Benchmark test suite structure
typedef struct {
    char *name;
    char *description;
    char *input_file;
    int expected_time_ms;
    int expected_memory_kb;
} BenchmarkCase;

BenchmarkCase benchmarks[] = {
    {"small_file", "Small C file", "small.c", 100, 1024},
    {"medium_file", "Medium C file", "medium.c", 500, 4096},
    {"large_file", "Large C file", "large.c", 2000, 16384},
    {"complex_template", "Complex template", "template.cpp", 5000, 32768}
};
\end{lstlisting}
