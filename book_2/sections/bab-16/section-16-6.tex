\section{Analisis Dampak Optimasi}

Optimasi tingkat lanjut tidak hanya bergantung pada logika kode, tetapi juga pada bagaimana kode tersebut dieksekusi di dunia nyata.

\subsection{Profile Guided Optimization (PGO)}
\compiler{PGO} (disebut juga \textit{Feedback-Directed Optimization}) menggunakan data performa dari ekosistem nyata untuk memandu kompilator \cite{fedoraproject2024compilers}.
Siklus PGO:
\begin{enumerate}
    \item \textbf{Instrumentation}: Kompilasi program dengan penanda (\textit{probes}) untuk mencatat jalur mana yang sering dilewati (\textit{hot paths}).
    \item \textbf{Profiling}: Menjalankan program yang telah ditandai dengan beban kerja representatif.
    \item \textbf{Optimized Build}: Kompilasi ulang menggunakan data profil tersebut untuk memprioritaskan optimasi pada fungsi yang paling sering digunakan.
\end{enumerate}

\subsection{BOLT (Binary Optimization and Layout Tool)}
Setelah kode ditautkan (\textit{post-link}), alat seperti \compiler{BOLT} dapat mengoptimalkan biner yang sudah jadi.
\begin{itemize}
    \item \textbf{Reorganisasi Biner}: BOLT mengatur ulang tata letak instruksi fisik di dalam berkas eksekusi agar fungsi-fungsi yang sering memanggil satu sama lain diletakkan berdekatan.
    \item \textbf{Manfaat}: Mengurangi \textit{Instruction Cache Misses} dan meningkatkan akurasi prediksi cabang secara signifikan melampaui kemampuan kompilator standar.
\end{itemize}

\begin{figure}[!htbp]
    \centering
    \adjustbox{max width=0.8\textwidth,center}{%
    \begin{tikzpicture}[
        node/.style={rectangle, draw=red!50, fill=red!10, text width=6cm, font=\tiny, align=center}
    ]
    \node[node] (pgo) {PGO: Higher-level code decisions (Inlining, Branching)};
    \node[node, below=0.2cm of pgo] (bolt) {BOLT: Lower-level binary layout (Cache locality)};
    \draw[dashed] (pgo) -- (bolt) node[midway, right] {Combined Benefit (up to 20\%+)};
    \end{tikzpicture}%
    }
    \caption{Optimasi Berbasis Umpan Balik (Feedback-Driven)}
\end{figure}
