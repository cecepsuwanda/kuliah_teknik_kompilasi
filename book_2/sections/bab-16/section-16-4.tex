\section{Perbandingan Kompilator dan Tingkat Optimasi}

\subsection{Tingkat Optimasi Standar}
Kompilator modern memiliki bendera (\textit{flags}) standar:
\begin{itemize}
    \item \code{-O0}: Tanpa optimasi (untuk debugging cepat).
    \item \code{-O2}: Optimasi moderat tanpa kompromi ukuran kode.
    \item \code{-O3}: Optimasi agresif (vektorisasi, unrolling).
    \item \code{-Os}: Optimasi untuk ukuran biner terkecil.
\end{itemize}

\subsection{Link Time Optimization (LTO)}
Kompilasi tradisional bekerja secara independen per berkas sumber. Hal ini mencegah optimasi lintas berkas (\textit{cross-module}).
\compiler{Link Time Optimization (LTO)} menunda pembuatan kode final hingga tahap penautan (\textit{linking}):
\begin{itemize}
    \item \textbf{Whole-Program Analysis}: Kompilator dapat melihat seluruh program sebagai satu kesatuan.
    \item \textbf{Cross-file Inlining}: Fungsi dari \code{file\_a.c} dapat disisipkan (\textit{inlined}) ke dalam \code{file\_b.c}.
    \item \textbf{Dead Code Elimination}: Menghapus kode yang benar-benar tidak pernah dipanggil di seluruh proyek.
\end{itemize}

\begin{figure}[!htbp]
    \centering
    \adjustbox{max width=0.8\textwidth,center}{%
    \begin{tikzpicture}[
        rect/.style={rectangle, draw=purple!50, fill=purple!10, text width=2.5cm, font=\tiny, align=center}
    ]
    \node[rect] (f1) {Source A};
    \node[rect, below=0.2cm of f1] (f2) {Source B};
    \node[rect, right=1.5cm of f1] (obj) {Intermediate (Bitcode)};
    \node[rect, right=1.5cm of obj, minimum height=1.5cm, fill=red!10] (lto) {\textbf{LTO Linker} (Optimize All)};
    
    \draw[->] (f1) -- (obj);
    \draw[->] (f2) -- (obj);
    \draw[->] (obj) -- (lto);
    \end{tikzpicture}%
    }
    \caption{Alur LTO untuk Optimasi Seluruh Program}
\end{figure}
