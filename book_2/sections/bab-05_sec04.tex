\section{BNF dan EBNF Notasi}

\subsection{Backus-Naur Form (BNF)}

BNF adalah notasi metalanguage yang dikembangkan oleh John Backus dan Peter Naur untuk mendefinisikan syntax bahasa pemrograman. BNF menggunakan simbol-simbol berikut:

\begin{itemize}
    \item \texttt{::=} atau \texttt{→}: Menandakan "didefinisikan sebagai"
    \item \texttt{|}: Menandakan alternatif (OR)
    \item \texttt{<nonterminal>}: Nonterminal symbol (biasanya dalam angle brackets)
    \item \texttt{terminal}: Terminal symbol (biasanya tanpa angle brackets)
\end{itemize}

Contoh grammar dalam BNF:

\begin{lstlisting}[language={},basicstyle=\ttfamily\footnotesize,breaklines=true,breakatwhitespace=false]
<expression> ::= <expression> + <term> 
               | <expression> - <term>
               | <term>

<term> ::= <term> * <factor>
        | <term> / <factor>
        | <factor>

<factor> ::= ( <expression> )
          | <number>
\end{lstlisting}

\subsection{Extended BNF (EBNF)}

EBNF memperluas BNF dengan konstruksi tambahan untuk membuat grammar lebih kompak dan mudah dibaca:

\begin{itemize}
    \item \textbf{Optional}: \texttt{[ ... ]} atau \texttt{?} - elemen opsional (nol atau satu kali)
    \item \textbf{Repetition}: 
    \begin{itemize}
        \item \texttt{*}: Nol atau lebih kali
        \item \texttt{+}: Satu atau lebih kali
    \end{itemize}
    \item \textbf{Grouping}: \texttt{( ... )} - untuk mengelompokkan
    \item \textbf{Terminal strings}: \texttt{"..."} atau \texttt{'...'} - untuk literal strings
\end{itemize}

Contoh grammar yang sama dalam EBNF:

\begin{verbatim}
expression = term { ("+" | "-") term }
term       = factor { ("*" | "/") factor }
factor     = "(" expression ")" | number
\end{verbatim}

EBNF lebih ringkas dan mudah dibaca. Banyak spesifikasi bahasa modern menggunakan EBNF, termasuk ISO standard untuk grammar notation.

Gambar \ref{fig:bnf-vs-ebnf} menunjukkan perbandingan notasi BNF dan EBNF.

\begin{figure}[H]
    \centering
    \adjustbox{max width=0.9\textwidth,center}{%
    \begin{tikzpicture}[
        box/.style={rectangle, draw=blue!50, fill=blue!10, text width=4cm, minimum height=0.6cm, font=\tiny\ttfamily, align=left, inner sep=4pt, rounded corners},
        label/.style={font=\footnotesize\bfseries, align=center},
        node distance=0.4cm
    ]
    
    \node[label] (bnf-title) {BNF};
    \node[box, below=of bnf-title] (bnf1) {<list> ::= <item>};
    \node[box, below=of bnf1] (bnf2) {| <list> , <item>};
    
    \node[label, right=4cm of bnf-title] (ebnf-title) {EBNF};
    \node[box, below=of ebnf-title] (ebnf1) {list = item \{ "," item \}};
    
    \node[below=0.3cm of bnf2, font=\tiny] {Lebih verbose};
    \node[below=0.3cm of ebnf1, font=\tiny] {Lebih ringkas};
    
    \end{tikzpicture}%
    }
    \caption{Perbandingan notasi BNF dan EBNF}
    \label{fig:bnf-vs-ebnf}
\end{figure}

\subsection{Contoh Grammar untuk Konstruksi Bahasa}

Berikut contoh grammar untuk beberapa konstruksi bahasa pemrograman:

\subsubsection{If-Statement}

\begin{lstlisting}[language={},basicstyle=\ttfamily\footnotesize,breaklines=true,breakatwhitespace=false]
<if_statement> ::= if ( <expression> ) <statement>
                 | if ( <expression> ) <statement> else <statement>
\end{lstlisting}

Dalam EBNF:
\begin{verbatim}
if_statement = "if" "(" expression ")" statement 
               [ "else" statement ]
\end{verbatim}

\subsubsection{While Loop}

\begin{verbatim}
<while_statement> ::= while ( <expression> ) <statement>
\end{verbatim}

Dalam EBNF:
\begin{verbatim}
while_statement = "while" "(" expression ")" statement
\end{verbatim}

\subsubsection{Variable Declaration}

\begin{verbatim}
<declaration> ::= <type> <identifier> [ = <expression> ] ;
\end{verbatim}

Dalam EBNF:
\begin{verbatim}
declaration = type identifier [ "=" expression ] ";"
\end{verbatim}