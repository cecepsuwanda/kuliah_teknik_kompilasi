\section{Referensi dan Bahan Bacaan Lanjutan}

Untuk memperdalam pemahaman tentang code generation, mahasiswa disarankan membaca:

\begin{itemize}
    \item \textbf{Dragon Book}: Aho, Lam, Sethi, \& Ullman (2006). \textit{Compilers: Principles, Techniques, and Tools} \cite{aho2006compilers} - Bab 8: Code Generation
    
    \item \textbf{Engineering a Compiler}: Cooper \& Torczon (2011) \cite{cooper2011engineering} - Bab 7-9: Code Shape, Introduction to Optimization, Scalar Optimizations
    
    \item \textbf{RISC-V Instruction Set Manual}: \url{https://riscv.org/technical/specifications/} - Dokumentasi lengkap instruksi RISC-V
    
    \item \textbf{LLVM Code Generator Documentation}: \url{https://llvm.org/docs/CodeGenerator.html} - Dokumentasi code generator LLVM
    
    \item \textbf{StudyLib - Outcomes-Based Education}: Materials tentang code generation dan runtime structures \cite{studylib2024obe}
    
    \item \textbf{Wikipedia - Register Allocation}: \url{https://en.wikipedia.org/wiki/Register_allocation} - Artikel tentang teknik register allocation
    
    \item \textbf{Wikipedia - Instruction Selection}: \url{https://en.wikipedia.org/wiki/Instruction_selection} - Artikel tentang instruction selection
\end{itemize}