\section{Activation Records (Stack Frames)}

Activation record (juga disebut stack frame) adalah struktur data yang digunakan untuk menyimpan informasi tentang eksekusi satu fungsi. Setiap kali fungsi dipanggil, activation record baru dibuat di stack.

\subsection{Komponen Activation Record}

Activation record biasanya berisi komponen-komponen berikut:

\begin{enumerate}
    \item \textbf{Return Address}: Alamat instruksi di caller yang harus dieksekusi setelah fungsi kembali
    
    \item \textbf{Control Link (Dynamic Link)}: Pointer ke activation record dari caller (fungsi yang memanggil)
    
    \item \textbf{Access Link (Static Link)}: Pointer ke activation record dari enclosing scope (untuk nested functions)
    
    \item \textbf{Saved Registers}: Nilai register yang harus disimpan dan dikembalikan setelah fungsi selesai
    
    \item \textbf{Parameters}: Nilai parameter yang diteruskan ke fungsi (actual parameters)
    
    \item \textbf{Local Variables}: Variabel lokal yang dideklarasikan dalam fungsi
    
    \item \textbf{Temporary Values}: Nilai sementara yang digunakan selama komputasi dalam fungsi
    
    \item \textbf{Return Value}: Nilai yang dikembalikan fungsi (jika ada)
\end{enumerate}

Gambar \ref{fig:activation-record-visual} menunjukkan struktur activation record secara visual.

\begin{figure}[H]
    \centering
    \adjustbox{max width=0.85\textwidth,center}{%
    \begin{tikzpicture}[
        comp/.style={rectangle, draw=blue!50, fill=blue!10, text width=3cm, minimum height=0.5cm, font=\tiny, align=center, inner sep=4pt, rounded corners},
        node distance=0.2cm
    ]
    
    \node[comp] (ret) {Return Address};
    \node[comp, below=of ret] (ctrl) {Control Link};
    \node[comp, below=of ctrl] (acc) {Access Link};
    \node[comp, below=of acc] (reg) {Saved Registers};
    \node[comp, below=of reg] (param) {Parameters};
    \node[comp, below=of param] (local) {Local Variables};
    \node[comp, below=of local] (temp) {Temporaries};
    \node[comp, below=of temp] (retval) {Return Value};
    
    \end{tikzpicture}%
    }
    \caption{Struktur activation record}
    \label{fig:activation-record-visual}
\end{figure}

Gambar \ref{fig:activation-record} menunjukkan struktur activation record yang khas:

\begin{figure}[H]
\centering
\begin{verbatim}
High Address
    +---------------------+
    |  Return Address     |
    +----------------------+
    |  Control Link (FP)  | -> Activation record caller
    +----------------------+
    |  Access Link        | -> Enclosing scope (if nested)
    +----------------------+
    |  Saved Registers    |
    +----------------------+
    |  Parameters         |
    |    param1           |
    |    param2           |
    +----------------------+
    |  Local Variables    |
    |    local1           |
    |    local2           |
    +----------------------+
    |  Temporaries        |
    |    temp1            |
    |    temp2            |
    +----------------------+
    |  Return Value       |
Low Address
\end{verbatim}
\caption{Struktur activation record (stack frame)}
\label{fig:activation-record}
\end{figure}

\subsection{Calling Sequence}

Calling sequence adalah urutan instruksi yang dihasilkan compiler untuk memanggil fungsi. Terdapat dua bagian:

\subsubsection{Caller Sequence (Prologue)}

Instruksi yang dijalankan oleh caller sebelum memanggil fungsi:
\begin{enumerate}
    \item Evaluasi actual parameters (dari kanan ke kiri atau kiri ke kanan, tergantung calling convention)
    \item Push parameters ke stack (atau pass melalui register)
    \item Save caller-saved registers
    \item Push return address
    \item Transfer control ke callee (CALL instruction)
\end{enumerate}

\subsubsection{Callee Sequence (Prologue)}

Instruksi yang dijalankan oleh callee di awal fungsi:
\begin{enumerate}
    \item Save frame pointer (FP) dari caller
    \item Set FP baru ke current stack pointer (SP)
    \item Allocate space untuk local variables (adjust SP)
    \item Save callee-saved registers (jika diperlukan)
\end{enumerate}

\subsubsection{Return Sequence (Epilogue)}

Instruksi yang dijalankan saat fungsi kembali:
\begin{enumerate}
    \item Place return value (di register atau stack)
    \item Restore callee-saved registers
    \item Restore SP (deallocate local variables)
    \item Restore FP dari control link
    \item Restore return address
    \item Return control ke caller (RET instruction)
\end{enumerate}

\subsection{Contoh Calling Sequence}

Mari kita lihat contoh calling sequence untuk program sederhana:

\begin{lstlisting}[language=C++, caption={Contoh program untuk analisis calling sequence}]
int add(int x, int y) {
    int sum = x + y;
    return sum;
}

int main() {
    int a = 5;
    int b = 10;
    int result = add(a, b);
    return 0;
}
\end{lstlisting}

Assembly code yang dihasilkan (simplified):

\begin{lstlisting}[language={[x86masm]Assembler},basicstyle=\ttfamily\footnotesize,breaklines=true,breakatwhitespace=false]
main:
    push rbp              ; Save caller's frame pointer
    mov rbp, rsp          ; Set new frame pointer
    sub rsp, 16           ; Allocate space for locals (a, b, result)
    
    mov [rbp-4], 5        ; a = 5
    mov [rbp-8], 10       ; b = 10
    
    ; Call add(a, b)
    mov eax, [rbp-8]      ; Load b
    push eax              ; Push parameter 2
    mov eax, [rbp-4]      ; Load a
    push eax              ; Push parameter 1
    call add              ; Call function
    add rsp, 8            ; Clean up parameters
    mov [rbp-12], eax     ; result = return value
    
    mov eax, 0            ; return 0
    mov rsp, rbp          ; Restore stack pointer
    pop rbp                ; Restore frame pointer
    ret                    ; Return

add:
    push rbp              ; Save caller's frame pointer
    mov rbp, rsp          ; Set new frame pointer
    sub rsp, 4            ; Allocate space for local (sum)
    
    mov eax, [rbp+8]      ; Load x (parameter 1)
    add eax, [rbp+12]     ; Add y (parameter 2)
    mov [rbp-4], eax      ; sum = x + y
    mov eax, [rbp-4]      ; Load sum for return
    
    mov rsp, rbp          ; Restore stack pointer
    pop rbp                ; Restore frame pointer
    ret                    ; Return
\end{lstlisting}