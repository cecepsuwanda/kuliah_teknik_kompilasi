\section{Demonstrasi Compiler}

Demo adalah bagian penting dari presentasi. Demo yang baik menunjukkan bahwa compiler benar-benar berfungsi dan dapat digunakan secara praktis.

Gambar \ref{fig:project-evaluation} menunjukkan aspek-aspek evaluasi project final.

\begin{figure}[H]
    \centering
    \adjustbox{max width=0.9\textwidth,center}{%
    \begin{tikzpicture}[
        aspect/.style={rectangle, draw=blue!50, fill=blue!10, text width=2.5cm, text centered, minimum height=0.7cm, rounded corners, font=\footnotesize, inner sep=4pt, align=center},
        arrow/.style={->, >=stealth, thick},
        node distance=1.2cm
    ]
    
    \node[aspect] (func) {Functionality};
    \node[aspect, right=of func] (design) {Design};
    \node[aspect, right=of design] (perf) {Performance};
    \node[aspect, below=of design] (doc) {Documentation};
    
    \draw[arrow] (func) -- (doc);
    \draw[arrow] (design) -- (doc);
    \draw[arrow] (perf) -- (doc);
    
    \end{tikzpicture}%
    }
    \caption{Aspek-aspek evaluasi project final}
    \label{fig:project-evaluation}
\end{figure}

\subsection{Preparing for Demo}

\textbf{1. Test Cases yang Komprehensif}
\begin{itemize}
    \item Program sederhana (hello world, arithmetic)
    \item Program dengan kontrol flow (if-else, loops)
    \item Program dengan fungsi dan scope
    \item Program dengan error (untuk menunjukkan error handling)
    \item Program yang lebih kompleks (menunjukkan kemampuan compiler)
\end{itemize}

\textbf{2. Environment Setup}
\begin{itemize}
    \item Pastikan semua dependencies terinstall
    \item Compile compiler terlebih dahulu
    \item Siapkan backup plan jika ada masalah teknis
    \item Test di environment yang sama dengan presentasi
\end{itemize}

\textbf{3. Demo Script}
\begin{itemize}
    \item Buat script demo yang terstruktur
    \item Siapkan penjelasan untuk setiap langkah
    \item Antisipasi kemungkinan error atau masalah
\end{itemize}

\subsection{Contoh Demo Flow}

Berikut adalah contoh alur demo yang efektif:

\begin{enumerate}
    \item \textbf{Hello World}: Menunjukkan compiler dapat menghasilkan executable sederhana
    \begin{verbatim}
    // hello.lang
    print("Hello, World!");
    \end{verbatim}
    
    \item \textbf{Arithmetic Operations}: Menunjukkan kemampuan menangani ekspresi
    \begin{verbatim}
    // calc.lang
    int x = 10;
    int y = 20;
    int result = x + y * 2;
    print(result);
    \end{verbatim}
    
    \item \textbf{Control Flow}: Menunjukkan if-else dan loops
    \begin{verbatim}
    // control.lang
    int i = 0;
    while (i < 10) {
        if (i % 2 == 0) {
            print(i);
        }
        i = i + 1;
    }
    \end{verbatim}
    
    \item \textbf{Error Handling}: Menunjukkan kualitas error messages
    \begin{verbatim}
    // error.lang
    int x = 10;
    int y = "string";  // Type error
    x = undefined_var; // Undefined variable
    \end{verbatim}
    
    \item \textbf{Complex Program}: Menunjukkan kemampuan compiler dengan program yang lebih kompleks
\end{enumerate}