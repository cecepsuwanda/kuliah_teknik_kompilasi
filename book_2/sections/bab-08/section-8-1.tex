\section{Dasar Basic Block dan Karakteristiknya}

\compiler{Basic Block} adalah sekumpulan instruksi kode antara (\textit{intermediate code}) yang dieksekusi secara berurutan tanpa adanya instruksi lompatan (\textit{jump/branch}) di tengah-tengahnya.

\subsection{Definisi Formal}
Sebuah blok instruksi disebut \textit{Basic Block} jika:
\begin{itemize}
    \item Hanya memiliki satu titik masuk (\textit{entry point}), yaitu instruksi pertama.
    \item Hanya memiliki satu titik keluar (\textit{exit point}), yaitu instruksi terakhir.
    \item Semua instruksi di dalamnya dieksekusi secara sekuensial.
\end{itemize}

\subsection{Contoh Basic Block}
Perhatikan potongan kode berikut:
\begin{lstlisting}
t1 = a + b
t2 = t1 * c
x = t2
\end{lstlisting}
Setelah eksekusi dimulai dari baris pertama, semua baris berikutnya dijamin akan dieksekusi secara berurutan. Ini adalah satu \textit{Basic Block}.
