\section{Dasar Basic Block dan Karakteristiknya}

\compiler{Basic Block} adalah unit atomik terkecil dalam analisis optimasi kompilator. Ia merupakan sekumpulan instruksi yang dieksekusi sebagai satu kesatuan: jika instruksi pertama dieksekusi, maka semua instruksi di bawahnya pasti dieksekusi hingga akhir blok.

\subsection{Definisi dan Sifat Utama}
Sebuah blok instruksi disebut sebagai \textit{Basic Block} jika memenuhi kriteria berikut:
\begin{itemize}
    \item \textbf{Single Entry}: Kontrol hanya bisa masuk ke blok melalui instruksi pertama. Tidak ada instruksi di tengah blok yang boleh menjadi target lompatan (\textit{jump label}) dari luar.
    \item \textbf{Single Exit}: Kontrol hanya bisa keluar dari blok melalui instruksi terakhir. Instruksi di tengah blok tidak boleh berupa \textit{jump} atau \textit{branch}.
    \item \textbf{High Cohesion}: Semua instruksi di dalamnya bersifat atomik dalam konteks kontrol alur.
\end{itemize}

\subsection{Variabel Live on Exit}
Dalam analisis blok, sangat penting untuk mengetahui variabel mana saja yang nilainya masih akan digunakan oleh blok lain setelah blok saat ini selesai dieksekusi. Variabel ini disebut \textbf{Live on Exit}. 
\begin{itemize}
    \item Jika sebuah variabel adalah \textit{dead} (tidak live), maka instruksi penugasan terakhir ke variabel tersebut di dalam blok dapat dihapus karena hasilnya tidak akan pernah dibaca lagi.
\end{itemize}

\begin{figure}[!htbp]
    \centering
    \adjustbox{max width=0.8\textwidth,center}{%
    \begin{tikzpicture}[
        node/.style={rectangle, draw=blue!50, fill=blue!10, font=\small, align=center, rounded corners, minimum height=1cm, text width=4cm},
        arrow/.style={->, >=stealth, thick}
    ]
    \node[node] (bb) {Basic Block $B$\\ \texttt{t1 = a + b}\\ \texttt{x = t1 * 2}};
    \node[below=0.2cm of bb, font=\itshape\footnotesize] {Entry Point (Fixed)};
    \node[above=0.2cm of bb, font=\itshape\footnotesize] {Exit Point (Fixed)};
    \end{tikzpicture}%
    }
    \caption{Abstraksi Basic Block sebagai Unit Terisolasi}
\end{figure}
