\section{Algoritma Identifikasi Leader}

Proses dekomposisi kode TAC menjadi sekumpulan \textit{Basic Blocks} diawali dengan menentukan \textbf{Leaders}.

\subsection{Prosedur Penentuan Leader}
Instruksi dalam deretan kode antara adalah \textit{leader} jika memenuhi salah satu aturan berikut:
\begin{enumerate}
    \item \textbf{Rule 1}: Instruksi pertama dalam kode program.
    \item \textbf{Rule 2}: Instruksi yang merupakan target (\textit{destination}) dari perintah lompatan (\texttt{goto}, \texttt{if...goto}).
    \item \textbf{Rule 3}: Instruksi yang berada tepat setelah perintah lompatan (\texttt{goto}, \texttt{if...goto}).
\end{enumerate}

\subsection{Contoh Trace Identifikasi}
Program TAC untuk menghitung faktorial:
\begin{lstlisting}[numbers=left, xleftmargin=2em]
(1) i = 1
(2) fact = 1
(3) if i > n goto (7)
(4) t1 = fact * i
(5) fact = t1
(6) i = i + 1
(7) goto (3)
(8) return fact
\end{lstlisting}

\textbf{Leader Analysis}:
\begin{itemize}
    \item \textbf{Line 1}: Leader (Rule 1).
    \item \textbf{Line 3}: Leader (Rule 2 - Target dari baris 7).
    \item \textbf{Line 4}: Leader (Rule 3 - Setelah \texttt{if} di baris 3).
    \item \textbf{Line 7}: Target dari baris 7 sendiri (Sudah tertutup Rule 2/3).
    \item \textbf{Line 8}: Leader (Rule 3 - Setelah \texttt{goto} di baris 7).
\end{itemize}

\subsection{Hasil Pembentukan Blok}
Berdasarkan leader tersebut, kita membagi kode menjadi:
\begin{itemize}
    \item \textbf{Block 1}: Baris 1-2.
    \item \textbf{Block 2}: Baris 3.
    \item \textbf{Block 3}: Baris 4-7.
    \item \textbf{Block 4}: Baris 8.
\end{itemize}

\begin{figure}[!htbp]
    \centering
    \adjustbox{max width=0.8\textwidth,center}{%
    \begin{tikzpicture}[
        rect/.style={rectangle, draw=blue!50, fill=blue!10, text width=2.5cm, minimum height=0.6cm, font=\tiny, align=center},
        arrow/.style={->, >=stealth, thick}
    ]
    \node[rect] (b1) {B1 (L1-2)\\Setup};
    \node[rect, below=0.5cm of b1] (b2) {B2 (L3)\\Cond Check};
    \node[rect, below=0.5cm of b2] (b3) {B3 (L4-7)\\Body \& Loop};
    \node[rect, below=0.5cm of b3] (b4) {B4 (L8)\\Exit};
    
    \draw[arrow] (b1) -- (b2);
    \draw[arrow] (b2) -- (b3);
    \draw[arrow] (b3) to[bend left=45] (b2);
    \draw[arrow] (b2) to[bend left=90, looseness=1.5] (b4);
    \end{tikzpicture}%
    }
    \caption{Visualisasi Segmentasi Kode menjadi Basic Blocks}
\end{figure}
