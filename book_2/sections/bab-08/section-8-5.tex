\section{Control Flow Graph (CFG) Construction}

\compiler{Control Flow Graph (CFG)} adalah representasi berarah di mana setiap \textit{node} adalah sebuah \textit{basic block} dan setiap \textit{edge} menunjukkan kemungkinan aliran kendali antar blok tersebut.

\subsection{Membangun Tepi (Edges)}
Sebuah tepi ditambahkan dari blok $B_1$ ke blok $B_2$ jika:
\begin{itemize}
    \item Terdapat instruksi lompatan (jump) dari akhir $B_1$ ke awal $B_2$.
    \item $B_2$ terletak tepat setelah $B_1$ dalam urutan program, dan $B_1$ tidak diakhiri dengan lompatan mutlak (\textit{unconditional jump}).
\end{itemize}

\subsection{Pentingnya CFG}
CFG adalah struktur data utama untuk melakukan \textit{Global Optimization} dan analisis aliran data (\textit{Data Flow Analysis}) yang mencakup seluruh fungsi/program.

\begin{figure}[!htbp]
    \centering
    \adjustbox{max width=0.8\textwidth,center}{%
    \begin{tikzpicture}[
        block/.style={rectangle, draw=green!50, fill=green!10, font=\tiny, align=center},
        arrow/.style={->, >=stealth, thick}
    ]
    \node[block] (b1) {Block 1 (Entry)};
    \node[block, below left=1cm of b1] (b2) {Block 2 (True)};
    \node[block, below right=1cm of b1] (b3) {Block 3 (False)};
    \node[block, below=2.5cm of b1] (b4) {Block 4 (Exit)};
    \draw[arrow] (b1) -- (b2);
    \draw[arrow] (b1) -- (b3);
    \draw[arrow] (b2) -- (b4);
    \draw[arrow] (b3) -- (b4);
    \end{tikzpicture}%
    }
    \caption{Contoh Control Flow Graph sederhana}
\end{figure}
