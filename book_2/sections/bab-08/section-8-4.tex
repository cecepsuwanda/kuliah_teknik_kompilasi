\section{Representasi DAG untuk Optimasi Lokal}

\textit{Directed Acyclic Graph} (\compiler{DAG}) adalah struktur data yang sangat kuat untuk menganalisis aliran data di dalam sebuah \textit{basic block}.

\subsection{Manfaat DAG}
Mengubah deretan instruksi ke dalam bentuk DAG memberikan beberapa keuntungan otomatis bagi kompilator:
\begin{enumerate}
    \item \textbf{Common Subexpression Elimination}: Jika sebuah perhitungan (misal \code{a + b}) muncul lebih dari sekali, DAG secara otomatis hanya akan membuat satu node untuk hasil tersebut.
    \item \textbf{Dead Code Elimination}: Node yang tidak memiliki label variabel keluar (output) dan tidak digunakan oleh node lain dapat langsung dihapus.
    \item \textbf{Instruction Reordering}: Struktur graf memungkinkan kompilator menata ulang instruksi tanpa merusak ketergantungan data.
\end{enumerate}

\subsection{Algoritma Konstruksi DAG}
Setiap instruksi TAC \texttt{x = y op z} diproses sebagai berikut:
\begin{enumerate}
    \item Cari node untuk operan \texttt{y} dan \texttt{z}. Jika belum ada, buat node daun (\textit{leaf}).
    \item Periksa apakah sudah ada node dengan operator \texttt{op} yang memiliki anak \texttt{y} dan \texttt{z}.
    \item Jika ada, beri label \texttt{x} pada node tersebut (hasil penggunaan kembali).
    \item Jika tidak ada, buat node baru dengan operator \texttt{op}, hubungkan ke \texttt{y} dan \texttt{z}, lalu beri label \texttt{x}.
\end{enumerate}

\begin{figure}[!htbp]
    \centering
    \adjustbox{max width=0.8\textwidth,center}{%
    \begin{tikzpicture}[
        node/.style={circle, draw=blue!50, fill=blue!10, minimum size=0.8cm, font=\small},
        arrow/.style={->, >=stealth, thick}
    ]
    % DAG for a = b+c; d = b+c
    \node[node] (plus) at (0,0) {+};
    \node[node] (b) at (-1,-1.5) {b};
    \node[node] (c) at (1,-1.5) {c};
    \draw[arrow] (plus) -- (b);
    \draw[arrow] (plus) -- (c);
    
    \node[right=0.2cm of plus, font=\small] {\{a, d\}};
    \node[below=1.8cm of plus, font=\itshape\footnotesize, text width=5cm, align=center] {
        Satu node '+' melayani variabel 'a' dan 'd'. Hilangkan redundansi.
    };
    \end{tikzpicture}%
    }
    \caption{Representasi DAG untuk Eliminasi Subekspresi Umum}
\end{figure}
