\section{Semantic Error Detection dan Reporting}

Semantic analyzer harus mendeteksi berbagai jenis error dan memberikan pesan error yang jelas dan informatif.

\subsection{Jenis-jenis Semantic Error}

\subsubsection{Undeclared Variable}

Error terjadi ketika variabel digunakan sebelum dideklarasi:

\begin{lstlisting}[language=C++,basicstyle=\ttfamily\footnotesize,breaklines=true,breakatwhitespace=false]
x = 42;  // Error: 'x' is not declared
int x;
\end{lstlisting}

\subsubsection{Type Mismatch}

Error terjadi ketika tipe tidak kompatibel:

\begin{lstlisting}[language=C++,basicstyle=\ttfamily\footnotesize,breaklines=true,breakatwhitespace=false]
int x = "hello";  // Error: cannot assign string to int
int y = 3.14;     // Error: cannot assign float to int (tanpa cast)
\end{lstlisting}

\subsubsection{Undefined Function}

Error terjadi ketika fungsi dipanggil tetapi tidak didefinisikan:

\begin{lstlisting}[language=C++,basicstyle=\ttfamily\footnotesize,breaklines=true,breakatwhitespace=false]
int result = add(5, 10);  // Error: function 'add' is not defined
\end{lstlisting}

\subsubsection{Wrong Number of Arguments}

Error terjadi ketika jumlah argumen tidak sesuai:

\begin{lstlisting}[language=C++,basicstyle=\ttfamily\footnotesize,breaklines=true,breakatwhitespace=false]
int add(int a, int b) { return a + b; }
int x = add(5);  // Error: expected 2 arguments, got 1
\end{lstlisting}

\subsubsection{Return Type Mismatch}

Error terjadi ketika return type tidak sesuai dengan deklarasi:

\begin{lstlisting}[language=C++,basicstyle=\ttfamily\footnotesize,breaklines=true,breakatwhitespace=false]
int getValue() {
    return "hello";  // Error: function should return int
}
\end{lstlisting}

\subsection{Error Reporting yang Informatif}

Pesan error yang baik harus:
\begin{itemize}
    \item Menunjukkan lokasi error (baris, kolom)
    \item Menjelaskan jenis error dengan jelas
    \item Memberikan konteks yang relevan
    \item Menyarankan solusi jika memungkinkan
\end{itemize}

Contoh implementasi error reporting:

\begin{lstlisting}[language=C++, caption=Error Reporting System]
class ErrorReporter {
private:
    std::vector<Error> errors;
    
public:
    void reportError(const std::string& message, 
                     int line, int column) {
        errors.push_back(Error{message, line, column});
        std::cerr << "Error at line " << line 
                  << ", column " << column 
                  << ": " << message << std::endl;
    }
    
    void reportTypeError(const std::string& expected,
                        const std::string& got,
                        int line, int column) {
        std::string msg = "Type mismatch: expected " + expected +
                         ", got " + got;
        reportError(msg, line, column);
    }
    
    bool hasErrors() const {
        return !errors.empty();
    }
    
    const std::vector<Error>& getErrors() const {
        return errors;
    }
};
\end{lstlisting}