\section{Error Handling dalam Bison}

\subsection{Error Token}

Bison menyediakan token khusus \texttt{error} untuk error recovery. Ketika parser menemukan error, ia dapat mencoba recovery dengan menggunakan produksi yang mengandung \texttt{error}.

\subsection{Contoh Error Recovery}

\begin{lstlisting}[language=C, caption={Error recovery dalam Bison}]
%%
input:
    /* empty */
  | input line
  ;

line:
    expr '\n' { printf("Result = %g\n", $1); }
  | error '\n' {
        yyerrok;  // Reset error state
        yyclearin; // Clear lookahead token
        printf("Syntax error, please try again.\n");
      }
  ;
\end{lstlisting}

\subsection{Fungsi yyerror}

Fungsi \texttt{yyerror()} dipanggil ketika terjadi syntax error. Kita dapat mengimplementasikannya untuk memberikan pesan error yang informatif:

\begin{lstlisting}[language=C, caption={Implementasi yyerror yang lebih informatif}]
void yyerror(const char *s) {
    extern int yylineno;
    fprintf(stderr, "Error at line %d: %s\n", yylineno, s);
}
\end{lstlisting}

\subsection{Location Tracking}

Untuk memberikan informasi lokasi yang lebih akurat, kita dapat menggunakan location tracking:

\begin{lstlisting}[language=C, caption={Location tracking dalam Bison}]
%locations

%%
expr: expr '+' expr {
    @$.first_line = @1.first_line;
    @$.first_column = @1.first_column;
    @$.last_line = @3.last_line;
    @$.last_column = @3.last_column;
    if (/* some error condition */) {
        yyerror("Error in expression");
    }
    $$ = $1 + $3;
}
\end{lstlisting}