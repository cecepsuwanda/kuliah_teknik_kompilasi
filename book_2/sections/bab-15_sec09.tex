\section{Evaluasi Efektivitas Optimasi}

Setelah mengimplementasikan optimasi, penting untuk mengevaluasi efektivitasnya.

\subsection{Metrics untuk Evaluasi}

\textbf{1. Code Size}
\begin{itemize}
    \item Ukuran executable sebelum dan sesudah optimasi
    \item Jumlah instruksi dalam intermediate code
    \item Ukuran object files
\end{itemize}

\textbf{2. Execution Time}
\begin{itemize}
    \item Waktu eksekusi program dengan benchmark
    \item Profiling untuk mengidentifikasi bottleneck
    \item Perbandingan before/after
\end{itemize}

\textbf{3. Memory Usage}
\begin{itemize}
    \item Peak memory consumption
    \item Stack usage
    \item Heap allocation patterns
\end{itemize}

\textbf{4. Compilation Time}
\begin{itemize}
    \item Waktu yang dibutuhkan untuk kompilasi
    \item Trade-off antara waktu kompilasi dan kualitas optimasi
\end{itemize}

\subsection{Benchmarking}

Langkah-langkah untuk benchmarking:

\begin{enumerate}
    \item \textbf{Prepare Test Cases}: Siapkan berbagai test case (small, medium, large programs)
    \item \textbf{Baseline Measurement}: Ukur metrik sebelum optimasi
    \item \textbf{Optimized Measurement}: Ukur metrik setelah optimasi
    \item \textbf{Compare Results}: Bandingkan dan hitung improvement percentage
    \item \textbf{Verify Correctness}: Pastikan program masih menghasilkan output yang benar
\end{enumerate}

\subsection{Contoh Evaluasi}

Berikut adalah contoh format laporan evaluasi:

\begin{table}[!htbp]
\centering
\begin{tabular}{|l|c|c|c|}
\hline
\textbf{Metric} & \textbf{Before} & \textbf{After} & \textbf{Improvement} \\
\hline
Code Size (bytes) & 1024 & 768 & 25\% reduction \\
\hline
Execution Time (ms) & 100 & 75 & 25\% faster \\
\hline
Instruction Count & 150 & 110 & 26.7\% reduction \\
\hline
Compilation Time (s) & 2.5 & 3.1 & 24\% slower \\
\hline
\end{tabular}
\caption{Contoh hasil evaluasi optimasi}
\label{tab:optimization-results}
\end{table}