\section{Referensi dan Bahan Bacaan Lanjutan}

Untuk memperdalam pemahaman tentang pengenalan kompilator, mahasiswa disarankan membaca:

\begin{itemize}
    \item \textbf{Dragon Book}: Aho, Lam, Sethi, \& Ullman (2006). \textit{Compilers: Principles, Techniques, and Tools} \cite{aho2006compilers} - Bab 1: Introduction
    
    \item \textbf{Engineering a Compiler}: Cooper \& Torczon (2011) \cite{cooper2011engineering} - Bab 1: Overview of Compilation
    
    \item \textbf{UC San Diego CSE 231}: Course materials tentang compiler construction \cite{ucsd2024compiler}
    
    \item \textbf{Northeastern University CS 4410}: Comprehensive compiler design course \cite{neu2024compiler}
    
    \item \textbf{Johns Hopkins University EN.601.428}: Course tentang compilers dan interpreters \cite{jhu2024compilers}
\end{itemize}
