\section{Referensi dan Bahan Bacaan Lanjutan}

Untuk memperdalam pemahaman tentang optimasi kompilator, mahasiswa disarankan membaca:

\begin{itemize}
    \item \textbf{Dragon Book}: Aho, Lam, Sethi, \& Ullman (2006). \textit{Compilers: Principles, Techniques, and Tools} \cite{aho2006compilers} - Bab 9: Machine-Independent Optimizations
    
    \item \textbf{Engineering a Compiler}: Cooper \& Torczon (2011) \cite{cooper2011engineering} - Bab 8: Introduction to Optimization, Bab 9: Data-Flow Analysis
    
    \item \textbf{GeeksforGeeks}: Tutorial tentang berbagai optimasi kompilator
    \begin{itemize}
        \item Constant Folding: \url{https://www.geeksforgeeks.org/compiler-design/constant-folding/}
        \item Dead Code Elimination: \url{https://www.geeksforgeeks.org/dead-code-elimination/}
        \item Data-Flow Analysis: \url{https://www.geeksforgeeks.org/data-flow-analysis-compiler/}
    \end{itemize}
    
    \item \textbf{University of Michigan}: Course materials tentang compiler optimization \footnote{\url{https://web.eecs.umich.edu/~weimerw/2015-4610/ca1/ca1.html}}
    
    \item \textbf{LLVM Documentation}: Advanced optimization techniques \footnote{\url{https://llvm.org/docs/Passes.html}}
\end{itemize}