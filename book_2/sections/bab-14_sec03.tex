\section{Target Architecture}

Target architecture adalah platform hardware yang menjadi tujuan kompilasi. Setiap architecture memiliki karakteristik yang berbeda yang mempengaruhi bagaimana code generator bekerja.

\subsection{Karakteristik Target Architecture}

Menurut dokumentasi LLVM\footnote{\url{https://llvm.org/docs/CodeGenerator.html}}, target architecture memiliki beberapa karakteristik penting:

\begin{enumerate}
    \item \textbf{Instruction Set Architecture (ISA)}: Kumpulan instruksi yang didukung oleh processor
    \begin{itemize}
        \item RISC (Reduced Instruction Set Computer): Instruksi sederhana, uniform, banyak register
        \item CISC (Complex Instruction Set Computer): Instruksi kompleks, berbagai format, addressing modes yang kaya
    \end{itemize}
    
    \item \textbf{Register Set}: Jumlah dan jenis register yang tersedia
    \begin{itemize}
        \item General-purpose registers
        \item Special-purpose registers (stack pointer, frame pointer, dll.)
        \item Floating-point registers
    \end{itemize}
    
    \item \textbf{Addressing Modes}: Cara mengakses operan (register, memory, immediate)
    \begin{itemize}
        \item Register addressing: \texttt{ADD R1, R2}
        \item Immediate addressing: \texttt{ADD R1, \#42}
        \item Memory addressing: \texttt{LOAD R1, [address]}
        \item Indexed addressing: \texttt{LOAD R1, [R2 + offset]}
    \end{itemize}
    
    \item \textbf{Memory Model}: Bagaimana memory diorganisir dan diakses
    \begin{itemize}
        \item Byte-addressable vs word-addressable
        \item Alignment requirements
        \item Endianness (little-endian vs big-endian)
    \end{itemize}
\end{enumerate}

\subsection{Contoh Target Architecture}

Dalam pembelajaran ini, kita akan fokus pada dua target architecture populer:

\textbf{1. x86-64 (AMD64)}
\begin{itemize}
    \item CISC architecture dengan instruksi kompleks
    \item 16 general-purpose registers (RAX, RBX, RCX, RDX, RSI, RDI, RBP, RSP, R8-R15)
    \item Berbagai addressing modes
    \item Variable-length instructions
    \item Little-endian
\end{itemize}

\textbf{2. RISC-V}
\begin{itemize}
    \item RISC architecture dengan instruksi sederhana
    \item 32 general-purpose registers (x0-x31)
    \item Fixed-length instructions (32-bit atau 16-bit untuk compressed)
    \item Simple addressing modes
    \item Little-endian atau big-endian (configurable)
    \item Open standard, populer untuk pembelajaran
\end{itemize}

Untuk pembelajaran, kita akan menggunakan subset sederhana dari RISC-V karena lebih mudah dipahami dan diimplementasikan.