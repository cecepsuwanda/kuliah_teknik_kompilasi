\section{Kesimpulan}

Bab ini telah membahas:

\begin{enumerate}
    \item \textbf{Project Final Presentation}: Struktur dan tips untuk presentasi yang efektif
    \item \textbf{Demonstrasi Compiler}: Cara mempersiapkan dan melakukan demo yang baik
    \item \textbf{Review Materi}: Review semua fase kompilasi yang telah dipelajari
    \item \textbf{Evaluasi Tools}: Perbandingan hand-written vs generator-based tools
    \item \textbf{Analisis Trade-Off}: Trade-off antara berbagai aspek compiler design
    \item \textbf{Benchmarking}: Metrik dan cara melakukan evaluasi kinerja
    \item \textbf{Dokumentasi}: Best practices untuk dokumentasi proyek
    \item \textbf{Refleksi Pembelajaran}: Cara melakukan refleksi yang konstruktif
\end{enumerate}

Project final adalah kesempatan untuk mengintegrasikan semua pengetahuan yang telah diperoleh selama semester. Melalui project ini, mahasiswa tidak hanya menunjukkan kemampuan teknis, tetapi juga kemampuan untuk membuat keputusan desain, mengevaluasi tools dan teknik, dan berkomunikasi secara efektif tentang hasil kerja.