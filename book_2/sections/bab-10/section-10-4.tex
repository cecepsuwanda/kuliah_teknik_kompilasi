\section{Manajemen Dinamis dan Strategi Heap}

Berbeda dengan stack yang memiliki struktur kaku, \compiler{Heap} adalah area memori bebas di mana blok-blok dapat dialokasikan dan dibebaskan dalam urutan apa pun.

\subsection{Strategi Alokasi Pemuatan}
Kompilator dan pustaka runtime menggunakan algoritma tertentu untuk mencari blok kosong yang cukup besar:
\begin{enumerate}
    \item \textbf{First-Fit}: Mencari dari awal heap dan mengambil blok kosong pertama yang ukurannya cukup. Algoritma ini sangat cepat namun cenderung meninggalkan fragmentasi di awal heap.
    \item \textbf{Best-Fit}: Mencari ke seluruh heap untuk menemukan blok kosong terkecil yang masih cukup menampung permintaan. Ini meminimalkan sisa ruang (\textit{leftover}) namun pencariannya lambat.
    \item \textbf{Next-Fit}: Mirip First-fit, tapi pencarian dimulai dari titik terakhir alokasi dilakukan. Ini mendistribusikan alokasi lebih merata di seluruh heap.
\end{enumerate}

\subsection{Masalah Fragmentasi}
Tujuan utama pengelola memori adalah menekan \textbf{External Fragmentation} (adanya banyak lubang kecil yang terpisah-pisah sehingga tidak bisa melayani alokasi satu blok besar). Solusinya adalah \textit{Coalescing} (menggabungkan blok kosong yang bertetangga menjadi satu) atau \textit{Compaction} (menggeser blok-blok yang terisi ke satu sisi).

\begin{figure}[!htbp]
    \centering
    \adjustbox{max width=0.8\textwidth,center}{%
    \begin{tikzpicture}[
        block/.style={rectangle, draw, minimum width=1cm, minimum height=0.5cm, font=\tiny}
    ]
    \node[block, fill=blue!20] (b1) {Used};
    \node[block, right=0pt of b1] (f1) {Free};
    \node[block, right=0pt of f1, fill=blue!20] (b2) {Used};
    \node[block, right=0pt of b2] (f2) {Free};
    \node[below=0.2cm of f1, font=\tiny] {Lubang Fragmentasi};
    \end{tikzpicture}%
    }
    \caption{Ilustrasi Fragmentasi Eksternal pada Heap}
\end{figure}
