\section{Activation Records dan Skoping Bertingkat}

Setiap kali sebuah fungsi dipanggil, struktur data yang disebut \compiler{Activation Record} (atau \textit{Stack Frame}) dibuat di puncak stack untuk mengelola eksekusi fungsi tersebut.

\subsection{Struktur Stack Frame}
Sebuah stack frame biasanya menampung informasi kritis berikut:
\begin{enumerate}
    \item \textbf{Return Address}: Alamat instruksi berikutnya di \textit{caller} yang harus dieksekusi setelah fungsi selesai.
    \item \textbf{Control Link (Dynamic Link)}: Pointer ke stack frame milik fungsi pemanggil (\textit{caller}). Digunakan untuk menghapus frame saat ini dan kembali ke frame sebelumnya.
    \item \textbf{Access Link (Static Link)}: Pointer ke stack frame milik fungsi yang secara leksikal (sintaksis) membungkus fungsi saat ini. Digunakan untuk mengakses variabel non-lokal dalam bahasa yang mendukung fungsi bersarang (\textit{nested functions}).
    \item \textbf{Saved Registers}: Nilai register prosesor yang harus dipulihkan sebelum kembali.
    \item \textbf{Local Variables}: Ruang untuk data yang dideklarasikan di dalam fungsi tersebut.
\end{enumerate}

\subsection{Menangani Akses Variabel Non-Lokal}
Dalam bahasa seperti Pascal atau Python, fungsi di dalam fungsi dapat mengakses variabel milik induknya. Ada dua teknik utama untuk mengelola ini:
\begin{enumerate}
    \item \textbf{Access Links}: Membentuk rantai pointer statis. Jika variabel berada di tingkat bersarang ke-3 di atasnya, kompilator harus mengikuti rantai link tersebut 3 kali. Ini sederhana tapi lambat jika nesting terlalu dalam.
    \item \textbf{Displays}: Menggunakan larik (\textit{array}) global berisi pointer ke frame yang sedang aktif di setiap tingkat bersarang. Dengan \textit{display}, akses ke tingkat mana pun selalu berbiaya konstan (satu kali akses larik).
\end{enumerate}

\begin{figure}[!htbp]
    \centering
    \adjustbox{max width=0.8\textwidth,center}{%
    \begin{tikzpicture}[
        node/.style={rectangle, draw=blue!50, fill=blue!10, text width=3cm, minimum height=0.6cm, font=\tiny, align=center}
    ]
    \node[node] (f1) {Frame A (Outer)};
    \node[node, below=0.5cm of f1] (f2) {Frame B (Inner)};
    \draw[->, >=stealth, bend left=45, blue, dashed] (f2.east) to node[right, font=\tiny] {Access Link} (f1.east);
    \draw[->, >=stealth, bend right=45, red] (f2.west) to node[left, font=\tiny] {Control Link} (f1.west);
    \end{tikzpicture}%
    }
    \caption{Perbedaan Aliran Control Link (Dinamis) vs Access Link (Statis)}
\end{figure}
