\section{Layout Memori dan Segmentasi}

\compiler{Runtime Environment} mengelola bagaimana program menggunakan memori saat dieksekusi. Manajemen memori saat runtime merupakan salah satu aspek yang paling krusial dalam performa dan keamanan sebuah program hasil kompilasi \cite{ucsd2024compiler}.

\subsection{Segmen Memori Utama}
Secara tradisional, memori proses dibagi menjadi beberapa segmen:
\begin{itemize}
    \item \textbf{Text/Code Segment}: Menyimpan instruksi biner hasil kompilasi. Biasanya bersifat \textit{read-only} untuk mencegah perubahan kode secara tidak sengaja.
    \item \textbf{Data Segment (Initialized)}: Menyimpan variabel global dan statis yang sudah diberi nilai awal (misal: \texttt{int x = 10;}).
    \item \textbf{BSS (Block Started by Symbol)}: Menyimpan variabel global dan statis yang belum diinisialisasi (misal: \texttt{int y;}). Segmen ini biasanya diisi dengan nol oleh sistem operasi saat program dimuat.
    \item \textbf{Shared Libraries / Memory Mapping}: Area di antara stack dan heap yang digunakan untuk memetakan \textit{shared objects} (.so atau .dll) dan file yang dipetakan ke memori.
    \item \textbf{Stack}: Segmen yang tumbuh ke bawah untuk mengelola pemanggilan fungsi dan variabel lokal secara LIFO.
    \item \textbf{Heap}: Segmen yang tumbuh ke atas untuk alokasi memori dinamis lewat instruksi seperti \texttt{malloc} atau \texttt{new}.
\end{itemize}

\begin{figure}[!htbp]
    \centering
    \adjustbox{max width=0.6\textwidth,center}{%
    \begin{tikzpicture}[
        seg/.style={rectangle, draw=blue!50, fill=blue!10, text width=4cm, minimum height=0.6cm, font=\tiny, align=center}
    ]
    \node[seg] (s) {Stack (↓)};
    \node[seg, below=0.5cm of s, fill=gray!20] (sh) {Shared Libraries / Mmap};
    \node[seg, below=0.5cm of sh] (h) {Heap (↑)};
    \node[seg, below=0.2cm of h] (b) {BSS (Uninitialized Data)};
    \node[seg, below=0.2cm of b] (d) {Data (Initialized Data)};
    \node[seg, below=0.2cm of d] (t) {Text (Code)};
    \end{tikzpicture}%
    }
    \caption{Model organisasi memori runtime yang diperluas}
\end{figure}
