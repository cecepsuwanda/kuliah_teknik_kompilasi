\section{Layout Memori dan Segmentasi}

\compiler{Runtime Environment} mengelola bagaimana program menggunakan memori saat dieksekusi. Manajemen memori saat runtime merupakan salah satu aspek yang paling krusial dalam performa dan keamanan sebuah program hasil kompilasi \cite{ucsd2024compiler}.

\subsection{Segmen Memori Utama}
\begin{itemize}
    \item \textbf{Text/Code Segment}: Menyimpan instruksi biner hasil kompilasi. Biasanya bersifat \textit{read-only}.
    \item \textbf{Data Segment}: Menyimpan variabel global dan statis yang sudah diinisialisasi.
    \item \textbf{Stack}: Segmen yang tumbuh ke bawah untuk mengelola pemanggilan fungsi dan variabel lokal.
    \item \textbf{Heap}: Segmen yang tumbuh ke atas untuk alokasi memori dinamis (\textit{runtime allocation}).
\end{itemize}

\begin{figure}[!htbp]
    \centering
    \adjustbox{max width=0.6\textwidth,center}{%
    \begin{tikzpicture}[
        seg/.style={rectangle, draw=blue!50, fill=blue!10, text width=3cm, minimum height=0.6cm, font=\tiny, align=center}
    ]
    \node[seg] (s) {Stack (↓)};
    \node[seg, below=1cm of s] (h) {Heap (↑)};
    \node[seg, below=0.2cm of h] (d) {Data};
    \node[seg, below=0.2cm of d] (t) {Text};
    \end{tikzpicture}%
    }
    \caption{Model konseptual organisasi memori runtime}
\end{figure}
