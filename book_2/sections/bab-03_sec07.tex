\section{Testing Lexer}

Unit testing sangat penting untuk memastikan lexer bekerja dengan benar. Berikut contoh test cases:

\subsection{Test Cases untuk Identifier dan Keyword}

\begin{lstlisting}[language=C++, caption=Test Cases: Identifiers dan Keywords]
void testIdentifiers() {
    Lexer lexer("int x = 42;");
    Token t1 = lexer.nextToken(); // Should be KEYWORD_INT
    Token t2 = lexer.nextToken(); // Should be IDENTIFIER "x"
    Token t3 = lexer.nextToken(); // Should be OP_ASSIGN
    // ...
}
\end{lstlisting}

\subsection{Test Cases untuk Numbers}

\begin{itemize}
    \item \texttt{42} → INTEGER\_LITERAL
    \item \texttt{3.14} → FLOAT\_LITERAL
    \item \texttt{123.456} → FLOAT\_LITERAL
    \item \texttt{0} → INTEGER\_LITERAL
\end{itemize}

\subsection{Test Cases untuk Strings}

\begin{itemize}
    \item \texttt{"hello"} → STRING\_LITERAL dengan value "hello"
    \item \texttt{"hello\textbackslash{}nworld"} → STRING\_LITERAL dengan escape sequence
    \item \texttt{"unclosed} → INVALID (unclosed string)
\end{itemize}

\subsection{Test Cases untuk Comments}

\begin{itemize}
    \item \texttt{// single line comment} $\rightarrow$ Di-skip, tidak menghasilkan token
    \item \texttt{/* multi-line comment */} $\rightarrow$ Di-skip
    \item \texttt{/* unclosed comment} → Error atau exception
\end{itemize}

Gambar \ref{fig:comment-handling} menunjukkan proses handling komentar dalam lexer.

\begin{figure}[!htbp]
    \centering
    \adjustbox{max width=0.85\textwidth,center}{%
    \begin{tikzpicture}[
        code/.style={rectangle, draw=gray!30, fill=gray!5, text width=6cm, minimum height=0.5cm, font=\footnotesize\ttfamily, align=left, inner sep=4pt},
        arrow/.style={->, >=stealth, thick, blue},
        result/.style={rectangle, draw=green!50, fill=green!10, minimum width=2.5cm, minimum height=0.6cm, font=\tiny, align=center},
        node distance=0.4cm
    ]
    
    % Single line comment
    \node[code] (code1) {int x = 10; // This is a comment};
    \node[result, below=0.3cm of code1] (r1) {Token: int, x, =, 10, ;\\Comment skipped};
    \draw[arrow] (code1) -- (r1);
    
    % Multi-line comment
    \node[code, below=1cm of code1] (code2) {int y = 20; /* Multi\\line comment */ int z = 30;};
    \node[result, below=0.3cm of code2] (r2) {Token: int, y, =, 20, ;\\int, z, =, 30, ;\\Comment skipped};
    \draw[arrow] (code2) -- (r2);
    
    \end{tikzpicture}%
    }
    \caption{Handling komentar dalam lexer}
    \label{fig:comment-handling}
\end{figure}