\section{GLR Parsing (Generalized LR)}

\subsection{Konsep GLR}

GLR (Generalized LR) adalah ekstensi dari LR parsing yang dapat menangani ambiguous grammar atau grammar yang akan menghasilkan conflict dalam tabel LR biasa.

Menurut Wikipedia:

\begin{quote}
``GLR extends LR parsing to handle ambiguous grammars or grammars that would cause conflicts in LR tables. It allows multiple possible parse actions in a state and pursues them in parallel.''\footnote{\url{https://en.wikipedia.org/wiki/GLR_parser}}
\end{quote}

GLR parser menjaga multiple stacks atau parse trees aktif secara bersamaan ketika terjadi conflict, dan merge stack prefixes yang mungkin untuk berbagi pekerjaan.

\subsection{Kapan Menggunakan GLR}

GLR parsing berguna untuk:
\begin{itemize}
    \item Grammar yang ambiguous (memiliki multiple parse trees valid)
    \item Grammar yang tidak LR(1) tetapi masih ingin di-parse secara deterministik
    \item Bahasa dengan syntax yang extensible
    \item Natural language processing
\end{itemize}