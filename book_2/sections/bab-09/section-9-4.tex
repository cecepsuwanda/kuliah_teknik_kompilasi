\section{Available Expressions dan CSE}

Ekspresi $x + y$ disebut \compiler{Available} di titik $p$ jika sudah pernah dihitung sebelumnya dan nilai $x$ maupun $y$ belum berubah sejak penghitungan tersebut.

\subsection{Common Subexpression Elimination (CSE)}
Jika sebuah ekspresi sudah \textit{available}, kompilator akan mengganti perhitungan ulang ekspresi tersebut dengan nilai yang sudah ada di variabel perantara.

\begin{lstlisting}
// Sebelum CSE
t1 = a + b
t2 = c * d
t3 = a + b  // Perhitungan berulang

// Sesudah CSE
t1 = a + b
t2 = c * d
t3 = t1
\end{lstlisting}

\begin{figure}[!htbp]
    \centering
    \adjustbox{max width=0.8\textwidth,center}{%
    \begin{tikzpicture}[
        node/.style={rectangle, draw=blue!50, fill=blue!10, font=\tiny, align=center},
        arrow/.style={->, >=stealth, thick}
    ]
    \node[node] (e1) {expr: a+b};
    \node[node, right=1.5cm of e1] (e2) {use cached result};
    \draw[arrow] (e1) -- (e2);
    \end{tikzpicture}%
    }
    \caption{Konsep eliminasi sub-ekspresi umum}
\end{figure}
