\section{Pipeline dan Interaksi Optimasi}

Optimasi kompilator tidak berjalan sekali saja, melainkan dalam beberapa lintasan (\textit{passes}). Satu optimasi seringkali menjadi pemicu untuk optimasi lainnya.

\subsection{Optimization Loop}
Karena satu perubahan sering memicu peluang baru, optimizer biasanya bekerja dalam sebuah loop iteratif:
\begin{verbatim}
while (terjadi perubahan):
    lakukan constant folding
    lakukan copy propagation
    lakukan dead code elimination
\end{verbatim}

\subsection{Trade-off}
Optimasi yang lebih agresif membutuhkan waktu kompilasi yang lebih lama. Kompilator modern menyediakan level optimasi (seperti \texttt{-O1}, \texttt{-O2}, \texttt{-O3} pada GCC) untuk memberikan fleksibilitas bagi pemrogram.
