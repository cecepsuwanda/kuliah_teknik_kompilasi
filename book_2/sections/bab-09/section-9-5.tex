\section{Global Constant Propagation}

Jika pada optimasi lokal (Bab 8) kita hanya melihat penyebaran konstanta di dalam blok, \compiler{Global Constant Propagation} menyebarkan nilai konstanta melalui percabangan dan loop di seluruh program.

\subsection{Mekanisme Aliran Data}
Analisis ini menggunakan informasi dari \textit{Reaching Definitions}. Jika di titik $p$, hanya terdapat satu definisi yang mencapai variabel $x$, dan definisi tersebut berupa konstanta (\code{x = 5}), maka semua penggunaan $x$ di titik $p$ dapat diganti dengan angka \texttt{5}.

\subsection{Analisis Nilai Konstan}
Beberapa alasan mengapa ini lebih kuat daripada versi lokal:
\begin{enumerate}
    \item \textbf{Branch Pruning}: Jika kondisi \texttt{if (x > 0)} dievaluasi mendapati $x$ selalu bernilai $10$, maka blok \textit{else} dapat dihapus sepenuhnya (\textit{Unreachable Code Elimination}).
    \item \textbf{Loop Invariant}: Membantu mengenali nilai yang tidak berubah selama iterasi loop.
\end{enumerate}

\begin{lstlisting}[language=C++]
void contohGlobal() {
    int x = 7;
    if (kondisi) {
        // x masih mencapai sini sebagai 7
    } else {
        // x masih mencapai sini sebagai 7
    }
    int y = x + 3; // Global Const Prop: y = 7 + 3 = 10
}
\end{lstlisting}
