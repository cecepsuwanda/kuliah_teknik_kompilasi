\section{Kerangka Kerja Data-Flow Analysis}

\compiler{Data-Flow Analysis (DFA)} adalah proses pengumpulan informasi tentang aliran data melalui graf kendali alir (\textit{Control Flow Graph}) \cite{nguyen2024semantic}. Informasi ini digunakan untuk menjawab pertanyaan global seperti: "Apakah variabel $x$ pasti memiliki nilai 10 di baris ini?" atau "Apakah nilai $y$ akan dibaca lagi di masa depan?".

\subsection{Iterative Data-Flow Framework}
Untuk menyelesaikan masalah DFA secara sistematis, kita menggunakan kerangka kerja iteratif. Setiap algoritma DFA dapat didefinisikan dengan empat komponen utama:
\begin{enumerate}
    \item \textbf{Direction}: Arah aliran (\textit{Forward} atau \textit{Backward}).
    \item \textbf{Domain}: Himpunan nilai yang dianalisis (misal: himpunan definisi variabel atau ekspresi).
    \item \textbf{Transfer Function} ($f_B$): Aturan bagaimana sebuah blok mengubah informasi data-flow. Format umumnya: $OUT[B] = f_B(IN[B])$.
    \item \textbf{Meet operator} ($\wedge$): Aturan penggabungan informasi ketika dua atau lebih jalur dalam CFG bertemu (biasanya berupa \textit{Union} $\cup$ atau \textit{Intersection} $\cap$).
\end{enumerate}

\subsection{Representasi Bit-Vector}
Agar proses DFA berjalan cepat, kompilator biasanya menggunakan representasi \textbf{Bit-Vector}. Setiap elemen dalam domain direpresentasikan oleh satu bit dalam sebuah \textit{array of bits}. 
\begin{itemize}
    \item Bit $1$ berarti properti tersebut \textit{true} atau ada dalam set.
    \item Bit $0$ berarti \textit{false} atau tidak ada.
\end{itemize}
Dengan bit-vector, operasi set seperti \texttt{Union} dapat dilakukan menggunakan instruksi bitwise \texttt{OR} ($|$), dan \texttt{Intersection} menggunakan bitwise \texttt{AND} ($\&$), yang didukung sangat cepat oleh perangkat keras CPU.

\begin{figure}[!htbp]
    \centering
    \adjustbox{max width=0.8\textwidth,center}{%
    \begin{tikzpicture}[
        node/.style={rectangle, draw=blue!50, fill=blue!10, font=\small, align=center, rounded corners, minimum height=0.8cm},
        arrow/.style={->, >=stealth, thick}
    ]
    \node[node] (in) {IN[B]};
    \node[node, right=1.5cm of in] (bb) {Basic Block $B$};
    \node[node, right=1.5cm of bb] (out) {OUT[B]};
    \draw[arrow] (in) -- (bb);
    \draw[arrow] (bb) -- node[above, font=\footnotesize] {$f_B$} (out);
    \end{tikzpicture}%
    }
    \caption{Model Aliran Data pada Satu Basic Block}
\end{figure}
