\section{Kompilator Modern: Multi-Pass vs Single-Pass}

Setelah memahami arsitektur dan fase-fase kompilasi, penting untuk mengetahui bahwa kompilator modern umumnya menggunakan pendekatan \textbf{multi-pass}, di mana setiap fase dijalankan dalam pass terpisah. Keuntungannya:
\begin{itemize}
    \item Memisahkan concern (tiap fase fokus pada tugas spesifik)
    \item Memudahkan maintenance dan debugging
    \item Memungkinkan optimasi yang lebih kompleks
\end{itemize}

Sebaliknya, kompilator \textbf{single-pass} mencoba menyelesaikan semua fase dalam satu pass. Pendekatan ini lebih cepat tetapi lebih sulit diimplementasikan dan terbatas dalam optimasi.

Gambar \ref{fig:multipass-vs-singlepass} menunjukkan perbedaan antara kedua pendekatan.

\begin{figure}[H]
    \centering
    \adjustbox{max width=0.85\textwidth,center}{%
    \begin{tikzpicture}[
        box/.style={rectangle, draw=blue!50, fill=blue!10, text width=2.5cm, text centered, minimum height=0.7cm, rounded corners, font=\footnotesize, inner sep=4pt},
        bigbox/.style={rectangle, draw=blue!50, fill=blue!10, text width=2.8cm, text centered, minimum height=3.6cm, rounded corners, font=\footnotesize, inner sep=6pt},
        arrow/.style={->, >=stealth, thick},
        title/.style={font=\bfseries\small},
        node distance=0.5cm and 3.5cm
    ]
    
    % ===== Multi-pass =====
    \node[title] (mp-title) {MULTI-PASS};
    
    \node[box, below=of mp-title] (mp1) {Pass 1: Lexical};
    \node[box, below=of mp1] (mp2) {Pass 2: Syntax};
    \node[box, below=of mp2] (mp3) {Pass 3: Semantic};
    \node[box, below=of mp3] (mp4) {Pass 4: IR Generation};
    \node[box, below=of mp4] (mp5) {Pass 5: Optimization};
    \node[box, below=of mp5] (mp6) {Pass 6: Code Generation};
    
    \draw[arrow] (mp1) -- (mp2);
    \draw[arrow] (mp2) -- (mp3);
    \draw[arrow] (mp3) -- (mp4);
    \draw[arrow] (mp4) -- (mp5);
    \draw[arrow] (mp5) -- (mp6);
    
    \node[below=0.3cm of mp6, font=\tiny, text width=2.8cm, align=center]
    (mp-note) {Lebih modular\\Lebih mudah dioptimasi};
    
    % ===== Single-pass =====
    \node[title, right=of mp-title] (sp-title) {SINGLE-PASS};
    
    \node[bigbox, below=of sp-title] (sp-all)
    {Semua fase\\dalam satu pass};
    
    \node[below=0.3cm of sp-all, font=\tiny, text width=2.8cm, align=center]
    (sp-note) {Lebih cepat\\Lebih sulit dikembangkan\\Optimasi terbatas};
    
    \end{tikzpicture}%
    }
    \caption{Perbandingan arsitektur Multi-Pass dan Single-Pass Compiler}
    \label{fig:multipass-vs-singlepass}
    \end{figure}
    
