\section{Portofolio Proyek}

Portofolio bukan hanya sekadar kumpulan kode, melainkan bukti kemampuan rekayasa (\textit{engineering capability}) Anda di hadapan dunia profesional atau akademis.

\subsection{Struktur Portofolio Teknis}
Untuk proyek \compiler{Subset C}, portofolio Anda sebaiknya mencakup elemen mendalam berikut:
\begin{enumerate}
  \item \textbf{Analisis Kedalaman Teknis}: Jelaskan satu algoritma kompleks yang Anda implementasikan (misalnya: \textit{Register Coloring} atau \textit{Liveness Analysis}). Mengapa algoritma itu dipilih?
  \item \textbf{Benchmarking Performa}: Sertakan grafik perbandingan antara biner yang dihasilkan dengan optimasi \textit{-O0} vs \textit{-O3}. Jelaskan mengapa biner yang satu lebih cepat (misal: karena berkurangnya instruksi memori).
  \item \textbf{Dokumentasi Arsitektur}: Gunakan diagram blok untuk menunjukkan bagaimana data mengalir dari \textit{Flex}, ke \textit{Bison}, melalui penganalisis semantik, hingga ke generator kode LLVM/Assembly.
  \item \textbf{Metodologi Pengujian}: Tunjukkan suite pengujian otomatis Anda, termasuk kasus uji untuk (\textit{Edge Cases}) seperti pembagian dengan nol atau \textit{scoping} yang tumpang tindih.
\end{enumerate}

\subsection{Kriteria Penilaian Portofolio}

\begin{table}[h]
\centering
\begin{tabular}{|l|c|}
\hline
\textbf{Aspek} & \textbf{Bobot} \\
\hline
Koreksi Output (Kepatuhan terhadap Spesifikasi) & 30\% \\
Kualitas Kode (Modularitas, Kerapuhan, Dokumentasi) & 20\% \\
Analisis Performa (Data Benchmarking Nyata) & 25\% \\
Kedalaman Refleksi (Pemahaman atas Keputusan Desain) & 25\% \\
\hline
\end{tabular}
\caption{Kriteria Penilaian Portofolio Akhir}
\end{table}
