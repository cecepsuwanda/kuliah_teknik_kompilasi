\section{Portofolio Proyek}

\subsection{Struktur Portofolio}

Portofolio proyek kompilator harus mencakup:

\begin{enumerate}
  \item \textbf{Deskripsi Proyek}: Tujuan, scope, dan requirements
  \item \textbf{Arsitektur}: Desain komponen dan interaksi
  \item \textbf{Implementasi}: Code snippets dan penjelasan
  \item \textbf{Testing}: Test cases dan hasil
  \item \textbf{Evaluasi}: Performance analysis dan benchmarking
  \item \textbf{Refleksi}: Pembelajaran dan improvement
\end{enumerate}

\subsection{Kriteria Penilaian Portofolio}

\begin{table}[h]
\centering
\begin{tabular}{|l|c|}
\hline
\textbf{Aspek} & \textbf{Bobot} \\
\hline
Koreksi dan kelengkapan & 20\% \\
Kualitas implementasi & 25\% \\
Pengujian dan validasi & 20\% \\
Dokumentasi & 15\% \\
Presentasi & 10\% \\
Refleksi dan pembelajaran & 10\% \\
\hline
\end{tabular}
\caption{Kriteria Penilaian Portofolio}
\end{table}
