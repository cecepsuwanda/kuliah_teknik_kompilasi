\section{Kesimpulan dan Langkah Selanjutnya}

\subsection{Pencapaian Dasar}
Teknik Kompilasi telah memberikan Anda fondasi untuk memahami bagaimana abstraksi tingkat tinggi diterjemahkan menjadi realitas fisik di dalam prosesor. Kemampuan ini adalah "kekuatan super" bagi pengembang perangkat lunak karena Anda tidak lagi melihat kompilator sebagai Kotak Hitam (\textit{Black Box}).

\subsection{Rekomendasi Lanjutan dan Tren Masa Depan}
Ekosistem pengembangan kompilator berkembang sangat pesat. Beberapa area yang layak untuk Anda jelajahi lebih lanjut meliputi:
\begin{itemize}
  \item \textbf{AI dalam Kompilator}: Penggunaan \textit{Large Language Models} (LLM) untuk membantu pembuatan kode otomatis, \textit{bug fixing}, dan penemuan strategi optimasi baru yang tidak terpikirkan oleh algoritma heuristik tradisional.
  \item \textbf{WebAssembly (WASM)}: Memahami bagaimana kompilator memungkinkan bahasa seperti C++ atau Rust berjalan di dalam peramban web dengan kecepatan mendekati aplikasi asli (\textit{near-native}).
  \item \textbf{GraalVM dan JIT modern}: Mempelajari bagaimana kompilator dapat melakukan optimasi dinamis saat program SEDANG berjalan untuk hasil yang bahkan lebih baik daripada kompilasi statis.
  \item \textbf{Verified Compilers (Contoh: CompCert)}: Bagi Anda yang tertarik pada keamanan kritis (\textit{mission-critical}), pelajari bagaimana kompilator dapat dibuktikan secara matematis tidak akan pernah menghasilkan kode biner yang salah atau tidak sesuai dengan spesifikasi aslinya.
\end{itemize}

\textit{"The best way to learn compiler construction is to build a compiler."} - Andrew Appel. Semoga perjalanan Anda di dunia \textit{Computer Science} semakin cemerlang setelah menaklukkan mata kuliah ini.
