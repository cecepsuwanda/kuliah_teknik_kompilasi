\section{Kesimpulan dan Rekomendasi}

\subsection{Pencapaian Pembelajaran}

Setelah menyelesaikan mata kuliah Teknik Kompilasi, mahasiswa diharapkan:

\begin{itemize}
  \item Memahami prinsip-prinsip fundamental kompilasi
  \item Mampu mengimplementasikan compiler components
  \item Dapat menggunakan tools modern untuk pengembangan compiler
  \item Memiliki kemampuan analisis dan evaluasi performance
  \item Siap untuk pengembangan software yang lebih kompleks
\end{itemize}

\subsection{Rekomendasi Lanjutan}

\begin{itemize}
  \item \textbf{Study Lanjut}: Advanced compiler design, optimization techniques
  \item \textbf{Aplikasi Praktis}: Bahasa programming, system programming
  \item \textbf{Research}: Compiler optimization, language design
  \item \textbf{Industry}: Compiler development, tool development
\end{itemize}

\subsection{Final Reflection}

Teknik Kompilasi adalah mata kuliah fundamental yang menghubungkan teori computer science dengan implementasi praktis. Pemahaman yang baik tentang kompilator memberikan fondasi kuat untuk pengembangan software yang efisien dan optimal.

\textit{"The best way to learn compiler construction is to build a compiler."} - Andrew Appel
