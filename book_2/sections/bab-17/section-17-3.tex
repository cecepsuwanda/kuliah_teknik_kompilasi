\section{Refleksi Pembelajaran}

\subsection{Pertanyaan Reflektif}

\textbf{Bagian 1: Pemahaman Konsep}
\begin{enumerate}
  \item Konsep kompilator mana yang paling menant bagi Anda? Mengapa?
  \item Bagaimana pemahaman Anda tentang kompilator berkembang selama semester ini?
  \item Konsep mana yang paling relevan dengan pengembangan software modern?
\end{enumerate}

\textbf{Bagian 2: Keterampilan Praktis}
\begin{enumerate}
  \item Implementasi mana yang memberikan insight terbesar tentang kompilasi?
  \item Kesulitan teknis apa yang Anda hadapi dan bagaimana mengatasinya?
  \item Keterampilan mana yang paling berharga untuk karir Anda?
\end{enumerate}

\textbf{Bagian 3: Pengembangan Diri}
\begin{enumerate}
  \item Bagaimana mata kuliah ini mengubah cara Anda berpikir tentang programming?
  \item Topik mana yang ingin Anda pelajari lebih dalam?
  \item Bagaimana Anda akan menerapkan pengetahuan ini di proyek masa depan?
\end{enumerate}

\subsection{Self-Assessment Checklist}

\begin{table}[h]
\centering
\small
\begin{tabular}{|l|c|c|c|}
\hline
\textbf{Kompetensi} & \textbf{Belum} & \textbf{Sedang} & \textbf{Menguasai} \\
\hline
Memahami fase kompilasi & $\square$ & $\square$ & $\square$ \\
Mengimplementasikan lexer & $\square$ & $\square$ & $\square$ \\
Membangun parser & $\square$ & $\square$ & $\square$ \\
Melakukan semantic analysis & $\square$ & $\square$ & $\square$ \\
Menghasilkan intermediate code & $\square$ & $\square$ & $\square$ \\
Melakukan optimasi & $\square$ & $\square$ & $\square$ \\
Menghasilkan target code & $\square$ & $\square$ & $\square$ \\
Menggunakan compiler tools & $\square$ & $\square$ & $\square$ \\
\hline
\end{tabular}
\caption{Self-Assessment Kompetensi}
\end{table}
