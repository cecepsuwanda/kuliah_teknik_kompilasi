\section{Refleksi Pembelajaran}

\subsection{Pertanyaan Reflektif Teknis}

\textbf{Bagian 1: Arsitektur dan Trade-offs}
\begin{enumerate}
  \item Setelah mempelajari interpretasi dan kompilasi, dalam skenario apa Anda akan memilih \textit{Just-In-Time} (JIT) compiler dibandingkan kompilasi AOT (\textit{Ahead-of-Time})?
  \item Analisis \textit{trade-off} antara kecepatan kompilasi (\code{-O0}) dan performa waktu eksekusi (\code{-O3}). Kapan kecepatan kompilasi menjadi lebih penting bagi pengembang?
  \item Mengapa transisi dari komponen buatan tangan (\textit{hand-written}) ke generator otomatis (seperti Flex/Bison) dianggap sebagai langkah krusial dalam produktivitas pengembangan kompilator?
\end{enumerate}

\textbf{Bagian 2: Keterampilan Praktis dan Implementasi}
\begin{enumerate}
  \item Implementasi fase mana (Lexer, Parser, atau Semantik) yang memberikan wawasan terdalam bagi Anda tentang cara kerja bahasa pemrograman?
  \item Bagaimana pengalaman menangani kesalahan (\textit{error handling}) di tingkat parser mengubah cara Anda menulis kode yang "aman" secara umum?
  \item Hubungan apa yang paling berharga antara teori \textit{Computer Science} (seperti Finite Automata) dengan praktik rekayasa yang Anda lakukan di laboratorium?
\end{enumerate}

\textbf{Bagian 3: Pandangan Visioner}
\begin{enumerate}
  \item Bagaimana pemahaman tentang representasi intermediet (IR) seperti SSA membantu Anda memahami cara kerja alat bantu modern seperti \textit{Linters} atau \textit{Static Analyzers}?
  \item Apa manfaat terbesar dari mempelajari manajemen stack dan register bagi Anda saat melakukan \textit{debugging} aplikasi tingkat sistem?
\end{enumerate}

\subsection{Self-Assessment Checklist Kompetensi}

\begin{table}[h]
\centering
\small
\begin{tabularx}{\textwidth}{|X|c|c|c|}
\hline
\textbf{Kompetensi Teknis} & \textbf{Belum} & \textbf{Sedang} & \textbf{Menguasai} \\
\hline
Memahami siklus hidup instruksi dari sumber ke biner & $\square$ & $\square$ & $\square$ \\
Membangun Parser yang mampu melakukan \textit{error recovery} & $\square$ & $\square$ & $\square$ \\
Melakukan \textit{Static Type Checking} pada AST & $\square$ & $\square$ & $\square$ \\
Menghasilkan IR dan menerapkan \textit{Optimization Passes} & $\square$ & $\square$ & $\square$ \\
Menganalisis performa biner menggunakan \code{perf} atau \code{valgrind} & $\square$ & $\square$ & $\square$ \\
\hline
\end{tabularx}
\caption{Self-Assessment Kompetensi Akhir Semester}
\end{table}
