\section{Struktur Data AST}

AST terdiri dari node-node yang merepresentasikan berbagai konstruk bahasa. Setiap node memiliki:
\begin{itemize}
    \item \textbf{Node Type}: Jenis node (expression, statement, declaration, dll.)
    \item \textbf{Children}: Node-node anak (untuk operasi biner, blok statement, dll.)
    \item \textbf{Attributes}: Informasi tambahan (nilai literal, nama identifier, operator, dll.)
    \item \textbf{Location}: Informasi posisi dalam source code (untuk error reporting)
\end{itemize}

\subsection{Node Types dalam AST}

Node types yang umum digunakan dalam AST:

\subsubsection{Expression Nodes}

\begin{itemize}
    \item \textbf{Literal Nodes}: Integer, float, string, boolean, null
    \item \textbf{Identifier Nodes}: Nama variabel, fungsi, tipe
    \item \textbf{Binary Expression Nodes}: Operasi biner (+, -, *, /, ==, !=, <, >, dll.)
    \item \textbf{Unary Expression Nodes}: Operasi unary (negation, logical NOT, address-of, dereference)
    \item \textbf{Function Call Nodes}: Pemanggilan fungsi dengan daftar argumen
    \item \textbf{Array Access Nodes}: Akses elemen array dengan index
    \item \textbf{Member Access Nodes}: Akses member struct/class (dot notation)
\end{itemize}

\subsubsection{Statement Nodes}

\begin{itemize}
    \item \textbf{Expression Statement}: Statement yang merupakan ekspresi (dengan semicolon)
    \item \textbf{Variable Declaration}: Deklarasi variabel dengan tipe dan optional initializer
    \item \textbf{Assignment Statement}: Assignment ke variabel atau l-value lainnya
    \item \textbf{If Statement}: Conditional statement dengan condition, then-branch, optional else-branch
    \item \textbf{While Statement}: Loop dengan condition dan body
    \item \textbf{For Statement}: Loop dengan init, condition, increment, dan body
    \item \textbf{Return Statement}: Return dengan optional expression
    \item \textbf{Block Statement}: Blok statement yang berisi daftar statement
    \item \textbf{Break/Continue Statement}: Control flow untuk loop
\end{itemize}

\subsubsection{Declaration Nodes}

\begin{itemize}
    \item \textbf{Function Declaration}: Deklarasi fungsi dengan parameter, return type, dan body
    \item \textbf{Type Declaration}: Deklarasi tipe baru (struct, enum, typedef)
    \item \textbf{Variable Declaration}: Deklarasi variabel global atau local
\end{itemize}

\subsubsection{Program Node}

\begin{itemize}
    \item \textbf{Program/Module Node}: Root node yang berisi semua deklarasi dan statement dalam program
\end{itemize}