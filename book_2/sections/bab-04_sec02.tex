\section{Pendahuluan}

Pada bab ini kita menambah \textbf{lexer proyek compiler subset C} menggunakan Flex. Spesifikasi token proyek telah didefinisikan di Bab 1 (Bagian~\ref{sec:spec-subset-c}); file \texttt{simplec.l} yang dibangun di bab ini (dan dipakai bersama parser di Bab 8) mengimplementasikan token set tersebut. Contoh lengkap \texttt{simplec.l} terintegrasi dengan parser \texttt{simplec.y} di Bab 8; token yang dihasilkan mengikuti spesifikasi Bab 1. Kode hasil generate menjadi bagian dari codebase proyek di folder \texttt{proyek-compiler-subset-c/}.

Pada bab sebelumnya, kita telah mempelajari implementasi hand-written lexer. Meskipun pendekatan tersebut memberikan kontrol penuh dan pemahaman mendalam, dalam praktik industri, penggunaan \textbf{lexer generator} lebih umum karena efisiensi dan kemudahan maintenance. Menurut sumber terbuka:

\begin{quote}
``re2c is a high-performance lexer generator for C/C++ that takes regex specifications and builds deterministic finite automata. It's used in real projects.''\cite{wikipedia2024re2c}
\end{quote}

Lexer generator adalah tools yang menerima specification file (berisi pattern dan action) dan menghasilkan kode lexer yang siap digunakan. Dua generator populer untuk C/C++ adalah \textbf{Flex} (Fast Lexical Analyzer) dan \textbf{re2c} (Regular Expressions to Code).

Gambar \ref{fig:lexer-generator-overview} menunjukkan alur kerja lexer generator secara umum.

\begin{figure}[!htbp]
    \centering
    \adjustbox{max width=0.9\textwidth,center}{%
    \begin{tikzpicture}[
        box/.style={rectangle, draw=blue!50, fill=blue!10, text width=2.5cm, text centered, minimum height=0.7cm, rounded corners, font=\footnotesize, inner sep=4pt, align=center},
        arrow/.style={->, >=stealth, thick},
        node distance=1.5cm
    ]
    
    \node[box] (spec) {Specification\\File};
    \node[box, right=of spec] (gen) {Lexer\\Generator};
    \node[box, right=of gen] (code) {Generated\\Lexer Code};
    \node[box, below=of code] (compile) {Compile};
    \node[box, left=of compile] (lexer) {Executable\\Lexer};
    
    \draw[arrow] (spec) -- node[above, font=\tiny] {Input} (gen);
    \draw[arrow] (gen) -- node[above, font=\tiny] {Generate} (code);
    \draw[arrow] (code) -- (compile);
    \draw[arrow] (compile) -- (lexer);
    
    \end{tikzpicture}%
    }
    \caption{Alur kerja lexer generator}
    \label{fig:lexer-generator-overview}
\end{figure}

Keuntungan menggunakan lexer generator:
\begin{itemize}
    \item \textbf{Produktivitas}: Lebih cepat dalam development karena tidak perlu menulis state machine manual
    \item \textbf{Maintainability}: Specification file lebih mudah dibaca dan dimodifikasi dibanding kode state machine
    \item \textbf{Optimasi Otomatis}: Generator menghasilkan kode yang sudah dioptimasi (DFA minimization, dll.)
    \item \textbf{Konsistensi}: Mengurangi bug karena generator sudah teruji
\end{itemize}