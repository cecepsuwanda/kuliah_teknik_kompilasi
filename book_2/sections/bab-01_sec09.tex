\section{Kesimpulan}

Dalam bab ini, kita telah mempelajari konsep dasar kompilator dan fase-fase kompilasi. Ringkasan materi yang telah dibahas:

\begin{enumerate}
    \item \textbf{Definisi Kompilator}: Kompilator adalah program yang menerjemahkan source code ke target code melalui beberapa fase yang terstruktur.
    
    \item \textbf{Perbedaan dengan Translator Lainnya}: Kompilator berbeda dengan interpreter dan translator lainnya (assembler, linker, loader, preprocessor, decompiler) dalam cara dan tujuan translasinya.
    
    \item \textbf{Arsitektur Kompilator}: Kompilator modern terdiri dari dua bagian utama:
    \begin{itemize}
        \item \textbf{Front-end (Analisis)}: Analisis Leksikal (Lexical Analysis), Analisis Sintaksis (Syntax Analysis), dan Analisis Semantik (Semantic Analysis)
        \item \textbf{Back-end (Sintesis)}: Generasi Kode Intermediate (Intermediate Code Generation), Optimasi Kode (Code Optimization), dan Generasi Kode (Code Generation)
    \end{itemize}
    
    \item \textbf{Enam Fase Utama Kompilasi}: Setiap fase memiliki peran spesifik dalam transformasi source code menjadi executable:
    \begin{itemize}
        \item Analisis Leksikal (Lexical Analysis): Tokenization
        \item Analisis Sintaksis (Syntax Analysis): Parsing dan pembangunan AST
        \item Analisis Semantik (Semantic Analysis): Type checking dan scope resolution
        \item Generasi Kode Intermediate (Intermediate Code Generation): Pembuatan IR
        \item Optimasi Kode (Code Optimization): Optimasi kode
        \item Generasi Kode (Code Generation): Generasi target code
    \end{itemize}
    
    \item \textbf{Alur Kerja Lengkap}: Proses dari source code hingga executable melibatkan preprocessing, enam fase kompilasi utama, assembling, dan linking.
    
    \item \textbf{Pendekatan Modern}: Kompilator modern menggunakan pendekatan multi-pass yang lebih modular dan memungkinkan optimasi yang lebih kompleks dibanding single-pass.
\end{enumerate}

Pemahaman terhadap arsitektur dan fase-fase kompilasi ini menjadi dasar penting untuk mempelajari implementasi praktis setiap fase dalam bab-bab selanjutnya. Setiap fase akan dibahas secara lebih mendalam dengan contoh implementasi menggunakan bahasa pemrograman C/C++.
