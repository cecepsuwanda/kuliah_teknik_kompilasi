\section{Parser Generator: Bison dan Yacc}

\subsection{Pengenalan Parser Generator}

Parser generator adalah tool yang secara otomatis menghasilkan parser dari specification grammar. Menurut sumber dari IT Trip:

\begin{quote}
``Bison / YACC: define grammar in a .y file, specify \%token s, grammar rules, actions, etc. Generates C parser (or C++ variants). Flex + Bison: use Flex to build the lexer (.l file), Bison for parser, integrate them via tokens.''\cite{ittrip2024bison}
\end{quote}

Keuntungan menggunakan parser generator:
\begin{itemize}
    \item Menghemat waktu development
    \item Mengurangi kemungkinan error
    \item Mudah di-maintain (ubah grammar, regenerate parser)
    \item Menghasilkan parser yang efisien
    \item Mendukung semantic actions untuk membangun AST
\end{itemize}

\subsection{Yacc (Yet Another Compiler Compiler)}

Yacc adalah parser generator yang dikembangkan di Bell Labs pada tahun 1970-an. Yacc menghasilkan LALR(1) parser dari grammar specification.

\subsection{Bison (GNU Yacc)}

Bison adalah implementasi open source dari Yacc yang dikembangkan oleh GNU Project. Bison lebih powerful dan memiliki fitur tambahan:
\begin{itemize}
    \item Mendukung LALR(1), LR(1), dan GLR parsing
    \item Mendukung C++ output
    \item Error recovery yang lebih baik
    \item Dokumentasi yang lebih lengkap
\end{itemize}

\subsection{Struktur File Bison (.y)}

File Bison memiliki struktur berikut:

\begin{verbatim}
%{
/* C/C++ code: includes, declarations */
%}

/* Bison declarations: tokens, types, precedence */
%token NUMBER IDENTIFIER
%left '+' '-'
%left '*' '/'

%%
/* Grammar rules */
expression:
    expression '+' term { /* semantic action */ }
    | term
    ;

term:
    term '*' factor { /* semantic action */ }
    | factor
    ;

factor:
    NUMBER { /* semantic action */ }
    | IDENTIFIER { /* semantic action */ }
    | '(' expression ')' { /* semantic action */ }
    ;

%%
/* User code: helper functions */
\end{verbatim}

\subsection{Integrasi Flex dan Bison}

Flex dan Bison dirancang untuk bekerja bersama:

\begin{enumerate}
    \item \textbf{Flex file (.l)}: Mendefinisikan token patterns
    \begin{verbatim}
    %{
    #include "parser.tab.h"  // Generated by Bison
    %}
    %%
    [0-9]+     { yylval = atoi(yytext); return NUMBER; }
    [a-zA-Z]+  { return IDENTIFIER; }
    \+         { return '+'; }
    \*         { return '*'; }
    %%
    \end{verbatim}

    \item \textbf{Bison file (.y)}: Mendefinisikan grammar dan semantic actions
    \begin{verbatim}
    %token NUMBER IDENTIFIER
    %%
    expression: expression '+' term | term;
    term: term '*' factor | factor;
    factor: NUMBER | IDENTIFIER | '(' expression ')';
    %%
    \end{verbatim}

    \item \textbf{Compilation}: 
    \begin{verbatim}
    flex lexer.l
    bison -d parser.y
    gcc lex.yy.c parser.tab.c -o parser
    \end{verbatim}
\end{enumerate}

\subsection{Semantic Actions}

Semantic actions adalah kode C/C++ yang dieksekusi ketika production rule di-reduce. Actions dapat:
\begin{itemize}
    \item Membangun AST nodes
    \item Mengevaluasi ekspresi
    \item Memvalidasi semantik
    \item Menghasilkan output
\end{itemize}

Contoh semantic action untuk membangun AST:

\begin{verbatim}
expression:
    expression '+' term 
    { 
        $$ = create_binary_op(PLUS, $1, $3); 
    }
    | term 
    { 
        $$ = $1; 
    }
    ;
\end{verbatim}

Di mana:
\begin{itemize}
    \item \texttt{\$\$}: Nilai yang dihasilkan oleh production (LHS)
    \item \texttt{\$1, \$2, ...}: Nilai dari simbol-simbol di RHS
\end{itemize}

\subsection{Error Handling dalam Bison}

Bison menyediakan mekanisme error handling:

\begin{verbatim}
%error-verbose  // Better error messages

expression:
    expression '+' term
    | error '+' term  // Error recovery: skip until '+'
    | term
    ;
\end{verbatim}

Error recovery rules memungkinkan parser untuk:
\begin{itemize}
    \item Mendeteksi error
    \item Melakukan recovery (skip tokens sampai synchronization point)
    \item Melanjutkan parsing
    \item Menghasilkan multiple error messages
\end{itemize}