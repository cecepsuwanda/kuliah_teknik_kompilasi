\section{Derivation}

Derivation adalah proses menerapkan aturan produksi untuk menghasilkan string terminal dari start symbol. Terdapat dua jenis derivation yang penting:

\subsection{Leftmost Derivation}

Leftmost derivation selalu mengganti nonterminal paling kiri terlebih dahulu pada setiap langkah.

Contoh untuk ekspresi \texttt{3 + 4 * 5} dengan grammar:
\begin{verbatim}
E → E + T | E - T | T
T → T * F | T / F | F
F → ( E ) | number
\end{verbatim}

Leftmost derivation:
\begin{align*}
E &\Rightarrow E + T \\
  &\Rightarrow T + T \\
  &\Rightarrow F + T \\
  &\Rightarrow \texttt{3} + T \\
  &\Rightarrow \texttt{3} + T * F \\
  &\Rightarrow \texttt{3} + F * F \\
  &\Rightarrow \texttt{3} + \texttt{4} * F \\
  &\Rightarrow \texttt{3} + \texttt{4} * \texttt{5}
\end{align*}

\subsection{Rightmost Derivation}

Rightmost derivation selalu mengganti nonterminal paling kanan terlebih dahulu pada setiap langkah.

Rightmost derivation untuk \texttt{3 + 4 * 5}:
\begin{align*}
E &\Rightarrow E + T \\
  &\Rightarrow E + T * F \\
  &\Rightarrow E + T * \texttt{5} \\
  &\Rightarrow E + F * \texttt{5} \\
  &\Rightarrow E + \texttt{4} * \texttt{5} \\
  &\Rightarrow T + \texttt{4} * \texttt{5} \\
  &\Rightarrow F + \texttt{4} * \texttt{5} \\
  &\Rightarrow \texttt{3} + \texttt{4} * \texttt{5}
\end{align*}

Gambar \ref{fig:derivation-comparison} menunjukkan perbandingan leftmost dan rightmost derivation.

\begin{figure}[!htbp]
    \centering
    \adjustbox{max width=0.85\textwidth,center}{%
    \begin{tikzpicture}[
        derivstep/.style={rectangle, draw=blue!50, fill=blue!10, text width=4cm, minimum height=0.5cm, font=\tiny\ttfamily, align=left, inner sep=4pt, rounded corners},
        label/.style={font=\footnotesize\bfseries, align=center},
        arrow/.style={->, >=stealth, thick},
        node distance=0.3cm
    ]
    
    \node[label] (left-title) {Leftmost Derivation};
    \node[derivstep, below=of left-title] (l1) {E};
    \node[derivstep, below=of l1] (l2) {E + T};
    \node[derivstep, below=of l2] (l3) {T + T};
    \node[derivstep, below=of l3] (l4) {3 + 4 * 5};
    
    \node[label, right=4cm of left-title] (right-title) {Rightmost Derivation};
    \node[derivstep, below=of right-title] (r1) {E};
    \node[derivstep, below=of r1] (r2) {E + T};
    \node[derivstep, below=of r2] (r3) {E + T * F};
    \node[derivstep, below=of r3] (r4) {3 + 4 * 5};
    
    \draw[arrow] (l1) -- (l2);
    \draw[arrow] (l2) -- (l3);
    \draw[arrow] (l3) -- (l4);
    
    \draw[arrow] (r1) -- (r2);
    \draw[arrow] (r2) -- (r3);
    \draw[arrow] (r3) -- (r4);
    
    \end{tikzpicture}%
    }
    \caption{Perbandingan leftmost dan rightmost derivation}
    \label{fig:derivation-comparison}
\end{figure}

\subsection{Pentingnya Derivation}

Derivation penting karena:
\begin{itemize}
    \item Menunjukkan bagaimana string dihasilkan dari grammar
    \item Menentukan urutan penggantian nonterminal (penting untuk parsing)
    \item Leftmost derivation digunakan dalam top-down parsing
    \item Rightmost derivation digunakan dalam bottom-up parsing
\end{itemize}