\section{Contoh Penggunaan}

Berikut contoh penggunaan lexer untuk tokenize source code sederhana:

\begin{lstlisting}[language=C++, caption=Contoh Penggunaan Lexer]
#include "lexer.h"
#include <iostream>

int main() {
    std::string source = R"(
        int x = 42;
        float y = 3.14;
        if (x > 10) {
            return y;
        }
    )";
    
    Lexer lexer(source);
    std::vector<Token> tokens = lexer.tokenize();
    
    for (const auto& token : tokens) {
        std::cout << "Token: " << token.lexeme 
                  << " Type: " << static_cast<int>(token.type)
                  << " Line: " << token.line 
                  << " Column: " << token.column << std::endl;
    }
    
    return 0;
}
\end{lstlisting}

Output yang diharapkan:
\begin{verbatim}
Token: int Type: 1 Line: 2 Column: 9
Token: x Type: 0 Line: 2 Column: 13
Token: = Type: 13 Line: 2 Column: 15
Token: 42 Type: 8 Line: 2 Column: 17
Token: ; Type: 20 Line: 2 Column: 19
...
\end{verbatim}

Gambar \ref{fig:complete-example} menunjukkan contoh lengkap tokenization untuk program sederhana.

\begin{figure}[H]
    \centering
    \adjustbox{max width=0.9\textwidth,center}{%
    \begin{tikzpicture}[
        code/.style={rectangle, draw=gray!30, fill=gray!5, text width=7cm, minimum height=0.6cm, font=\footnotesize\ttfamily, align=left, inner sep=4pt},
        token/.style={rectangle, draw=blue!50, fill=blue!10, text width=1.3cm, minimum height=0.5cm, font=\tiny, align=center, inner sep=2pt},
        arrow/.style={->, >=stealth, thick, gray},
        node distance=0.3cm and 0.15cm
    ]
    
    % Source code
    \node[code] (source) {int x = 42; float y = 3.14;};
    
    % Tokens row 1
    \node[token, below=0.5cm of source, xshift=-3cm] (t1) {int\\KEYWORD};
    \node[token, right=of t1] (t2) {x\\IDENTIFIER};
    \node[token, right=of t2] (t3) {=\\OP\_ASSIGN};
    \node[token, right=of t3] (t4) {42\\INTEGER};
    \node[token, right=of t4] (t5) {;\\SEMICOLON};
    
    % Tokens row 2
    \node[token, below=0.3cm of t1] (t6) {float\\KEYWORD};
    \node[token, right=of t6] (t7) {y\\IDENTIFIER};
    \node[token, right=of t7] (t8) {=\\OP\_ASSIGN};
    \node[token, right=of t8] (t9) {3.14\\FLOAT};
    \node[token, right=of t9] (t10) {;\\SEMICOLON};
    
    % Arrows
    \draw[arrow] (source.south) to[out=-90, in=90] (t1.north);
    \draw[arrow] (source.south) to[out=-90, in=90] (t2.north);
    \draw[arrow] (source.south) to[out=-90, in=90] (t3.north);
    \draw[arrow] (source.south) to[out=-90, in=90] (t4.north);
    \draw[arrow] (source.south) to[out=-90, in=90] (t5.north);
    \draw[arrow] (source.south) to[out=-90, in=90] (t6.north);
    \draw[arrow] (source.south) to[out=-90, in=90] (t7.north);
    \draw[arrow] (source.south) to[out=-90, in=90] (t8.north);
    \draw[arrow] (source.south) to[out=-90, in=90] (t9.north);
    \draw[arrow] (source.south) to[out=-90, in=90] (t10.north);
    
    \node[below=0.2cm of t8, font=\small\bfseries] {Complete Token Stream};
    
    \end{tikzpicture}%
    }
    \caption{Contoh lengkap tokenization untuk program sederhana}
    \label{fig:complete-example}
\end{figure}

Gambar \ref{fig:token-stream-example} menunjukkan visualisasi token stream untuk contoh kode sederhana.

\begin{figure}[H]
    \centering
    \adjustbox{max width=0.9\textwidth,center}{%
    \begin{tikzpicture}[
        code/.style={rectangle, draw=gray!30, fill=gray!5, text width=8cm, minimum height=0.6cm, font=\footnotesize\ttfamily, align=left, inner sep=4pt},
        token/.style={rectangle, draw=blue!50, fill=blue!10, text width=1.3cm, minimum height=0.5cm, font=\tiny, align=center, inner sep=2pt},
        arrow/.style={->, >=stealth, thick, gray},
        node distance=0.3cm and 0.2cm
    ]
    
    % Source code
    \node[code] (source) {int x = 42;};
    
    % Tokens
    \node[token, below=0.5cm of source, xshift=-2.5cm] (t1) {int\\KEYWORD};
    \node[token, right=of t1] (t2) {x\\IDENTIFIER};
    \node[token, right=of t2] (t3) {=\\OP\_ASSIGN};
    \node[token, right=of t3] (t4) {42\\INTEGER};
    \node[token, right=of t4] (t5) {;\\SEMICOLON};
    
    % Arrows from source to tokens
    \draw[arrow] (source.south) to[out=-90, in=90] (t1.north);
    \draw[arrow] (source.south) to[out=-90, in=90] (t2.north);
    \draw[arrow] (source.south) to[out=-90, in=90] (t3.north);
    \draw[arrow] (source.south) to[out=-90, in=90] (t4.north);
    \draw[arrow] (source.south) to[out=-90, in=90] (t5.north);
    
    % Token stream label
    \node[below=0.3cm of t3, font=\small\bfseries] {Token Stream};
    
    \end{tikzpicture}%
    }
    \caption{Contoh tokenization: \texttt{int x = 42;} menjadi token stream}
    \label{fig:token-stream-example}
\end{figure}

Gambar \ref{fig:lexer-example-formal} menunjukkan proses scanning secara detail untuk string \texttt{"hello"}.

\begin{figure}[H]
    \centering
    \adjustbox{max width=0.85\textwidth,center}{%
    \begin{tikzpicture}[
        char/.style={rectangle, draw=blue!60, fill=blue!5, minimum width=0.7cm, minimum height=0.7cm, font=\footnotesize\ttfamily, rounded corners=2pt},
        state/.style={rectangle, draw=green!60, fill=green!8, minimum width=1.8cm, minimum height=0.7cm, font=\scriptsize\sffamily, rounded corners=4pt, align=center},
        arrow/.style={->, >=stealth, thick, color=black!70},
        trans/.style={->, >=stealth, dashed, thick, color=black!50},
        node distance=0.5cm and 0.5cm
    ]
    
    % =========================
    % Row 1: Input Characters
    % =========================
    \node[char] (c1) {"};
    \node[char, right=1.5cm of c1] (c2) {h};
    \node[char, right=1.5cm of c2] (c3) {e};
    \node[char, right=1.5cm of c3] (c4) {l};
    \node[char, right=1.5cm of c4] (c5) {l};
    \node[char, right=1.5cm of c5] (c6) {o};
    \node[char, right=1.5cm of c6] (c7) {"};
    
    % =========================
    % Row 2: Lexer States
    % =========================
    \node[state, below=0.9cm of c1] (s0) {START};
    \node[state, below=0.9cm of c2] (s1) {IN\_STRING};
    \node[state, below=0.9cm of c3] (s2) {IN\_STRING};
    \node[state, below=0.9cm of c4] (s3) {IN\_STRING};
    \node[state, below=0.9cm of c5] (s4) {IN\_STRING};
    \node[state, below=0.9cm of c6] (s5) {IN\_STRING};
    \node[state, below=0.9cm of c7] (s6) {ACCEPT};
    
    % =========================
    % Vertical Mapping (Input -> State)
    % =========================
    \draw[arrow] (c1) -- (s0);
    \draw[arrow] (c2) -- (s1);
    \draw[arrow] (c3) -- (s2);
    \draw[arrow] (c4) -- (s3);
    \draw[arrow] (c5) -- (s4);
    \draw[arrow] (c6) -- (s5);
    \draw[arrow] (c7) -- (s6);
    
    % =========================
    % Horizontal State Transitions
    % =========================
    \draw[trans] (s0) -- (s1);
    \draw[trans] (s1) -- (s2);
    \draw[trans] (s2) -- (s3);
    \draw[trans] (s3) -- (s4);
    \draw[trans] (s4) -- (s5);
    \draw[trans] (s5) -- (s6);
    
    % =========================
    % Result Annotation
    % =========================
    \node[below=0.7cm of s3, font=\small\bfseries, align=center] (res)
    {Token dihasilkan: \texttt{STRING\_LITERAL("hello")}};
    
    \end{tikzpicture}%
    }
    \caption{Representasi formal proses analisis leksikal pada string literal \texttt{"hello"}}
    \label{fig:lexer-example-formal}
    \end{figure}
    

Gambar \ref{fig:error-recovery-strategy} menunjukkan strategi error recovery dalam lexer.

\begin{figure}[H]
    \centering
    \adjustbox{max width=0.85\textwidth,center}{%
    \begin{tikzpicture}[
        box/.style={rectangle, draw=blue!50, fill=blue!10, text width=2.5cm, minimum height=0.6cm, font=\footnotesize, align=center, rounded corners},
        error/.style={rectangle, draw=red!50, fill=red!10, text width=2.5cm, minimum height=0.6cm, font=\footnotesize, align=center, rounded corners},
        decision/.style={diamond, draw=orange!50, fill=orange!10, text width=1.5cm, minimum height=0.6cm, font=\footnotesize, align=center},
        arrow/.style={->, >=stealth, thick},
        node distance=0.6cm
    ]
    
    \node[box] (input) {Input\\Character};
    \node[decision, below=of input] (valid) {Valid?};
    \node[box, below left=of valid] (token) {Generate\\Token};
    \node[error, below right=of valid] (skip) {Skip\\Character};
    \node[box, below=of token] (continue) {Continue\\Scanning};
    \node[box, below=of skip] (report) {Report\\Error};
    
    \draw[arrow] (input) -- (valid);
    \draw[arrow] (valid) -- node[left, font=\tiny] {Yes} (token);
    \draw[arrow] (valid) -- node[right, font=\tiny] {No} (skip);
    \draw[arrow] (token) -- (continue);
    \draw[arrow] (skip) -- (report);
    \draw[arrow, dashed] (report) to[out=180, in=270] (continue);
    
    \end{tikzpicture}%
    }
    \caption{Strategi error recovery dalam lexer}
    \label{fig:error-recovery-strategy}
\end{figure}