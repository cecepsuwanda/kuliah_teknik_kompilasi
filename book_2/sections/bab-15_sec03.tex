\section{Basic Blocks}

Basic block adalah fondasi untuk banyak optimasi kompilator. Menurut University of Michigan\footnote{\url{https://web.eecs.umich.edu/~weimerw/2015-4610/ca1/ca1.html}}, basic block didefinisikan sebagai:

\begin{quote}
``A basic block is a straight-line sequence of code with no jumps in except at the entry, and no jumps out except at the exit. Once execution enters it, all instructions execute sequentially.''
\end{quote}

\subsection{Karakteristik Basic Block}

Sebuah basic block memiliki karakteristik berikut:

\begin{enumerate}
    \item \textbf{Single Entry Point}: Hanya ada satu titik masuk (entry point)
    \item \textbf{Single Exit Point}: Hanya ada satu titik keluar (exit point)
    \item \textbf{Sequential Execution}: Semua instruksi dieksekusi secara berurutan tanpa branching
    \item \textbf{No Internal Control Flow}: Tidak ada jump, branch, atau call di tengah-tengah block
\end{enumerate}

Gambar \ref{fig:basic-block-example} menunjukkan contoh basic block.

\begin{figure}[H]
    \centering
    \adjustbox{max width=0.85\textwidth,center}{%
    \begin{tikzpicture}[
        inst/.style={rectangle, draw=blue!50, fill=blue!10, text width=4cm, minimum height=0.5cm, font=\tiny\ttfamily, align=left, inner sep=4pt, rounded corners},
        node distance=0.2cm
    ]
    
    \node[inst] (i1) {t1 = a + b};
    \node[inst, below=of i1] (i2) {t2 = t1 * c};
    \node[inst, below=of i2] (i3) {x = t2};
    
    \node[left=0.3cm of i1, font=\tiny] {Entry};
    \node[left=0.3cm of i3, font=\tiny] {Exit};
    
    \end{tikzpicture}%
    }
    \caption{Contoh basic block}
    \label{fig:basic-block-example}
\end{figure}

\subsection{Identifikasi Basic Blocks}

Algoritma untuk mengidentifikasi basic blocks dalam intermediate code:

\begin{enumerate}
    \item \textbf{Leader Identification}: Tentukan leader (instruksi pertama dalam basic block)
    \begin{itemize}
        \item Instruksi pertama dalam program adalah leader
        \item Instruksi yang merupakan target dari jump/branch adalah leader
        \item Instruksi setelah jump/branch/call adalah leader
    \end{itemize}
    
    \item \textbf{Block Construction}: Untuk setiap leader, buat basic block yang berisi:
    \begin{itemize}
        \item Leader instruction
        \item Semua instruksi berikutnya hingga menemukan leader berikutnya atau instruksi control flow
    \end{itemize}
\end{enumerate}

\subsection{Contoh Identifikasi Basic Block}

Perhatikan contoh three-address code berikut:

\begin{verbatim}
L1: t1 = a + b
    t2 = c * d
    t3 = t1 + t2
    if t3 > 0 goto L2
    t4 = t1 - t2
    goto L3
L2: t5 = t1 * t2
L3: t6 = t5 + 1
    return t6
\end{verbatim}

Basic blocks yang diidentifikasi:

\textbf{Block 1 (L1):}
\begin{verbatim}
    t1 = a + b
    t2 = c * d
    t3 = t1 + t2
    if t3 > 0 goto L2
\end{verbatim}

\textbf{Block 2 (L2):}
\begin{verbatim}
    t5 = t1 * t2
\end{verbatim}

\textbf{Block 3 (setelah goto L3):}
\begin{verbatim}
    t4 = t1 - t2
    goto L3
\end{verbatim}

\textbf{Block 4 (L3):}
\begin{verbatim}
    t6 = t5 + 1
    return t6
\end{verbatim}

\subsection{Control Flow Graph (CFG)}

Control Flow Graph adalah representasi grafis dari alur kontrol program. Menurut Wikipedia\footnote{\url{https://en.wikipedia.org/wiki/Control-flow_graph}}:

\begin{quote}
``A control-flow graph (CFG) is a representation of a function where each node is a basic block, and edges represent possible flow of control from one block to another.''
\end{quote}

CFG membantu dalam:
\begin{itemize}
    \item Memahami struktur program
    \item Melakukan data-flow analysis
    \item Mengidentifikasi loop dan struktur kontrol lainnya
    \item Mengoptimasi lintas basic blocks
\end{itemize}