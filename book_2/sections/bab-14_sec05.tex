\section{Register Allocation}

Register allocation adalah proses menentukan variabel dan temporary values mana yang disimpan di register (fast storage) dan mana yang di-spill ke memory. Karena jumlah register fisik terbatas, ini adalah optimasi yang constrained.

\subsection{Mengapa Register Allocation Penting?}

Menurut Wikipedia\footnote{\url{https://en.wikipedia.org/wiki/Register_allocation}}:

\begin{itemize}
    \item Register jauh lebih cepat daripada memory access
    \item Jumlah register fisik terbatas (biasanya 16-32 register)
    \item Program mungkin memiliki lebih banyak variabel aktif daripada jumlah register
    \item Register allocation yang baik dapat meningkatkan performa secara signifikan
\end{itemize}

\subsection{Dua Fase Register Allocation}

\subsubsection{1. Allocation Phase}

Memutuskan \textit{mana} nilai yang harus disimpan di register pada setiap program point. Ini melibatkan:
\begin{itemize}
    \item Live range analysis: Kapan variabel hidup (live) dan kapan mati (dead)
    \item Interference graph: Grafik yang menunjukkan variabel mana yang tidak bisa menggunakan register yang sama secara bersamaan
\end{itemize}

\subsubsection{2. Assignment Phase}

Memetakan nilai yang dialokasikan ke register fisik spesifik. Jika lebih banyak nilai yang perlu register daripada register yang tersedia, beberapa nilai harus di-spill ke memory.

\subsection{Local Register Allocation}

Local register allocation bekerja dalam satu basic block (satu entry, satu exit, tidak ada branching). Ini lebih sederhana karena control flow linear.

\subsubsection{Algoritma Simple Local Allocation}

Algoritma sederhana untuk local allocation:

\begin{enumerate}
    \item Scan basic block dari awal hingga akhir
    \item Untuk setiap instruksi:
    \begin{itemize}
        \item Jika operan tidak di register, load dari memory
        \item Eksekusi operasi menggunakan register
        \item Jika hasil perlu disimpan dan register penuh, spill register yang paling lama tidak digunakan
    \end{itemize}
    \item Di akhir block, store semua register yang modified ke memory
\end{enumerate}

\subsubsection{Contoh Local Allocation}

Misalkan kita memiliki basic block dengan TAC berikut:
\begin{verbatim}
t1 = a + b
t2 = t1 * c
d = t2
\end{verbatim}

Dengan 3 register tersedia (R1, R2, R3), allocation bisa seperti ini:
\begin{verbatim}
LOAD R1, a        ; Load a ke R1
LOAD R2, b        ; Load b ke R2
ADD R1, R1, R2    ; R1 = a + b (t1)
LOAD R2, c        ; Load c ke R2 (b tidak lagi diperlukan)
MUL R1, R1, R2    ; R1 = t1 * c (t2)
STORE R1, d       ; Store hasil ke d
\end{verbatim}

\subsection{Global Register Allocation}

Global register allocation bekerja lintas basic blocks atau seluruh function. Ini lebih kompleks karena perlu mempertimbangkan control flow.

Metode populer:
\begin{itemize}
    \item \textbf{Graph Coloring}: Membangun interference graph dan mewarnainya dengan k warna (k = jumlah register)
    \item \textbf{Linear Scan}: Lebih cepat, menghitung live intervals dan assign register dalam satu pass
\end{itemize}

Gambar \ref{fig:register-allocation-methods} menunjukkan perbandingan metode register allocation.

\begin{figure}[!htbp]
    \centering
    \adjustbox{max width=0.85\textwidth,center}{%
    \begin{tikzpicture}[
        method/.style={rectangle, draw=blue!50, fill=blue!10, text width=3cm, minimum height=0.7cm, font=\footnotesize, align=center, rounded corners, inner sep=4pt},
        arrow/.style={->, >=stealth, thick},
        node distance=1.2cm
    ]
    
    \node[method] (graph) {Graph\\Coloring};
    \node[method, right=of graph] (linear) {Linear\\Scan};
    
    \node[below=0.3cm of graph, font=\tiny, align=center] {Optimal\\Slower};
    \node[below=0.3cm of linear, font=\tiny, align=center] {Fast\\Good};
    
    \end{tikzpicture}%
    }
    \caption{Perbandingan metode register allocation}
    \label{fig:register-allocation-methods}
\end{figure}

Gambar \ref{fig:register-allocation-2} menunjukkan konsep register allocation.

\begin{figure}[!htbp]
    \centering
    \adjustbox{max width=0.85\textwidth,center}{%
    \begin{tikzpicture}[
        var/.style={rectangle, draw=blue!50, fill=blue!10, text width=1.5cm, minimum height=0.5cm, font=\tiny, align=center, inner sep=4pt, rounded corners},
        reg/.style={rectangle, draw=green!50, fill=green!10, text width=1.5cm, minimum height=0.5cm, font=\tiny, align=center, inner sep=4pt, rounded corners},
        mem/.style={rectangle, draw=orange!50, fill=orange!10, text width=1.5cm, minimum height=0.5cm, font=\tiny, align=center, inner sep=4pt, rounded corners},
        arrow/.style={->, >=stealth, thick},
        node distance=0.6cm
    ]
    
    \node[var] (v1) {a};
    \node[var, right=of v1] (v2) {b};
    \node[var, right=of v2] (v3) {c};
    \node[var, right=of v3] (v4) {d};
    
    \node[reg, below=of v1] (r1) {R1};
    \node[reg, right=of r1] (r2) {R2};
    \node[mem, right=of r2] (m1) {Memory};
    
    \draw[arrow] (v1) -- (r1);
    \draw[arrow] (v2) -- (r2);
    \draw[arrow] (v3) -- (m1);
    \draw[arrow] (v4) -- (m1);
    
    \end{tikzpicture}%
    }
    \caption{Register allocation: variabel dialokasikan ke register atau memory}
    \label{fig:register-allocation-2}
\end{figure}

Untuk pembelajaran, kita akan fokus pada local allocation yang lebih sederhana.