\section{Pendahuluan}

Dalam proyek compiler subset C, optimasi diterapkan pada IR (Bab 12) atau pada output code generation (Bab 14): basic block, constant folding, constant propagation, dead code elimination. Konteks ``compiler subset C kita'' memastikan bahwa optimasi konsisten dengan AST, symbol table, dan IR proyek.

Optimasi kompilator adalah proses transformasi kode intermediate untuk meningkatkan kualitas kode yang dihasilkan tanpa mengubah semantik program. Menurut sumber dari Scribd OBE CSE Document:

\begin{quote}
``Perform machine-independent optimizations (basic block optimizations, data-flow analysis). Local and global optimization; data-flow analysis.''\cite{aho2006compilers}
\end{quote}

Tujuan optimasi kompilator meliputi:
\begin{itemize}
    \item \textbf{Meningkatkan Performa}: Mengurangi waktu eksekusi program
    \item \textbf{Mengurangi Ukuran Kode}: Menghasilkan executable yang lebih kecil
    \item \textbf{Mengurangi Konsumsi Memory}: Mengoptimasi penggunaan memory
    \item \textbf{Meningkatkan Efisiensi Energy}: Mengurangi konsumsi daya (penting untuk embedded systems)
\end{itemize}

Gambar \ref{fig:optimization-levels} menunjukkan level-level optimasi.

\begin{figure}[!htbp]
    \centering
    \adjustbox{max width=0.9\textwidth,center}{%
    \begin{tikzpicture}[
        level/.style={rectangle, draw=blue!50, fill=blue!10, text width=3cm, text centered, minimum height=0.7cm, rounded corners, font=\footnotesize, inner sep=4pt, align=center},
        arrow/.style={->, >=stealth, thick},
        node distance=0.6cm
    ]
    
    \node[level] (local) {Optimasi\\Lokal};
    \node[level, below=of local] (global) {Optimasi\\Global};
    \node[level, below=of global] (inter) {Optimasi\\Interprosedural};
    
    \draw[arrow] (local) -- (global);
    \draw[arrow] (global) -- (inter);
    
    \end{tikzpicture}%
    }
    \caption{Level-level optimasi kompilator}
    \label{fig:optimization-levels}
\end{figure}

Namun, optimasi harus dilakukan dengan hati-hati karena:
\begin{itemize}
    \item Optimasi yang terlalu agresif dapat meningkatkan waktu kompilasi
    \item Beberapa optimasi dapat membuat kode lebih sulit di-debug
    \item Optimasi yang salah dapat mengubah semantik program (bug)
\end{itemize}

\subsection{Prinsip Optimasi}

Menurut Dragon Book\cite{aho2006compilers}, optimasi harus mematuhi prinsip-prinsip berikut:

\begin{enumerate}
    \item \textbf{Correctness}: Optimasi tidak boleh mengubah semantik program
    \item \textbf{Benefit}: Optimasi harus memberikan manfaat yang signifikan
    \item \textbf{Speed}: Proses optimasi tidak boleh terlalu lambat
    \item \textbf{Simplicity}: Optimasi harus mudah diimplementasikan dan di-maintain
\end{enumerate}

\subsection{Level Optimasi}

Optimasi dapat dikategorikan berdasarkan scope-nya:

\begin{itemize}
    \item \textbf{Optimasi Lokal (Local Optimization)}: Optimasi dalam satu basic block
    \begin{itemize}
        \item Constant folding
        \item Constant propagation
        \item Algebraic simplification
        \item Strength reduction
    \end{itemize}
    
    \item \textbf{Optimasi Global (Global Optimization)}: Optimasi lintas basic blocks
    \begin{itemize}
        \item Common subexpression elimination
        \item Loop optimization
        \item Dead code elimination (global)
        \item Constant propagation (global)
    \end{itemize}
    
    \item \textbf{Optimasi Interprosedural (Interprocedural Optimization)}: Optimasi lintas fungsi/prosedur
    \begin{itemize}
        \item Inlining
        \item Interprocedural constant propagation
        \item Whole-program optimization
    \end{itemize}
\end{itemize}

Gambar \ref{fig:optimization-pipeline} menunjukkan pipeline optimasi dalam kompilator.

\begin{figure}[!htbp]
    \centering
    \adjustbox{max width=0.9\textwidth,center}{%
    \begin{tikzpicture}[
        box/.style={rectangle, draw=blue!50, fill=blue!10, text width=2.5cm, text centered, minimum height=0.7cm, rounded corners, font=\footnotesize, inner sep=4pt, align=center},
        arrow/.style={->, >=stealth, thick},
        node distance=1.2cm
    ]
    
    \node[box] (ir) {IR};
    \node[box, right=of ir] (local) {Optimasi\\Lokal};
    \node[box, right=of local] (global) {Optimasi\\Global};
    \node[box, below=of local] (opt-ir) {IR\\Teroptimasi};
    
    \draw[arrow] (ir) -- (local);
    \draw[arrow] (local) -- (global);
    \draw[arrow] (global) -- (opt-ir);
    
    \end{tikzpicture}%
    }
    \caption{Pipeline optimasi kompilator}
    \label{fig:optimization-pipeline}
\end{figure}