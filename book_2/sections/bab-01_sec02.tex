\section{Apa itu Kompilator?}

Kompilator adalah program komputer yang menerjemahkan source code (kode sumber) yang ditulis dalam bahasa pemrograman tingkat tinggi menjadi target code (kode target) dalam bahasa yang dapat dieksekusi oleh komputer, seperti assembly language atau machine code.

\subsection{Definisi Kompilator}

Secara formal, kompilator adalah program yang melakukan translasi dari bahasa sumber (source language) ke bahasa target (target language), dengan mempertahankan makna semantik dari program sumber. Proses translasi ini disebut \textbf{kompilasi}.

Menurut Aho, Lam, Sethi, dan Ullman dalam buku klasik "Compilers: Principles, Techniques, and Tools"\cite{aho2006compilers}:

\begin{quote}
``A compiler is a program that can read a program in one language (the source language) and translate it into an equivalent program in another language (the target language).''
\end{quote}

\subsection{Karakteristik Kompilator}

Kompilator memiliki beberapa karakteristik penting:

\begin{itemize}
    \item \textbf{Translasi Lengkap}: Kompilator membaca seluruh program sumber sebelum menghasilkan output, berbeda dengan interpreter yang mengeksekusi baris demi baris.
    \item \textbf{Analisis Mendalam}: Kompilator melakukan analisis mendalam terhadap struktur program, termasuk pengecekan syntax, semantic, dan optimasi.
    \item \textbf{Output Terpisah}: Hasil kompilasi adalah file terpisah (executable atau object file) yang dapat dieksekusi tanpa perlu source code.
    \item \textbf{Optimasi}: Kompilator dapat melakukan berbagai optimasi untuk meningkatkan efisiensi program yang dihasilkan.
\end{itemize}

\subsection{Perbedaan Kompilator dengan Interpreter}

Meskipun keduanya menerjemahkan kode, kompilator dan interpreter memiliki perbedaan fundamental:

\begin{itemize}
    \item \textbf{Kompilator}: Menerjemahkan seluruh program sekaligus menjadi executable file sebelum eksekusi. Contoh: C, C++, Rust.
    \item \textbf{Interpreter}: Menerjemahkan dan mengeksekusi program baris demi baris secara langsung tanpa menghasilkan file terpisah. Contoh: Python, JavaScript, Ruby.
    \item \textbf{Hybrid}: Beberapa bahasa menggunakan kombinasi keduanya, seperti Java (compile ke bytecode, kemudian diinterpretasi oleh JVM).
\end{itemize}

\subsection{Peran Kompilator dalam Pengembangan Software}

Kompilator memainkan peran penting dalam ekosistem pengembangan software modern:

\begin{enumerate}
    \item \textbf{Bridge antara High-Level dan Low-Level}: Memungkinkan programmer menulis kode dalam bahasa yang mudah dipahami manusia, sementara komputer menjalankan instruksi tingkat rendah yang efisien.
    \item \textbf{Error Detection}: Mendeteksi berbagai jenis error (syntax, semantic, type) sebelum program dieksekusi.
    \item \textbf{Optimization}: Menerapkan berbagai teknik optimasi untuk menghasilkan kode yang lebih efisien.
    \item \textbf{Portability}: Memungkinkan program yang sama dikompilasi untuk berbagai platform dan arsitektur.
\end{enumerate}