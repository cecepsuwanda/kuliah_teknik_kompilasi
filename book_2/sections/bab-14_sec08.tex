\section{Implementasi Code Generator Lengkap}

Mari kita buat implementasi code generator yang lebih lengkap yang dapat menangani berbagai operasi.

\subsection{Code Generator untuk TAC}

Berikut adalah contoh code generator yang mengambil TAC dan menghasilkan RISC-V assembly:

\begin{lstlisting}[language=C++, caption={Code Generator untuk TAC ke RISC-V}]
class TACCodeGenerator {
private:
    std::vector<std::string> assembly;
    int labelCounter;
    std::map<std::string, int> varOffset;  // Offset di stack frame
    int stackOffset;
    
public:
    TACCodeGenerator() : labelCounter(0), stackOffset(0) {}
    
    void generateTAC(const TACInstruction& tac) {
        switch (tac.op) {
            case TAC_OP::ADD:
                generateAdd(tac.result, tac.arg1, tac.arg2);
                break;
            case TAC_OP::SUB:
                generateSub(tac.result, tac.arg1, tac.arg2);
                break;
            case TAC_OP::MUL:
                generateMul(tac.result, tac.arg1, tac.arg2);
                break;
            case TAC_OP::DIV:
                generateDiv(tac.result, tac.arg1, tac.arg2);
                break;
            case TAC_OP::ASSIGN:
                generateAssign(tac.result, tac.arg1);
                break;
            case TAC_OP::LOAD:
                generateLoad(tac.result, tac.arg1);
                break;
            case TAC_OP::STORE:
                generateStore(tac.arg1, tac.result);
                break;
            // ... operasi lainnya
        }
    }
    
    void generateAdd(const std::string& result,
                     const std::string& arg1,
                     const std::string& arg2) {
        std::string reg1 = loadToRegister(arg1);
        std::string reg2 = loadToRegister(arg2);
        std::string regResult = allocateRegister(result);
        
        assembly.push_back("ADD " + regResult + ", " + 
                          reg1 + ", " + reg2);
        
        releaseRegister(reg1);
        releaseRegister(reg2);
    }
    
    // ... implementasi operasi lainnya
};
\end{lstlisting}