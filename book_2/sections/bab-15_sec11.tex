\section{Kesimpulan}

Dalam bab ini, kita telah mempelajari:

\begin{enumerate}
    \item Optimasi kompilator bertujuan meningkatkan kualitas kode tanpa mengubah semantik
    \item Basic blocks adalah unit fundamental untuk optimasi lokal
    \item Constant folding dan constant propagation adalah optimasi dasar yang efektif
    \item Dead code elimination menghapus kode yang tidak berguna
    \item Data-flow analysis adalah fondasi untuk optimasi global
    \item Evaluasi efektivitas optimasi penting untuk memastikan optimasi memberikan manfaat
\end{enumerate}

Optimasi kompilator adalah bidang yang luas dan kompleks. Bab ini memberikan dasar-dasar optimasi lokal. Untuk optimasi yang lebih advanced seperti loop optimization, interprocedural optimization, dan machine-specific optimization, diperlukan pemahaman yang lebih mendalam tentang data-flow analysis dan teknik optimasi lainnya. Optimasi proyek subset C diterapkan pada IR dan/atau kode yang dihasilkan; compiler lengkap (lexer, parser, AST, symbol table, type check, IR, codegen, optimasi) dipresentasikan di Bab 16.