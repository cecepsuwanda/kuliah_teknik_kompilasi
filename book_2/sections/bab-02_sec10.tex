\section{Contoh Praktis: Implementasi Lengkap}

Sebagai contoh praktis, berikut adalah implementasi lengkap DFA sederhana untuk mengenali identifier dan number:

\begin{lstlisting}[language=C++, caption={Implementasi Lengkap DFA untuk Identifier dan Number}]
#include <iostream>
#include <string>
#include <map>
#include <set>

class SimpleDFA {
private:
    int start_state;
    std::set<int> accept_states;
    std::map<std::pair<int, char>, int> transitions;
    
    bool isLetter(char c) {
        return (c >= 'a' && c <= 'z') || (c >= 'A' && c <= 'Z') || c == '_';
    }
    
    bool isDigit(char c) {
        return c >= '0' && c <= '9';
    }
    
    bool isAlphanumeric(char c) {
        return isLetter(c) || isDigit(c);
    }
    
public:
    SimpleDFA() {
        // DFA untuk identifier: [a-zA-Z_][a-zA-Z0-9_]*
        start_state = 0;
        accept_states.insert(1);
        
        // State 0 -> State 1 dengan letter atau underscore
        for (char c = 'a'; c <= 'z'; c++) {
            transitions[{0, c}] = 1;
            transitions[{1, c}] = 1;  // Loop di state 1
        }
        for (char c = 'A'; c <= 'Z'; c++) {
            transitions[{0, c}] = 1;
            transitions[{1, c}] = 1;
        }
        transitions[{0, '_'}] = 1;
        transitions[{1, '_'}] = 1;
        
        // State 1 -> State 1 dengan digit
        for (char c = '0'; c <= '9'; c++) {
            transitions[{1, c}] = 1;
        }
    }
    
    bool simulate(const std::string& input) {
        int current_state = start_state;
        
        for (char c : input) {
            auto key = std::make_pair(current_state, c);
            if (transitions.find(key) == transitions.end()) {
                return false;
            }
            current_state = transitions[key];
        }
        
        return accept_states.find(current_state) != accept_states.end();
    }
};

int main() {
    SimpleDFA dfa;
    
    std::vector<std::string> test_cases = {
        "variable",
        "var123",
        "_private",
        "123var",      // Invalid (dimulai dengan digit)
        "var_name",
        "VarName123"
    };
    
    std::cout << "Testing DFA untuk Identifier:\n";
    std::cout << "==============================\n";
    for (const auto& test : test_cases) {
        bool result = dfa.simulate(test);
        std::cout << "\"" << test << "\" -> " 
                  << (result ? "ACCEPT" : "REJECT") << "\n";
    }
    
    return 0;
}
\end{lstlisting}

Output program di atas:

\begin{verbatim}
Testing DFA untuk Identifier:
==============================
"variable" -> ACCEPT
"var123" -> ACCEPT
"_private" -> ACCEPT
"123var" -> REJECT
"var_name" -> ACCEPT
"VarName123" -> ACCEPT
\end{verbatim}