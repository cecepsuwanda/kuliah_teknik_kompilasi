\section{Precedence dan Error Recovery}

\subsection{Menangani Precedence}
Dalam recursive descent, precedence diatur melalui level pemanggilan fungsi. Operator dengan precedence lebih tinggi (misal: perkalian) dipanggil lebih dalam dibanding operator dengan precedence rendah (misal: penjumlahan).

\subsection{Error Recovery: Panic Mode}
Agar parser tidak berhenti pada kesalahan pertama, kita menerapkan \textit{Panic Mode Recovery}. Jika terjadi kesalahan, parser akan membuang token (\textit{skipping}) sampai menemukan token sinkronisasi seperti \texttt{;} atau \texttt{\}}.

\begin{lstlisting}[language=C++]
void Parser::synchronize() {
    while (!isAtEnd()) {
        if (peek().type == SEMICOLON) return;
        switch (peek().type) {
            case CLASS: case FUN: case VAR: case IF: return;
        }
        advance();
    }
}
\end{lstlisting}
