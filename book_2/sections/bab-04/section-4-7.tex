\section{Precedence dan Error Recovery}

\subsection{Menangani Operator Precedence}
Dalam \textit{Recursive Descent}, precedence diatur secara implisit melalui struktur fungsi yang berlapis. Fungsi yang dipanggil paling dalam (misal: \texttt{parseFactor}) memiliki prioritas eksekusi paling tinggi (mengikat operan lebih kuat).
\begin{itemize}
    \item \textbf{Lowest Precedence}: \texttt{parseExpression} (menangani \texttt{+ -}).
    \item \textbf{Medium Precedence}: \texttt{parseTerm} (menangani \texttt{* /}).
    \item \textbf{Highest Precedence}: \texttt{parseFactor} (menangani \texttt{( )}, angka, variabel).
\end{itemize}

\subsection{Error Recovery: Panic Mode}
Parser yang baik tidak boleh berhenti (\textit{abort}) begitu menemukan satu kesalahan sintaks.
\begin{enumerate}
    \item \textbf{Deteksi Error}: Parser menemukan token tak terduga (misal: \texttt{Expected ';'}).
    \item \textbf{Masuk Mode Panik}: Parser membuang semua produksi yang sedang dikerjakan.
    \item \textbf{Sinkronisasi}: Parser terus membuang token input (\textit{eating tokens}) sampai menemukan token "jangkar" yang aman, biasanya titik koma (\texttt{;}) atau kurung kurawal tutup (\texttt{\}}).
\end{enumerate}

\begin{lstlisting}[language=C++]
void Parser::panic() {
    while (!isAtEnd()) {
        if (previous().type == SEMICOLON) return;
        switch (peek().type) {
            case CLASS: case FUN: case VAR: case FOR: case IF: return;
        }
        advance();
    }
}
\end{lstlisting}
