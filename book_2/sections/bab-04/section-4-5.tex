\section{Integrasi Lexer dan Parser (Bison)}

Dalam ekosistem GNU, \textbf{Bison} (Parser) adalah tuan dan \textbf{Flex} (Lexer) adalah pelayan. Parser meminta token, Lexer menyediakannya.

\subsection{Mekanisme Komunikasi: yylval}
Lexer dan Parser berbagi variabel global (atau anggota struct dalam mode \textit{reentrant}) bernama \texttt{yylval}.
\begin{enumerate}
    \item Lexer mencocokkan string "100".
    \item Lexer mengonversi string "100" menjadi integer 100.
    \item Lexer menyimpan nilai 100 ke dalam \texttt{yylval.ival}.
    \item Lexer mengembalikan token ID \texttt{T\_INT} ke parser.
    \item Parser menerima token \texttt{T\_INT}, dan mengambil nilai 100 dari \texttt{yylval} untuk membangun node AST.
\end{enumerate}

\begin{lstlisting}[language=C]
// Di dalam file .l (Flex)
[0-9]+  { yylval.ival = atoi(yytext); return T_INT; }
[a-z]+  { yylval.sval = strdup(yytext); return T_ID; }
\end{lstlisting}

\subsection{Struktur File Bison (.y)}
File Bison memiliki tiga segmen yang dipisahkan oleh \texttt{\%\%}:
\begin{enumerate}
    \item \textbf{Deklarasi}: Definisi token, tipe data union \texttt{yylval}, dan precedence.
    \item \textbf{Rules}: Grammar dalam format BNF beserta \textit{semantic actions} (kode C dalam \texttt{\{ \}}).
    \item \textbf{User Code}: Fungsi \texttt{main()}, \texttt{yyerror()}, dll.
\end{enumerate}
