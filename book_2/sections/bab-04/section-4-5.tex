\section{Integrasi Lexer dan Parser (Bison)}

Parser generator modern seperti \textbf{Bison} bekerja sama dengan \textbf{Flex} melalui mekanisme integrasi token.

\subsection{Mekanisme yylval}
Lexer mengirimkan nilai semantik token (misal: nilai konstanta atau nama variabel) ke parser melalui variabel global \texttt{yylval}.

\subsection{Struktur File Bison (.y)}
\begin{lstlisting}[language=C]
%token NUMBER IDENTIFIER
%left '+' '-'
%left '*' '/'
%%
expr: expr '+' expr | NUMBER ;
%%
\end{lstlisting}

\begin{figure}[!htbp]
    \centering
    \adjustbox{max width=0.8\textwidth,center}{%
    \begin{tikzpicture}[
        box/.style={rectangle, draw=blue!50, fill=blue!10, font=\tiny, align=center},
        arrow/.style={->, >=stealth, thick}
    ]
    \node[box] (flex) {Flex (Lexer)};
    \node[box, right=2cm of flex] (bison) {Bison (Parser)};
    \draw[arrow] (flex) -- node[above, font=\tiny] {Tokens + yylval} (bison);
    \end{tikzpicture}%
    }
    \caption{Integrasi data antara Flex dan Bison}
\end{figure}
