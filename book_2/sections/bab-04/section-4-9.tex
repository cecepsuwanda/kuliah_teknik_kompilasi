\section{Parser Generator: Bison Deep Dive}

\subsection{Shift/Reduce Conflict}
Konflik ini terjadi ketika parser bingung memilih antara menggeser (\textit{shift}) token baru atau mereduksi (\textit{reduce}) stack sekarang.
Contoh klasik: \texttt{if E1 then if E2 then S1 else S2}.
Di sini, parser memiliki \texttt{if E then S} di stack dan melihat \texttt{else}.
\begin{itemize}
    \item \textbf{Shift}: Membawa \texttt{else} masuk (berarti \texttt{else} milik \texttt{if} dalam).
    \item \textbf{Reduce}: Mengubah \texttt{if E then S} menjadi \textit{Stmt} (berarti \texttt{else} milik \texttt{if} luar).
\end{itemize}
Default Bison adalah \textbf{Shift} (yang benar untuk kasus ini).

\subsection{Operator Precedence Declarations}
Tanpa mengubah grammar menjadi rumit, kita bisa memberi tahu Bison prioritas operator:
\begin{lstlisting}[language=C]
%left '+' '-'   // Precedence rendah, asosiasi kiri
%left '*' '/'   // Precedence tinggi, asosiasi kiri
%right '^'      // Precedence lebih tinggi, asosiasi kanan
%nonassoc UMINUS // Untuk operator unary minus
\end{lstlisting}
Aturan yang dideklarasikan lebih bawah memiliki precedence lebih tinggi.

\subsection{Error Handling dengan Token 'error'}
Bison memiliki token spesial \texttt{error} untuk pemulihan.
\begin{lstlisting}[language=C]
stmt: 
    expr ';' 
  | error ';' { yyerror("Syntax error, skipping to semicolon"); yyerrok; }
;
\end{lstlisting}
Jika terjadi kesalahan dalam \textit{stmt}, parser akan membuang token sampai menemukan titik koma, lalu mengeksekusi aksi pemulihan, dan melanjutkan parsing seolah-olah tidak ada masalah.
