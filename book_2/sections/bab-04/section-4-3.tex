\section{Struktur Derivasi dan Ambiguitas}

\subsection{Leftmost vs Rightmost Derivation}
\begin{itemize}
    \item \textbf{Leftmost Derivation}: Selalu memperluas non-terminal paling kiri. Ini yang dilakukan oleh parser \textit{Top-Down} (LL). Parser "menebak" produksi apa yang dipakai sebelum melihat seluruh isi produksinya.
    \item \textbf{Rightmost Derivation}: Selalu memperluas non-terminal paling kanan. Jika urutannya dibalik (\textit{Reverse Rightmost}), ini yang dilakukan parser \textit{Bottom-Up} (LR). Parser menunggu sampai seluruh bagian produksi terlihat (\textit{handle}) baru melakukan reduksi.
\end{itemize}

\subsection{Bahaya Ambiguitas: The Dangling Else}
Ambiguitas terjadi ketika satu string kode memiliki lebih dari satu pohon sintaks yang valid. Kasus klasik adalah \textit{Dangling Else}:
\begin{lstlisting}
if E1 then if E2 then S1 else S2
\end{lstlisting}
Apakah \texttt{else S2} milik \texttt{if E1} atau \texttt{if E2}?
\begin{enumerate}
    \item \textbf{Interpretasi 1}: \texttt{if E1 then (if E2 then S1 else S2)} (Else milik inner if).
    \item \textbf{Interpretasi 2}: \texttt{if E1 then (if E2 then S1) else S2} (Else milik outer if).
\end{enumerate}
Secara konvensi, bahasa pemrograman (C, Java, Pascal) memilih interpretasi 1 (\textit{nearest-match}): else dipasangkan dengan if terdekat yang belum punya pasangan. Parser generator seperti Bison biasanya menyelesaikan konflik ini dengan aturan \textit{Shift over Reduce}.

\begin{figure}[!htbp]
    \centering
    \adjustbox{max width=0.8\textwidth,center}{%
    \begin{tikzpicture}[
        tree/.style={rectangle, draw=red!50, fill=red!10, font=\tiny, align=center},
        arrow/.style={->, >=stealth, thick}
    ]
    \node (in) at (0,0) {3 + 4 * 5};
    \node[tree] (t1) at (-2,-1.5) {Tree A (Precedence Salah):\\(3+4)*5 = 35};
    \node[tree] (t2) at (2,-1.5) {Tree B (Precedence Benar):\\3+(4*5) = 23};
    \draw[arrow] (in) -- (t1);
    \draw[arrow] (in) -- (t2);
    \node[below=2cm of in, font=\footnotesize] {Grammar ambigu $E \rightarrow E+E \mid E*E$ memungkinkan kedua struktur di atas.};
    \end{tikzpicture}%
    }
    \caption{Ambiguitas pada ekspresi aritmatika}
\end{figure}
