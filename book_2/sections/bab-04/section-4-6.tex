\section{Implementasi: Hand-written Recursive Descent}

Penerapan top-down parsing yang paling umum secara manual adalah melalui \textit{Recursive Descent Parser}. Setiap \textit{non-terminal} dalam grammar direpresentasikan oleh satu fungsi C++.

\subsection{Arsitektur Parser}
Fungsi-fungsi parser akan saling memanggil secara rekursif mengikuti struktur grammar. Terminal dicocokkan menggunakan fungsi \texttt{match(tokenType)}.

\subsection{Contoh Implementasi}
Berikut adalah kerangka parser sederhana untuk ekspresi:

\begin{lstlisting}[language=C++]
void Parser::parseE() {
    parseT();
    parseEPrime();
}

void Parser::parseEPrime() {
    if (lookahead == '+') {
        match('+');
        parseT();
        parseEPrime();
    }
    // Epsilon case: do nothing
}
\end{lstlisting}

\begin{figure}[!htbp]
    \centering
    \adjustbox{max width=0.8\textwidth,center}{%
    \begin{tikzpicture}[
        func/.style={rectangle, draw=blue!50, fill=blue!10, font=\tiny, align=center},
        arrow/.style={->, >=stealth, thick}
    ]
    \node[func] (e) {parseE()};
    \node[func, below left=1cm of e] (t) {parseT()};
    \node[func, below right=1cm of e] (ep) {parseE'()};
    \draw[arrow] (e) -- (t);
    \draw[arrow] (e) -- (ep);
    \end{tikzpicture}%
    }
    \caption{Hierarki pemanggilan fungsi Recursive Descent}
\end{figure}
