\section{Parse Tree dan Abstract Syntax Tree (AST)}

\subsection{Parse Tree (Concrete Syntax Tree)}
Parse Tree adalah representasi visual lengkap dari proses derivasi. Ia mencakup semua detail grammar, termasuk simbol-simbol non-terminal perantara yang mungkin tidak relevan untuk eksekusi program.
\begin{itemize}
    \item \textbf{Kelebihan}: Merefleksikan grammar secara presisi.
    \item \textbf{Kekurangan}: Sangat boros memori dan sulit ditraversasi karena terlalu dalam.
\end{itemize}

\subsection{Abstract Syntax Tree (AST)}
AST adalah versi ringkas dan bersih dari Parse Tree. AST membuang semua token sintaksis yang tidak perlu (seperti kurung \texttt{(}, \texttt{)}, titik koma \texttt{;}, keyword \texttt{then}) dan hanya menyimpan struktur logis operasi.
\begin{itemize}
    \item \textbf{Simpul Dalam}: Operator atau struktur kontrol (\texttt{+}, \texttt{if}, \texttt{while}).
    \item \textbf{Daun}: Operan atau nilai (\texttt{3.14}, \texttt{counter}).
\end{itemize}

\begin{figure}[!htbp]
    \centering
    \adjustbox{max width=0.8\textwidth,center}{%
    \begin{tikzpicture}[
        node/.style={circle, draw=blue!50, fill=blue!10, minimum size=0.8cm, font=\small},
        op/.style={rectangle, draw=red!50, fill=red!10, minimum size=0.8cm, font=\small}
    ]
    % AST for 3 + 4 * 5
    \node[op] (plus) at (0,0) {+};
    \node[node] (n3) at (-1.5,-1.5) {3};
    \node[op] (mul) at (1.5,-1.5) {*};
    \node[node] (n4) at (0.5,-3) {4};
    \node[node] (n5) at (2.5,-3) {5};
    
    \draw[->, thick] (plus) -- (n3);
    \draw[->, thick] (plus) -- (mul);
    \draw[->, thick] (mul) -- (n4);
    \draw[->, thick] (mul) -- (n5);
    
    \node[right=3cm of plus, align=left, font=\footnotesize] {
        \textbf{Struktur Node}:\\
        \texttt{struct Expr \{ ... \};}\\
        \texttt{struct BinOp : Expr \{} \\
        \texttt{  Expr* left;}\\
        \texttt{  Token op;}\\
        \texttt{  Expr* right;}\\
        \texttt{\};}
    };
    \end{tikzpicture}%
    }
    \caption{AST untuk ungkapan $3 + 4 * 5$}
\end{figure}
