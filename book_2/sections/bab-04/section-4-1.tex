\section{Dasar Syntax Analysis dan CFG}

\subsection{Peran Parser}
Parser adalah jantung dari front-end kompilator. Jika Lexer mengolah ''kata per kata'', Parser mengolah ''kalimat per kalimat'' \cite{ittrip2024bison}. Parser mengubah deretan token datar menjadi struktur hierarkis (pohon) yang merepresentasikan struktur gramatikal program.

\subsection{Context-Free Grammar (CFG)}
Bahasa pemrograman umumnya didefinisikan menggunakan Context-Free Grammar (CFG). CFG cukup kuat untuk mengekspresikan struktur bersarang (\textit{nested structures}) seperti kurung \texttt{((...))} dan blok \texttt{\{...\}} yang tidak bisa ditangani oleh Regular Expression \cite{aho2006compilers}.

Secara formal, CFG didefinisikan sebagai 4-tuple $G = (V, \Sigma, R, S)$:
\begin{itemize}
    \item $V$ (\textit{Non-terminals}): Simbol variabel yang bisa diturunkan lebih lanjut (misal: \textit{Statement}, \textit{Expression}).
    \item $\Sigma$ (\textit{Terminals}): Simbol dasar atau token dari lexer (misal: \texttt{ID}, \texttt{IF}, \texttt{+}).
    \item $R$ (\textit{Production Rules}): Aturan substitusi berbentuk $A \rightarrow \alpha$, di mana $A \in V$ dan $\alpha \in (V \cup \Sigma)^*$.
    \item $S$ (\textit{Start Symbol}): Non-terminal awal dimulainya derivasi.
\end{itemize}

\subsection{Derivasi: Dari Simbol ke String}
Derivasi adalah proses penggantian non-terminal secara beruntun hingga menghasilkan string terminal.
Contoh Grammar: $E \rightarrow E + E \mid \textbf{id}$
Derivasi string "\textbf{id} + \textbf{id}":
\begin{enumerate}
    \item $E$ (Start)
    \item $\Rightarrow E + E$ (Gunakan aturan $E \rightarrow E+E$)
    \item $\Rightarrow \textbf{id} + E$ (Ganti $E$ pertama dengan \textbf{id})
    \item $\Rightarrow \textbf{id} + \textbf{id}$ (Ganti $E$ kedua dengan \textbf{id})
\end{enumerate}
