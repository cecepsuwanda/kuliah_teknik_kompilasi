\section{Top-Down Parsing dan LL(1)}

\subsection{Recursive Descent Parser}
Merupakan parser top-down yang menggunakan sekumpulan prosedur rekursif untuk mengenali struktur bahasa pemrograman.

\subsection{Eliminasi Masalah Grammar}
Recursive descent tidak dapat menangani \textbf{Left Recursion} dan seringkali membutuhkan \textbf{Left Factoring} agar keputusan lookahead menjadi deterministik.

\begin{figure}[!htbp]
    \centering
    \adjustbox{max width=0.8\textwidth,center}{%
    \begin{tikzpicture}[
        box/.style={rectangle, draw=blue!50, fill=blue!10, font=\tiny, align=center},
        arrow/.style={->, >=stealth, thick}
    ]
    \node[box] (orig) {E → E + T | T};
    \node[box, right=2cm of orig] (new) {E → T E' \\ E' → + T E' | ε};
    \draw[arrow] (orig) -- node[above, font=\tiny] {Eliminate LR} (new);
    \end{tikzpicture}%
    }
    \caption{Transformasi eliminasi Left Recursion}
\end{figure}
