\section{Top-Down Parsing dan LL(1)}

\subsection{Recursive Descent Parser}
Metode ini paling intuitif untuk ditulis tangan. Idenya sederhana: \textit{setiap non-terminal dalam grammar menjadi satu fungsi dalam kode program} \cite{gnu2024bison}. Jika grammar memiliki aturan $S \rightarrow a B c$, maka fungsi \texttt{parseS()} akan memanggil \texttt{match('a')}, lalu \texttt{parseB()}, dan akhirnya \texttt{match('c')}.

\subsection{Masalah pada Grammar Top-Down}
Agar Recursive Descent bekerja, grammar harus memenuhi syarat \textbf{LL(1)}: membaca input dari Kiri (\textbf{L}eft), menghasilkan derivasi Kiri (\textbf{L}eftmost), dengan \textbf{1} token lookahead. Dua musuh utama LL(1) adalah:

\subsubsection{1. Left Recursion}
Aturan seperti $A \rightarrow A\alpha$ akan menyebabkan fungsi \texttt{parseA()} memanggil dirinya sendiri tanpa henti (\textit{infinite loop}) sebelum membaca input apa pun.
\textbf{Solusi:} Ubah menjadi \textit{Right Recursion}.
\[ A \rightarrow A\alpha \mid \beta \quad \Longrightarrow \quad A \rightarrow \beta A', \quad A' \rightarrow \alpha A' \mid \epsilon \]

\subsubsection{2. Common Prefix (Butuh Left Factoring)}
Jika parser melihat aturan $S \rightarrow \textbf{if} E \textbf{then} S \mid \textbf{if} E \textbf{then} S \textbf{else} S$, parser bingung memilih cabang mana saat melihat token \textbf{if}.
\textbf{Solusi:} Faktorkan prefiks yang sama keluar.
\[ A \rightarrow \alpha\beta_1 \mid \alpha\beta_2 \quad \Longrightarrow \quad A \rightarrow \alpha A', \quad A' \rightarrow \beta_1 \mid \beta_2 \]

\begin{figure}[!htbp]
    \centering
    \adjustbox{max width=0.9\textwidth,center}{%
    \begin{tikzpicture}[
        box/.style={rectangle, draw=blue!50, fill=blue!10, font=\small, align=center, rounded corners, minimum height=1cm},
        arrow/.style={->, >=stealth, thick}
    ]
    \node[box] (orig) {\textbf{Masalah: Left Recursion}\\$E \rightarrow E + T \mid T$};
    \node[box, right=1.5cm of orig] (fix) {\textbf{Solusi: Right Recursion}\\$E \rightarrow T E'$\\$E' \rightarrow + T E' \mid \epsilon$};
    \draw[arrow] (orig) -- node[above, font=\footnotesize] {Eliminasi} (fix);
    
    \node[below=0.5cm of orig, font=\itshape\footnotesize] {Menyebabkan Infinite Loop};
    \node[below=0.5cm of fix, font=\itshape\footnotesize] {Aman untuk Recursive Descent};
    \end{tikzpicture}%
    }
    \caption{Transformasi Grammar untuk LL(1)}
\end{figure}
