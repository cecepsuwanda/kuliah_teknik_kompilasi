\section{Kesimpulan}

Dalam bab ini, kita telah mempelajari:

\begin{enumerate}
    \item Abstract Syntax Tree (AST) adalah representasi abstrak dari struktur program yang menghilangkan detail sintaksis yang tidak relevan
    
    \item AST terdiri dari berbagai node types: expression nodes, statement nodes, declaration nodes, dan program node
    
    \item Implementasi AST dalam C++ menggunakan inheritance dan virtual functions, dengan smart pointers untuk memory management
    
    \item Visitor pattern memisahkan algoritma traversal dari struktur node, memungkinkan ekstensibilitas tanpa modifikasi node classes
    
    \item Tree traversal dapat dilakukan dengan pre-order, post-order, atau in-order, masing-masing berguna untuk tujuan berbeda
    
    \item Parser dapat dibangun untuk menghasilkan AST selama proses parsing menggunakan semantic actions
\end{enumerate}

Pemahaman tentang AST sangat penting karena AST menjadi input untuk fase-fase selanjutnya: semantic analysis, optimization, dan code generation.