\section{Generasi Kode Antara (Three-Address Code)}

Representasi Antara atau \textit{Intermediate Representation} (IR) menyederhanakan proses optimasi dan portabilitas antar arsitektur.

\subsection{Format Three-Address Code (TAC)}
TAC memecah ekspresi kompleks menjadi urutan instruksi sederhana dengan maksimal tiga operand.
Contoh untuk \code{x = a + b * c}:
\begin{lstlisting}
t1 = b * c
t2 = a + t1
x = t2
\end{lstlisting}

\subsection{Pembangkitan TAC dari AST}
Fungsi rekursif berjalan menyusuri AST dan menghasilkan instruksi TAC:

\begin{lstlisting}[language=C]
char* generateTAC(ASTNode *node) {
    if (node->type == NODE_NUMBER) {
        return intToString(node->value);
    }
    if (node->type == NODE_BINOP) {
        char *left = generateTAC(node->left);
        char *right = generateTAC(node->right);
        char *temp = nextTemp();
        printf("%s = %s %c %s\n", temp, left, node->op, right);
        return temp;
    }
    // ... handling other nodes
}
\end{lstlisting}

Dengan IR, kita memiliki abstraksi yang bersih sebelum masuk ke detail instruksi mesin yang sangat spesifik.
