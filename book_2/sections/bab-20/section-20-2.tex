\section{Analisis Sintaksis dan Abstract Syntax Tree (Bison)}

Setelah token diidentifikasi, kita perlu memahami struktur program dan membangun \textit{Abstract Syntax Tree} (AST) \cite{ittrip2024bison}.

\subsection{Struktur Data AST (ast.h)}
Definisikan node pohon untuk mewakili konstruksi bahasa pemrograman.

\begin{lstlisting}[language=C]
typedef enum { NODE_PRINT, NODE_ASSIGN, NODE_NUMBER, NODE_ID, NODE_BINOP } NodeType;

struct ASTNode {
    NodeType type;
    int value;           // Untuk NUMBER
    char *name;          // Untuk ID
    char op;             // Untuk BINOP (+, -)
    struct ASTNode *left;
    struct ASTNode *right;
};
\end{lstlisting}

\subsection{Grammar Bison (parser.y)}
Aturan tata bahasa menentukan bagaimana token disusun menjadi pernyataan.

\begin{lstlisting}[language=C]
%token PRINT NUMBER ID IF ELSE WHILE PLUS MINUS LBRACE RBRACE LPAREN RPAREN SEMI ASSIGN
%%
program:
    program statement
    | /* empty */
    ;

statement:
    PRINT LPAREN expression RPAREN SEMI { 
        $$ = createNode(NODE_PRINT, $3, NULL); 
    }
    | ID ASSIGN expression SEMI { 
        $$ = createNode(NODE_ASSIGN, createIDNode($1), $3); 
    }
    ;

expression:
    NUMBER { $$ = createNumNode($1); }
    | ID { $$ = createIDNode($1); }
    | expression PLUS expression { $$ = createOpNode('+', $1, $3); }
    ;
%%
\end{lstlisting}

Bison akan memanggil fungsi \code{yyparse()} yang secara rekursif membangun AST selama proses pembacaan token.
