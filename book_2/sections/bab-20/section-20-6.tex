\section{Integrasi dan Pengujian Akhir}

Untuk menyatukan semua komponen, kita memerlukan sistem build otomatis menggunakan \code{Makefile}.

\subsection{Otomasi Build (Makefile)}
\begin{lstlisting}[language=sh]
all: compiler

compiler: lex.yy.c parser.tab.c main.c
	gcc lex.yy.c parser.tab.c main.c -o my_compiler

lex.yy.c: lex_spec.l
	flex lex_spec.l

parser.tab.c: parser.y
	bison -d parser.y

clean:
	rm -f my_compiler lex.yy.c parser.tab.h parser.tab.c
\end{lstlisting}

\subsection{Alur Pengujian (End-to-End)}
Ikuti langkah berikut untuk menguji kompilator buatan Anda:

\begin{enumerate}
    \item Buat file sumber \code{test.c}: \code{int x = 10; print(x + 5);}
    \item Jalankan kompilator: \code{./my\_compiler test.c > out.asm}
    \item Assembling dengan NASM: \code{nasm -f elf64 out.asm -o out.o}
    \item Linking dengan GCC: \code{gcc out.o -o program}
    \item Eksekusi: \code{./program} (Output seharusnya \code{15}).
\end{enumerate}

\textbf{Selamat!} Anda telah berhasil membangun kompilator sederhana dari nol hingga menghasilkan biner yang dapat dieksekusi.
