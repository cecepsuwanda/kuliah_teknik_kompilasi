\section{Arsitektur dan Analisis Leksikal (Flex)}

Dalam tutorial ini, kita akan membangun kompilator untuk \textit{Subset C} yang mendukung operasi aritmatika dasar dan fungsi \code{print}.

\subsection{Spesifikasi Bahasa}
Bahasa target kita mendukung:
\begin{itemize}
    \item Tipe data: \code{int}.
    \item Kontrol aliran: \code{if-else} dan \code{while}.
    \item Fungsi bawaan: \code{print(ekspresi)}.
    \item Komentar: Gaya C (\code{//}).
\end{itemize}

\subsection{Implementasi Lexer (lex\_spec.l)}
Gunakan alat \textit{Flex} untuk mendefinisikan token. Berikut adalah cuplikan file spesifikasi leksikalnya:

\begin{lstlisting}[language=C]
%{
#include "parser.tab.h"
%}

%%
"print"      { return PRINT; }
"if"         { return IF; }
"else"       { return ELSE; }
"while"      { return WHILE; }
[0-9]+       { yylval.num = atoi(yytext); return NUMBER; }
[a-zA-Z_][a-zA-Z0-9_]* { yylval.id = strdup(yytext); return ID; }
"+"          { return PLUS; }
"-"          { return MINUS; }
"{"          { return LBRACE; }
"}"          { return RBRACE; }
"("          { return LPAREN; }
")"          { return RPAREN; }
";"          { return SEMI; }
"="          { return ASSIGN; }
[ \t\n]      ; // Abaikan whitespace
.            { printf("Error: %s\n", yytext); }
%%

int yywrap() { return 1; }
\end{lstlisting}

Lexer ini akan mengubah aliran karakter mentah menjadi urutan token yang dapat diproses oleh parser pada tahap berikutnya.
