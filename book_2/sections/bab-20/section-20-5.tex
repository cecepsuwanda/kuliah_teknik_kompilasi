\section{Generasi Kode Target (Assembly x86-64)}

Tahap akhir adalah menerjemahkan IR (TAC) menjadi bahasa rakitan yang dapat diproses oleh assembler.

\subsection{Pemetaan TAC ke x86-64}
Gunakan register \code{RAX} sebagai akumulator utama.

\begin{lstlisting}[language=C]
void emitAssembly(char *op, char *src, char *dest) {
    if (strcmp(op, "+") == 0) {
        printf("  mov rax, [%s]\n", src);
        printf("  add rax, [%s]\n", dest);
        printf("  mov [temp], rax\n");
    }
}
\end{lstlisting}

\subsection{Implementasi Fungsi Print}
Untuk mencetak output, kita memanggil fungsi \code{printf} dari pustaka standar C melalui instruksi \code{CALL}.

\begin{lstlisting}
; Assembly NASM x86-64
extern printf
section .data
    fmt db "%d", 10, 0

section .text
global main
main:
    push rbp        ; Align stack
    mov rdi, fmt    ; Parameter 1
    mov rsi, [rax]  ; Parameter 2 (hasil perhitungan)
    call printf     ; Cetak ke layar
    pop rbp
    ret
\end{lstlisting}

Instruksi \code{push rbp} dan \code{pop rbp} sangat penting untuk menjaga perataan stack (\textit{stack alignment}) 16-byte sesuai standar ABI x86-64.
