\section{Shift-Reduce Parsing}

\subsection{Konsep Shift-Reduce}

Gambar \ref{fig:shift-reduce-operations} menunjukkan empat operasi dasar dalam shift-reduce parsing.

\begin{figure}[!htbp]
    \centering
    \adjustbox{max width=0.9\textwidth,center}{%
    \begin{tikzpicture}[
        op/.style={rectangle, draw=blue!50, fill=blue!10, text width=2.5cm, minimum height=0.7cm, font=\footnotesize, align=center, rounded corners, inner sep=4pt},
        arrow/.style={->, >=stealth, thick},
        node distance=1.2cm
    ]
    
    \node[op] (shift) {SHIFT\\Push token};
    \node[op, right=of shift] (reduce) {REDUCE\\Apply rule};
    \node[op, right=of reduce] (accept) {ACCEPT\\Success};
    \node[op, below=of reduce] (error) {ERROR\\Invalid};
    
    \draw[arrow] (shift) to[out=45, in=135] (reduce);
    \draw[arrow] (reduce) to[out=45, in=135] (accept);
    \draw[arrow] (reduce) to[out=-45, in=45] (error);
    
    \end{tikzpicture}%
    }
    \caption{Operasi-operasi dalam shift-reduce parsing}
    \label{fig:shift-reduce-operations}
\end{figure}

Shift-reduce parsing adalah implementasi dasar dari bottom-up parsing yang menggunakan stack dan empat operasi dasar. Menurut GeeksforGeeks:

\begin{quote}
``Shift-Reduce Parser uses a stack and four basic operations:
1. Shift: push the next input symbol onto the stack.
2. Reduce: when the top of stack matches RHS of some grammar rule, pop it and push the LHS nonterminal.
3. Accept: if the stack has start symbol and input is exhausted.
4. Error: no valid shift/reduce possible.''\footnote{\url{https://www.geeksforgeeks.org/compiler-design/shift-reduce-parser-compiler/}}
\end{quote}

\subsection{Operasi Shift}

Operasi \textbf{shift} memindahkan token berikutnya dari input ke stack. Ini dilakukan ketika parser belum menemukan handle yang lengkap di stack.

Contoh: Jika stack berisi \texttt{[E, +]} dan input berikutnya adalah \texttt{id}, maka operasi shift akan menghasilkan stack \texttt{[E, +, id]}.

\subsection{Operasi Reduce}

Operasi \textbf{reduce} mengganti handle di top of stack dengan left-hand side (LHS) dari production rule yang sesuai. Handle harus cocok persis dengan RHS dari suatu production.

Contoh: Jika stack berisi \texttt{[E, +, T, *, F]} dan kita memiliki production \texttt{F -> id}, dan top of stack adalah \texttt{id} yang cocok dengan RHS, maka reduce akan menghasilkan stack \texttt{[E, +, T, *, F]}.

\subsection{Operasi Accept}

Operasi \textbf{accept} terjadi ketika:
\begin{itemize}
    \item Stack hanya berisi start symbol (atau augmented start symbol)
    \item Input sudah habis (hanya end marker \$ tersisa)
\end{itemize}

Ini menandakan bahwa parsing berhasil dan input valid.

\subsection{Operasi Error}

Operasi \textbf{error} terjadi ketika tidak ada operasi shift atau reduce yang valid. Ini berarti input tidak valid menurut grammar.

\subsection{Contoh Shift-Reduce Parsing}

Mari kita lihat contoh parsing ekspresi \texttt{id + id} dengan grammar sederhana:

\begin{verbatim}
E -> E + T | T
T -> id
\end{verbatim}

\begin{table}[!htbp]
\centering
\begin{tabular}{|c|c|c|c|}
\hline
\textbf{Stack} & \textbf{Input} & \textbf{Action} & \textbf{Production} \\
\hline
\$ & id + id \$ & Shift & \\
\hline
\$ id & + id \$ & Reduce & T -> id \\
\hline
\$ T & + id \$ & Reduce & E -> T \\
\hline
\$ E & + id \$ & Shift & \\
\hline
\$ E + & id \$ & Shift & \\
\hline
\$ E + id & \$ & Reduce & T -> id \\
\hline
\$ E + T & \$ & Reduce & E -> E + T \\
\hline
\$ E & \$ & Accept & \\
\hline
\end{tabular}
\caption{Contoh shift-reduce parsing untuk \texttt{id + id}}
\label{tab:shift-reduce-example}
\end{table}