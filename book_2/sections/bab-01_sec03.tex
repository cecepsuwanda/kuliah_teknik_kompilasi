\section{Alur Kerja Kompilator: Gambaran Umum}

Sebelum membahas detail arsitektur dan fase-fase kompilasi, mari kita lihat gambaran umum alur kerja kompilator dari source code hingga executable. Gambar \ref{fig:compiler-flow} menunjukkan alur kerja kompilator secara keseluruhan:

\begin{figure}[!htbp]
\centering
\adjustbox{max width=0.92\textwidth,center}{%
\begin{tikzpicture}[
    process/.style={rectangle, draw=blue!50, fill=blue!10, text width=2.5cm, text centered, minimum height=0.6cm, rounded corners, font=\footnotesize},
    output/.style={rectangle, draw=green!50, fill=green!10, text width=2cm, text centered, minimum height=0.5cm, rounded corners, font=\tiny},
    arrow/.style={->, >=stealth, thick}
]
    % Source code
    \node[process] (source) {\textbf{Source Code}\\\footnotesize(C/C++, Java, dll.)};
    
    % Preprocessing
    \node[process, below=0.7cm of source] (preproc) {\textbf{Preprocessing}\\\footnotesize(\#include, \#define)};
    
    % Lexical
    \node[process, below=0.65cm of preproc] (lexical) {\textbf{Analisis Leksikal}};
    \node[output, right=1cm of lexical] (tokens) {Tokens};
    
    % Syntax
    \node[process, below=0.65cm of lexical] (syntax) {\textbf{Analisis Sintaksis}};
    \node[output, right=1cm of syntax] (ast) {Parse Tree\\AST};
    
    % Semantic
    \node[process, below=0.65cm of syntax] (semantic) {\textbf{Analisis Semantik}};
    \node[output, right=1cm of semantic] (annotated) {Annotated AST\\+ Symbol Table};
    
    % IR Generation
    \node[process, below=0.65cm of semantic] (irgen) {\textbf{Generasi IR}};
    \node[output, right=1cm of irgen] (ir) {Representasi\\Intermediate};
    
    % Optimization
    \node[process, below=0.65cm of irgen] (optimize) {\textbf{Optimasi}};
    \node[output, right=1cm of optimize] (optir) {IR Teroptimasi};
    
    % Code Generation
    \node[process, below=0.65cm of optimize] (codegen) {\textbf{Generasi Kode}};
    \node[output, right=1cm of codegen] (asm) {Assembly\\Machine Code};
    
    % Assembling
    \node[process, below=0.65cm of codegen] (assemble) {\textbf{Assembling}};
    \node[output, right=1cm of assemble] (obj) {Object Files};
    
    % Linking
    \node[process, below=0.65cm of assemble] (link) {\textbf{Linking}};
    \node[output, right=1cm of link] (exe) {\textbf{Executable}};
    
    % Arrows
    \draw[arrow] (source) -- (preproc);
    \draw[arrow] (preproc) -- (lexical);
    \draw[arrow] (lexical) -- (tokens);
    \draw[arrow] (lexical) -- (syntax);
    \draw[arrow] (syntax) -- (ast);
    \draw[arrow] (syntax) -- (semantic);
    \draw[arrow] (semantic) -- (annotated);
    \draw[arrow] (semantic) -- (irgen);
    \draw[arrow] (irgen) -- (ir);
    \draw[arrow] (irgen) -- (optimize);
    \draw[arrow] (optimize) -- (optir);
    \draw[arrow] (optimize) -- (codegen);
    \draw[arrow] (codegen) -- (asm);
    \draw[arrow] (codegen) -- (assemble);
    \draw[arrow] (assemble) -- (obj);
    \draw[arrow] (assemble) -- (link);
    \draw[arrow] (link) -- (exe);
\end{tikzpicture}%
}
\caption{Alur kerja kompilator dari source code ke executable}
\label{fig:compiler-flow}
\end{figure}

Dari gambar di atas, dapat dilihat bahwa proses kompilasi melibatkan beberapa tahap utama:
\begin{enumerate}
    \item \textbf{Preprocessing}: Memproses directive khusus sebelum kompilasi
    \item \textbf{Analisis} (Front-end): Analisis Leksikal (Lexical Analysis), Analisis Sintaksis (Syntax Analysis), dan Analisis Semantik (Semantic Analysis)
    \item \textbf{Sintesis} (Back-end): Generasi IR (IR Generation), Optimasi (Optimization), dan Generasi Kode (Code Generation)
    \item \textbf{Assembling dan Linking}: Mengubah assembly menjadi executable
\end{enumerate}\par

\subsection{Preprocessing}

Sebelum kompilasi dimulai, preprocessor memproses directive khusus seperti:
\begin{itemize}
    \item \texttt{\#include}: Menyisipkan konten file header
    \item \texttt{\#define}: Makro definisi
    \item \texttt{\#ifdef}, \texttt{\#ifndef}: Conditional compilation
\end{itemize}

\subsection{Assembling dan Linking}

Setelah Generasi Kode (Code Generation), assembler mengubah assembly code menjadi object code (file .o atau .obj). Linker kemudian menyatukan:
\begin{itemize}
    \item Object files dari source code yang dikompilasi
    \item Library files (static atau dynamic libraries)
    \item Startup code
\end{itemize}

Menjadi satu executable file yang siap dieksekusi.
