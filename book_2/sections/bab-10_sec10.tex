\section{Kesimpulan}

Dalam bab ini, kita telah mempelajari:

\begin{enumerate}
    \item Symbol table adalah struktur data penting yang memetakan identifier ke informasi deklarasinya
    \item Hash table adalah implementasi efisien untuk symbol table dengan waktu akses O(1) rata-rata
    \item Nested scopes diimplementasikan menggunakan stack of hash tables, di mana setiap scope memiliki parent pointer
    \item Name resolution mengikuti aturan static scoping: pencarian dimulai dari scope saat ini dan naik ke enclosing scopes
    \item Shadowing terjadi ketika identifier dalam scope dalam memiliki nama yang sama dengan identifier di scope luar
    \item Scope entry/exit harus ditangani dengan benar untuk berbagai konstruk bahasa (function, block, loop)
    \item Visualisasi symbol table membantu dalam debugging dan pembelajaran
\end{enumerate}

Pemahaman tentang symbol table dan scope management adalah dasar penting untuk implementasi semantic analysis yang akan dibahas dalam bab selanjutnya.