\section{Praktikum: Membuat Lexer dengan Flex}

\subsection{Tugas Praktikum}

Buatlah lexer menggunakan Flex untuk bahasa mini dengan minimal 10 token types:

\begin{enumerate}
    \item \textbf{Keywords}: \texttt{if}, \texttt{else}, \texttt{while}, \texttt{int}, \texttt{float}, \texttt{return}
    \item \textbf{Identifiers}: Nama variabel dan fungsi
    \item \textbf{Literals}: Integer dan float numbers
    \item \textbf{Operators}: \texttt{+}, \texttt{-}, \texttt{*}, \texttt{/}, \texttt{=}, \texttt{==}, \texttt{!=}, \texttt{<}, \texttt{>}
    \item \textbf{Punctuation}: \texttt{(}, \texttt{)}, \texttt{\{}, \texttt{\}}, \texttt{;}, \texttt{,}
    \item \textbf{Comments}: Single-line (\texttt{//}) dan multi-line (\texttt{/* */})
\end{enumerate}

\subsection{Langkah-langkah}

\begin{enumerate}
    \item Buat file \texttt{lexer.l} dengan specification sesuai requirement
    \item Generate lexer: \texttt{flex lexer.l}
    \item Buat program test sederhana yang menggunakan \texttt{yylex()}
    \item Test dengan berbagai input (valid dan invalid)
    \item Dokumentasikan token types dan behavior lexer
\end{enumerate}

\subsection{Expected Output}

Lexer harus dapat:
\begin{itemize}
    \item Mengenali semua token types yang didefinisikan
    \item Menangani whitespace dan comments dengan benar
    \item Memberikan error message yang informatif untuk invalid input
    \item Melacak line number untuk error reporting
\end{itemize}

Gambar \ref{fig:flex-error-handling} menunjukkan contoh error handling dalam Flex.

\begin{figure}[!htbp]
    \centering
    \adjustbox{max width=0.85\textwidth,center}{%
    \begin{tikzpicture}[
        input/.style={rectangle, draw=red!50, fill=red!10, text width=5cm, minimum height=0.6cm, font=\footnotesize\ttfamily, align=left, inner sep=4pt},
        error/.style={rectangle, draw=red!50, fill=red!20, text width=5cm, minimum height=0.6cm, font=\tiny, align=left, inner sep=4pt, rounded corners},
        arrow/.style={->, >=stealth, thick, red},
        node distance=0.5cm
    ]
    
    \node[input] (in) {Input: \texttt{int x @ y;}};
    \node[error, below=of in] (err) {Error: unexpected character '@'\\at line 1, column 7};
    
    \draw[arrow] (in) -- (err);
    
    \node[below=0.3cm of err, font=\tiny, align=center] {Flex dapat menangani\\error dengan memberikan\\informasi posisi yang tepat};
    
    \end{tikzpicture}%
    }
    \caption{Error handling dalam Flex}
    \label{fig:flex-error-handling}
\end{figure}

Gambar \ref{fig:flex-comment-handling} menunjukkan bagaimana Flex menangani komentar multi-line.

\begin{figure}[!htbp]
    \centering
    \adjustbox{max width=0.85\textwidth,center}{%
    \begin{tikzpicture}[
        code/.style={rectangle, draw=gray!30, fill=gray!5, text width=7cm, minimum height=0.6cm, font=\footnotesize\ttfamily, align=left, inner sep=4pt},
        comment/.style={rectangle, draw=green!50, fill=green!10, text width=7cm, minimum height=0.6cm, font=\footnotesize\ttfamily, align=left, inner sep=4pt},
        arrow/.style={->, >=stealth, thick, green},
        node distance=0.4cm
    ]
    
    \node[code] (in) {\texttt{int x = 42; /* comment */ int y;}};
    \node[comment, below=of in] (out) {\texttt{int x = 42; [SKIPPED] int y;}};
    
    \draw[arrow] (in) -- node[right, font=\tiny] {Flex skips} (out);
    
    \node[below=0.3cm of out, font=\tiny, align=center] {Komentar multi-line di-skip\\oleh Flex menggunakan start conditions};
    
    \end{tikzpicture}%
    }
    \caption{Handling komentar multi-line dalam Flex}
    \label{fig:flex-comment-handling}
\end{figure}

Gambar \ref{fig:real-world-usage} menunjukkan penggunaan Flex dan re2c dalam project nyata.

\begin{figure}[!htbp]
    \centering
    \adjustbox{max width=0.9\textwidth,center}{%
    \begin{tikzpicture}[
        project/.style={rectangle, draw=blue!50, fill=blue!10, text width=3cm, minimum height=0.8cm, font=\footnotesize, align=center, rounded corners, inner sep=4pt},
        tool/.style={rectangle, draw=green!50, fill=green!10, text width=2.5cm, minimum height=0.6cm, font=\tiny, align=center, rounded corners, inner sep=4pt},
        arrow/.style={->, >=stealth, thick},
        node distance=0.6cm and 0.3cm
    ]
    
    % Projects using Flex
    \node[project] (p1) {GCC\\Compiler};
    \node[tool, below=of p1] (t1) {Flex};
    
    \node[project, right=of p1] (p2) {PostgreSQL};
    \node[tool, below=of p2] (t2) {Flex};
    
    \node[project, right=of p2] (p3) {PHP};
    \node[tool, below=of p3] (t3) {Flex};
    
    % Projects using re2c
    \node[project, below=1.5cm of p1] (p4) {PHP 7+};
    \node[tool, below=of p4] (t4) {re2c};
    
    \node[project, right=of p4] (p5) {MariaDB};
    \node[tool, below=of p5] (t5) {re2c};
    
    \node[project, right=of p5] (p6) {Ragel};
    \node[tool, below=of p6] (t6) {re2c};
    
    \draw[arrow] (p1) -- (t1);
    \draw[arrow] (p2) -- (t2);
    \draw[arrow] (p3) -- (t3);
    \draw[arrow] (p4) -- (t4);
    \draw[arrow] (p5) -- (t5);
    \draw[arrow] (p6) -- (t6);
    
    \end{tikzpicture}%
    }
    \caption{Contoh penggunaan Flex dan re2c dalam project nyata}
    \label{fig:real-world-usage}
\end{figure}