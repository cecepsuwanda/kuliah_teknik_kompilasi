\section{Precedence dan Associativity}

Untuk menangani precedence dan associativity tanpa membuat grammar yang ambigu, Bison menyediakan deklarasi khusus:

\subsection{Deklarasi Precedence}

\begin{verbatim}
%left '+' '-'      // Left-associative, precedence rendah
%left '*' '/'      // Left-associative, precedence tinggi
%right '^'         // Right-associative (exponentiation)
%nonassoc '<' '>'  // Non-associative (comparison)
\end{verbatim}

Urutan deklarasi menentukan precedence: yang dideklarasikan terakhir memiliki precedence tertinggi.

\subsection{Contoh Penggunaan}

\begin{lstlisting}[language=C, caption={Grammar dengan precedence}]
%left PLUS MINUS
%left MULTIPLY DIVIDE
%right POWER

%%
expr:
    NUMBER
  | expr PLUS expr    // Precedence rendah
  | expr MINUS expr
  | expr MULTIPLY expr // Precedence tinggi
  | expr DIVIDE expr
  | expr POWER expr   // Precedence tertinggi, right-associative
  ;
\end{lstlisting}

Dengan deklarasi ini, ekspresi \texttt{2 + 3 * 4} akan di-parse sebagai \texttt{2 + (3 * 4)}, dan ekspresi \verb|2^3^4| akan di-parse sebagai \verb|2^(3^4)|.