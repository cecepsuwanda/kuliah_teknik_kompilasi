\section{Layout Memori Array Multidimensi}

Penyimpanan array multidimensi dalam memori linear (satu dimensi) memerlukan pemetaan indeks yang konsisten. Ada dua pendekatan utama yang digunakan oleh berbagai bahasa pemrograman.

\subsection{Row-Major vs Column-Major}
\begin{itemize}
    \item \textbf{Row-Major Order} (digunakan di C, C++, Java): Elemen-elemen dalam satu baris disimpan secara berturutan. Indeks paling kanan berubah paling cepat.
    \item \textbf{Column-Major Order} (digunakan di Fortran, MATLAB, Julia): Elemen-elemen dalam satu kolom disimpan secara berturutan. Indeks paling kiri berubah paling cepat.
\end{itemize}

Rumus pemetaan alamat untuk \code{A[i][j]} pada array $M \times N$:
\begin{itemize}
    \item \textbf{Row-Major}: $\text{Base}(A) + (i \times N + j) \times \text{ukuran\_elemen}$
    \item \textbf{Column-Major}: $\text{Base}(A) + (j \times M + i) \times \text{ukuran\_elemen}$
\end{itemize}

\subsection{Larik Petunjuk (Iliffe Vectors)}
Alih-alih menyatukan baris dalam satu blok kontinu, beberapa bahasa (seperti Python untuk \textit{list of lists}) menggunakan \textit{Iliffe Vectors}.
\begin{itemize}
    \item Array utama berisi sekumpulan pointer. 
    \item Setiap pointer merujuk ke blok memori baris yang terpisah.
    \item \textbf{Keuntungan}: Mendukung \textit{Jagged Arrays} (panjang baris bisa berbeda-beda).
    \item \textbf{Kerugian}: Memerlukan akses memori ganda (dereferensi pointer baris, baru kemudian ambil datanya).
\end{itemize}

\subsection{Rumus Umum N-Dimensi (Row-Major)}
Untuk array $A[d_1][d_2]...[d_n]$, alamat elemen $A[i_1][i_2]...[i_n]$ dihitung sebagai:
\[ \text{Offset} = i_1 \cdot (d_2 \cdot d_3 \cdots d_n) + i_2 \cdot (d_3 \cdot d_4 \cdots d_n) + \cdots + i_n \]

\begin{figure}[!htbp]
    \centering
    \adjustbox{max width=0.8\textwidth,center}{%
    \begin{tikzpicture}[
        node/.style={rectangle, draw=blue!50, fill=blue!10, text width=2cm, font=\tiny, align=center}
    ]
    \node[node] (c1) {Contiguous (C/C++)};
    \node[node, right=1cm of c1] (i1) {Iliffe (Java/JS)};
    \draw[->, >=stealth] (c1) edge node[above, font=\tiny] {Static} (c1);
    \draw[->, >=stealth] (i1) edge node[above, font=\tiny] {Jagged} (i1);
    \end{tikzpicture}%
    }
    \caption{Visualisasi Struktur Kontinu vs Layar Petunjuk}
\end{figure}
