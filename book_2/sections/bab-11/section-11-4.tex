\section{Layout Struktur dan Perataan Memori}

Kompilator tidak selalu menyimpan anggota struktur secara berdampingan tanpa celah. Ada aturan perangkat keras yang memaksa data tertentu berada di alamat yang "selaras" (\textit{aligned}).

\subsection{Perataan Memori (Memory Alignment)}
Kebanyakan prosesor modern bekerja paling efisien jika data diakses pada alamat yang merupakan kelipatan dari ukuran data tersebut.
\begin{itemize}
    \item \textbf{Natural Alignment}: \texttt{int} (4 byte) harus berada di alamat yang habis dibagi 4; \texttt{double} (8 byte) di alamat yang habis dibagi 8.
    \item Jika data tidak selaras (\textit{misaligned}), CPU mungkin perlu melakukan dua kali akses memori untuk satu data, atau bahkan menyebabkan \textit{exception}.
\end{itemize}

\subsection{Padding pada Struct}
Untuk memenuhi syarat perataan, kompilator menyelipkan byte kosong yang disebut \textit{padding}.
\begin{lstlisting}[language=C]
struct Contoh {
    char a;     // 1 byte
    // Padding: 3 byte (agar 'b' mulai di offset 4)
    int b;      // 4 byte
    short c;    // 2 byte
    // Padding: 2 byte (agar total struct kelipatan 4 atau 8)
};
\end{lstlisting}

\subsection{32-bit vs 64-bit}
Ukuran pointer dan tipe data dasar dapat berubah antar arsitektur:
\begin{itemize}
    \item \textbf{Pointer}: 4 byte pada sistem 32-bit, 8 byte pada sistem 64-bit.
    \item \textbf{Alignment Requirement}: Sistem 64-bit seringkali lebih ketat, mengharuskan lebih banyak padding untuk tipe data \texttt{long} atau pointer.
\end{itemize}

\subsection{Packed Structures}
Kompilator menyediakan direktif (seperti \texttt{\#pragma pack(1)}) untuk menghilangkan padding. Ini berguna untuk protokol jaringan atau format file, namun berisiko menurunkan performa akses memori.

\begin{figure}[!htbp]
    \centering
    \adjustbox{max width=0.8\textwidth,center}{%
    \begin{tikzpicture}[
        byte/.style={rectangle, draw, minimum width=0.5cm, minimum height=0.5cm, font=\tiny}
    ]
    \node[byte, fill=blue!20] (b1) {A};
    \node[byte, right=0pt of b1, fill=gray!20] (p1) {P};
    \node[byte, right=0pt of p1, fill=gray!20] (p2) {P};
    \node[byte, right=0pt of p2, fill=gray!20] (p3) {P};
    \node[byte, right=0pt of p3, fill=green!20, minimum width=2cm] (i1) {Integer B};
    \end{tikzpicture}%
    }
    \caption{Visualisasi Padding (P) untuk Menyelaraskan Integer B}
\end{figure}
