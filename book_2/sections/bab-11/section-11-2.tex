\section{Filosofi Pengalamatan: RISC vs CISC}

Arsitektur target sangat mempengaruhi bagaimana kompilator menghasilkan kode untuk mengakses data. Dua filosofi utama adalah \compiler{CISC} (x86) dan \compiler{RISC} (RISC-V, ARM).

\subsection{CISC: Kaya dan Kompleks}
Pada x86, filosofinya adalah menyediakan instruksi yang mampu melakukan banyak hal sekaligus.
\begin{itemize}
    \item \textbf{Memory-to-Memory}: Memungkinkan operan diambil dari memori, diproses, dan hasilnya disimpan kembali ke memori dalam satu instruksi.
    \item \textbf{Variable Length}: Instruksi bisa berukuran mulai dari 1 hingga 15 byte, tergantung kompleksitas pengalamatannya.
\end{itemize}

\subsection{RISC: Sederhana dan Cepat}
Pada RISC-V atau ARM, filosofinya adalah "Load/Store Architecture".
\begin{itemize}
    \item \textbf{Terisolasi}: Hanya instruksi khusus (\code{LOAD} dan \code{STORE}) yang boleh mengakses memori. Instruksi aritmatika hanya boleh bekerja pada register.
    \item \textbf{Decomposisi}: Kompilator harus memecah akses kompleks menjadi urutan instruksi sederhana.
\end{itemize}

\subsection{Perbandingan Dekomposisi Kode}
Akses ke \code{x = a[i]} di mana \code{a} adalah array global:

\textbf{x86 (CISC)}:
\begin{lstlisting}[language={[x86masm]Assembler}]
MOV EAX, [base_a + ECX*4] ; 1 instruksi langsung
\end{lstlisting}

\textbf{RISC-V (RISC)}:
\begin{lstlisting}[language=bash]
SLLI t1, t0, 2      # t1 = i * 4 (shift left logical)
LA   t2, base_a     # Muat alamat dasar a ke t2
ADD  t3, t1, t2     # t3 = alamat elemen yang dituju
LW   a0, 0(t3)      # Muat nilai dari memori ke register
\end{lstlisting}

\begin{figure}[!htbp]
    \centering
    \adjustbox{max width=0.8\textwidth,center}{%
    \begin{tikzpicture}[
        node/.style={rectangle, draw=purple!50, fill=purple!10, text width=5cm, font=\tiny, align=center}
    ]
    \node[node] (cisc) {CISC: Instruksi Sedikit, Kompilasi Mudah, Hardware Kompleks};
    \node[node, below=0.5cm of cisc] (risc) {RISC: Instruksi Banyak, Kompilasi Berat, Hardware Sederhana};
    \end{tikzpicture}%
    }
    \caption{Trade-off antara Kompleksitas Kompiler dan Desain CPU}
\end{figure}
