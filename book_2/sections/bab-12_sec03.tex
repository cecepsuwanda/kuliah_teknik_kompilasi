\section{Format Intermediate Representation}

Terdapat berbagai format IR yang umum digunakan dalam kompilator modern:

\subsection{Three-Address Code (TAC)}

Three-address code adalah format IR di mana setiap instruksi memiliki paling banyak tiga operand (dua sumber dan satu tujuan). Format ini sangat populer karena:

\begin{itemize}
    \item Sederhana dan mudah dipahami
    \item Mirip dengan assembly code
    \item Memudahkan optimasi
    \item Mudah di-generate dari AST
\end{itemize}

Contoh three-address code untuk ekspresi \texttt{x = a + b * c}:

\begin{verbatim}
t1 = b * c
t2 = a + t1
x = t2
\end{verbatim}

Setiap baris adalah satu instruksi dengan format: \texttt{result = operand1 operator operand2}

Gambar \ref{fig:tac-example-basic} menunjukkan contoh three-address code untuk ekspresi kompleks.

\begin{figure}[H]
    \centering
    \adjustbox{max width=0.85\textwidth,center}{%
    \begin{tikzpicture}[
        code/.style={rectangle, draw=blue!50, fill=blue!10, text width=6cm, minimum height=0.6cm, font=\footnotesize\ttfamily, align=left, inner sep=4pt, rounded corners},
        arrow/.style={->, >=stealth, thick},
        node distance=0.3cm
    ]
    
    \node[code] (expr) {Expression: x = a + b * c};
    \node[code, below=of expr] (tac1) {t1 = b * c};
    \node[code, below=of tac1] (tac2) {t2 = a + t1};
    \node[code, below=of tac2] (tac3) {x = t2};
    
    \draw[arrow] (expr) -- (tac1);
    \draw[arrow] (tac1) -- (tac2);
    \draw[arrow] (tac2) -- (tac3);
    
    \end{tikzpicture}%
    }
    \caption{Contoh three-address code}
    \label{fig:tac-example-basic}
\end{figure}

\subsection{Quadruples}

Quadruples adalah representasi struktural dari three-address code. Setiap instruksi direpresentasikan sebagai record dengan empat field:

\begin{itemize}
    \item \textbf{op}: Operator (+, -, *, /, =, jmp, jmpf, dll.)
    \item \textbf{arg1}: Operand pertama
    \item \textbf{arg2}: Operand kedua (kosong untuk unary operations)
    \item \textbf{result}: Hasil operasi (temporary atau variabel)
\end{itemize}

Contoh quadruple untuk \texttt{x = a + b * c}:

\begin{verbatim}
Quad 1: (op: *, arg1: b, arg2: c, result: t1)
Quad 2: (op: +, arg1: a, arg2: t1, result: t2)
Quad 3: (op: =, arg1: t2, arg2: _, result: x)
\end{verbatim}

Keuntungan quadruples:
\begin{itemize}
    \item Mudah untuk di-reorder (optimasi)
    \item Mudah untuk di-optimize (common subexpression elimination)
    \item Struktur data yang jelas untuk manipulasi
\end{itemize}

Gambar \ref{fig:quadruple-structure} menunjukkan struktur quadruple.

\begin{figure}[H]
    \centering
    \adjustbox{max width=0.85\textwidth,center}{%
    \begin{tikzpicture}[
        quad/.style={rectangle, draw=blue!50, fill=blue!10, text width=5cm, minimum height=0.6cm, font=\tiny\ttfamily, align=left, inner sep=4pt, rounded corners},
        field/.style={rectangle, draw=green!50, fill=green!10, text width=1cm, minimum height=0.5cm, font=\tiny, align=center, inner sep=2pt},
        node distance=0.3cm
    ]
    
    \node[quad] (q1) {Quad 1: (op: *, arg1: b, arg2: c, result: t1)};
    \node[field, below=0.2cm of q1, xshift=-2cm] (op) {op};
    \node[field, right=of op] (a1) {arg1};
    \node[field, right=of a1] (a2) {arg2};
    \node[field, right=of a2] (res) {result};
    
    \draw[->, >=stealth, thick] (op) -- (q1);
    \draw[->, >=stealth, thick] (a1) -- (q1);
    \draw[->, >=stealth, thick] (a2) -- (q1);
    \draw[->, >=stealth, thick] (res) -- (q1);
    
    \end{tikzpicture}%
    }
    \caption{Struktur quadruple}
    \label{fig:quadruple-structure}
\end{figure}

\subsection{Format IR Lainnya}

Selain TAC dan quadruples, terdapat format IR lainnya:

\begin{itemize}
    \item \textbf{Static Single Assignment (SSA)}: Setiap variabel hanya di-assign sekali, memudahkan optimasi data-flow
    \item \textbf{Bytecode}: Untuk bahasa yang diinterpretasi (Java bytecode, Python bytecode)
    \item \textbf{DAG (Directed Acyclic Graph)}: Representasi graf untuk ekspresi, memudahkan common subexpression elimination
    \item \textbf{LLVM IR}: Format IR modern yang digunakan oleh LLVM compiler infrastructure
\end{itemize}