\section{Contoh Praktis: Alur Kompilasi Program C Sederhana}

Untuk memperkuat pemahaman tentang fase-fase kompilasi yang telah dipelajari, mari kita lihat contoh konkret bagaimana sebuah program C sederhana diproses melalui setiap fase kompilasi. Contoh ini akan menunjukkan aplikasi praktis dari konsep-konsep yang telah dibahas.

Mari kita lihat contoh konkret bagaimana sebuah program C sederhana diproses melalui fase-fase kompilasi:

\begin{lstlisting}[language=C, caption={Program C sederhana: hello.c}, label={lst:hello-c}]
#include <stdio.h>

int main() {
    int x = 10;
    int y = 20;
    int sum = x + y;
    printf("Sum = %d\n", sum);
    return 0;
}
\end{lstlisting}

\subsubsection{Setelah Preprocessing}

Preprocessor akan mengganti \texttt{\#include <stdio.h>} dengan isi file header tersebut (biasanya ratusan baris deklarasi fungsi).

\subsubsection{Setelah Lexical Analysis}

Source code dipecah menjadi token-token seperti yang ditunjukkan pada Tabel \ref{tab:tokens-example}.

\begin{longtable}{@{}p{4cm}p{6cm}@{}}
\caption{Token stream hasil lexical analysis untuk program hello.c}
\label{tab:tokens-example} \\
\toprule
\textbf{Token} & \textbf{Token Type} \\
\midrule
\endfirsthead
\multicolumn{2}{c}%
{{\bfseries \tablename\ \thetable{} -- lanjutan dari halaman sebelumnya}} \\
\toprule
\textbf{Token} & \textbf{Token Type} \\
\midrule
\endhead
\midrule \multicolumn{2}{r}{{Dilanjutkan pada halaman berikutnya}} \\
\endfoot
\bottomrule
\endlastfoot
\texttt{int} & KEYWORD \\
\texttt{main} & IDENTIFIER \\
\texttt{(} & PUNCTUATION (LPAREN) \\
\texttt{)} & PUNCTUATION (RPAREN) \\
\texttt{\{} & PUNCTUATION (LBRACE) \\
\texttt{int} & KEYWORD \\
\texttt{x} & IDENTIFIER \\
\texttt{=} & OPERATOR (ASSIGN) \\
\texttt{10} & INTEGER\_LITERAL \\
\texttt{;} & PUNCTUATION (SEMICOLON) \\
\texttt{int} & KEYWORD \\
\texttt{y} & IDENTIFIER \\
\texttt{=} & OPERATOR (ASSIGN) \\
\texttt{20} & INTEGER\_LITERAL \\
\texttt{;} & PUNCTUATION (SEMICOLON) \\
\texttt{int} & KEYWORD \\
\texttt{sum} & IDENTIFIER \\
\texttt{=} & OPERATOR (ASSIGN) \\
\texttt{x} & IDENTIFIER \\
\texttt{+} & OPERATOR (PLUS) \\
\texttt{y} & IDENTIFIER \\
\texttt{;} & PUNCTUATION (SEMICOLON) \\
\texttt{printf} & IDENTIFIER \\
\texttt{(} & PUNCTUATION (LPAREN) \\
\texttt{"Sum = \%d\textbackslash n"} & STRING\_LITERAL \\
\texttt{,} & PUNCTUATION (COMMA) \\
\texttt{sum} & IDENTIFIER \\
\texttt{)} & PUNCTUATION (RPAREN) \\
\texttt{;} & PUNCTUATION (SEMICOLON) \\
\texttt{return} & KEYWORD \\
\texttt{0} & INTEGER\_LITERAL \\
\texttt{;} & PUNCTUATION (SEMICOLON) \\
\texttt{\}} & PUNCTUATION (RBRACE) \\
\end{longtable}

\subsubsection{Setelah Syntax Analysis}

Parser membangun AST yang menunjukkan struktur program. Bagian AST untuk statement \texttt{int sum = x + y;} ditunjukkan pada Gambar \ref{fig:ast-sum-example}.

\begin{figure}[!htbp]
\centering
\adjustbox{max width=0.8\textwidth,center}{%
\begin{forest}
for tree={
    grow'=0,
    child anchor=west,
    parent anchor=east,
    anchor=west,
    calign=first,
    s sep=8mm,
    l sep=12mm,
    edge path={
        \noexpand\path[\forestoption{edge}]
        (!u.south west) +(7.5pt,0) |- (.child anchor) \forestoption{edge label};
    },
    before typesetting nodes={
        if n=1
          {insert before={[,phantom]}}
          {}
    },
    fit=band,
    before computing xy={l=15pt},
    font=\footnotesize
}
[Declaration
    [Type [int]]
    [Variable [sum]]
    [Initializer
        [Expression
            [BinaryOp [+]
                [Identifier [x]]
                [Identifier [y]]
            ]
        ]
    ]
]
\end{forest}%
}
\caption{AST untuk deklarasi \texttt{int sum = x + y;}}
\label{fig:ast-sum-example}
\end{figure}

\subsubsection{Setelah Semantic Analysis}

Semantic analyzer memeriksa:
\begin{itemize}
    \item Variabel \texttt{x} dan \texttt{y} sudah dideklarasi sebelum digunakan
    \item Tipe data \texttt{int} kompatibel untuk operasi penjumlahan
    \item Fungsi \texttt{printf} dideklarasi di \texttt{stdio.h}
    \item Format string \texttt{"Sum = \%d\textbackslash n"} sesuai dengan parameter \texttt{sum} (tipe \texttt{int})
\end{itemize}

\subsubsection{Setelah Intermediate Code Generation}

TAC yang dihasilkan untuk bagian \texttt{sum = x + y;}:
\begin{verbatim}
t1 = x + y
sum = t1
\end{verbatim}

\subsubsection{Setelah Code Generation}

Assembly code yang dihasilkan (contoh untuk x86-64):
\begin{verbatim}
mov eax, DWORD PTR [rbp-4]    ; Load x
add eax, DWORD PTR [rbp-8]    ; Add y
mov DWORD PTR [rbp-12], eax   ; Store to sum
\end{verbatim}
