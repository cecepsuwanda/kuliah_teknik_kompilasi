\section{Tujuan Pembelajaran dan Pendahuluan}

Tutorial ini menyajikan \textbf{versi minimal yang dapat dijalankan} dari proyek compiler subset C. Konsep dan fase-fase (lexical analysis, parsing, AST, code generation, assembling, linking) mengikuti materi Bab 2--16; spesifikasi token dan grammar selaras dengan Bab 1 (Bagian~\ref{sec:spec-subset-c}). Karena fokus pada ``build dari nol'' yang singkat, subset di sini dibatasi pada \texttt{print("string");}; setelah menyelesaikan Bab 2--16, pembaca dapat memperluas ke deklarasi, assignment, dan ekspresi sesuai grammar proyek. Kode dalam bab ini dapat dipandang sebagai langkah pertama atau snapshot minimal dari codebase proyek di \texttt{proyek-compiler-subset-c/}.

\subsection{Tujuan Pembelajaran}

Setelah mempelajari bab ini, mahasiswa diharapkan mampu:
\begin{enumerate}
    \item Memahami arsitektur compiler sederhana dari awal hingga akhir
    \item Mengimplementasikan lexer sederhana dalam bahasa C
    \item Mengimplementasikan parser recursive descent dalam bahasa C
    \item Mengimplementasikan code generator yang menghasilkan assembly code
    \item Mengintegrasikan assembler dan linker untuk menghasilkan executable file
    \item Membangun compiler lengkap yang dapat mengkompilasi program sederhana menjadi .exe
\end{enumerate}

\subsection{Overview Compiler yang Akan Dibuat}

Dalam tutorial ini, kita akan membuat compiler sederhana yang dapat mengkompilasi subset minimal bahasa C menjadi executable Windows (.exe). Compiler ini mengimplementasikan fase-fase yang sama dengan proyek Bab 2--16 (lexer, parser, codegen, assemble, link), dengan subset terbatas pada \texttt{print("string");} agar tutorial singkat. Compiler ini akan mengimplementasikan semua fase kompilasi yang telah dipelajari:

\begin{enumerate}
    \item \textbf{Lexical Analysis}: Memecah source code menjadi token-token
    \item \textbf{Syntax Analysis}: Membangun AST dari token stream
    \item \textbf{Code Generation}: Menghasilkan assembly code dari AST
    \item \textbf{Assembling}: Mengubah assembly menjadi object file
    \item \textbf{Linking}: Menghasilkan executable file
\end{enumerate}

\subsection{Fitur yang Didukung}

Compiler sederhana ini akan mendukung:
\begin{itemize}
    \item \textbf{Print statement}: \texttt{print("string");}
    \item \textbf{String literals}: String dalam tanda kutip ganda
    \item \textbf{Minimal syntax}: Semicolon dan parentheses
\end{itemize}

Meskipun sangat sederhana, compiler ini akan menunjukkan semua fase kompilasi secara lengkap dan dapat menghasilkan executable yang benar-benar dapat dijalankan.

\subsection{Contoh Program Target}

Program yang akan kita kompilasi adalah:

\begin{lstlisting}[language=C, caption={Program hello.c}]
print("hello world !!!");
\end{lstlisting}

Program ini akan dikompilasi menjadi \texttt{hello.exe} yang ketika dijalankan akan menampilkan:
\begin{verbatim}
hello world !!!
\end{verbatim}

\subsection{Tools yang Diperlukan}

Untuk mengikuti tutorial ini, Anda memerlukan:

\begin{enumerate}
    \item \textbf{C Compiler}: 
    \begin{itemize}
        \item GCC (MinGW untuk Windows)
        \item Atau Microsoft Visual C++ Compiler (cl.exe)
        \item Atau TCC (Tiny C Compiler) - lebih sederhana
    \end{itemize}
    
    \item \textbf{NASM (Netwide Assembler)}:
    \begin{itemize}
        \item Download dari: \url{https://www.nasm.us/}
        \item Versi untuk Windows (nasm.exe)
        \item Digunakan untuk meng-assemble kode assembly menjadi object file
    \end{itemize}
    
    \item \textbf{Linker}:
    \begin{itemize}
        \item Microsoft Linker (link.exe) - biasanya sudah termasuk dengan Visual Studio
        \item Atau MinGW linker (ld.exe) - sudah termasuk dengan GCC
        \item Digunakan untuk mengubah object file menjadi executable
    \end{itemize}
    
    \item \textbf{Text Editor}: Editor teks apapun untuk menulis kode C
\end{enumerate}

\subsection{Struktur Project}

Struktur project compiler yang akan kita buat:

\begin{verbatim}
compiler/
├── lexer.h          # Header file untuk lexer
├── lexer.c          # Implementasi lexer
├── parser.h         # Header file untuk parser
├── parser.c         # Implementasi parser
├── codegen.h        # Header file untuk code generator
├── codegen.c        # Implementasi code generator
├── main.c           # Driver program utama
├── build.bat        # Script untuk build compiler
├── test.bat         # Script untuk test compiler
└── hello.c          # Program test: print("hello world !!!");
\end{verbatim}

\subsection{Arsitektur Compiler}

Diagram berikut menunjukkan arsitektur compiler sederhana yang akan kita buat:

\begin{figure}[!htbp]
    \centering
    \adjustbox{max width=0.9\textwidth,center}{%
    \begin{tikzpicture}[
        box/.style={rectangle, draw=blue!50, fill=blue!10, text width=3cm, text centered, minimum height=1cm, rounded corners, font=\footnotesize},
        file/.style={rectangle, draw=green!50, fill=green!10, text width=2.5cm, text centered, minimum height=0.8cm, rounded corners, font=\tiny},
        arrow/.style={->, >=stealth, thick},
        node distance=1.5cm and 2cm
    ]
    
    \node[file] (source) {hello.c\\Source Code};
    
    \node[box, below=of source] (lexer) {Lexer\\Tokenization};
    \node[file, right=of lexer] (tokens) {Tokens};
    
    \node[box, below=of lexer] (parser) {Parser\\AST Building};
    \node[file, right=of parser] (ast) {AST};
    
    \node[box, below=of parser] (codegen) {Code Generator\\Assembly};
    \node[file, right=of codegen] (asm) {hello.asm};
    
    \node[box, below=of codegen] (nasm) {NASM\\Assembler};
    \node[file, right=of nasm] (obj) {hello.obj};
    
    \node[box, below=of nasm] (linker) {Linker};
    \node[file, right=of linker] (exe) {hello.exe\\Executable};
    
    \draw[arrow] (source) -- (lexer);
    \draw[arrow] (lexer) -- (tokens);
    \draw[arrow] (tokens) -- (parser);
    \draw[arrow] (parser) -- (ast);
    \draw[arrow] (ast) -- (codegen);
    \draw[arrow] (codegen) -- (asm);
    \draw[arrow] (asm) -- (nasm);
    \draw[arrow] (nasm) -- (obj);
    \draw[arrow] (obj) -- (linker);
    \draw[arrow] (linker) -- (exe);
    
    \end{tikzpicture}%
    }
    \caption{Arsitektur compiler sederhana}
    \label{fig:compiler-architecture}
\end{figure}

\subsection{Langkah-Langkah Pembuatan}

Tutorial ini akan dibagi menjadi beberapa bagian:

\begin{enumerate}
    \item \textbf{Bagian 1}: Implementasi Lexer - mengenali token dari source code
    \item \textbf{Bagian 2}: Implementasi Parser - membangun AST dari token stream
    \item \textbf{Bagian 3}: Code Generation - menghasilkan assembly code
    \item \textbf{Bagian 4}: Assembling dan Linking - menghasilkan executable
    \item \textbf{Bagian 5}: Testing dan Verifikasi - memastikan compiler bekerja dengan benar
    \item \textbf{Bagian 6}: Extensions - menambahkan fitur tambahan
\end{enumerate}

Setiap bagian akan dilengkapi dengan kode lengkap yang dapat langsung digunakan dan di-compile.
