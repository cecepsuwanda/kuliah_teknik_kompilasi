\section{Kesimpulan}

Dalam bab ini, kita telah mempelajari:

\begin{enumerate}
    \item Struktur token dan token types untuk subset bahasa C
    \item Konsep finite state machine dalam konteks lexical analysis
    \item Implementasi hand-written lexer dalam C++ dengan handling:
    \begin{itemize}
        \item Identifier dan keyword recognition
        \item Number literals (integer dan float)
        \item String dan character literals dengan escape sequences
        \item Operators (single dan multi-character)
        \item Whitespace dan komentar (single-line dan multi-line)
    \end{itemize}
    \item Error handling untuk edge cases
    \item Testing strategies untuk lexer
\end{enumerate}

Implementasi hand-written lexer memberikan pemahaman mendalam tentang proses tokenization dan menjadi dasar untuk memahami bagaimana lexer generator seperti Flex bekerja di belakang layar. Lexer yang dibahas di bab ini mengimplementasikan spesifikasi token proyek subset C (Bab 1); lexer proyek dengan Flex dibangun di Bab 4 (\texttt{simplec.l}).