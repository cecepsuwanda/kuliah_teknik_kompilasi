\section{Pendahuluan}

Spesifikasi token untuk proyek compiler subset C telah didefinisikan di Bab 1 (Bagian~\ref{sec:spec-subset-c}). Dalam bab ini kita mempelajari teori formal yang mendasari lexical analysis---regular expression dan finite automata---dan mengaitkan contoh ke token-token proyek (identifier, number, keyword, operator, punctuator) bila relevan.

Sebagai landasan untuk memahami lexical analysis, kita perlu mempelajari teori formal yang mendasarinya. Menurut sumber dari Aoyama Gakuin University:

\begin{quote}
``Lexical analysis breaks input text into lexemes which correspond to tokens. Usually implemented using regular languages → regex → NFA → DFA → (minimized) DFA for efficiency.''\cite{aoyama2024lexical}
\end{quote}

Alur ini menunjukkan bahwa lexical analysis dalam kompilator modern menggunakan teori formal language, khususnya regular languages, yang direpresentasikan sebagai regular expressions dan kemudian diimplementasikan sebagai finite automata untuk efisiensi.

Gambar \ref{fig:lexical-theory-overview} menunjukkan alur lengkap dari regular expression hingga implementasi praktis dalam lexical analyzer.

\begin{figure}[!htbp]
    \centering
    \adjustbox{max width=0.9\textwidth,center}{%
    \begin{tikzpicture}[
        box/.style={rectangle, draw=blue!50, fill=blue!10, text width=1.8cm, text centered, minimum height=0.8cm, rounded corners, font=\footnotesize, inner sep=4pt, align=center},
        arrow/.style={->, >=stealth, thick},
        label/.style={font=\tiny, above, align=center},
        node distance=1.8cm
    ]
    
    \node[box] (regex) {Regular\\Expression};
    \node[box, right=of regex] (nfa) {$\epsilon$-NFA\\(Thompson)};
    \node[box, right=of nfa] (dfa) {DFA\\(Subset)};
    \node[box, right=of dfa] (min) {Minimized\\DFA};
    \node[box, right=of min] (impl) {Scanner};
    
    \draw[arrow] (regex) -- node[label, align=center] {Algoritma\\Thompson} (nfa);
    \draw[arrow] (nfa) -- node[label, align=center] {Subset\\Construction} (dfa);
    \draw[arrow] (dfa) -- node[label] {Minimization} (min);
    \draw[arrow] (min) -- node[label, align=center] {Code\\Generation} (impl);
    
    \end{tikzpicture}%
    }
    \caption{Alur konversi dari regular expression ke implementasi scanner}
    \label{fig:lexical-theory-overview}
\end{figure}