\section{Sistem Tipe (Type System)}

Sistem tipe adalah kumpulan aturan yang menentukan bagaimana tipe data ditetapkan pada konstruksi program dan operasi apa yang "legal" untuk setiap tipe.

\subsection{Jenis-jenis Type System}

\subsubsection{Static vs Dynamic Typing}

\begin{itemize}
    \item \textbf{Static Typing}: Pengecekan tipe dilakukan pada waktu kompilasi. Bahasa seperti C, C++, Java, dan Rust menggunakan static typing. Keuntungannya adalah deteksi error lebih awal dan performa runtime yang lebih baik.
    
    \item \textbf{Dynamic Typing}: Pengecekan tipe dilakukan pada waktu eksekusi. Bahasa seperti Python, JavaScript, dan Ruby menggunakan dynamic typing. Keuntungannya adalah fleksibilitas lebih tinggi, tetapi error baru terdeteksi saat runtime.
\end{itemize}

\subsubsection{Nominal vs Structural Typing}

\begin{itemize}
    \item \textbf{Nominal Typing}: Kompatibilitas tipe ditentukan berdasarkan nama tipe yang dideklarasikan. Dua tipe dengan struktur yang sama tetapi nama berbeda dianggap tidak kompatibel. Contoh: Java, C++.
    
    \item \textbf{Structural Typing}: Kompatibilitas tipe ditentukan berdasarkan struktur tipe (field, method). Jika struktur cocok, tipe dianggap kompatibel meskipun nama berbeda. Contoh: TypeScript, OCaml.
\end{itemize}

\subsection{Type Hierarchy}

Dalam bahasa berorientasi objek, tipe-tipe membentuk hierarki melalui inheritance. Misalnya:

\begin{lstlisting}[language={},basicstyle=\ttfamily\footnotesize,breaklines=true,breakatwhitespace=false]
Object
|-- Number
|   |-- Integer
|   `-- Float
|-- String
`-- Boolean
\end{lstlisting}

Hierarki ini memungkinkan subtyping, di mana tipe turunan dapat digunakan di tempat tipe induk (substitution principle).

Gambar \ref{fig:type-checking-process} menunjukkan proses type checking.

\begin{figure}[!htbp]
    \centering
    \adjustbox{max width=0.9\textwidth,center}{%
    \begin{tikzpicture}[
        box/.style={rectangle, draw=blue!50, fill=blue!10, text width=2.5cm, text centered, minimum height=0.7cm, rounded corners, font=\footnotesize, inner sep=4pt, align=center},
        arrow/.style={->, >=stealth, thick},
        node distance=1.2cm
    ]
    
    \node[box] (ast) {AST};
    \node[box, right=of ast] (check) {Type\\Check};
    \node[box, right=of check] (annotate) {Annotated\\AST};
    \node[box, below=of check] (errors) {Type\\Errors};
    
    \draw[arrow] (ast) -- (check);
    \draw[arrow] (check) -- (annotate);
    \draw[arrow] (check) -- (errors);
    
    \end{tikzpicture}%
    }
    \caption{Proses type checking}
    \label{fig:type-checking-process}
\end{figure}