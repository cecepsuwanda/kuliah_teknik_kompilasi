\section{Referensi dan Bahan Bacaan Lanjutan}

Untuk memperdalam pemahaman tentang intermediate code generation, mahasiswa disarankan membaca:

\begin{itemize}
    \item \textbf{Dragon Book}: Aho, Lam, Sethi, \& Ullman (2006). \textit{Compilers: Principles, Techniques, and Tools} \cite{aho2006compilers} - Bab 6: Intermediate-Code Generation
    
    \item \textbf{Engineering a Compiler}: Cooper \& Torczon (2011) \cite{cooper2011engineering} - Bab 6: The Procedure Abstraction dan Bab 7: Code Shape
    
    \item \textbf{SDSU CS 524}: Intermediate Code Generation \footnote{\url{https://stewart.sdsu.edu/cs524/spr08/lects/ch6_IntermediateCodeGen.html}}
    
    \item \textbf{GeeksforGeeks}: Three Address Code \footnote{\url{https://www.geeksforgeeks.org/three-address-code-compiler/}}
    
    \item \textbf{Linköping University}: Lab 6 - Intermediate Code Generation \footnote{\url{https://www.ida.liu.se/~TDDB44/laboratories/instructions/lab6.html}}
    
    \item \textbf{Shasank's Engineering Notes}: Module 7 - Intermediate Code Generation \footnote{\url{https://shasankp000.github.io/CSE-Engineering-Notes/Compiler_Design/Module-7----Intermediate-Code-Generation}}
    
    \item \textbf{Wikipedia - Intermediate Representation}: \footnote{\url{https://en.wikipedia.org/wiki/Intermediate_representation}}
    
    \item \textbf{LLVM Language Reference}: \footnote{\url{https://llvm.org/docs/LangRef.html}} - Untuk mempelajari format IR modern
\end{itemize}