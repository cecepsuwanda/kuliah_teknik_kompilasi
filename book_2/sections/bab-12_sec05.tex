\section{Handling Control Flow Statements}

Generasi kode untuk control flow statements (if, while, for) memerlukan label dan jump instructions.

\subsection{If-Then-Else Statement}

Untuk if statement, kita perlu:
\begin{itemize}
    \item Label untuk else branch (jika ada)
    \item Label untuk end of if statement
    \item Conditional jump berdasarkan kondisi
    \item Unconditional jump untuk skip else branch
\end{itemize}

\begin{lstlisting}[language=C++, caption=Generator TAC untuk if statement]
class ASTIf : public ASTNode {
    ASTNode* condition;
    ASTNode* thenBranch;
    ASTNode* elseBranch;  // bisa null
public:
    std::string genCode(QuadList& quads, SymbolTable& symtab) override {
        // Generate code untuk kondisi
        std::string condTemp = condition->genCode(quads, symtab);
        
        std::string elseLabel = quads.newLabel();
        std::string endLabel = quads.newLabel();
        
        // Jump to else jika kondisi false
        quads.emit(Quad("jmpf", condTemp, "", elseLabel));
        
        // Generate code untuk then branch
        thenBranch->genCode(quads, symtab);
        
        // Jump to end (skip else)
        quads.emit(Quad("jmp", "", "", endLabel));
        
        // Else label
        quads.emit(Quad("label", "", "", elseLabel));
        
        // Generate code untuk else branch (jika ada)
        if (elseBranch != nullptr) {
            elseBranch->genCode(quads, symtab);
        }
        
        // End label
        quads.emit(Quad("label", "", "", endLabel));
        
        return "";  // if statement tidak menghasilkan nilai
    }
};
\end{lstlisting}

Contoh output untuk \texttt{if (x > 0) y = 1; else y = 0;}:

\begin{verbatim}
t1 = x > 0
jmpf t1, L0
t2 = 1
y = t2
jmp L1
label L0
t3 = 0
y = t3
label L1
\end{verbatim}

\subsection{While Loop}

While loop memerlukan:
\begin{itemize}
    \item Label untuk start of loop
    \item Label untuk end of loop
    \item Conditional jump untuk exit loop
    \item Unconditional jump untuk kembali ke start
\end{itemize}

\begin{lstlisting}[language=C++, caption=Generator TAC untuk while loop]
class ASTWhile : public ASTNode {
    ASTNode* condition;
    ASTNode* body;
public:
    std::string genCode(QuadList& quads, SymbolTable& symtab) override {
        std::string startLabel = quads.newLabel();
        std::string endLabel = quads.newLabel();
        
        // Start label
        quads.emit(Quad("label", "", "", startLabel));
        
        // Generate code untuk kondisi
        std::string condTemp = condition->genCode(quads, symtab);
        
        // Jump to end jika kondisi false
        quads.emit(Quad("jmpf", condTemp, "", endLabel));
        
        // Generate code untuk body
        body->genCode(quads, symtab);
        
        // Jump back to start
        quads.emit(Quad("jmp", "", "", startLabel));
        
        // End label
        quads.emit(Quad("label", "", "", endLabel));
        
        return "";
    }
};
\end{lstlisting}

Contoh output untuk \texttt{while (i < 10) \{ i = i + 1; \}}:

\begin{verbatim}
label L0
t1 = i < 10
jmpf t1, L1
t2 = i + 1
i = t2
jmp L0
label L1
\end{verbatim}

\subsection{For Loop}

For loop dapat di-translate menjadi while loop atau di-generate langsung. Implementasi sebagai while loop:

\begin{verbatim}
for (init; condition; update) {
    body
}
\end{verbatim}

Menjadi:

\begin{verbatim}
init
label L0
jmpf condition, L1
body
update
jmp L0
label L1
\end{verbatim}