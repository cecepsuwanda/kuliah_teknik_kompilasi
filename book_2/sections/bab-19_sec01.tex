\section{Soal Ujian Tengah Semester (UTS)}

\subsection{Petunjuk}
\begin{itemize}
    \item Waktu ujian: 120 menit
    \item Materi: Bab 1 sampai Bab 7
    \item Jawablah semua soal dengan jelas dan lengkap
    \item Gunakan diagram atau ilustrasi jika diperlukan
\end{itemize}

\subsection{Soal UTS}

\begin{enumerate}
    \item \textbf{[Bab 1 - 15 poin]} Jelaskan secara detail perbedaan antara kompilator dan interpreter. Berikan contoh minimal 2 bahasa pemrograman untuk masing-masing pendekatan dan jelaskan mengapa bahasa tersebut menggunakan pendekatan tersebut. Selain itu, jelaskan juga pendekatan hybrid yang digunakan oleh beberapa bahasa modern.
    
    \item \textbf{[Bab 1 - 15 poin]} Buatlah diagram lengkap yang menunjukkan alur kerja kompilator dari source code C sampai menjadi executable file. Sertakan semua fase kompilasi yang telah dipelajari beserta output dari setiap fase. Jelaskan juga peran linker dan assembler dalam proses kompilasi.
    
    \item \textbf{[Bab 2 - 20 poin]} 
    \begin{enumerate}
        \item Buatlah regular expression untuk mengenali:
        \begin{itemize}
            \item Identifier dalam bahasa C (dimulai dengan huruf atau underscore, diikuti huruf, angka, atau underscore)
            \item Floating point number (format: \texttt{123.456} atau \texttt{1.23e-4})
            \item C-style multi-line comment (\texttt{/* ... */})
        \end{itemize}
        
        \item Konstruksi NFA untuk regular expression \texttt{(a|b)*abb} menggunakan algoritma Thompson. Gambarkan state diagram lengkap dengan semua transisi.
        
        \item Konversi NFA dari poin (b) menjadi DFA menggunakan subset construction. Tunjukkan proses konversi secara detail dan gambarkan DFA yang dihasilkan.
    \end{enumerate}
    
    \item \textbf{[Bab 3 - 15 poin]} Implementasikan lexer sederhana dalam C++ yang dapat mengenali token-token berikut:
    \begin{itemize}
        \item Keywords: \texttt{if}, \texttt{else}, \texttt{while}, \texttt{int}, \texttt{float}
        \item Identifier: \texttt{[a-zA-Z\_][a-zA-Z0-9\_]*}
        \item Integer literal: \texttt{[0-9]+}
        \item Float literal: \texttt{[0-9]+\textbackslash.[0-9]+}
        \item Operators: \texttt{+}, \texttt{-}, \texttt{*}, \texttt{/}, \texttt{=}, \texttt{==}, \texttt{!=}
        \item Punctuation: \texttt{;}, \texttt{,}, \texttt{(}, \texttt{)}, \texttt{\{}, \texttt{\}}
    \end{itemize}
    Sertakan juga error handling untuk karakter yang tidak valid dan tracking posisi (line dan column number).
    
    \item \textbf{[Bab 4 - 15 poin]} 
    \begin{enumerate}
        \item Buatlah specification file Flex untuk mengenali semua token yang sama dengan soal nomor 4. Bandingkan kompleksitas implementasi antara hand-written lexer dan Flex-generated lexer.
        
        \item Jelaskan kapan sebaiknya menggunakan hand-written lexer dan kapan menggunakan lexer generator seperti Flex atau re2c. Berikan contoh kasus konkret untuk masing-masing pendekatan.
    \end{enumerate}
    
    \item \textbf{[Bab 5 - 20 poin]}
    \begin{enumerate}
        \item Tuliskan grammar dalam BNF untuk ekspresi aritmatika yang mendukung:
        \begin{itemize}
            \item Operasi: \texttt{+}, \texttt{-}, \texttt{*}, \texttt{/}
            \item Precedence: \texttt{*} dan \texttt{/} lebih tinggi dari \texttt{+} dan \texttt{-}
            \item Associativity: semua operator left-associative
            \item Parentheses untuk mengubah precedence
        \end{enumerate}
        
        \item Buatlah leftmost derivation untuk ekspresi \texttt{2 + 3 * 4} menggunakan grammar yang dibuat di poin (a).
        
        \item Gambarkan parse tree lengkap untuk ekspresi \texttt{(5 + 3) * 2 - 1} menggunakan grammar yang sama.
        
        \item Identifikasi apakah grammar berikut ambiguous. Jika ya, berikan contoh string yang dapat di-parse dengan lebih dari satu cara dan jelaskan mengapa terjadi ambiguity:
        \begin{verbatim}
        E → E + E | E * E | ( E ) | number
        \end{verbatim}
    \end{enumerate}
    
    \item \textbf{[Bab 6 - 20 poin]}
    \begin{enumerate}
        \item Jelaskan perbedaan antara top-down parsing dan bottom-up parsing. Berikan contoh grammar yang cocok untuk masing-masing pendekatan.
        
        \item Implementasikan recursive descent parser untuk grammar ekspresi aritmatika sederhana:
        \begin{verbatim}
        E → T | E + T | E - T
        T → F | T * F | T / F
        F → number | ( E )
        \end{verbatim}
        Tuliskan fungsi C++ untuk setiap non-terminal dengan error handling yang baik.
        
        \item Eliminasi left recursion dari grammar di poin (b) agar dapat digunakan dengan recursive descent parser. Tunjukkan grammar yang sudah dimodifikasi.
    \end{enumerate}
    
    \item \textbf{[Bab 7 - 20 poin]}
    \begin{enumerate}
        \item Jelaskan konsep shift-reduce parsing dalam bottom-up parsing. Berikan contoh langkah-langkah parsing untuk string \texttt{id + id * id} menggunakan grammar:
        \begin{verbatim}
        E → E + T | T
        T → T * F | F
        F → id
        \end{verbatim}
        
        \item Jelaskan perbedaan antara LR(0), SLR(1), LALR(1), dan CLR(1) parser. Berikan contoh grammar yang dapat di-parse oleh LALR(1) tetapi tidak dapat di-parse oleh SLR(1).
        
        \item Buatlah parsing table LR(0) untuk grammar sederhana berikut:
        \begin{verbatim}
        S → A a
        A → b
        \end{verbatim}
        Tunjukkan semua state dan transisi yang terjadi.
    \end{enumerate}
    
    \item \textbf{[Integratif - 20 poin]} Rancanglah sistem kompilator sederhana yang terdiri dari:
    \begin{itemize}
        \item Lexer (hand-written atau menggunakan Flex) yang dapat mengenali token-token dasar
        \item Parser (recursive descent atau menggunakan Bison) yang dapat mem-parse ekspresi aritmatika
        \item Evaluator sederhana yang dapat mengevaluasi ekspresi yang sudah di-parse
    \end{itemize}
    Buatlah diagram arsitektur sistem dan jelaskan bagaimana ketiga komponen tersebut berinteraksi. Sertakan contoh input/output untuk ekspresi \texttt{2 + 3 * 4}.
    
    \item \textbf{[Integratif - 20 poin]} Analisis dan bandingkan pendekatan berikut untuk membangun kompilator:
    \begin{itemize}
        \item Hand-written lexer + Hand-written parser
        \item Flex-generated lexer + Hand-written parser
        \item Flex-generated lexer + Bison-generated parser
    \end{itemize}
    Berikan kelebihan dan kekurangan masing-masing pendekatan, serta rekomendasi kapan sebaiknya menggunakan masing-masing pendekatan. Sertakan pertimbangan performa, maintainability, dan kompleksitas development.
\end{enumerate}

\vspace{1cm}
\textbf{Total: 180 poin}
