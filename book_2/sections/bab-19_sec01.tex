\section{Soal Ujian Tengah Semester (UTS)}

\subsection{Petunjuk}
\begin{itemize}
    \item Waktu ujian: 120 menit
    \item Materi: Bab 1 sampai Bab 7
    \item Jawablah semua soal dengan jelas dan lengkap
    \item Gunakan diagram atau ilustrasi jika diperlukan
\end{itemize}

\subsection{Soal UTS}

\begin{enumerate}
    \item \textbf{[Bab 1 - 15 poin]} Jelaskan secara detail perbedaan antara kompilator dan interpreter. Berikan contoh minimal 2 bahasa pemrograman untuk masing-masing pendekatan dan jelaskan mengapa bahasa tersebut menggunakan pendekatan tersebut. Selain itu, jelaskan juga pendekatan hybrid yang digunakan oleh beberapa bahasa modern.
    
    \item \textbf{[Bab 1 - 15 poin]} Buatlah diagram lengkap yang menunjukkan alur kerja kompilator dari source code C sampai menjadi executable binary. Sertakan semua fase kompilasi yang telah dipelajari beserta output dari setiap fase. Jelaskan juga peran linker dan assembler dalam proses kompilasi.
    
    \item \textbf{[Bab 2 - 20 poin]} 
    \begin{enumerate}
        \item Buatlah regular expression untuk mengenali:
        \begin{itemize}
            \item Identifier dalam bahasa C (dimulai dengan huruf atau underscore, diikuti huruf, angka, atau underscore)
            \item Floating point number (format: \texttt{123.456} atau \texttt{1.23e-4})
            \item C-style multi-line comment (\texttt{/* ... */})
        \end{itemize}
        
        \item Konstruksi NFA untuk regular expression \texttt{(a|b)*abb} menggunakan algoritma Thompson. Gambarkan state diagram lengkap dengan semua transisi.
        
        \item Konversi NFA dari poin (b) menjadi DFA menggunakan subset construction. Tunjukkan proses konversi secara detail dan gambarkan DFA yang dihasilkan.
    \end{enumerate}
    
    \item \textbf{[Bab 3 - 15 poin]} Implementasikan lexer sederhana dalam C++ yang dapat mengenali token-token berikut:
    \begin{itemize}
        \item Keywords: \texttt{if}, \texttt{else}, \texttt{while}, \texttt{int}, \texttt{float}
        \item Identifier: \texttt{[a-zA-Z\_][a-zA-Z0-9\_]*}
        \item Integer literal: \texttt{[0-9]+}
        \item Float literal: \texttt{[0-9]+\.[0-9]+}
        \item Operators: \texttt{+}, \texttt{-}, \texttt{*}, \texttt{/}, \texttt{=}, \texttt{==}, \texttt{!=}
        \item Punctuation: \texttt{;}, \texttt{,}, \texttt{(}, \texttt{)}, \texttt{\{}, \texttt{\}}
    \end{itemize}
    Sertakan juga error handling untuk karakter yang tidak valid dan tracking posisi (line dan column number).
    
    \item \textbf{[Bab 4 - 15 poin]} 
    \begin{enumerate}
        \item Buatlah specification file Flex untuk mengenali semua token yang sama dengan soal nomor 3. Bandingkan kompleksitas implementasi antara hand-written lexer dan Flex-generated lexer.
        
        \item Jelaskan kapan sebaiknya menggunakan hand-written lexer dan kapan menggunakan lexer generator seperti Flex atau re2c. Berikan contoh kasus konkret untuk masing-masing pendekatan.
    \end{enumerate}
\end{enumerate}