\section{Pendahuluan}

Setelah lexical analysis menghasilkan stream token, fase berikutnya dalam kompilator adalah syntax analysis atau parsing. Menurut sumber terbuka:

\begin{quote}
``Given the stream of tokens from the lexer, syntax analysis checks whether they form a valid sequence under the language grammar. Builds a parse tree or AST that represents nested structure of language constructs.''\cite{diznr2024phases}
\end{quote}

Syntax analysis membutuhkan formal grammar untuk mendefinisikan struktur yang valid dalam bahasa pemrograman. Context-free grammar (CFG) adalah alat formal yang paling umum digunakan untuk tujuan ini karena kemampuannya dalam menangani struktur nested dan recursive yang umum ditemui dalam bahasa pemrograman.

Gambar \ref{fig:parsing-overview} menunjukkan posisi syntax analysis dalam pipeline kompilator.

\begin{figure}[H]
    \centering
    \adjustbox{max width=0.9\textwidth,center}{%
    \begin{tikzpicture}[
        box/.style={rectangle, draw=blue!50, fill=blue!10, text width=2.5cm, text centered, minimum height=0.7cm, rounded corners, font=\footnotesize, inner sep=4pt, align=center},
        arrow/.style={->, >=stealth, thick},
        node distance=1.2cm
    ]
    
    \node[box] (source) {Source\\Code};
    \node[box, right=of source] (lexer) {Lexical\\Analysis};
    \node[box, right=of lexer] (tokens) {Token\\Stream};
    \node[box, right=of tokens] (parser) {Syntax\\Analysis};
    \node[box, below=of parser] (ast) {Parse Tree\\/ AST};
    
    \draw[arrow] (source) -- (lexer);
    \draw[arrow] (lexer) -- (tokens);
    \draw[arrow] (tokens) -- (parser);
    \draw[arrow] (parser) -- (ast);
    
    \node[below=0.3cm of parser, font=\tiny, align=center] {CFG-based\\Parsing};
    
    \end{tikzpicture}%
    }
    \caption{Posisi syntax analysis dalam pipeline kompilator}
    \label{fig:parsing-overview}
\end{figure}