\section{Integrasi dengan Semantic Analyzer}

Symbol table diintegrasikan dengan semantic analyzer untuk melakukan berbagai pemeriksaan:

\begin{enumerate}
    \item \textbf{Declaration Check}: Memastikan setiap identifier dideklarasi sebelum digunakan
    \item \textbf{Type Checking}: Memverifikasi tipe data dalam operasi dan assignment
    \item \textbf{Scope Resolution}: Menyelesaikan referensi identifier ke deklarasi yang benar
    \item \textbf{Duplicate Detection}: Mendeteksi deklarasi ganda dalam scope yang sama
\end{enumerate}

Contoh integrasi dengan semantic analyzer:

\begin{lstlisting}[language=C++, caption={Contoh penggunaan symbol table dalam semantic analysis}]
void semanticAnalyzeIdentifier(ASTNode* node) {
    std::string name = node->getName();
    
    // Lookup identifier
    Symbol* sym = symbolTable.lookup(name);
    
    if (sym == nullptr) {
        // Error: identifier tidak dideklarasi
        error("Undeclared identifier: " + name, 
              node->getLineNumber());
        return;
    }
    
    // Annotate AST node dengan symbol info
    node->setSymbol(sym);
    node->setType(sym->type);
}

void semanticAnalyzeAssignment(ASTNode* node) {
    ASTNode* lhs = node->getLeft();
    ASTNode* rhs = node->getRight();
    
    // Analyze kedua sisi
    semanticAnalyzeExpression(lhs);
    semanticAnalyzeExpression(rhs);
    
    // Type checking
    if (lhs->getType() != rhs->getType()) {
        error("Type mismatch in assignment", 
              node->getLineNumber());
    }
}
\end{lstlisting}