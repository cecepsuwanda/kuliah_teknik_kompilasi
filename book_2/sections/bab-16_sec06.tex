\section{Evaluasi Tools: Hand-Written vs Generator-Based}

Salah satu aspek penting dalam project final adalah evaluasi tools yang digunakan. Mahasiswa perlu membandingkan pendekatan hand-written dengan generator-based tools.

\subsection{Perbandingan Lexer: Hand-Written vs Flex/re2c}

\textbf{Hand-Written Lexer}

\textbf{Keuntungan:}
\begin{itemize}
    \item Kontrol penuh atas implementasi
    \item Error messages yang lebih informatif dan customizable
    \item Tidak ada dependency eksternal
    \item Dapat dioptimasi untuk kasus khusus
    \item Lebih mudah di-debug karena kode lebih readable
\end{itemize}

\textbf{Kekurangan:}
\begin{itemize}
    \item Lebih banyak waktu development
    \item Lebih banyak kode boilerplate
    \item Lebih mudah terjadi error manual
    \item Perlu implementasi state machine secara manual
\end{itemize}

\textbf{Generator-Based Lexer (Flex/re2c)}

\textbf{Keuntungan:}
\begin{itemize}
    \item Development lebih cepat
    \item Grammar specification lebih declarative
    \item Automatically generates efficient DFA
    \item Less boilerplate code
    \item Proven algorithms (Thompson's construction, subset construction)
\end{itemize}

\textbf{Kekurangan:}
\begin{itemize}
    \item Generated code sulit di-debug
    \item Error messages kurang informatif
    \item Dependency pada tool eksternal
    \item Kurang fleksibel untuk kasus edge yang kompleks
    \item Build process lebih kompleks
\end{itemize}

\subsection{Perbandingan Parser: Hand-Written vs Bison/Yacc}

\textbf{Hand-Written Parser (Recursive Descent)}

\textbf{Keuntungan:}
\begin{itemize}
    \item Kode lebih readable dan mudah di-maintain
    \item Error recovery yang lebih baik dan customizable
    \item Tidak ada dependency eksternal
    \item Dapat menangani grammar yang kompleks dengan mudah
    \item Lebih mudah di-debug
\end{itemize}

\textbf{Kekurangan:}
\begin{itemize}
    \item Hanya cocok untuk LL(1) grammar
    \item Lebih banyak kode manual
    \item Lebih mudah terjadi error dalam implementasi
\end{itemize}

\textbf{Generator-Based Parser (Bison/Yacc)}

\textbf{Keuntungan:}
\begin{itemize}
    \item Mendukung LR, LALR, GLR parsing
    \item Automatic table generation
    \item Grammar specification lebih declarative
    \item Proven algorithms
    \item Mendukung grammar yang lebih kompleks
\end{itemize}

\textbf{Kekurangan:}
\begin{itemize}
    \item Generated code sulit di-debug
    \item Error messages kurang informatif
    \item Dependency pada tool eksternal
    \item Build process lebih kompleks
    \item Kurang fleksibel untuk error recovery yang kompleks
\end{itemize}

\subsection{Case Study: Real-World Compilers}

Banyak compiler production menggunakan pendekatan hybrid atau beralih dari generator ke hand-written:

\textbf{GCC (GNU Compiler Collection)}
\begin{itemize}
    \item Menggunakan hand-written recursive descent parser untuk C dan C++
    \item Alasan: Better error messages, easier maintenance, better handling of complex grammar
    \item Trade-off: Lebih banyak kode manual, tetapi lebih maintainable
\end{itemize}

\textbf{Clang (LLVM Compiler)}
\begin{itemize}
    \item Menggunakan hand-written parser
    \item Alasan: Superior error diagnostics, better recovery
    \item Hasil: Error messages yang sangat informatif dan helpful
\end{itemize}

\textbf{Many Educational Compilers}
\begin{itemize}
    \item Menggunakan Flex + Bison untuk pembelajaran
    \item Alasan: Lebih cepat untuk development, fokus pada konsep bukan implementasi detail
    \item Cocok untuk: Prototyping, learning, small languages
\end{itemize}