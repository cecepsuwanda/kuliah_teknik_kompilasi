\section{Sistem Tipe dan Type Checking}

\subsection{Formalisme Sistem Tipe}
Dalam teori bahasa pemrograman, aturan tipe sering dituliskan menggunakan \textit{Inference Rules}. Notasi $\Gamma \vdash e : T$ dibaca "Dalam konteks $\Gamma$, ekspresi $e$ memiliki tipe $T$".

Contoh aturan untuk penjumlahan integer:
\[
\frac{\Gamma \vdash e_1 : \textbf{int} \quad \Gamma \vdash e_2 : \textbf{int}}
{\Gamma \vdash e_1 + e_2 : \textbf{int}}
\]
Artinya: "Jika $e_1$ bertipe int DAN $e_2$ bertipe int, MAKA $e_1 + e_2$ bertipe int".

\subsection{Type Checking vs Type Inference}
\begin{itemize}
    \item \textbf{Type Checking}: Programmer menulis tipe eksplisit (\code{int x = 5;}), kompilator memverifikasi kebenarannya.
    \item \textbf{Type Inference}: Programmer tidak menulis tipe (\code{auto x = 5;}), kompilator \textit{menyimpulkan} tipe berdasarkan nilai inisialisasi ($5 \rightarrow \texttt{int}$, maka $x \rightarrow \texttt{int}$).
\end{itemize}

\subsection{Static vs Dynamic Typing}
\begin{itemize}
    \item \textbf{Static Typing} (C, Java): Tipe variabel diketahui saat \textit{compile-time} dan tidak berubah. Lebih aman dan cepat dieksekusi.
    \item \textbf{Dynamic Typing} (Python, JS): Tipe variabel ditentukan saat \textit{run-time} berdasarkan nilai yang saat itu disimpannya. Lebih fleksibel tapi rentan \textit{runtime error}.
\end{itemize}
