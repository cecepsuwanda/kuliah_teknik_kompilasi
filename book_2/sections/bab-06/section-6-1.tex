\section{Abstract Syntax Tree (AST) Deep Dive}

\compiler{Abstract Syntax Tree (AST)} adalah representasi internal program yang telah disederhanakan. Analisis semantik bertugas memastikan bahwa program memiliki makna yang valid sesuai dengan aturan bahasa pemrogramannya, di luar sekadar kebenaran struktur sintaksisnya \cite{nguyen2024semantic}.

\subsection{Struktur Node AST}
Setiap node dalam AST mewakili konstruk bahasa (misal: \code{IfStmt}, \code{BinaryExpr}).
\begin{enumerate}
    \item \textbf{Expression Nodes}: Literal, Identifier, Unary/Binary operations.
    \item \textbf{Statement Nodes}: Assignment, Loop, Conditional branches.
    \item \textbf{Declaration Nodes}: Variabel, Fungsi, Tipe.
\end{enumerate}

\begin{figure}[!htbp]
    \centering
    \adjustbox{max width=0.8\textwidth,center}{%
    \begin{tikzpicture}[
        node/.style={circle, draw=blue!50, fill=blue!10, minimum size=0.6cm, font=\tiny}
    ]
    \node[node] (plus) at (0,0) {+};
    \node[node, below left=of plus] (a) {a};
    \node[node, below right=of plus] (mul) {*};
    \node[node, below left=of mul] (b) {b};
    \node[node, below right=of mul] (c) {c};
    \draw (plus) -- (a); \draw (plus) -- (mul);
    \draw (mul) -- (b); \draw (mul) -- (c);
    \end{tikzpicture}%
    }
    \caption{Representasi AST untuk ekspresi \code{a + b * c}}
\end{figure}
