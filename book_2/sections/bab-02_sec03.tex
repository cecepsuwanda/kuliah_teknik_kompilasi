\section{Regular Expression dan Regular Language}

\subsection{Definisi Regular Expression}

Regular expression (regex) adalah notasi formal untuk mendeskripsikan pola string dalam suatu bahasa. Regular expression menggunakan operasi-operasi dasar untuk membangun pattern yang lebih kompleks.

Operasi-operasi dasar dalam regular expression meliputi:

\begin{enumerate}
    \item \textbf{Literal}: Karakter tunggal, misalnya \texttt{a} mencocokkan string ``a''
    \item \textbf{Concatenation}: Penggabungan, misalnya \texttt{ab} mencocokkan string ``ab''
    \item \textbf{Union/Alternation}: Pilihan, misalnya \texttt{a|b} mencocokkan ``a'' atau ``b''
    \item \textbf{Kleene Star}: Nol atau lebih pengulangan, misalnya \texttt{a*} mencocokkan ``'', ``a'', ``aa'', ``aaa'', dll.
    \item \textbf{Kleene Plus}: Satu atau lebih pengulangan, misalnya \texttt{a+} mencocokkan ``a'', ``aa'', ``aaa'', dll.
    \item \textbf{Optional}: Nol atau satu, misalnya \texttt{a?} mencocokkan ``'' atau ``a''
    \item \textbf{Character Class}: Set karakter, misalnya \texttt{[0-9]} mencocokkan digit 0-9
\end{enumerate}

\subsection{Contoh Regular Expression untuk Token}

Dalam lexical analysis, setiap jenis token didefinisikan menggunakan regular expression. Berikut beberapa contoh:

\begin{itemize}
    \item \textbf{Identifier}: \texttt{[a-zA-Z\_][a-zA-Z0-9\_]*}
    \begin{itemize}
        \item Dimulai dengan huruf atau underscore
        \item Diikuti oleh nol atau lebih huruf, digit, atau underscore
    \end{itemize}
    
    \item \textbf{Integer Literal}: \texttt{[0-9]+}
    \begin{itemize}
        \item Satu atau lebih digit
    \end{itemize}
    
    \item \textbf{Floating Point}: \texttt{[0-9]+\textbackslash.[0-9]+}
    \begin{itemize}
        \item Digit, titik desimal, digit
    \end{itemize}
    
    \item \textbf{String Literal}: \texttt{"([\textasciicircum"\\]|\textbackslash\textbackslash.)*"}
    \begin{itemize}
        \item Dimulai dan diakhiri dengan tanda kutip
        \item Berisi karakter apapun kecuali tanda kutip (atau escape sequence)
    \end{itemize}
    
    \item \textbf{Whitespace}: \texttt{[ \textbackslash t\textbackslash n]+}
    \begin{itemize}
        \item Satu atau lebih spasi, tab, atau newline
    \end{itemize}
    
    \item \textbf{Operator}: \texttt{+|-|*|/|=|==|!=}
    \begin{itemize}
        \item Operator aritmatika dan perbandingan
    \end{itemize}
\end{itemize}

\subsection{Regular Language}

Bahasa yang dapat dinyatakan dengan regular expression disebut \textbf{regular language}. Regular language memiliki sifat-sifat penting:

\begin{itemize}
    \item Dapat dikenali oleh finite automata (NFA atau DFA)
    \item Tertutup terhadap operasi union, concatenation, dan Kleene star
    \item Tidak dapat mengekspresikan struktur nested (seperti matching parentheses)
    \item Cukup untuk mendeskripsikan sebagian besar token dalam bahasa pemrograman
\end{itemize}