\section{Referensi dan Bahan Bacaan Lanjutan}

Untuk memperdalam pemahaman tentang type checking dan semantic analysis, mahasiswa disarankan membaca:

\begin{itemize}
    \item \textbf{Dragon Book}: Aho, Lam, Sethi, \& Ullman (2006). \textit{Compilers: Principles, Techniques, and Tools} \cite{aho2006compilers} - Bab 6: Type Checking
    
    \item \textbf{Engineering a Compiler}: Cooper \& Torczon (2011) \cite{cooper2011engineering} - Bab 4: Context-Sensitive Analysis
    
    \item \textbf{Nguyen Thanh Vu - Compiler Class Notes}: Semantic Analysis \cite{nguyen2024semantic}
    
    \item \textbf{GeeksforGeeks}: Type Checking in Compiler Design \footnote{\url{https://www.geeksforgeeks.org/type-checking-in-compiler-design/}}
    
    \item \textbf{Wikipedia - Type Inference}: \footnote{\url{https://en.wikipedia.org/wiki/Type_inference}}
    
    \item \textbf{Wikipedia - Type System}: \footnote{\url{https://en.wikipedia.org/wiki/Type_system}}
    
    \item \textbf{TypeScript Handbook - Type Compatibility}: \footnote{\url{https://www.typescriptlang.org/docs/handbook/type-compatibility.html}}
\end{itemize}