\section{Pendahuluan}

Pada bab ini kita menyelesaikan \textbf{parser proyek compiler subset C} dengan Bison. Lexer proyek sudah ada di Bab 4 (file \texttt{simplec.l}); grammar proyek telah didefinisikan di Bab 1 dan Bab 5 (Bagian~\ref{sec:spec-subset-c}, \ref{sec:grammar-proyek-subset-c}). File \texttt{simplec.y} yang dibangun di bab ini mengimplementasikan grammar tersebut dan bersama \texttt{simplec.l} menjadi tulang punggung codebase proyek.

Setelah mempelajari implementasi parser secara manual (recursive descent) dan memahami konsep LR parsing, kita akan mempelajari cara menggunakan \textbf{parser generator} untuk menghasilkan parser secara otomatis dari grammar specification. Pendekatan ini memiliki beberapa keuntungan:

\begin{itemize}
    \item \textbf{Produktivitas}: Menghemat waktu dan mengurangi kesalahan dalam implementasi parser
    \item \textbf{Maintainability}: Grammar dapat diubah dengan mudah tanpa menulis ulang parser secara manual
    \item \textbf{Konsistensi}: Parser yang dihasilkan konsisten dengan grammar specification
    \item \textbf{Error Handling}: Parser generator menyediakan mekanisme error handling yang lebih baik
\end{itemize}

Menurut sumber dari IT Trip:

\begin{quote}
``Bison / YACC: define grammar in a .y file, specify \%token s, grammar rules, actions, etc. Generates C parser (or C++ variants). Flex + Bison: use Flex to build the lexer (.l file), Bison for parser, integrate them via tokens.''\cite{ittrip2024bison}
\end{quote}

\subsection{Bison vs Yacc}

\textbf{Yacc} (Yet Another Compiler Compiler) adalah parser generator yang dikembangkan di Bell Labs pada tahun 1970-an. \textbf{Bison} adalah implementasi GNU dari Yacc dengan beberapa peningkatan:

\begin{itemize}
    \item Dukungan untuk C++ output
    \item Fitur tambahan untuk error recovery
    \item Dukungan untuk GLR (Generalized LR) parsing
    \item Dokumentasi yang lebih lengkap
    \item Open source dan aktif dikembangkan
\end{itemize}

Dalam buku ini, kita akan menggunakan \textbf{Bison} karena lebih modern dan tersedia secara luas. Namun, konsep yang dipelajari juga berlaku untuk Yacc.

Gambar \ref{fig:bison-overview} menunjukkan alur kerja Bison parser generator.

\begin{figure}[!htbp]
    \centering
    \adjustbox{max width=0.9\textwidth,center}{%
    \begin{tikzpicture}[
        file/.style={rectangle, draw=blue!50, fill=blue!10, text width=2.5cm, text centered, minimum height=0.7cm, rounded corners, font=\footnotesize, inner sep=4pt, align=center},
        process/.style={rectangle, draw=green!50, fill=green!10, text width=2.5cm, text centered, minimum height=0.7cm, rounded corners, font=\footnotesize, inner sep=4pt, align=center},
        arrow/.style={->, >=stealth, thick},
        node distance=1.2cm
    ]
    
    \node[file] (spec) {parser.y\\Grammar};
    \node[process, right=of spec] (bison) {bison\\Generator};
    \node[file, right=of bison] (code) {parser.tab.c\\Generated};
    \node[process, below=of code] (gcc) {gcc\\Compiler};
    \node[file, left=of gcc] (exe) {parser\\Executable};
    
    \draw[arrow] (spec) -- node[above, font=\tiny] {Input} (bison);
    \draw[arrow] (bison) -- node[above, font=\tiny] {Generate} (code);
    \draw[arrow] (code) -- node[right, font=\tiny] {Compile} (gcc);
    \draw[arrow] (gcc) -- node[below, font=\tiny] {Link} (exe);
    
    \end{tikzpicture}%
    }
    \caption{Workflow Bison parser generator}
    \label{fig:bison-overview}
\end{figure}