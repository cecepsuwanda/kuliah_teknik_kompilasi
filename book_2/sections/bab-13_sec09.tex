\section{Referensi dan Bahan Bacaan Lanjutan}

Untuk memperdalam pemahaman tentang runtime environment dan memory management, mahasiswa disarankan membaca:

\begin{itemize}
    \item \textbf{Dragon Book}: Aho, Lam, Sethi, \& Ullman (2006). \textit{Compilers: Principles, Techniques, and Tools} \cite{aho2006compilers} - Bab 7: Run-Time Environments
    
    \item \textbf{Engineering a Compiler}: Cooper \& Torczon (2011) \cite{cooper2011engineering} - Bab 6: The Procedure Abstraction
    
    \item \textbf{StudyLib - Outcomes-Based Education}: Materials tentang runtime environment dan activation records \cite{studylib2024obe}
    
    \item \textbf{UC San Diego CSE 231}: Course materials tentang compiler construction dan runtime organization \cite{ucsd2024compiler}
    
    \item \textbf{Northeastern University CS 4410}: Comprehensive compiler design course dengan coverage runtime issues \cite{neu2024compiler}
\end{itemize}