\section{Praktikum: TAC Generator dari AST}

Mahasiswa akan mengimplementasikan fungsi \texttt{genTAC()} pada setiap kelas node AST untuk menghasilkan instruksi IR secara otomatis.

\subsection{Pengelolaan Temp Table}
Penting untuk memastikan nama variabel \textit{temporary} (\texttt{t1, t2, ...}) unik di seluruh program dan didaftarkan ke \textit{temporary table} untuk manajemen alokasi memori nantinya.

\begin{lstlisting}[language=C++]
void genTAC(ASTNode* node) {
    if (node->type == ASSIGN) {
        string src = translateExpr(node->right);
        emit("STORE", src, "-", node->left->name);
    }
}
\end{lstlisting}
