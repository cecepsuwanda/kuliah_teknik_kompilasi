\section{Representasi: Quadruples, Triples, dan Indirect Triples}

Ada tiga cara utama untuk merepresentasikan instruksi TAC dalam struktur data memori. Pemilihan representasi ini sangat mempengaruhi performa fase optimasi.

\subsection{1. Quadruples}
Setiap instruksi disimpan dalam objek atau struct dengan empat \textit{fields}: \texttt{(op, arg1, arg2, result)}. 
\begin{itemize}
    \item \textbf{Kelebihan}: Sangat fleksibel untuk optimasi. Kita bisa memindahkan atau menghapus instruksi tanpa merusak referensi instruksi lainnya karena tujuan (\textit{result}) ditulis secara eksplisit sebagai nama variabel temporary.
    \item \textbf{Kekurangan}: Memerlukan memori lebih banyak untuk menyimpan nama variabel temporary di setiap entri.
\end{itemize}

\subsection{2. Triples}
Struktur ini hanya memiliki tiga \textit{fields}: \texttt{(op, arg1, arg2)}. Hasil dari operasi tidak diberi nama variabel, melainkan dirujuk menggunakan ID atau indeks instruksi tersebut.
\begin{itemize}
    \item \textbf{Kelebihan}: Lebih hemat memori karena tidak ada field \textit{result}.
    \item \textbf{Kekurangan}: Sangat sulit untuk dioptimasi. Jika kita memindahkan baris \#10 ke baris \#15, semua instruksi lain yang merujuk pada hasil baris \#10 (menggunakan indeks (10)) harus diperbarui secara manual.
\end{itemize}

\subsection{3. Indirect Triples}
Merupakan pengembangan dari Triples. Kita menyimpan Triples di satu tempat, dan memiliki array tambahan yang berisi \textit{pointers} ke Triples tersebut dalam urutan eksekusi yang diinginkan.
\begin{itemize}
    \item \textbf{Kelebihan}: Mendapatkan efisiensi memori Triples namun tetap mudah dioptimasi seperti Quadruples. Jika kita ingin menukar urutan kode, kita cukup menukar pointer di array tambahan tersebut.
\end{itemize}

\begin{table}[!htbp]
\centering
\begin{tabular}{|l|c|c|c|}
\hline
\textbf{Fitur} & \textbf{Quadruples} & \textbf{Triples} & \textbf{Indirect Triples} \\
\hline
Field Result & Eksplisit & Implisit (Indeks) & Implisit (Indeks) \\
Optimasi & Sangat Mudah & Sulit & Mudah (via Pointer) \\
Konsumsi Memori & Tinggi & Rendah & Menengah \\
Standar Industri & Tinggi & Rendah & Menengah \\
\hline
\end{tabular}
\caption{Perbandingan Representasi TAC}
\end{table}

\begin{figure}[!htbp]
    \centering
    \adjustbox{max width=0.9\textwidth,center}{%
    \begin{tikzpicture}[
        rect/.style={rectangle, draw=black, minimum width=2.5cm, minimum height=0.6cm, font=\small},
        row/.style={rectangle split, rectangle split parts=4, draw, rectangle split horizontal, minimum height=0.6cm, font=\small}
    ]
    % Quadruple representation
    \node (q0) at (0,0) [row] {+ \nodepart{two} a \nodepart{three} b \nodepart{four} t1};
    \node (q1) at (0,-0.7) [row] {* \nodepart{two} t1 \nodepart{three} c \nodepart{four} t2};
    \node[above=0.2cm of q0, font=\bfseries] {Quadruples};

    % Triple representation
    \node (t0) at (5,0) [row, rectangle split parts=3] {+ \nodepart{two} a \nodepart{three} b};
    \node (t1) at (5,-0.7) [row, rectangle split parts=3] {* \nodepart{two} (0) \nodepart{three} c};
    \node[above=0.2cm of t0, font=\bfseries] {Triples};
    \end{tikzpicture}%
    }
    \caption{Perbandingan Struktur Data Quadruples vs Triples}
\end{figure}
