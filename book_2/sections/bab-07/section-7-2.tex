\section{Representasi: Quadruples dan Triples}

Ada beberapa cara menyimpan instruksi TAC di dalam memori:

\subsection{Quadruples}
Setiap instruksi disimpan dalam struktur dengan empat \textit{fields}: \texttt{(op, arg1, arg2, result)}. Keunggulannya adalah kemudahan dalam optimasi karena setiap instruksi memiliki lokasi eksplisit.

\begin{figure}[!htbp]
    \centering
    \begin{tabular}{|c|c|c|c|c|}
    \hline
    \textbf{ID} & \textbf{op} & \textbf{arg1} & \textbf{arg2} & \textbf{result} \\
    \hline
    (0) & * & b & c & t1 \\
    (1) & + & a & t1 & t2 \\
    \hline
    \end{tabular}
    \caption{Representasi Quadruples untuk \code{a + b * c}}
\end{figure}

\subsection{Triples}
Representasi yang lebih efisien memori dengan mengacu pada ID instruksi sebelumnya sebagai operan: \texttt{(op, arg1, arg2)}. Hasil antara direferensikan melalui index array instruksi tersebut.
