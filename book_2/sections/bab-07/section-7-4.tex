\section{Struktur Kontrol: Label dan Jump}

\subsection{Translasi If-Then-Else}
Struktur kondisional diterjemahkan menggunakan instruksi lompatan bersyarat (\textit{conditional jump}) dan tanpa syarat (\textit{unconditional jump}).

\begin{lstlisting}
    if (condition) goto L1
    // Code for False branch
    goto L2
L1: // Code for True branch
L2: // Next statement
\end{lstlisting}

\subsection{Translasi While-Loop}
Loop memerlukan label di awal untuk pengulangan dan label di akhir untuk terminasi berdasarkan evaluasi kondisi.

\begin{figure}[!htbp]
    \centering
    \adjustbox{max width=0.8\textwidth,center}{%
    \begin{tikzpicture}[
        node/.style={rectangle, draw=blue!50, fill=blue!10, font=\tiny, align=center},
        arrow/.style={->, >=stealth, thick}
    ]
    \node[node] (start) {Label L\_start};
    \node[node, below=0.5cm of start] (cond) {If !cond goto L\_end};
    \node[node, below=0.5cm of cond] (body) {Loop Body};
    \node[node, below=0.5cm of body] (jmp) {goto L\_start};
    \node[node, below=0.5cm of jmp] (end) {Label L\_end};
    \draw[arrow] (start) -- (cond);
    \draw[arrow] (cond) -- (body);
    \draw[arrow] (body) -- (jmp);
    \draw[arrow] (jmp) to[out=180, in=180] (start);
    \draw[arrow] (cond) to[out=0, in=0] (end);
    \end{tikzpicture}%
    }
    \caption{Struktur aliran TAC untuk statement While}
\end{figure}
