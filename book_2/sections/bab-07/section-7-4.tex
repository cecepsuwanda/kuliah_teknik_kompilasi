\section{Struktur Kontrol: Label dan Jump Alternatif}

Penerjemahan struktur kontrol (\textit{If}, \textit{While}, \textit{For}) menggunakan teknik backpatching menghasilkan kode TAC yang lebih efisien karena meminimalisir jumlah instruksi \texttt{goto} yang tidak perlu.

\subsection{1. Translasi If-Then-Else}
Skema translasi untuk statement \code{if ($E$) $S_1$ else $S_2$}:
\begin{enumerate}
    \item Evaluasi ekspresi boolean $E$. Hasilnya adalah \texttt{truelist} dan \texttt{falselist}.
    \item \texttt{backpatch(E.truelist, M1)}, di mana $M1$ adalah indeks awal kode untuk $S_1$.
    \item \texttt{backpatch(E.falselist, M2)}, di mana $M2$ adalah indeks awal kode untuk $S_2$.
    \item Tambahkan instruksi \texttt{goto} di akhir $S_1$ untuk melompati $S_2$. Alamat ini disimpan dalam list \texttt{nextlist}.
\end{enumerate}

\subsection{2. Translasi While Loop}
Untuk statement \code{while ($M_1$ $E$) $M_2$ $S_1$}:
\begin{enumerate}
    \item Simpan indeks $M_1$ (awal pengecekan kondisi).
    \item Evaluasi $E$, menghasilkan \texttt{truelist} dan \texttt{falselist}.
    \item \texttt{backpatch(E.truelist, M2)}, di mana $M2$ adalah awal tubuh loop $S_1$.
    \item Di akhir $S_1$, tambahkan \texttt{goto M1} untuk pengulangan.
    \item Hasil akhir dari statement \texttt{while} adalah \texttt{E.falselist} (kontrol keluar lewat sini).
\end{enumerate}

\begin{figure}[!htbp]
    \centering
    \adjustbox{max width=0.8\textwidth,center}{%
    \begin{tikzpicture}[
        rect/.style={rectangle, draw=red!50, fill=red!10, text width=3.5cm, minimum height=0.6cm, font=\tiny, align=center},
        arrow/.style={->, >=stealth, thick}
    ]
    \node[rect] (cond) {Evaluasi E\\(truelist, falselist)};
    \node[rect, below left=1cm and 0.5cm of cond] (s1) {Statement S1\\(True Branch)};
    \node[rect, below right=1cm and 0.5cm of cond] (s2) {Statement S2\\(False Branch)};
    \node[rect, below=3cm of cond] (next) {Next Statement};
    
    \draw[arrow] (cond) -- node[left, font=\tiny] {backpatch @E.true} (s1);
    \draw[arrow] (cond) -- node[right, font=\tiny] {backpatch @E.false} (s2);
    \draw[arrow] (s1) -- (next);
    \draw[arrow] (s2) -- (next);
    \end{tikzpicture}%
    }
    \caption{Alur Backpatching pada Struktur Kontrol If-Else}
\end{figure}

\subsection{Tabel Simbol dan Label}
Kompilator menyimpan pemetaan label ke alamat memori instruksi dalam tabel khusus. Ini memungkinkan optimasi \textit{Jump-to-Jump} (menghapus lompatan yang menuju ke instruksi lompatan lain).
