\section{Postfix Notation}

Postfix notation (Reverse Polish Notation) menuliskan operator setelah operand. Representasi ini mempermudah evaluasi dengan stack dan sering digunakan sebagai jembatan menuju three-address code.

\subsection{Contoh Konversi}
Ekspresi infix:
\begin{verbatim}
(a + b) * c
\end{verbatim}
Postfix:
\begin{verbatim}
a b + c *
\end{verbatim}

\subsection{Evaluasi dengan Stack}
Setiap operand didorong ke stack; saat operator ditemui, dua operand teratas di-pop, dioperasikan, lalu hasilnya di-push kembali. Model ini sejalan dengan proses pembentukan kode intermediate.

\subsection{Keterkaitan dengan IR}
Postfix notation mempermudah pembuatan TAC karena urutan operasinya sudah eksplisit dan tidak memerlukan penanganan prioritas operator secara khusus.

\subsection{Contoh Evaluasi dengan Stack}
\begin{table}[!htbp]
\centering
\begin{tabular}{|l|l|l|}
\hline
\textbf{Token} & \textbf{Stack Sebelum} & \textbf{Stack Sesudah} \\
\hline
\texttt{a} & \texttt{[]} & \texttt{[a]} \\
\texttt{b} & \texttt{[a]} & \texttt{[a, b]} \\
\texttt{+} & \texttt{[a, b]} & \texttt{[t1]} \\
\texttt{c} & \texttt{[t1]} & \texttt{[t1, c]} \\
\texttt{*} & \texttt{[t1, c]} & \texttt{[t2]} \\
\hline
\end{tabular}
\caption{Evaluasi postfix \texttt{a b + c *} dengan stack}
\end{table}

\subsection{Algoritma Konversi Infix ke Postfix}
\begin{lstlisting}[language=C++]
// Shunting-yard (ringkas)
for (token in input) {
  if (operand) output.push(token);
  else if (token == '(') opstack.push(token);
  else if (token == ')') {
    while (opstack.top() != '(') output.push(opstack.pop());
    opstack.pop(); // buang '('
  } else { // operator
    while (!opstack.empty() &&
           prec(opstack.top()) >= prec(token)) {
      output.push(opstack.pop());
    }
    opstack.push(token);
  }
}
while (!opstack.empty()) output.push(opstack.pop());
\end{lstlisting}
