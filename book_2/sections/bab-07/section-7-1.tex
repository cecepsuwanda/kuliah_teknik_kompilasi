\section{Pengenalan Three-Address Code (TAC)}

\compiler{Three-Address Code (TAC)} adalah bentuk Intermediate Representation (IR) yang setiap instruksinya memiliki maksimal tiga operandi (biasanya satu hasil dan dua operan).

\subsection{Format Instruksi TAC}
Format umum instruksi TAC adalah: \texttt{x = y op z}.
\begin{itemize}
    \item \texttt{x}: Tujuan (biasanya variabel \textit{temporary}).
    \item \texttt{y}, \texttt{z}: Operan (variabel, konstanta, atau temporary).
    \item \texttt{op}: Operator (aritmatika, logika, atau relasional).
\end{itemize}

\subsection{Temporary Variables}
Kompilator secara otomatis membangkitkan variabel sementara (\textit{temporaries}) untuk menyimpan hasil antara dari ekspresi kompleks. Misal, \code{a + b * c} ditransformasikan menjadi:
\begin{lstlisting}
t1 = b * c
t2 = a + t1
\end{lstlisting}
