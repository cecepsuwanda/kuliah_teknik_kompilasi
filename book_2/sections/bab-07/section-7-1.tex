\section{Pengenalan Three-Address Code (TAC)}

\compiler{Three-Address Code (TAC)} adalah salah satu bentuk Intermediate Representation (IR) yang paling populer dalam desain kompilator. Nama ini merujuk pada format instruksinya yang secara formal dibatasi memiliki maksimal tiga buah alamat (\textit{addresses}) atau operan: satu tujuan (\textit{result}) dan dua sumber (\textit{arguments}).

\subsection{Mengapa Kita Memerlukan IR?}
IR seperti TAC berfungsi sebagai jembatan penting antara \textit{front-end} (yang memahami bahasa manusia) dan \textit{back-end} (yang memahami mesin spesifik). Keuntungannya meliputi:
\begin{itemize}
    \item \textbf{Retargetability}: Kita bisa menargetkan arsitektur CPU yang berbeda (x86, ARM, RISC-V) cukup dengan mengganti \textit{back-end}, tanpa menyentuh \textit{front-end}.
    \item \textbf{Machine-Independent Optimization}: Banyak optimasi seperti \textit{Constant Folding} atau \textit{Common Subexpression Elimination} jauh lebih mudah dilakukan pada level IR daripada pada kode mesin mentah.
    \item \textbf{Simplification}: Mengecilkan struktur program yang kompleks (seperti nested loops atau ekspresi rumit) menjadi urutan instruksi linier yang sederhana.
\end{itemize}

\subsection{Format Instruksi dan Variabel Temporary}
Format umum: \texttt{result = arg1 op arg2}. 
Ekspresi kompleks seperti \code{x + y * z / w} tidak bisa ditulis langsung dalam satu baris TAC. Kompilator mendekomposisinya menjadi urutan langkah kecil menggunakan variabel \textit{temporary} (\texttt{t1, t2, ...}):
\begin{lstlisting}
t1 = y * z
t2 = t1 / w
t3 = x + t2
\end{lstlisting}

\begin{figure}[!htbp]
    \centering
    \adjustbox{max width=0.8\textwidth,center}{%
    \begin{tikzpicture}[
        node/.style={rectangle, draw=blue!50, fill=blue!10, font=\small, align=center, rounded corners, minimum height=0.8cm},
        arrow/.style={->, >=stealth, thick}
    ]
    \node[node] (ast) {Annotated AST\\(Internal Representation)};
    \node[node, right=1.5cm of ast] (tac) {Three-Address Code\\(Linear IR)};
    \node[node, right=1.5cm of tac] (asm) {Target Machine Code\\(Assembly)};
    \draw[arrow] (ast) -- node[above, font=\footnotesize] {TAC Generator} (tac);
    \draw[arrow] (tac) -- node[above, font=\footnotesize] {Code Generator} (asm);
    \end{tikzpicture}%
    }
    \caption{Posisi TAC dalam Alur Kompilasi}
\end{figure}
