\section{Translasi Ekspresi Boolean dan Backpatching}

Penerjemahan ekspresi boolean (seperti \texttt{a < b || c > d}) seringkali memerlukan teknik khusus karena sifatnya yang mempengaruhi alur kontrol program, terutama jika bahasa mendukung \textit{Short-Circuit Evaluation}.

\subsection{Short-Circuit Evaluation}
Kompilator tidak perlu mengevaluasi seluruh ekspresi jika hasil akhirnya sudah pasti. 
\begin{itemize}
    \item Pada \code{A || B}, jika \code{A} benar, maka \code{B} tidak perlu dieksekusi.
    \item Pada \code{A \&\& B}, jika \code{A} salah, maka \code{B} diabaikan.
\end{itemize}
Ini diterjemahkan ke dalam TAC sebagai serangkaian \textit{conditional jumps}.

\subsection{Teknik Backpatching}
Saat membangkitkan kode dalam satu fase (\textit{one-pass}), seringkali kita belum tahu target alamat dari sebuah instruksi \texttt{goto}. Misalnya, saat melihat \texttt{if (cond)}, kita tahu harus lompat jika salah, tapi kita belum tahu baris mana yang merupakan akhir dari blok \texttt{if} tersebut. \textbf{Backpatching} menyelesaikan ini dengan meninggalkan target alamat kosong untuk sementara, lalu ''menambalnya'' kemudian.

Fungsi utama dalam backpatching:
\begin{enumerate}
    \item \textbf{makelist(i)}: Membuat list baru berisi indeks instruksi ke-$i$.
    \item \textbf{merge(p1, p2)}: Menggabungkan dua list jump menjadi satu.
    \item \textbf{backpatch(p, target)}: Mengisi semua instruksi jump dalam list $p$ dengan alamat \textit{target}.
\end{enumerate}

\begin{lstlisting}[language=C++]
// Contoh manipulasi list jump
struct BoolResult {
    list<int> truelist;  // Instruksi jump jika benar
    list<int> falselist; // Instruksi jump jika salah
};

BoolResult translateBoolean(ASTNode* node) {
    if (node->op == OR) {
        BoolResult left = translateBoolean(node->left);
        // Tandai titik masuk untuk evaluasi bagian kanan
        int M = nextInstructionID(); 
        backpatch(left.falselist, M);
        
        BoolResult right = translateBoolean(node->right);
        return { merge(left.truelist, right.truelist), right.falselist };
    }
}
\end{lstlisting}
Dengan teknik ini, list \texttt{truelist} dan \texttt{falselist} diteruskan ke atas hingga akhirnya di-backpatch oleh struktur kontrol (seperti \texttt{if} atau \texttt{while}).
