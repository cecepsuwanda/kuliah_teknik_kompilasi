\section{Kunci Jawaban}

\begin{table}[H]
\centering
\small
\begin{tabular}{|c|l|c|c|c|c|c|}
\hline
\textbf{Bab} & \textbf{Materi} & \textbf{1} & \textbf{2} & \textbf{3} & \textbf{4} & \textbf{5} \\
\hline
1 & Pengenalan Kompilator & b & c & b & d & c \\
2 & Regular Expression dan FA & b & b & b & b & b \\
3 & Implementasi Lexer & b & b & c & b & b \\
4 & Lexer Generator & b & b & b & b & b \\
5 & CFG dan Parsing & c & c & b & b & b \\
6 & Top-Down Parsing & b & b & b & a & b \\
7 & Bottom-Up Parsing & b & b & b & b & b \\
8 & Parser Generator (Bison) & b & b & b & b & b \\
9 & AST & b & b & b & b & b \\
10 & Symbol Table & b & b & b & b & b \\
11 & Type Checking & b & b & b & b & b \\
12 & Intermediate Code & a & b & b & a & b \\
13 & Runtime Environment & b & b & c & b & b \\
14 & Code Generation & a & b & b & b & b \\
15 & Optimasi & a & b & b & b & b \\
\hline
\end{tabular}
\caption{Kunci jawaban quiz Bab 1--15. Kolom 1--5 = nomor soal; isi = pilihan benar (a, b, c, atau d).}
\label{tab:kunci-quiz}
\end{table}

\noindent\textbf{Cara membaca:} Untuk Quiz Bab 1, jawaban benar soal 1 = \textbf{b}, soal 2 = \textbf{c}, soal 3 = \textbf{b}, soal 4 = \textbf{d}, soal 5 = \textbf{c}. Demikian seterusnya untuk bab lainnya.