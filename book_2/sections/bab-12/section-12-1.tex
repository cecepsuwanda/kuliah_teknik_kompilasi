\section{Pengenalan Target Machine (RISC vs CISC)}

\compiler{Target Machine} adalah arsitektur komputer tujuan di mana kode hasil kompilasi akan dijalankan. Memahami karakteristik perangkat keras sangat krusial bagi \textit{code generator} untuk menghasilkan instruksi yang optimal \cite{jhu2024compilers}.

\subsection{Arsitektur RISC (Reduced Instruction Set Computer)}
Contoh: RISC-V, ARM.
\begin{itemize}
    \item \textbf{Load-Store Architecture}: Hanya instruksi \code{Load} dan \code{Store} yang bisa mengakses memori. Instruksi aritmatika hanya bekerja pada register.
    \item \textbf{Register Pressure}: Karena semua operan harus berada di register, RISC membutuhkan lebih banyak register fisik dan temporer, yang meningkatkan "tekanan" pada pengalokasi register.
    \item \textbf{Fixed-Length}: Instruksi selalu berukuran tetap (32-bit), memudahkan \textit{pipelining}.
\end{itemize}

\subsection{Arsitektur CISC (Complex Instruction Set Computer)}
Contoh: x86 (Intel/AMD).
\begin{itemize}
    \item \textbf{Orthogonality}: Kemampuan instruksi untuk menggunakan berbagai mode pengalamatan secara bebas. Misalnya, instruksi \code{ADD} pada x86 bisa menjumlahkan register dengan memori secara langsung.
    \item \textbf{Variable-Length}: Instruksi berukuran 1-15 byte, menghemat ruang memori tapi mempersulit pendekodean instruksi (\textit{decoding}).
\end{itemize}

\subsection{Dampak pada Code Generation}
Code generator harus memilih strategi yang sesuai dengan arsitektur:
\begin{enumerate}
    \item Pada \textbf{RISC}, fokus pada jadwal instruksi (\textit{scheduling}) untuk menghindari \textit{pipeline stall} dan manajemen register yang agresif.
    \item Pada \textbf{CISC}, fokus pada pemilihan instruksi kompleks yang dapat menggabungkan beberapa operasi TAC menjadi satu instruksi mesin untuk mengurangi ukuran kode (\textit{code density}).
\end{enumerate}

\begin{figure}[!htbp]
    \centering
    \adjustbox{max width=0.8\textwidth,center}{%
    \begin{tikzpicture}[
        node/.style={rectangle, draw=blue!50, fill=blue!10, text width=6cm, font=\tiny, align=center}
    ]
    \node[node] (risc) {RISC: Banyak Temporary $\rightarrow$ Register Allocator harus cerdas.};
    \node[node, below=0.5cm of risc] (cisc) {CISC: Instruksi Kompleks $\rightarrow$ Instruction Selector harus cerdas.};
    \end{tikzpicture}%
    }
    \caption{Prioritas Optimasi berdasarkan Arsitektur Target}
\end{figure}
