\section{Pengenalan Target Machine (RISC vs CISC)}

\subsection{Arsitektur RISC (Reduced Instruction Set Computer)}
Contoh: RISC-V, ARM.
\begin{itemize}
    \item Instruksi berukuran tetap (biasanya 32-bit).
    \item Hanya instruksi \code{Load} dan \code{Store} yang bisa mengakses memori.
    \item Memiliki banyak register umum (\textit{general purpose}).
\end{itemize}

\subsection{Arsitektur CISC (Complex Instruction Set Computer)}
Contoh: x86 (Intel/AMD).
\begin{itemize}
    \item Instruksi berukuran variabel (1-15 byte).
    \item Instruksi aritmatika bisa langsung mengakses operan di memori.
    \item Jumlah register fisik lebih terbatas dibanding RISC.
\end{itemize}

\subsection{Dampak pada Code Generation}
Code generator harus menyadari arsitektur target. Pada RISC, operasi aritmatika yang kompleks harus diurai menjadi beberapa instruksi dasar \code{load-load-op-store}.
