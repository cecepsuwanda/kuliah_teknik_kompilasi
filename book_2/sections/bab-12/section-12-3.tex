\section{Register Allocation dan Graph Coloring}

Register adalah sumber daya paling berharga dalam CPU karena kecepatannya. \compiler{Register Allocation} berupaya menempatkan sebanyak mungkin variabel ke dalam register fisik dan meminimalkan \textit{spilling} (pemindahan data ke memori).

\subsection{Masalah Pewarnaan Graf (Graph Coloring)}
Alokasi register global sering dimodelkan sebagai masalah pewarnaan graf.
\begin{enumerate}
    \item \textbf{Interference Graph}: Setiap simpul mewakili sebuah variabel. Sebuah sisi menghubungkan dua variabel jika mereka "hidup" (\textit{live}) secara bersamaan. Variabel yang terhubung tidak boleh menggunakan register yang sama.
    \item \textbf{Warna (Register)}: Jika CPU memiliki $k$ register, tantangannya adalah mewarnai graf tersebut dengan maksimal $k$ warna sehingga tidak ada dua simpul bertetangga yang memiliki warna yang sama.
\end{enumerate}

\subsection{Algoritma Chaitin-Briggs}
Langkah-langkah umum algoritma ini adalah:
\begin{itemize}
    \item \textbf{Build}: Bangun graf interferensi berdasarkan analisis \textit{liveness}.
    \item \textbf{Simplify}: Cari simpul dengan derajat $< k$, hapus dari graf, dan masukkan ke dalam stack.
    \item \textbf{Spill}: Jika semua simpul berderajat $\ge k$, pilih satu variabel untuk disimpan di memori (\textit{spilling}), hapus dari graf, dan lanjutkan.
    \item \textbf{Select}: Ambil simpul dari stack satu per satu dan berikan warna yang belum digunakan oleh tetangganya.
\end{itemize}

\begin{figure}[!htbp]
    \centering
    \adjustbox{max width=0.8\textwidth,center}{%
    \begin{tikzpicture}[
        node/.style={circle, draw, minimum size=0.6cm, font=\tiny}
    ]
    \node[node] (a) {a};
    \node[node, right=1cm of a] (b) {b};
    \node[node, below=1cm of a] (c) {c};
    \node[node, right=1cm of c] (d) {d};
    
    \draw (a) -- (b);
    \draw (b) -- (d);
    \draw (c) -- (d);
    \draw (a) -- (c);
    \draw (a) -- (d);
    \end{tikzpicture}%
    }
    \caption{Graf Interferensi: Interaksi antar variabel yang tidak boleh berbagi register}
\end{figure}
