\section{Pemilihan Instruksi (Instruction Selection)}

\compiler{Instruction Selection} adalah proses memetakan instruksi tingkat menengah (\textit{TAC}) ke instruksi spesifik mesin target yang memberikan performa terbaik.

\subsection{Tiling (Pengubinan)}
Proses pemilihan instruksi sering divisualisasikan sebagai "menutupi" pohon ekspresi (\textit{Expression Tree}) dengan "ubin" (\textit{tiles}). Setiap ubin mewakili satu instruksi mesin yang dapat menggantikan satu atau lebih simpul pada pohon tersebut.

\subsection{Algoritma Pilihan}
Ada dua strategi populer untuk melakukan \textit{tiling}:
\begin{enumerate}
    \item \textbf{Maximal Munch}: Strategi \textit{greedy} (rakus). Dimulai dari akar pohon, pilih ubin terbesar yang cocok (\textit{match}). Jika ada ubin berukuran 3 simpul dan 1 simpul, ubin 3 simpul akan dipilih. Sangat efektif untuk arsitektur RISC.
    \item \textbf{Dynamic Programming}: Strategi optimal. Menghitung biaya (\textit{cost}) minimum untuk menutupi setiap sub-pohon. Algoritma ini memastikan total biaya seluruh pohon adalah yang terendah. Sangat berguna jika CPU memiliki banyak instruksi kompleks dengan biaya yang bervariasi.
\end{enumerate}

\begin{figure}[!htbp]
    \centering
    \adjustbox{max width=0.8\textwidth,center}{%
    \begin{tikzpicture}[
        node/.style={circle, draw, minimum size=0.6cm, font=\tiny}
    ]
    \node[node] (add) {+};
    \node[node, below left=0.5cm and 0.3cm of add] (mul) {*};
    \node[node, below right=0.5cm and 0.3cm of add] (d) {d};
    \node[node, below left=0.5cm and 0.2cm of mul] (b) {b};
    \node[node, below right=0.5cm and 0.2cm of mul] (c) {c};
    
    \draw[blue, thick, dashed] (-1, -1.5) rectangle (1.2, 0.5);
    \node[blue, font=\tiny] at (1.5, 0) {Tile: MADD};
    \end{tikzpicture}%
    }
    \caption{Representasi Tiling: Satu instruksi MADD menutupi operasi Multiply dan Add}
\end{figure}
