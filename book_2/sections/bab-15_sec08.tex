\section{Kombinasi Optimasi}

Dalam praktik, optimasi biasanya dilakukan dalam beberapa pass dan saling berinteraksi:

\subsection{Order of Optimization}

Urutan optimasi yang umum:

\begin{enumerate}
    \item \textbf{Constant Folding \& Propagation}: Simplifikasi ekspresi konstanta
    \item \textbf{Dead Code Elimination}: Hapus kode yang tidak digunakan
    \item \textbf{Common Subexpression Elimination}: Hapus komputasi duplikat
    \item \textbf{Loop Optimizations}: Optimasi khusus untuk loop
    \item \textbf{Register Allocation}: Alokasi register yang efisien
\end{enumerate}

\subsection{Iterative Optimization}

Optimizer biasanya menjalankan beberapa pass hingga tidak ada lagi perubahan:

\begin{verbatim}
do {
    changed = false
    changed |= constantFolding()
    changed |= constantPropagation()
    changed |= deadCodeElimination()
    // ... optimasi lainnya
} while (changed)
\end{verbatim}