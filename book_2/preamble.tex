% File konfigurasi terpusat untuk semua dokumen
% Digunakan oleh main.tex dan semua file standalone
% File ini berisi semua package dan setting yang sama

% ============================================
% PACKAGES DASAR
% ============================================
\usepackage[utf8]{inputenc}
\usepackage[T1]{fontenc}
\usepackage[bahasa]{babel}

% Font standar buku ajar Indonesia (Times New Roman atau serif)
\usepackage{times}  % Times New Roman untuk teks utama
% Alternatif: \usepackage{mathptmx} untuk Times dengan math support

% ============================================
% PACKAGES UNTUK LAYOUT DAN FORMATTING
% ============================================
\usepackage{geometry}
\usepackage{enumitem}
\usepackage{setspace}

% ============================================
% PACKAGES UNTUK HYPERLINK DAN URL
% ============================================
\usepackage[pdfencoding=auto,unicode=true,breaklinks=true]{hyperref}
\hypersetup{
    pdfstartview={FitH},
    pdfpagelayout=SinglePage,
    colorlinks=true,
    linkcolor=blue,
    urlcolor=blue,
    citecolor=blue,
    bookmarksopen=true,
    bookmarksnumbered=true
}
\usepackage{url}
\usepackage{xurl}

% ============================================
% PACKAGES UNTUK TABEL
% ============================================
\usepackage{tabularx}
\usepackage{booktabs}
\usepackage{longtable}

% ============================================
% PACKAGES UNTUK GRAFIK DAN GAMBAR
% ============================================
\usepackage{graphicx}
\usepackage{adjustbox}
\usepackage{float}
\usepackage{placeins}

% ============================================
% PACKAGES UNTUK MATEMATIKA
% ============================================
\usepackage{amsmath}
\usepackage{amssymb}
\usepackage{textcomp}
\usepackage{textgreek}

% ============================================
% PACKAGES UNTUK KODE PROGRAM
% ============================================
\usepackage{listings}
\usepackage{xcolor}
\usepackage{fancyvrb}

% ============================================
% PACKAGES UNTUK DIAGRAM DAN TREE
% ============================================
\usepackage{forest}
\usepackage{tikz}
\usetikzlibrary{shapes.geometric, arrows.meta, positioning, calc}

% ============================================
% PACKAGES UNTUK BIBLIOGRAFI
% ============================================
\usepackage{natbib}

% ============================================
% PACKAGES KHUSUS UNTUK BOOK (hanya untuk main.tex)
% ============================================
% Package ini hanya digunakan di main.tex, tidak di standalone
% \usepackage[toc]{appendix}  % Uncomment di main.tex jika diperlukan
% \usepackage{fancyhdr}        % Uncomment di main.tex jika diperlukan

% ============================================
% DUKUNGAN UNICODE KHUSUS
% ============================================
\DeclareUnicodeCharacter{2192}{$\rightarrow$}  % →
\DeclareUnicodeCharacter{2190}{$\leftarrow$}  % ←
\DeclareUnicodeCharacter{2713}{$\checkmark$}   % ✓
\DeclareUnicodeCharacter{2717}{$\times$}      % ✗
\DeclareUnicodeCharacter{8320}{$_0$}           % ₀
\DeclareUnicodeCharacter{8321}{$_1$}           % ₁
\DeclareUnicodeCharacter{8322}{$_2$}           % ₂

% ============================================
% KONFIGURASI GEOMETRY
% ============================================
% Konfigurasi geometry sesuai standar buku ajar Indonesia (Kementerian Pendidikan Tinggi)
% Untuk kertas A4 (29,7 x 21 cm):
% - Margin atas: 2,5-3 cm
% - Margin kiri: 3 cm (untuk penjilidan)
% - Margin kanan: 2 cm
% - Margin bawah: 2-2,5 cm
\geometry{
    a4paper,
    top=2.5cm,
    bottom=2.5cm,
    left=3cm,
    right=2cm,
    bindingoffset=0.5cm,
    headheight=14pt,
    headsep=0.5cm,
    footskip=1cm
}

% ============================================
% KONFIGURASI SPACING
% ============================================
% Spacing 1.5 sesuai standar Indonesia
\onehalfspacing

% ============================================
% KONFIGURASI UNTUK MENGURANGI WARNING
% ============================================
% Mengurangi overfull hbox warnings
\sloppy
\emergencystretch=3em
\hbadness=10000
\vbadness=10000
\tolerance=10000
\pretolerance=10000
% Memungkinkan breaking URL di mana saja
\Urlmuskip=0mu plus 1mu

% ============================================
% KONFIGURASI LISTINGS UNTUK KODE
% ============================================
\lstset{
    language=C++,
    basicstyle=\ttfamily\footnotesize,
    keywordstyle=\color{blue}\bfseries,
    commentstyle=\color{green!60!black},
    stringstyle=\color{red},
    numbers=left,
    numberstyle=\tiny,
    stepnumber=1,
    numbersep=4pt,
    frame=single,
    breaklines=true,
    breakatwhitespace=false,
    breakindent=0pt,
    postbreak=\mbox{\textcolor{red}{$\hookrightarrow$}\space},
    showstringspaces=false,
    tabsize=2,
    extendedchars=true,
    upquote=true,
    columns=flexible,
    keepspaces=true,
    aboveskip=3pt,
    belowskip=3pt,
    lineskip=1pt
}

% ============================================
% KONFIGURASI VERBATIM UNTUK UKURAN LEBIH KECIL
% ============================================
% Redefine verbatim environment untuk menggunakan font lebih kecil
\let\oldverbatim\verbatim
\let\endoldverbatim\endverbatim
\renewenvironment{verbatim}{%
    \footnotesize
    \oldverbatim
}{%
    \endoldverbatim
}
