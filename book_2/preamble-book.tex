% Konfigurasi khusus untuk document class 'book' (main.tex)
% File ini di-include setelah preamble.tex di main.tex

% ============================================
% PACKAGES KHUSUS UNTUK BOOK
% ============================================
\usepackage[toc]{appendix}
\usepackage{fancyhdr}

% ============================================
% KONFIGURASI UKURAN FONT UNTUK JUDUL BAB DAN SUBJUDUL
% ============================================
% Judul bab: 16pt, Section: 14pt (sesuai standar Kementerian Pendidikan Tinggi)
\makeatletter
\renewcommand{\@makechapterhead}[1]{%
  \vspace*{50\p@}%
  {\parindent \z@ \raggedright \normalfont
    \ifnum \c@secnumdepth >\m@ne
        \huge\bfseries \@chapapp\space \thechapter
        \par\nobreak
        \vskip 20\p@
    \fi
    \interlinepenalty\@M
    \fontsize{16}{20}\selectfont\bfseries #1\par\nobreak
    \vskip 40\p@
  }}
% Konfigurasi section (14pt sesuai standar)
\renewcommand{\section}{\@startsection {section}{1}{\z@}%
                                   {-3.5ex \@plus -1ex \@minus -.2ex}%
                                   {2.3ex \@plus.2ex}%
                                   {\normalfont\fontsize{14}{18}\selectfont\bfseries}}
\makeatother

% ============================================
% KONFIGURASI HEADER DAN FOOTER
% ============================================
\pagestyle{fancy}
\fancyhf{}
\fancyhead[LE]{\small\leftmark}
\fancyhead[RO]{\small\rightmark}
\fancyfoot[C]{\small\thepage}
\renewcommand{\headrulewidth}{0.4pt}
\renewcommand{\footrulewidth}{0pt}
