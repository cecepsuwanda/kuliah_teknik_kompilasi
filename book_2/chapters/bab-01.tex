\documentclass[../main.tex]{subfiles}

\addbibresource{\subfix{../references.bib}}

\begin{document}

\ifSubfilesClassLoaded{%
    \setcounter{chapter}{0}%
    \begin{refsection}
}{}

\chapter{Pendahuluan dan Orientasi OBE}
\label{chap:pendahuluan}

\begin{subcpmk}
  \item Memahami keterkaitan antara kurikulum \textit{Outcome-Based Education} (OBE) dengan desain sistem kompilator.
  \item Mengidentifikasi alur pembelajaran Teknik Kompilasi melalui peta konsep buku ajar.
  \item Merencanakan strategi belajar mandiri untuk menguasai setiap fase kompilasi secara sistematis.
\end{subcpmk}

% ============================================================
% MATERI POKOK
% ============================================================
\section{Tujuan Buku Ajar}

Buku ajar ini dirancang sebagai panduan komprehensif untuk menguasai \compiler{Teknik Kompilasi} secara sistematis dan terukur, selaras dengan standar \textit{Outcome-Based Education} (OBE) \cite{studylib2024obe}. Fokus utama buku ini adalah pada pembangunan fondasi teoretis dan keterampilan praktis dalam membangun arsitektur kompilator modern. Menurut \cite{aho2006compilers}, kompilasi adalah proses transformasi dari bahasa sumber ke bahasa sasaran.
\begin{enumerate}
  \item Memberikan pemahaman mendalam tentang setiap fase kompilasi, mulai dari analisis leksikal hingga generasi kode target.
  \item Mengembangkan kemampuan merancang dan mengimplementasikan komponen-komponen utama kompilator seperti \textit{lexer}, \textit{parser}, dan \textit{semantic analyzer}.
  \item Membangun keterampilan dalam optimasi kode dan manajemen memori pada \textit{runtime}.
  \item Memfasilitasi pencapaian \textit{Capaian Pembelajaran Lulusan} (CPL) dan \textit{Capaian Pembelajaran Mata Kuliah} (CPMK) yang telah ditetapkan dalam kurikulum.
\end{enumerate}

Setelah mempelajari buku ini secara menyeluruh, mahasiswa diharapkan mampu:
\begin{itemize}
  \item Menjelaskan arsitektur kompilator dan fungsi setiap fasenya.
  \item Membangun pemroses bahasa (\textit{language processor}) menggunakan teknik manual maupun generator (\textit{Flex/Bison}).
  \item Mengelola struktur data kompleks seperti \textit{Symbol Table} dan \textit{Abstract Syntax Tree} (AST).
  \item Menghasilkan kode target yang efisien untuk arsitektur mesin tertentu.
  \item Melakukan evaluasi performa dan optimasi pada tingkat \textit{intermediate code} dan \textit{target code}.
\end{itemize}

\section[Keterkaitan Buku Ajar dengan RPS Berbasis OBE]{Keterkaitan Buku Ajar dengan RPS \protect\\ Berbasis OBE}

Buku ajar ini disusun dengan penyelarasan yang ketat terhadap Rencana Pembelajaran Semester (RPS) mata kuliah Teknik Kompilasi yang menerapkan kerangka kerja \textit{Outcome-Based Education} (OBE). Keterkaitan ini menjamin bahwa setiap aktivitas kognitif yang dilakukan mahasiswa—mulai dari memahami definisi hingga merancang sistem—memiliki kontribusi langsung terhadap pencapaian Capaian Pembelajaran Lulusan (CPL). Integrasi yang kuat antara konten buku dan CPL memastikan bahwa keterampilan yang dikuasai mahasiswa relevan dengan kebutuhan dan standar kompetensi di industri teknologi informasi modern \cite{neu2024compiler}.

Secara spesifik, buku ini dirancang untuk memenuhi CPL-1 (Pengetahuan Rekayasa) dengan menyajikan teori otoritatif mengenai bahasa formal, automata, dan tata bahasa bebas konteks. Di sisi lain, buku ini juga secara intensif menargetkan CPL-3 (Perancangan dan Pengembangan Solusi) dengan membimbing mahasiswa melalui proses iteratif pembangunan kompilator yang fungsional. Keseimbangan ini memastikan mahasiswa tidak hanya tumbuh sebagai teoretisi yang memahami konsep abstrak, tetapi juga sebagai insinyur perangkat lunak yang mampu menghasilkan solusi konkret untuk masalah yang kompleks.

Aktivitas pembelajaran dalam buku ini disusun secara bertingkat mengikuti hierarki Taksonomi Bloom untuk memandu perkembangan kognitif mahasiswa. Dimulai dari level dasar seperti memahami sintaksis dan semantik bahasa, mahasiswa dibimbing menuju level analisis untuk memecahkan konflik \textit{parsing}, dan akhirnya mencapai level tertinggi yaitu menciptakan (\textit{creating}) sebuah pemroses bahasa yang utuh. Progresi bertahap ini sangat krusial dalam pendidikan teknik untuk membantu mahasiswa menguasai permasalahan rekayasa yang rumit secara terstruktur dan percaya diri.

\section{Arsitektur Kompilator Modern}

Arsitektur kompilator modern umumnya terbagi menjadi front-end dan back-end \cite{cooper2011engineering}.

\subsection{Alignment dengan CPL dan CPMK}

Struktur buku ini disusun untuk mendukung pencapaian indikator-indikator berikut:
\begin{itemize}
  \item \textbf{CPL-1 (Pengetahuan)}: Menguasai konsep teoretis analisis leksikal, sintaksis, semantik, dan generasi kode secara mendalam.
  \item \textbf{CPL-3 (Keterampilan Khusus)}: Mampu merancang, mengimplementasikan, dan mengevaluasi sistem kompilator lengkap.
  \item \textbf{CPMK-1 s.d CPMK-6}: Meliputi seluruh spektrum pengembangan kompilator dari arsitektur awal hingga evaluasi performa akhir.
\end{itemize}

Setiap bab dalam buku ini memuat daftar \textbf{Sub-CPMK} di bagian awal untuk memberikan fokus yang jelas bagi mahasiswa mengenai kompetensi spesifik yang akan dikuasai.

\subsection{Integrasi Metode Pembelajaran Aktif}

Sesuai dengan RPS berbasis OBE, buku ini mendukung berbagai metode pembelajaran:
\begin{itemize}
  \item \textbf{Problem-Based Learning}: Melalui studi kasus penanganan \textit{semantic errors} dan optimasi lokal.
  \item \textbf{Project-Based Learning}: Pengembangan kompilator secara bertahap dalam setiap bab.
  \item \textbf{Praktikum Terbimbing}: Implementasi komponen menggunakan \textit{tooling} industri seperti LLVM atau Clang.
\end{itemize}

\subsection{Sistem Evaluasi Berbasis Kompetensi}

Komponen asesmen yang disediakan di setiap akhir bab (latihan, asesmen, dan \textit{checklist}) dirancang untuk mengukur pencapaian \textit{Sub-CPMK} secara objektif, yang nantinya akan menjadi bobot penilaian utama dalam UTS dan UAS (total 35\% sesuai RPS).

\section{Petunjuk Penggunaan Buku Ajar}

\subsection{Untuk Mahasiswa}

Mengingat kompleksitas pengembangan kompilator, mahasiswa disarankan untuk menggunakan buku ini dengan langkah-langkah berikut:

\textbf{Tahap Persiapan:}
\begin{enumerate}
  \item Pahami target \textit{Sub-CPMK} di awal bab agar fokus belajar tetap terjaga.
  \item Tinjau kembali materi prasyarat (Struktur Data dan Algoritma) jika diperlukan.
\end{enumerate}

\textbf{Tahap Implementasi (Eksperimental):}
\begin{enumerate}
  \item Pelajari kode contoh yang disediakan dan jalankan menggunakan \textit{compiler} atau \textit{interpreter} yang sesuai.
  \item Lakukan modifikasi pada parameter \textit{lexer} atau aturan \textit{grammar} untuk melihat dampaknya terhadap proses \textit{parsing}.
  \item Gunakan perangkat lunak pendukung seperti \textit{Flex}, \textit{Bison}, atau \textit{Graphviz} untuk memvisualisasikan AST.
\end{enumerate}

\textbf{Tahap Evaluasi:}
\begin{enumerate}
  \item Kerjakan latihan refleksi untuk memperdalam pemahaman teoretis.
  \item Lakukan penilaian mandiri menggunakan \textit{checklist} kompetensi di akhir bab.
  \item Gabungkan komponen yang telah dibuat di setiap bab menjadi satu proyek kompilator utuh.
\end{enumerate}

\subsection{Untuk Dosen}

Dosen dapat memanfaatkan buku ini sebagai instrumen pembelajaran utama:
\begin{itemize}
  \item \textbf{Modul Praktikum}: Gunakan aktivitas pembelajaran di setiap bab sebagai panduan tugas mingguan.
  \item \textbf{Bank Soal}: Manfaatkan bagian asesmen sebagai referensi dalam menyusun soal UTS (CPMK-1, 2) dan UAS (CPMK 1 s.d 6).
  \item \textbf{Alat Ukur Capaian}: Gunakan rubrik penilaian dan indikator dalam buku untuk mengukur ketercapaian outcomes mahasiswa.
\end{itemize}

\section{Konteks Kurikulum OBE}

\subsection{Filosofi OBE dalam Teknik Kompilasi}

\compiler{Outcome-Based Education (OBE)} adalah pendekatan yang menekankan pada apa yang bisa dilakukan oleh mahasiswa di akhir masa studi, bukan sekadar apa yang diajarkan. Dalam Teknik Kompilasi, hal ini berarti mahasiswa tidak hanya menghafal algoritma \textit{parsing}, tetapi mampu membangun sebuah program yang secara nyata dapat menerjemahkan sebuah bahasa ke bahasa lain.

\textbf{Empat Prinsip Utama OBE dalam Buku Ini:}
\begin{enumerate}
  \item \textbf{Clarity of Focus}: Fokus pada hasil akhir berupa kompilator yang berfungsi.
  \item \textbf{Designing Down}: Materi disusun mundur dari kebutuhan akhir sebuah sistem \textit{backend} kompilator.
  \item \textbf{High Expectations}: Mahasiswa didorong untuk mengimplementasikan optimasi kode yang efisien.
  \item \textbf{Expanded Opportunity}: Menyediakan berbagai aktivitas belajar mulai dari teori hingga proyek tim.
\end{enumerate}

\subsection{Implementasi Tahapan OBE}

Buku ini membagi proses pencapaian kompetensi dalam empat pilar utama:

\begin{table}[!htbp]
\centering
\begin{tabular}{|l|p{10.5cm}|}
\hline
\textbf{Komponen OBE} & \textbf{Implementasi dalam Teknik Kompilasi} \\
\hline
\textit{Defined Outcomes} & Sub-CPMK eksplisit untuk setiap fase (Leksikal, Sintaksis, dst). \\
\hline
\textit{Designing Down} & Kurikulum dimulai dari pengenalan bahasa ke deteksi kesalahan hingga emisi kode. \\
\hline
\textit{Student Activity} & Implementasi manual mesin \textit{state} dan penggunaan alat otomatisasi generator. \\
\hline
\textit{Continuous Assessment} & \textit{Weekly reflection} dan audit kualitas kode secara berkala. \\
\hline
\end{tabular}
\caption{Penerapan OBE dalam Pengembangan Kompilator}
\end{table}

\subsection{Hierarki Capaian Pembelajaran}

\begin{konsep}
Pemahaman mahasiswa terhadap Teknik Kompilasi divalidasi melalui hierarki capaian:
\begin{itemize}
    \item \textbf{CPL}: Mahasiswa menguasai teori kompilasi secara utuh sebagai sarjana teknik.
    \item \textbf{CPMK}: Mahasiswa mampu mengintegrasikan fase-fase kompilasi menjadi sistem yang koheren.
    \item \textbf{Sub-CPMK}: Mahasiswa mahir dalam satu spesialisasi fase (misalnya: \textit{Register Allocation}).
\end{itemize}
\end{konsep}

\section{Peta Konsep Teknik Kompilasi}

Buku ini disusun dalam 19 bab yang mencakup seluruh spektrum pengembangan kompilator, yang dapat dipandang sebagai sebuah ``pipa transformasi'' data. Proses dimulai dari aliran karakter mentah (\textit{source code}), yang kemudian diubah menjadi token linear, disusun menjadi struktur pohon hierarkis (\textit{Syntax Tree}), diperkaya dengan informasi semantik, diterjemahkan menjadi kode antara (\textit{Intermediate Representation}) yang agnostik terhadap mesin, dan akhirnya diekspansi menjadi instruksi mesin spesifik (\textit{Assembly}) yang optimal. Setiap bab dalam buku ini membedah satu ruas dari pipa tersebut secara mendalam.

Secara holistik, peta konsep ini tidak hanya mengajarkan teknik isolasi komponen, tetapi juga integrasi sistem. Mahasiswa diajak untuk melihat bagaimana keputusan desain di satu fase (misalnya, desain IR) berdampak signifikan pada fase berikutnya (optimasi dan generasi kode). Pemahaman lintas-fase ini adalah esensi dari pemikiran sistem (\textit{systems thinking}) yang ingin dibangun melalui mata kuliah ini.

Berikut adalah rincian materi per bab:
\begin{enumerate}
  \item \textbf{Bab I}: Pengenalan dan Konteks OBE
  \item \textbf{Bab II}: Arsitektur Kompilator - Gambaran umum sistem
  \item \textbf{Bab III-IV}: \textit{Front-end} - Analisis leksikal dan representasi regular
  \item \textbf{Bab V-VI}: \textit{Syntax Analysis} - \textit{Parsing} dan \textit{grammar} formal
  \item \textbf{Bab VII-X}: \textit{Middle-end} - \textit{Intermediate code}, tabel simbol, analisis semantik, dan penanganan kesalahan
  \item \textbf{Bab XI-XIV}: \textit{Back-end} - Tata letak memori, \textit{code generation}, alokasi register, dan manajemen \textit{stack}
  \item \textbf{Bab XV-XVI}: \textit{Analysis \& Evaluation} - \textit{Compiler tools} dan evaluasi performa
  \item \textbf{Bab XVII-XIX}: \textit{Assessment \& Resources} - Evaluasi kompetensi, lampiran, dan daftar referensi
\end{enumerate}

\textbf{Alur Pembelajaran:}
\begin{itemize}
  \item \textbf{Fase Analisis (Bab II-VI)}: Memahami bagaimana bahasa manusia diterjemahkan menjadi token dan pohon hirarki.
  \item \textbf{Fase Transformasi (Bab VII-X)}: Memastikan kebenaran makna dan mengubahnya menjadi representasi antara.
  \item \textbf{Fase Sintesis (Bab XI-XIV)}: Membangun instruksi mesin yang optimal sesuai arsitektur target.
  \item \textbf{Fase Profesional (Bab XV-XIX)}: Menggunakan alat bantu modern dan melakukan standarisasi kualitas.
\end{itemize}

\section{Proyek Buku: Compiler Subset C}
\label{sec:spec-subset-c}

Salah satu kekuatan utama buku ajar ini adalah penggunaan proyek pengembangan tunggal yang berkelanjutan (\textit{continuous project}) berupa kompilator untuk subset bahasa C. Pendekatan ini dipilih karena bahasa C merupakan \textit{lingua franca} sistem pemrograman yang memiliki karakteristik imperatif, prosedural, dan \textit{statically typed} yang representatif. Dengan membangun kompilator untuk subset C, mahasiswa akan menghadapi tantangan nyata yang relevan dengan industri, namun dalam lingkup yang terkendali sehingga tetap dapat diselesaikan dalam satu semester.

Proyek ini dibangun menggunakan filosofi pengembangan inkremental (\textit{incremental development}). Kita tidak mencoba membangun seluruh kompilator sekaligus. Sebaliknya, kita membangunnya lapis demi lapis—dimulai dari lexer sederhana, kemudian parser, lalu AST, dan seterusnya. Setiap bab menambahkan fungsionalitas baru ke atas fondasi yang telah dibangun sebelumnya. Metode ini tidak hanya memudahkan proses \textit{debugging}, tetapi juga mengajarkan disiplin rekayasa perangkat lunak tentang bagaimana mengelola kompleksitas melalui modularitas.

Sepanjang Bab 2 hingga Bab 16, kita secara bertahap membangun \textbf{satu compiler untuk subset bahasa C}. Setiap bab menambah satu lapis ke proyek yang sama: spesifikasi token (Bab 3), lexer hand-written (Bab 3), lexer Flex (Bab 4), grammar (Bab 5), parser hand-written (Bab 6), teori bottom-up (Bab 7), parser Bison (Bab 8), AST (Bab 9), symbol table (Bab 10), type checking (Bab 11), IR (Bab 12), runtime (Bab 13), code generation (Bab 14), optimasi (Bab 15), dan integrasi (Bab 16). Spesifikasi berikut menjadi acuan tunggal agar semua contoh dan kode mengacu ke bahasa yang sama.

\subsection{Spesifikasi Token Proyek Subset C}

Token yang dikenali oleh compiler proyek (untuk Bab 3--4):

\begin{itemize}
    \item \textbf{Identifier}: huruf atau underscore diikuti huruf, angka, atau underscore. Pola: \texttt{[a-zA-Z\_][a-zA-Z0-9\_]*}
    \item \textbf{Kata kunci}: \texttt{int}, \texttt{float}, \texttt{print}. (Nanti dapat diperluas: \texttt{if}, \texttt{else}, \texttt{while}.)
    \item \textbf{Literal}: integer \texttt{[0-9]+}, float \texttt{[0-9]+.[0-9]+}, string \texttt{"..."} dalam tanda kutip ganda.
    \item \textbf{Operator}: \texttt{+}, \texttt{-}, \texttt{*}, \texttt{/}, \texttt{=}, \texttt{==}, \texttt{!=}, \texttt{<}, \texttt{>}, \texttt{<=}, \texttt{>=}.
    \item \textbf{Punctuator}: \texttt{;}, \texttt{,}, \texttt{(} \texttt{)}, kurung kurawal \texttt{\char`\{\char`\}}.
    \item \textbf{Komentar}: satu baris \texttt{//} dan banyak baris \texttt{/* */}; serta whitespace (spasi, tab, newline) diabaikan.
\end{itemize}

\subsection{Spesifikasi Grammar Proyek Subset C}

Grammar dalam BNF untuk Bab 5--8 (dan parser proyek):

\begin{itemize}
    \item \textbf{Program}: barisan statement.
    \item \textbf{Statement}: deklarasi \texttt{;} \textbar\ assignment \texttt{;} \textbar\ print-statement \texttt{;}
    \item \textbf{Deklarasi}: \texttt{int} identifier \textbar\ \texttt{float} identifier
    \item \textbf{Assignment}: identifier \texttt{=} ekspresi
    \item \textbf{Print-statement}: \texttt{print} \texttt{(} string-literal \texttt{)} \textbar\ \texttt{print} \texttt{(} ekspresi \texttt{)}
    \item \textbf{Ekspresi}: term \textbar\ ekspresi \texttt{+} term \textbar\ ekspresi \texttt{-} term
    \item \textbf{Term}: factor \textbar\ term \texttt{*} factor \textbar\ term \texttt{/} factor
    \item \textbf{Factor}: literal \textbar\ identifier \textbar\ \texttt{(} ekspresi \texttt{)}
\end{itemize}

Precedence: \texttt{*} dan \texttt{/} lebih tinggi dari \texttt{+} dan \texttt{-}; associativity kiri untuk semuanya.

\subsection{Peta Bab ke Lapis Proyek}

\begin{center}
\begin{tabular}{cl}
\toprule
\textbf{Bab} & \textbf{Lapis proyek} \\
\midrule
3 & Spesifikasi token + teori RE/FA \\
3 & Lexer hand-written (mengikuti spec token) \\
4 & Lexer proyek (Flex, file \texttt{simplec.l}) \\
5 & Grammar proyek (BNF/EBNF di atas) \\
6 & Parser hand-written (mengikuti grammar proyek) \\
7 & Teori LR; grammar proyek termasuk kelas LR \\
8 & Parser proyek (Bison, file \texttt{simplec.y}) \\
9 & AST proyek (\texttt{ast.h}/\texttt{ast.c}) \\
10 & Symbol table proyek (\texttt{symtab.h}/\texttt{symtab.c}) \\
11 & Type checking proyek \\
12 & IR proyek (TAC/quadruples dari AST) \\
13 & Runtime; asumsi proyek untuk stack/activation record \\
14 & Code generation proyek (IR $\to$ assembly) \\
15 & Optimasi proyek (basic block, constant folding, dll.) \\
16 & Integrasi dan presentasi compiler subset C lengkap \\
\bottomrule
\end{tabular}
\end{center}

Semua bab dari Bab 3 sampai Bab 16 merujuk ke spesifikasi ini. Kode dan contoh dalam bab tersebut mengacu ke token set dan grammar di atas, serta ke file proyek (\texttt{simplec.l}, \texttt{simplec.y}, dan seterusnya) yang tumbuh di folder \texttt{proyek-compiler-subset-c/}.


% ============================================================
% AKTIVITAS PEMBELAJARAN
% ============================================================
\begin{aktivitas}
  \item \textbf{Analisis RPS}: Pelajari RPS Teknik Kompilasi dan identifikasi bagaimana CPL Pengetahuan (CPL-1) diukur melalui proyek pengembangan kompilator.
  \item \textbf{Pemetaan Fase}: Buat diagram alir yang memetakan bab-bab dalam buku ini ke dalam tiga pilar utama: \textit{Front-end}, \textit{Middle-end}, dan \textit{Back-end}.
  \item \textbf{Tooling Audit}: Identifikasi \textit{software} pendukung (GCC, Flex, Bison, Clang) yang akan digunakan di setiap bab berdasarkan peta konsep.
  \item \textbf{Diskusi Kurikulum}: Diskusikan mengapa penguasaan Teknik Kompilasi sangat krusial dalam mencapai standar kompetensi lulusan Teknik Informatika di industri modern.
\end{aktivitas}

% ============================================================
% LATIHAN DAN REFLEKSI
% ============================================================
\begin{latihan}
  \item Jelaskan secara spesifik bagaimana pendekatan OBE membantu mahasiswa dalam menghadapi kompleksitas algoritma \textit{parsing}!
  \item Mengapa setiap aktivitas praktikum harus memiliki rubrik penilaian yang eksplisit dalam konteks OBE?
  \item Bagaimana cara Anda memantau kemajuan pembangunan proyek kompilator Anda menggunakan \textit{checklist} kompetensi?
  \item Hubungkan antara \textit{Target Architecture} (Bab XII) dengan tujuan akhir pencapaian kompetensi dalam RPS!
  \item \textbf{Refleksi}: Sejauh mana Anda memahami bahwa membangun sebuah kompilator adalah bukti nyata pencapaian kompetensi teknik yang utuh?
\end{latihan}

% ============================================================
% ASESMEN
% ============================================================
\begin{asesmen}
\textbf{Instrumen Penilaian untuk Orientasi OBE Kompilator}

\textbf{A. Pilihan Ganda}

\begin{enumerate}
  \item Manakah yang merupakan contoh \textit{Outcome} nyata dalam mata kuliah ini?
  \begin{enumerate}
    \item Membaca buku Naga (Dragon Book)
    \item Lulus ujian teori
    \item Menghasilkan kode assembly yang dapat dijalankan
    \item Mengetahui sejarah FORTRAN
  \end{enumerate}
  
  \item \textit{Designing Down} dalam buku ini berarti:
  \begin{enumerate}
    \item Mendesain dari tingkat mesin ke bahasa tingkat tinggi
    \item Menyusun materi berdasarkan urutan fase kompilasi untuk mencapai produk akhir
    \item Mengurangi beban materi yang sulit
    \item Hanya fokus pada latihan praktak
  \end{enumerate}
\end{enumerate}

\textbf{B. Essay}

\begin{enumerate}
  \item Jelaskan keterkaitan antara peta konsep (Bab I s.d XIX) dengan pencapaian kompetensi profesional seorang \textit{System Programmer}!
  \item Desainlah satu indikator pencapaian kompetensi untuk Sub-CPMK "Membangun Lexer Hand-written"!
\end{enumerate}

\textbf{Rubrik Penilaian}: Lihat Lampiran A
\end{asesmen}

% ============================================================
% CHECKLIST KOMPETENSI
% ============================================================
\begin{checklist}
  \item Saya memahami filosofi OBE dalam konteks pengembangan sistem kompilator.
  \item Saya dapat memetakan hierarki CPL, CPMK, dan Sub-CPMK Teknik Kompilasi ke dalam konten buku.
  \item Saya memahami alur kerja \textit{Front-end}, \textit{Middle-end}, dan \textit{Back-end} melalui peta konsep.
  \item Saya telah menyiapkan \textit{software stack} yang diperlukan untuk proyek semester ini.
  \item Saya berkomitmen untuk melakukan \textit{self-assessment} secara berkala di setiap akhir bab.
\end{checklist}

% ============================================================
% RANGKUMAN BAB
% ============================================================
\begin{rangkuman}
Bab ini memberikan fondasi bagi mahasiswa untuk memahami bagaimana buku ajar ini disusun menggunakan standar OBE guna menjamin penguasaan Teknik Kompilasi yang utuh dan profesional.

\textbf{Poin Kunci:}
\begin{itemize}
  \item Fokus utama adalah pencapaian \textit{outcome} berupa sistem kompilator yang berfungsi.
  \item Kurikulum dirancang mundur (\textit{designing down}) untuk memandu mahasiswa dari analisis ke sintesis kode.
  \item Peta konsep menunjukkan integrasi 19 bab sebagai satu kesatuan ekosistem pembelajaran.
  \item Peran aktif mahasiswa dalam evaluasi diri adalah kunci keberhasilan dalam sistem OBE.
\end{itemize}

\textbf{Kata Kunci}: \compiler{OBE}, \compiler{Teknik Kompilasi}, \compiler{Peta Konsep}, \compiler{CPL}, \compiler{Outcome}, \compiler{Compiler Design}
\end{rangkuman}

\ifSubfilesClassLoaded{%
    \clearpage
    \printbibliography[title={Daftar Pustaka}]
    \end{refsection}
}{}

\end{document}
