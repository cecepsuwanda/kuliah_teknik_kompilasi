\documentclass[../main.tex]{subfiles}

\addbibresource{\subfix{../references.bib}}

\begin{document}

\ifSubfilesClassLoaded{%
    \setcounter{chapter}{16}%
    \begin{refsection}
}{}

\chapter{Evaluasi dan Refleksi Kompetensi}
\label{chap:evaluasi-kompetensi}

\begin{subcpmk}
  \item \textbf{Sub-CPMK 1.1-6.2:} Mengevaluasi seluruh capaian pembelajaran mata kuliah Teknik Kompilasi
\end{subcpmk}

% ============================================================
% MATERI POKOK
% ============================================================
\section{Latihan Mandiri Komprehensif}

Bagian ini menyajikan kumpulan soal latihan untuk menguji pemahaman mahasiswa terhadap seluruh fase kompilasi yang telah dipelajari.

\subsection{Analisis Leksikal dan Sintaksis}
Identifikasi token dan gambarkan \textit{parse tree} untuk ekspresi berikut: \code{x = (10 + y) * 2;}.

\subsection{Semantik dan Tabel Simbol}
Gambarkan hierarki tabel simbol untuk kode program yang memiliki fungsi di dalam lingkup global dan variabel lokal dengan nama yang sama (\textit{shadowing}).

\subsection{Generasi IR dan Optimasi}
Tuliskan \textit{Three-Address Code} (TAC) untuk struktur \code{while} loop dan lakukan \textit{Constant Folding} jika terdapat ekspresi statis.

\section{Evaluasi Capaian Pembelajaran Mata Kuliah}

\subsection{CPMK-1: Arsitektur Kompilator}

\begin{table}[h]
\centering
\begin{tabular}{|l|c|c|c|c|}
\hline
\textbf{Sub-CPMK} & \textbf{Teori} & \textbf{Praktik} & \textbf{Proyek} & \textbf{Nilai} \\
\hline
1.1: Fase kompilasi & 20\% & 15\% & 10\% & 45\% \\
1.2: Struktur kompilator & 20\% & 15\% & 10\% & 45\% \\
1.3: Intermediate code & 20\% & 15\% & 10\% & 45\% \\
\hline
\end{tabular}
\caption{Komponen Penilaian CPMK-1}
\end{table}

\subsection{CPMK-2: Lexer dan Parser}

\begin{table}[h]
\centering
\begin{tabular}{|l|c|c|c|c|}
\hline
\textbf{Sub-CPMK} & \textbf{Teori} & \textbf{Praktik} & \textbf{Proyek} & \textbf{Nilai} \\
\hline
2.1: Regular expression & 15\% & 25\% & 15\% & 55\% \\
2.2: Finite automata & 15\% & 25\% & 15\% & 55\% \\
2.3: Parsing techniques & 20\% & 20\% & 15\% & 55\% \\
2.4: Parser generators & 15\% & 25\% & 15\% & 55\% \\
\hline
\end{tabular}
\caption{Komponen Penilaian CPMK-2}
\end{table}

\subsection{CPMK-3: Semantic Analysis}

\begin{table}[h]
\centering
\begin{tabular}{|l|c|c|c|c|}
\hline
\textbf{Sub-CPMK} & \textbf{Teori} & \textbf{Praktik} & \textbf{Proyek} & \textbf{Nilai} \\
\hline
3.1: Symbol table & 15\% & 25\% & 15\% & 55\% \\
3.2: Scope management & 15\% & 25\% & 15\% & 55\% \\
3.3: Type checking & 20\% & 20\% & 15\% & 55\% \\
\hline
\end{tabular}
\caption{Komponen Penilaian CPMK-3}
\end{table}

\subsection{CPMK-4: Intermediate Code dan Optimasi}

\begin{table}[h]
\centering
\begin{tabular}{|l|c|c|c|c|}
\hline
\textbf{Sub-CPMK} & \textbf{Teori} & \textbf{Praktik} & \textbf{Proyek} & \textbf{Nilai} \\
\hline
4.1: Three-address code & 15\% & 25\% & 15\% & 55\% \\
4.2: Basic blocks & 15\% & 25\% & 15\% & 55\% \\
4.3: Local optimization & 20\% & 20\% & 15\% & 55\% \\
\hline
\end{tabular}
\caption{Komponen Penilaian CPMK-4}
\end{table}

\subsection{CPMK-5: Code Generation dan Runtime}

\begin{table}[h]
\centering
\begin{tabular}{|l|c|c|c|c|}
\hline
\textbf{Sub-CPMK} & \textbf{Teori} & \textbf{Praktik} & \textbf{Proyek} & \textbf{Nilai} \\
\hline
5.1: Runtime environment & 15\% & 25\% & 15\% & 55\% \\
5.2: Memory layout & 15\% & 25\% & 15\% & 55\% \\
5.3: Code generation & 20\% & 20\% & 15\% & 55\% \\
\hline
\end{tabular}
\caption{Komponen Penilaian CPMK-5}
\end{table}

\subsection{CPMK-6: Tools dan Evaluasi}

\begin{table}[h]
\centering
\begin{tabular}{|l|c|c|c|c|}
\hline
\textbf{Sub-CPMK} & \textbf{Teori} & \textbf{Praktik} & \textbf{Proyek} & \textbf{Nilai} \\
\hline
6.1: Compiler tools & 15\% & 25\% & 15\% & 55\% \\
6.2: Performance evaluation & 15\% & 25\% & 15\% & 55\% \\
\hline
\end{tabular}
\caption{Komponen Penilaian CPMK-6}
\end{table}

\section{Refleksi Pembelajaran}

\subsection{Pertanyaan Reflektif Teknis}

\textbf{Bagian 1: Arsitektur dan Trade-offs}
\begin{enumerate}
  \item Setelah mempelajari interpretasi dan kompilasi, dalam skenario apa Anda akan memilih \textit{Just-In-Time} (JIT) compiler dibandingkan kompilasi AOT (\textit{Ahead-of-Time})?
  \item Analisis \textit{trade-off} antara kecepatan kompilasi (\code{-O0}) dan performa waktu eksekusi (\code{-O3}). Kapan kecepatan kompilasi menjadi lebih penting bagi pengembang?
  \item Mengapa transisi dari komponen buatan tangan (\textit{hand-written}) ke generator otomatis (seperti Flex/Bison) dianggap sebagai langkah krusial dalam produktivitas pengembangan kompilator?
\end{enumerate}

\textbf{Bagian 2: Keterampilan Praktis dan Implementasi}
\begin{enumerate}
  \item Implementasi fase mana (Lexer, Parser, atau Semantik) yang memberikan wawasan terdalam bagi Anda tentang cara kerja bahasa pemrograman?
  \item Bagaimana pengalaman menangani kesalahan (\textit{error handling}) di tingkat parser mengubah cara Anda menulis kode yang "aman" secara umum?
  \item Hubungan apa yang paling berharga antara teori \textit{Computer Science} (seperti Finite Automata) dengan praktik rekayasa yang Anda lakukan di laboratorium?
\end{enumerate}

\textbf{Bagian 3: Pandangan Visioner}
\begin{enumerate}
  \item Bagaimana pemahaman tentang representasi intermediet (IR) seperti SSA membantu Anda memahami cara kerja alat bantu modern seperti \textit{Linters} atau \textit{Static Analyzers}?
  \item Apa manfaat terbesar dari mempelajari manajemen stack dan register bagi Anda saat melakukan \textit{debugging} aplikasi tingkat sistem?
\end{enumerate}

\subsection{Self-Assessment Checklist Kompetensi}

\begin{table}[h]
\centering
\small
\begin{tabularx}{\textwidth}{|X|c|c|c|}
\hline
\textbf{Kompetensi Teknis} & \textbf{Belum} & \textbf{Sedang} & \textbf{Menguasai} \\
\hline
Memahami siklus hidup instruksi dari sumber ke biner & $\square$ & $\square$ & $\square$ \\
Membangun Parser yang mampu melakukan \textit{error recovery} & $\square$ & $\square$ & $\square$ \\
Melakukan \textit{Static Type Checking} pada AST & $\square$ & $\square$ & $\square$ \\
Menghasilkan IR dan menerapkan \textit{Optimization Passes} & $\square$ & $\square$ & $\square$ \\
Menganalisis performa biner menggunakan \code{perf} atau \code{valgrind} & $\square$ & $\square$ & $\square$ \\
\hline
\end{tabularx}
\caption{Self-Assessment Kompetensi Akhir Semester}
\end{table}

\section{Portofolio Proyek}

Portofolio bukan hanya sekadar kumpulan kode, melainkan bukti kemampuan rekayasa (\textit{engineering capability}) Anda di hadapan dunia profesional atau akademis.

\subsection{Struktur Portofolio Teknis}
Untuk proyek \compiler{Subset C}, portofolio Anda sebaiknya mencakup elemen mendalam berikut:
\begin{enumerate}
  \item \textbf{Analisis Kedalaman Teknis}: Jelaskan satu algoritma kompleks yang Anda implementasikan (misalnya: \textit{Register Coloring} atau \textit{Liveness Analysis}). Mengapa algoritma itu dipilih?
  \item \textbf{Benchmarking Performa}: Sertakan grafik perbandingan antara biner yang dihasilkan dengan optimasi \textit{-O0} vs \textit{-O3}. Jelaskan mengapa biner yang satu lebih cepat (misal: karena berkurangnya instruksi memori).
  \item \textbf{Dokumentasi Arsitektur}: Gunakan diagram blok untuk menunjukkan bagaimana data mengalir dari \textit{Flex}, ke \textit{Bison}, melalui penganalisis semantik, hingga ke generator kode LLVM/Assembly.
  \item \textbf{Metodologi Pengujian}: Tunjukkan suite pengujian otomatis Anda, termasuk kasus uji untuk (\textit{Edge Cases}) seperti pembagian dengan nol atau \textit{scoping} yang tumpang tindih.
\end{enumerate}

\subsection{Kriteria Penilaian Portofolio}

\begin{table}[h]
\centering
\begin{tabular}{|l|c|}
\hline
\textbf{Aspek} & \textbf{Bobot} \\
\hline
Koreksi Output (Kepatuhan terhadap Spesifikasi) & 30\% \\
Kualitas Kode (Modularitas, Kerapuhan, Dokumentasi) & 20\% \\
Analisis Performa (Data Benchmarking Nyata) & 25\% \\
Kedalaman Refleksi (Pemahaman atas Keputusan Desain) & 25\% \\
\hline
\end{tabular}
\caption{Kriteria Penilaian Portofolio Akhir}
\end{table}

\section{Feedback dan Continuous Improvement}

\subsection{Mekanisme Feedback}

\begin{itemize}
  \item \textbf{Formative Assessment}: Feedback selama pembelajaran
  \item \textbf{Peer Review}: Feedback dari teman sekelas
  \item \textbf{Self-Reflection}: Evaluasi diri
  \item \textbf{Instructor Feedback}: Feedback dari dosen
\end{itemize}

\subsection{Action Plan for Improvement}

\begin{enumerate}
  \item \textbf{Identifikasi Gap}: Area yang perlu improvement
  \item \textbf{Set Goals}: Target pencapaian yang spesifik
  \item \textbf{Action Steps}: Langkah konkret untuk mencapai goals
  \item \textbf{Timeline}: Jadwal implementasi
  \item \textbf{Evaluation}: Cara mengukur progress
\end{enumerate}

\section{Kesimpulan dan Langkah Selanjutnya}

\subsection{Pencapaian Dasar}
Teknik Kompilasi telah memberikan Anda fondasi untuk memahami bagaimana abstraksi tingkat tinggi diterjemahkan menjadi realitas fisik di dalam prosesor. Kemampuan ini adalah "kekuatan super" bagi pengembang perangkat lunak karena Anda tidak lagi melihat kompilator sebagai Kotak Hitam (\textit{Black Box}).

\subsection{Rekomendasi Lanjutan dan Tren Masa Depan}
Ekosistem pengembangan kompilator berkembang sangat pesat. Beberapa area yang layak untuk Anda jelajahi lebih lanjut meliputi:
\begin{itemize}
  \item \textbf{AI dalam Kompilator}: Penggunaan \textit{Large Language Models} (LLM) untuk membantu pembuatan kode otomatis, \textit{bug fixing}, dan penemuan strategi optimasi baru yang tidak terpikirkan oleh algoritma heuristik tradisional.
  \item \textbf{WebAssembly (WASM)}: Memahami bagaimana kompilator memungkinkan bahasa seperti C++ atau Rust berjalan di dalam peramban web dengan kecepatan mendekati aplikasi asli (\textit{near-native}).
  \item \textbf{GraalVM dan JIT modern}: Mempelajari bagaimana kompilator dapat melakukan optimasi dinamis saat program SEDANG berjalan untuk hasil yang bahkan lebih baik daripada kompilasi statis.
  \item \textbf{Verified Compilers (Contoh: CompCert)}: Bagi Anda yang tertarik pada keamanan kritis (\textit{mission-critical}), pelajari bagaimana kompilator dapat dibuktikan secara matematis tidak akan pernah menghasilkan kode biner yang salah atau tidak sesuai dengan spesifikasi aslinya.
\end{itemize}

\textit{"The best way to learn compiler construction is to build a compiler."} - Andrew Appel. Semoga perjalanan Anda di dunia \textit{Computer Science} semakin cemerlang setelah menaklukkan mata kuliah ini.


% ============================================================
% AKTIVITAS PEMBELAJARAN
% ============================================================
\begin{aktivitas}
  \item \textbf{Self-Assessment}: Isi checklist kompetensi mandiri.
  \item \textbf{Portfolio Construction}: Susun portofolio proyek semester.
  \item \textbf{Reflective Essay}: Tulis refleksi pembelajaran satu semester.
  \item \textbf{Peer Review}: Evaluasi portofolio rekan sekelas.
  \item \textbf{Action Planning}: Buat rencana pengembangan diri pasca mata kuliah.
\end{aktivitas}

% ============================================================
% LATIHAN DAN REFLEKSI
% ============================================================
\begin{latihan}
  \item Identifikasi Sub-CPMK yang paling sulit dicapai!
  \item Susun bukti-bukti pencapaian kompetensi dalam portofolio!
  \item Analisis kaitan antar modul dalam pengembangan compiler!
  \item Evaluasi perkembangan keterampilan programming Anda!
  \item Rencanakan langkah belajar selanjutnya di bidang software engineering!
  \item \textbf{Refleksi}: Apa insight terbesar yang Anda dapatkan dari mata kuliah ini?
\end{latihan}

% ============================================================
% ASESMEN
% ============================================================
\begin{asesmen}
\textbf{Instrumen Penilaian Akhir Semester}

\textbf{A. Evaluasi Portofolio}
\begin{enumerate}
  \item Kelengkapan komponen compiler (Lexer, Parser, Semantic, dsb)
  \item Kualitas dokumentasi dan analisis
  \item Hasil pengujian dan benchmarking
\end{enumerate}

\textbf{B. Presentasi Proyek}
\begin{enumerate}
  \item Demonstrasi fungsionalitas compiler
  \item Penjelasan arsitektur dan design decisions
  \item Tanya jawab teknis
\end{enumerate}

\textbf{Rubrik Penilaian}: Lihat Lampiran A dan B
\end{asesmen}

% ============================================================
% CHECKLIST KOMPETENSI
% ============================================================
\begin{checklist}
  \item Saya memahami seluruh fase kompilasi dan interaksinya
  \item Saya dapat mengimplementasikan front-end dan back-end compiler
  \item Saya dapat melakukan optimasi dan performance evaluation
  \item Saya mahir menggunakan compiler tools modern
  \item Saya dapat mendokumentasikan proyek pengembangan compiler dengan baik
  \item Saya siap menerapkan prinsip kompilasi dalam rekayasa perangkat lunak
\end{checklist}

% ============================================================
% RANGKUMAN
% ============================================================
\begin{rangkuman}
Bab ini merangkum seluruh perjalanan pembelajaran Teknik Kompilasi, melalui evaluasi CPMK, refleksi diri, dan penyusunan portofolio. Mahasiswa belajar mengevaluasi pencapaian kompetensi secara komprehensif.

\textbf{Poin Kunci:}
\begin{itemize}
  \item Evaluasi kompetensi mencakup teori dan praktik implementasi
  \item Refleksi membantu internalisasi pemahaman konsep
  \item Portofolio proyek menjadi bukti nyata pencapaian Sub-CPMK
  \item Continuous improvement penting untuk pengembangan karir professional
  \item Mata kuliah ini memberikan fondasi kuat dalam computer science
\end{itemize}

\textbf{Kata Kunci}: \compiler{Evaluasi Kompetensi}, \compiler{Refleksi}, \compiler{Portofolio}, \compiler{CPMK}, \compiler{Self-Assessment}, \compiler{Continuous Improvement}
\end{rangkuman}

\ifSubfilesClassLoaded{%
    \clearpage
    \printbibliography[title={Daftar Pustaka}]
    \end{refsection}
}{}

\end{document}
