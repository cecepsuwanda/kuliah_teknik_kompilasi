\documentclass[../main.tex]{subfiles}

\addbibresource{\subfix{../references.bib}}

\begin{document}

\ifSubfilesClassLoaded{%
    \setcounter{chapter}{6}%
    \begin{refsection}
}{}

\chapter{Three-Address Code Generation}
\label{chap:three-address-code}

\begin{subcpmk}
  \item \textbf{Sub-CPMK 4.1:} Merancang three-address code representation
\end{subcpmk}

% ============================================================
% MATERI POKOK
% ============================================================
\section{Pengenalan Three-Address Code (TAC)}

\compiler{Three-Address Code (TAC)} adalah bentuk Intermediate Representation (IR) yang setiap instruksinya memiliki maksimal tiga operandi (biasanya satu hasil dan dua operan).

\subsection{Format Instruksi TAC}
Format umum instruksi TAC adalah: \texttt{x = y op z}.
\begin{itemize}
    \item \texttt{x}: Tujuan (biasanya variabel \textit{temporary}).
    \item \texttt{y}, \texttt{z}: Operan (variabel, konstanta, atau temporary).
    \item \texttt{op}: Operator (aritmatika, logika, atau relasional).
\end{itemize}

\subsection{Temporary Variables}
Kompilator secara otomatis membangkitkan variabel sementara (\textit{temporaries}) untuk menyimpan hasil antara dari ekspresi kompleks. Misal, \code{a + b * c} ditransformasikan menjadi:
\begin{lstlisting}
t1 = b * c
t2 = a + t1
\end{lstlisting}

\section{Representasi: Quadruples, Triples, dan Indirect Triples}

Ada tiga cara utama untuk merepresentasikan instruksi TAC dalam struktur data memori. Pemilihan representasi ini sangat mempengaruhi performa fase optimasi.

\subsection{1. Quadruples}
Setiap instruksi disimpan dalam objek atau struct dengan empat \textit{fields}: \texttt{(op, arg1, arg2, result)}. 
\begin{itemize}
    \item \textbf{Kelebihan}: Sangat fleksibel untuk optimasi. Kita bisa memindahkan atau menghapus instruksi tanpa merusak referensi instruksi lainnya karena tujuan (\textit{result}) ditulis secara eksplisit sebagai nama variabel temporary.
    \item \textbf{Kekurangan}: Memerlukan memori lebih banyak untuk menyimpan nama variabel temporary di setiap entri.
\end{itemize}

\subsection{2. Triples}
Struktur ini hanya memiliki tiga \textit{fields}: \texttt{(op, arg1, arg2)}. Hasil dari operasi tidak diberi nama variabel, melainkan dirujuk menggunakan ID atau indeks instruksi tersebut.
\begin{itemize}
    \item \textbf{Kelebihan}: Lebih hemat memori karena tidak ada field \textit{result}.
    \item \textbf{Kekurangan}: Sangat sulit untuk dioptimasi. Jika kita memindahkan baris \#10 ke baris \#15, semua instruksi lain yang merujuk pada hasil baris \#10 (menggunakan indeks (10)) harus diperbarui secara manual.
\end{itemize}

\subsection{3. Indirect Triples}
Merupakan pengembangan dari Triples. Kita menyimpan Triples di satu tempat, dan memiliki array tambahan yang berisi \textit{pointers} ke Triples tersebut dalam urutan eksekusi yang diinginkan.
\begin{itemize}
    \item \textbf{Kelebihan}: Mendapatkan efisiensi memori Triples namun tetap mudah dioptimasi seperti Quadruples. Jika kita ingin menukar urutan kode, kita cukup menukar pointer di array tambahan tersebut.
\end{itemize}

\begin{table}[!htbp]
\centering
\begin{tabular}{|l|c|c|c|}
\hline
\textbf{Fitur} & \textbf{Quadruples} & \textbf{Triples} & \textbf{Indirect Triples} \\
\hline
Field Result & Eksplisit & Implisit (Indeks) & Implisit (Indeks) \\
Optimasi & Sangat Mudah & Sulit & Mudah (via Pointer) \\
Konsumsi Memori & Tinggi & Rendah & Menengah \\
Standar Industri & Tinggi & Rendah & Menengah \\
\hline
\end{tabular}
\caption{Perbandingan Representasi TAC}
\end{table}

\begin{figure}[!htbp]
    \centering
    \adjustbox{max width=0.9\textwidth,center}{%
    \begin{tikzpicture}[
        rect/.style={rectangle, draw=black, minimum width=2.5cm, minimum height=0.6cm, font=\small},
        row/.style={rectangle split, rectangle split parts=4, draw, rectangle split horizontal, minimum height=0.6cm, font=\small}
    ]
    % Quadruple representation
    \node (q0) at (0,0) [row] {+ \nodepart{two} a \nodepart{three} b \nodepart{four} t1};
    \node (q1) at (0,-0.7) [row] {* \nodepart{two} t1 \nodepart{three} c \nodepart{four} t2};
    \node[above=0.2cm of q0, font=\bfseries] {Quadruples};

    % Triple representation
    \node (t0) at (5,0) [row, rectangle split parts=3] {+ \nodepart{two} a \nodepart{three} b};
    \node (t1) at (5,-0.7) [row, rectangle split parts=3] {* \nodepart{two} (0) \nodepart{three} c};
    \node[above=0.2cm of t0, font=\bfseries] {Triples};
    \end{tikzpicture}%
    }
    \caption{Perbandingan Struktur Data Quadruples vs Triples}
\end{figure}

\section{Translasi Ekspresi dan Penugasan}

\subsection{Skema Translasi AST ke TAC}
Proses ini biasanya dilakukan melalui penelusuran AST (\textit{recursive traversal}). Fungsi translasi untuk ekspresi biner mengembalikan variabel \textit{temporary} yang menampung hasilnya.

\begin{lstlisting}[language=C++]
string translateExpr(ASTNode* node) {
    if (node->isLiteral()) return node->value;
    if (node->isBinary()) {
        string t1 = translateExpr(node->left);
        string t2 = translateExpr(node->right);
        string temp = generateTemp();
        emit(node->op, t1, t2, temp);
        return temp;
    }
}
\end{lstlisting}

\subsection{Manajemen Label}
Untuk struktur kontrol, kita memerlukan label unik agar parser dapat melakukan \textit{jump} (lompatan) antar blok kode.

\section{Struktur Kontrol: Label dan Jump}

\subsection{Translasi If-Then-Else}
Struktur kondisional diterjemahkan menggunakan instruksi lompatan bersyarat (\textit{conditional jump}) dan tanpa syarat (\textit{unconditional jump}).

\begin{lstlisting}
    if (condition) goto L1
    // Code for False branch
    goto L2
L1: // Code for True branch
L2: // Next statement
\end{lstlisting}

\subsection{Translasi While-Loop}
Loop memerlukan label di awal untuk pengulangan dan label di akhir untuk terminasi berdasarkan evaluasi kondisi.

\begin{figure}[!htbp]
    \centering
    \adjustbox{max width=0.8\textwidth,center}{%
    \begin{tikzpicture}[
        node/.style={rectangle, draw=blue!50, fill=blue!10, font=\tiny, align=center},
        arrow/.style={->, >=stealth, thick}
    ]
    \node[node] (start) {Label L\_start};
    \node[node, below=0.5cm of start] (cond) {If !cond goto L\_end};
    \node[node, below=0.5cm of cond] (body) {Loop Body};
    \node[node, below=0.5cm of body] (jmp) {goto L\_start};
    \node[node, below=0.5cm of jmp] (end) {Label L\_end};
    \draw[arrow] (start) -- (cond);
    \draw[arrow] (cond) -- (body);
    \draw[arrow] (body) -- (jmp);
    \draw[arrow] (jmp) to[out=180, in=180] (start);
    \draw[arrow] (cond) to[out=0, in=0] (end);
    \end{tikzpicture}%
    }
    \caption{Struktur aliran TAC untuk statement While}
\end{figure}

\section{Translasi Array dan Pemanggilan Fungsi}

\subsection{1. Alamat Array Multidimensi}
Mengakses elemen array seperti \code{A[i][j]} memerlukan perhitungan alamat memori pada saat \textit{runtime}. Untuk array 2D berukuran $M \times N$ dengan urutan \textit{Row-Major} (baris demi baris):
\begin{enumerate}
    \item \textbf{Rumus Dasar}: $Address(A[i][j]) = Base_A + ((i \times N) + j) \times w$
    \item \textbf{TAC Decompositon}:
\end{enumerate}
\begin{lstlisting}
t1 = i * N
t2 = t1 + j
t3 = t2 * w
t4 = base_A + t3
x = *t4  // Load value dari alamat t4
\end{lstlisting}
Di mana $N$ adalah jumlah kolom dan $w$ adalah ukuran tipe data (misal 4 byte untuk \texttt{int}). Perhitungan ini menjadi lebih kompleks untuk array 3D atau lebih, namun pola dekomposisi TAC tetap sama: hitung indeks linear, kalikan ukuran, tambahkan basis.

\subsection{2. Pemanggilan Fungsi (\textit{Function Calls})}
Translasi fungsi dalam TAC menggunakan instruksi \texttt{param}, \texttt{call}, dan \texttt{return}.
Skema untuk \code{x = f(a, b+c)}:
\begin{lstlisting}
t1 = b + c
param a
param t1
t2 = call f, 2  // '2' adalah jumlah argumen
x = t2
\end{lstlisting}

\begin{figure}[!htbp]
    \centering
    \adjustbox{max width=0.8\textwidth,center}{%
    \begin{tikzpicture}[
        node/.style={rectangle, draw=green!50, fill=green!10, font=\tiny, align=center},
        arrow/.style={->, >=stealth, thick}
    ]
    \node[node] (p1) {Push param 1};
    \node[node, below=0.3cm of p1] (p2) {Push param 2};
    \node[node, below=0.3cm of p2] (call) {Call addr, count};
    \node[node, below=0.3cm of call] (ret) {Get return value};
    
    \draw[arrow] (p1) -- (p2);
    \draw[arrow] (p2) -- (call);
    \draw[arrow] (call) -- (ret);
    
    \node[right=1cm of call, font=\itshape\tiny, text width=3cm] {Caller menyimpan register dan menyiapkan stack frame};
    \end{tikzpicture}%
    }
    \caption{Urutan Operasi TAC untuk Pemanggilan Prosedur}
\end{figure}

\subsection{Struktur Akhir TAC}
Hasil akhir dari Chapter 7 adalah deretan instruksi TAC yang merepresentasikan logika program secara utuh. Kode ini kini siap untuk dioptimasi (Chapter 8) sebelum akhirnya diubah menjadi instruksi mesin yang sebenarnya.


% ============================================================
% AKTIVITAS PEMBELAJARAN
% ============================================================
\begin{aktivitas}
  \item \textbf{Expression Translation}: Konversi berbagai ekspresi kompleks ke three-address code.
  \item \textbf{Control Flow}: Implementasikan translator untuk if-then-else dan while loops.
  \item \textbf{Function Translation}: Bangun translator untuk function calls dan parameter passing.
  \item \textbf{Array Handling}: Implementasikan array access dan assignment dalam three-address code.
  \item \textbf{TAC Generator}: Buat generator lengkap dari AST ke three-address code.
\end{aktivitas}

% ============================================================
% LATIHAN DAN REFLEKSI
% ============================================================
\begin{latihan}
  \item Konversi ekspresi \code{a * (b + c) - d / e} ke three-address code!
  \item Terjemahkan statement \code{for(i=0; i<10; i++) x = x + i;} ke three-address code!
  \item Implementasikan temporary variable management dengan optimalisasi!
  \item Analisis keuntungan three-address code untuk optimasi!
  \item Bandingkan three-address code dengan SSA form!
  \item \textbf{Refleksi}: Bagaimana three-address code menyederhanakan code generation?
\end{latihan}

% ============================================================
% ASESMEN
% ============================================================
\begin{asesmen}
\textbf{Instrumen Penilaian untuk Sub-CPMK 4.1}

\textbf{A. Pilihan Ganda}

\begin{enumerate}
  \item Three-address code memiliki maksimal:
  \begin{enumerate}
    \item 2 operands
    \item 3 operands
    \item 4 operands
    \item Tidak terbatas
  \end{enumerate}
  
  \item Temporary variables digunakan untuk:
  \begin{enumerate}
    \item Menyimpan hasil sementara
    \item Optimasi kode
    \item Error handling
    \item Debugging
  \end{enumerate}
  
  \item Keuntungan utama three-address code adalah:
  \begin{enumerate}
    \item Eksekusi lebih cepat
    \item Ukuran kode lebih kecil
    \item Memudahkan optimasi
    \item Debugging lebih mudah
  \end{enumerate}
\end{enumerate}

\textbf{B. Essay}

\begin{enumerate}
  \item Jelaskan proses konversi dari AST ke three-address code dengan contoh kompleks!
  \item Desain dan implementasikan three-address code generator untuk bahasa dengan arrays dan function calls!
\end{enumerate}

\textbf{Rubrik Penilaian}: Lihat Lampiran A
\end{asesmen}

% ============================================================
% CHECKLIST KOMPETENSI
% ============================================================
\begin{checklist}
  \item Saya dapat merancang three-address code representation
  \item Saya dapat mengkonversi ekspresi aritmatika ke three-address code
  \item Saya dapat menerjemahkan control flow structures
  \item Saya dapat mengimplementasikan temporary variable management
  \item Saya dapat menangani arrays dan pointers
  \item Saya dapat mengimplementasikan function calls
\end{checklist}

% ============================================================
% RANGKUMAN
% ============================================================
\begin{rangkuman}
Bab ini membahas three-address code generation, termasuk format instruksi, translation dari AST, implementasi data structures, dan penanganan konstruksi kompleks. Mahasiswa belajar membangun intermediate code generator.

\textbf{Poin Kunci:}
\begin{itemize}
  \item Three-address code adalah IR dengan maksimal 3 operands per instruksi
  \item Translation dari AST ke TAC memerlukan temporary variables
  \item Control flow structures memerlukan label dan jump instructions
  \item TAC memudahkan optimasi dan code generation
  \item Arrays dan pointers memerlukan address arithmetic
\end{itemize}

\textbf{Kata Kunci}: \compiler{Three-Address Code}, \compiler{Intermediate Representation}, \compiler{TAC}, \compiler{Temporary Variables}, \compiler{AST Translation}, \compiler{Control Flow}
\end{rangkuman}

\ifSubfilesClassLoaded{%
    \clearpage
    \printbibliography[title={Daftar Pustaka}]
    \end{refsection}
}{}

\end{document}
