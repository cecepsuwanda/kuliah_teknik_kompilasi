\documentclass[../main.tex]{subfiles}

\addbibresource{\subfix{../references.bib}}

\begin{document}

\ifSubfilesClassLoaded{%
    \setcounter{chapter}{17}%
    \begin{refsection}
}{}

\chapter{Lampiran dan Rubrik Penilaian}
\label{chap:lampiran}

\begin{subcpmk}
  \item \textbf{Sub-CPMK 1.1-6.2:} Menyediakan instrumen penilaian dan panduan praktis untuk seluruh capaian pembelajaran
\end{subcpmk}

% ============================================================
% MATERI POKOK
% ============================================================
\section{Koleksi Quiz dan Uji Kompetensi}

Uji pemahaman cepat untuk setiap topik utama:

\subsection{Quiz 1: Arsitektur Kompilator}
\begin{enumerate}
    \item Apa perbedaan utama antara \textit{front-end} dan \textit{back-end} kompilator?
    \item Mengapa kita memerlukan \textit{Intermediate Representation}?
\end{enumerate}

\subsection{Quiz 2: Optimasi dan Code Gen}
\begin{enumerate}
    \item Apa resiko melakukan optimasi yang terlalu agresif?
    \item Jelaskan istilah \textit{Register Spilling}!
\end{enumerate}

\section{Template Dokumentasi Kode}

\subsection{Template Laporan Praktikum}

\begin{lstlisting}
NAMA MAHASISWA: [Nama Lengkap]
NIM: [Nomor Induk Mahasiswa]
KELAS: [Kelas]
MATA KULIAH: Teknik Kompilasi
TUGAS: [Judul Tugas]
TANGGAL: [Tanggal]

1. TUJUAN
   [Jelaskan tujuan praktikum]

2. TEORI DASAR
   [Jelaskan teori yang relevan]

3. IMPLEMENTASI
   [Jelaskan implementasi yang dilakukan]

4. HASIL DAN ANALISIS
   [Tampilkan hasil dan analisis]

5. KESULITAN DAN SOLUSI
   [Jelaskan kesulitan yang dihadapi dan solusinya]

6. KESIMPULAN
   [Buat kesimpulan dari praktikum]

7. LAMPIRAN
   [Lampirkan kode sumber lengkap]
\end{lstlisting}

\subsection{Template Dokumentasi Kode}

\begin{lstlisting}[language=C]
/**
 * @file [nama_file].c
 * @brief [deskripsi singkat file]
 * @author [nama author]
 * @date [tanggal]
 * 
 * [deskripsi detail file]
 */

/**
 * @brief [deskripsi fungsi]
 * @param [param1] [deskripsi parameter 1]
 * @param [param2] [deskripsi parameter 2]
 * @return [deskripsi return value]
 * 
 * [detail implementasi fungsi]
 */
int function_name(int param1, char *param2) {
    // Implementation here
}
\end{lstlisting}

\section{Checklist Kualitas Teknis}

Checklist ini dirancang untuk membantu mahasiswa dan asisten praktikum dalam memverifikasi kedalaman teknis dari implementasi kompilator.

\subsection{Checklist Analisis Semantik (Sub-CPMK 3)}
\begin{itemize}
  \item[$\square$] \textbf{Symbol Table Hierarchy}: Apakah tabel simbol menangani lingkup berjenjang (\textit{nested scopes}) dengan benar?
  \item[$\square$] \textbf{Type Coercion}: Apakah ada aturan untuk konversi otomatis (misal: \code{int} ke \code{float}) atau penolakan tipe yang tidak kompatibel?
  \item[$\square$] \textbf{Recursive Definitions}: Apakah fungsi rekursif terdaftar di tabel simbol sebelum tubuh fungsinya dianalisis?
  \item[$\square$] \textbf{Contextual Constraints}: Verifikasi apakah variabel digunakan sebelum dideklarasikan dan apakah fungsi dipanggil dengan jumlah argumen yang tepat.
\end{itemize}

\subsection{Checklist Optimasi dan IR (Sub-CPMK 4)}
\begin{itemize}
  \item[$\square$] \textbf{Semantics Preserving}: Apakah kode hasil optimasi menghasilkan output yang identik dengan kode asli untuk input yang sama?
  \item[$\square$] \textbf{Constant Folding}: Apakah ekspresi seperti \code{3 + 4} dievaluasi pada saat kompilasi menjadi \code{7}?
  \item[$\square$] \textbf{Dead Code Elimination}: Apakah blok kode yang tidak terjangkau (\textit{unreachable}) atau variabel yang tidak digunakan berhasil dihapus?
  \item[$\square$] \textbf{Loop Invariant}: Verifikasi apakah perhitungan yang tidak berubah di dalam loop berhasil dipindahkan ke luar loop.
\end{itemize}

\subsection{Checklist Generasi Kode (Sub-CPMK 5)}
\begin{itemize}
  \item[$\square$] \textbf{Activation Records}: Apakah alokasi stack frame sesuai dengan ABI (misal: perataan 16-byte pada x86-64)?
  \item[$\square$] \textbf{Register Handling}: Apakah kompilator menangani \textit{Register Spilling} jika jumlah variabel melebihi jumlah register fisik yang tersedia?
  \item[$\square$] \textbf{System Calls}: Apakah instruksi \code{print} atau \code{input} terhubung dengan benar ke standar \code{libc} atau \textit{syscall} sistem operasi?
\end{itemize}

\section{Format Submisi Tugas}

\subsection{Struktur Folder Proyek}

\begin{lstlisting}
[NIM]_[Nama]_[Tugas]/
|-- src/
|   |-- lexer.c
|   |-- parser.c
|   |-- ast.c
|   |-- symtab.c
|   |-- main.c
|   `-- util.h
|-- include/
|   |-- ast.h
|   `-- symtab.h
|-- tests/
|   |-- test1.c
|   `-- test2.c
|-- docs/
|   |-- README.md
|   `-- laporan.pdf
|-- Makefile
`-- README.md
\end{lstlisting}

\subsection{Format Penamaan File}

\begin{itemize}
  \item Source code: \code{[nim]\_[nama]\_[komponen].c}
  \item Header files: \code{[nim]\_[nama]\_[komponen].h}
  \item Test files: \code{test\_[komponen].c}
  \item Documentation: \code{laporan\_[nim]\_[nama].pdf}
  \item Executable: \code{[nim]\_[nama]\_[proyek]}
\end{itemize}

\section{Best Practices Pengembangan Compiler}

\subsection{Coding Standards}

\begin{itemize}
  \item Gunakan consistent naming conventions
  \item Comment code yang kompleks
  \item Modular design dengan single responsibility
  \item Error handling yang robust
  \item Memory management yang safe
  \item Testing yang komprehensif
\end{itemize}

\subsection{Version Control}

\begin{lstlisting}[language=sh]
# Git workflow untuk proyek compiler
git init
git add .
git commit -m "Initial commit: lexer implementation"
git branch feature-parser
git checkout feature-parser
# Implement parser
git add .
git commit -m "Add parser with LL(1) algorithm"
git checkout main
git merge feature-parser
git tag v1.0.0
\end{lstlisting}

\subsection{Testing Strategies}

\begin{itemize}
  \item Unit testing untuk setiap komponen
  \item Integration testing untuk komponen interaksi
  \item End-to-end testing untuk complete compiler
  \item Performance testing untuk optimization validation
  \item Regression testing untuk maintenance
\end{itemize}

\section{Resources Tambahan}

\subsection{Online Resources}

\begin{itemize}
  \item \textbf{Compiler Design}: \url{https://www.cs.cornell.edu/courses/cs4120/2018fa/}
  \item \textbf{LLVM Tutorial}: \url{https://llvm.org/docs/tutorial/}
  \item \textbf{Flex/Bison Manual}: \url{https://www.gnu.org/software/flex/manual/}
  \item \textbf{ANTLR}: \url{https://www.antlr.org/}
\end{itemize}

\subsection{Recommended Books}

\begin{itemize}
  \item \textbf{Compilers: Principles, Techniques, and Tools} - Aho, Lam, Sethi, Ullman
  \item \textbf{Modern Compiler Implementation} - Andrew Appel
  \item \textbf{Engineering a Compiler} - Cooper and Torczon
  \item \textbf{Programming Language Pragmatics} - Michael Scott
\end{itemize}

\subsection{Useful Tools}

\begin{itemize}
  \item \textbf{IDE}: VS Code, CLion, Eclipse CDT
  \item \textbf{Debugging}: GDB, Valgrind, AddressSanitizer
  \item \textbf{Profiling}: Perf, gprof, Intel VTune
  \item \textbf{Documentation}: Doxygen, Sphinx
\end{itemize}


% ============================================================
% AKTIVITAS PEMBELAJARAN
% ============================================================
\begin{aktivitas}
  \item \textbf{Rubrik Review}: Pelajari kriteria penilaian untuk setiap jenis tugas.
  \item \textbf{Template Usage}: Gunakan template dokumentasi untuk laporan praktikum.
  \item \textbf{Code Quality Check}: Lakukan self-audit menggunakan checklist kualitas kode.
  \item \textbf{Workflow Implementation}: Terapkan best practices dalam pengembangan tugas.
  \item \textbf{Resource Exploration}: Telusuri referensi tambahan untuk pendalaman materi.
\end{aktivitas}

% ============================================================
% LATIHAN DAN REFLEKSI
% ============================================================
\begin{latihan}
  \item Buat draf laporan praktikum menggunakan template yang disediakan!
  \item Lakukan review kode menggunakan checklist kualitas untuk Lexer!
  \item Simulasikan Git workflow untuk proyek kecil!
  \item Susun struktur folder proyek sesuai standar submisi!
  \item Analisis kriteria penilaian untuk mendapatkan nilai maksimal (Grade A)!
  \item \textbf{Refleksi}: Bagaimana rubrik dan standar dokumentasi membantu efektivitas belajar Anda?
\end{latihan}

% ============================================================
% ASESMEN
% ============================================================
\begin{asesmen}
\textbf{Instrumen Evaluasi Kualitas Dokumentasi dan Praktik}

\textbf{A. Audit Dokumentasi}
\begin{enumerate}
  \item Kesesuaian dengan template laporan
  \item Kejelasan penjelasan implementasi
  \item Kualitas dokumentasi dalam kode (comments/doxygen)
\end{enumerate}

\textbf{B. Audit Standar Kode}
\begin{enumerate}
  \item Kepatuhan terhadap coding standards
  \item Penggunaan version control (commit messages/history)
  \item Struktur folder dan penamaan file
\end{enumerate}

\textbf{Rubrik Penilaian}: Lihat Section 18.1
\end{asesmen}

% ============================================================
% CHECKLIST KOMPETENSI
% ============================================================
\begin{checklist}
  \item Saya memahami kriteria penilaian untuk setiap komponen evaluasi
  \item Saya mahir menggunakan template dokumentasi yang disediakan
  \item Saya dapat menerapkan standar kualitas kode dalam pengembangan compiler
  \item Saya memahami format submisi tugas yang benar
  \item Saya dapat menerapkan best practices dalam version control dan testing
  \item Saya dapat memanfaatkan resources tambahan untuk belajar mandiri
\end{checklist}

% ============================================================
% RANGKUMAN
% ============================================================
\begin{rangkuman}
Bab ini menyediakan berbagai instrumen pendukung pembelajaran, mulai dari rubrik penilaian, template dokumentasi, checklist kualitas, hingga best practices dan referensi tambahan. Mahasiswa menggunakannya sebagai panduan standar dalam menyelesaikan tugas dan proyek.

\textbf{Poin Kunci:}
\begin{itemize}
  \item Rubrik memberikan transparansi dalam penilaian kompetensi
  \item Template dan checklist memastikan konsistensi dan kualitas output
  \item Standar submisi mempermudah integrasi dan manajemen proyek
  \item Best practices meningkatkan profesionalisme dalam pengembangan software
  \item Resources tambahan membuka peluang eksplorasi lebih dalam
\end{itemize}

\textbf{Kata Kunci}: \compiler{Rubrik Penilaian}, \compiler{Template Dokumentasi}, \compiler{Checklist Kualitas}, \compiler{Best Practices}, \compiler{Standard Submisi}, \compiler{Learning Resources}
\end{rangkuman}

\ifSubfilesClassLoaded{%
    \clearpage
    \printbibliography[title={Daftar Pustaka}]
    \end{refsection}
}{}

\end{document}
