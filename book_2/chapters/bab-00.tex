\documentclass[../main.tex]{subfiles}
\begin{document}

\chapter*{Prakata}
\addcontentsline{toc}{chapter}{Prakata}

Buku ajar ini disusun sebagai bahan pembelajaran untuk mata kuliah \textbf{Teknik Kompilasi} pada Program Studi S1 Teknik Informatika. Buku ini dirancang mengikuti pendekatan \textit{Outcome-Based Education (OBE)} dengan fokus pada pembelajaran berbasis praktik.

Buku ini bertujuan untuk memberikan pemahaman dan keterampilan praktis dalam merancang dan mengimplementasikan kompilator untuk bahasa pemrograman. Setiap bab dilengkapi dengan contoh praktis menggunakan bahasa pemrograman C atau C++, serta latihan yang mengarah pada pembangunan komponen-komponen kompilator secara bertahap.

Mata kuliah ini mencakup fase-fase kompilator dari analisis leksikal, analisis sintaksis, analisis semantik, hingga generasi kode. Sesuai dengan pendekatan OBE, mahasiswa diharapkan tidak hanya memahami teori, tetapi juga mampu mengimplementasikan setiap fase kompilator secara praktis.

\vspace{1cm}
\noindent\textit{Penyusun}\\
\noindent\textit{\today}

\end{document}
