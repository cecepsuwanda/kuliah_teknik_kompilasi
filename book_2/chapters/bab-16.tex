\documentclass[../main.tex]{subfiles}

\addbibresource{\subfix{../references.bib}}

\begin{document}

\ifSubfilesClassLoaded{%
    \setcounter{chapter}{15}%
    \begin{refsection}
}{}

\chapter{Performance Evaluation dan Benchmarking}
\label{chap:performance-evaluation}

\begin{subcpmk}
  \item \textbf{Sub-CPMK 6.2:} Mengevaluasi kinerja compiler dan optimasi
\end{subcpmk}

% ============================================================
% MATERI POKOK
% ============================================================
\section{Studi Kasus: Proyek Compiler Subset C}

Sebagai bentuk evaluasi kinerja dan integrasi seluruh fase, kita akan meninjau spesifikasi proyek \compiler{Subset C} yang telah kita bangun secara bertahap.

\subsection{Spesifikasi Grammar dan AST}
Proyek ini mengimplementasikan grammar \textit{top-down} untuk mengakomodasi \textit{recursive descent parser} serta grammar \textit{bottom-up} untuk \textit{Bison}. Token yang didukung meliputi tipe data \code{int}/\code{float}, kontrol \code{if}/\code{while}, dan fungsi \code{print}.

\subsection{Manajemen Runtime dan Memori}
Pada fase awal, proyek ini menggunakan alokasi statis untuk variabel sederhana. Saat fungsi ditambahkan, proyek mengadopsi \textit{activation record} standar x86-64 untuk memastikan kompatibilitas dengan \textit{runtime C library} (\code{libc}).

\subsection{Analisis Kinerja}
Mahasiswa diharapkan melakukan \textit{benchmarking} terhadap kode yang dihasilkan, membandingkan antara kode tanpa optimasi dengan kode yang telah melalui fase \textit{constant folding} dan \textit{dead code elimination}.

\section{Benchmarking Methodology}

\subsection{Benchmark Design}

Prinsip benchmark yang baik:

\begin{itemize}
  \item \textbf{Reproducibility}: Hasil dapat direproduksi
  \item \textbf{Fairness}: Perbandingan yang adil
  \item \textbf{Representative}: Mewakili kasus nyata
  \item \textbf{Statistical significance}: Hasil statistik valid
\end{itemize}

\subsection{Test Suite}

\begin{lstlisting}[language=C]
// Benchmark test suite structure
typedef struct {
    char *name;
    char *description;
    char *input_file;
    int expected_time_ms;
    int expected_memory_kb;
} BenchmarkCase;

BenchmarkCase benchmarks[] = {
    {"small_file", "Small C file", "small.c", 100, 1024},
    {"medium_file", "Medium C file", "medium.c", 500, 4096},
    {"large_file", "Large C file", "large.c", 2000, 16384},
    {"complex_template", "Complex template", "template.cpp", 5000, 32768}
};
\end{lstlisting}

\section{Alat Ukur Kinerja (Measurement Tools)}

Untuk mendapatkan analisis yang mendalam, kita memerlukan alat yang dapat melihat ke dalam mesin (perangkat keras) maupun perangkat lunak.

\subsection{Pengukuran Waktu dan Memori}
Kompilator diukur berdasarkan seberapa cepat ia memproses kode (\textit{compilation speed}) dan seberapa cepat kode yang dihasilkannya berjalan (\textit{runtime speed}). Penggunaan memori puncak (\textit{Peak RSS}) sangat penting untuk memastikan kompilasi tidak menyebabkan sistem kehabisan RAM.

\subsection{Hardware Performance Counters (HPC)}
Alat profil perangkat lunak standar memberi tahu kita fungsi mana yang lambat, tetapi \compiler{Hardware Performance Counters (HPC)} memberi tahu kita \textbf{mengapa} fungsi tersebut lambat \cite{jhu2024compilers}.
Alat seperti \code{perf} pada Linux memungkinkan akses ke register perangkat keras di dalam CPU untuk mengukur:
\begin{itemize}
    \item \textbf{Cache Misses}: Seberapa sering CPU harus menunggu data dari RAM karena tidak ada di cache L1/L2.
    \item \textbf{Branch Mispredictions}: Seberapa sering unit prediksi cabang CPU salah menebak alur instruksi (sangat berguna untuk mengoptimalkan \textit{switch-case} pada parser).
    \item \textbf{IPC (Instructions Per Cycle)}: Menunjukkan efisiensi penggunaan instruksi per siklus detak CPU.
\end{itemize}

\begin{figure}[!htbp]
    \centering
    \adjustbox{max width=0.8\textwidth,center}{%
    \begin{tikzpicture}[
        node/.style={rectangle, draw=gray, fill=gray!10, text width=6cm, font=\tiny, align=center}
    ]
    \node[node] (sw) {Software Profiling: "Function \code{parse()} takes 60\% of time"};
    \node[node, below=0.2cm of sw, fill=green!10] (hw) {Hardware Counters: "\code{parse()} is slow due to 80\% L2 Cache Misses"};
    \draw[->] (sw) -- (hw);
    \end{tikzpicture}%
    }
    \caption{Keunggulan Analisis Perangkat Keras (HPC)}
\end{figure}

\section{Compiler Comparison}

\subsection{Compiler Matrix}

\begin{table}[h]
\centering
\begin{tabular}{|l|l|l|l|l|}
\hline
\textbf{Compiler} & \textbf{Versi} & \textbf{Speed} & \textbf{Size} & \textbf{Memory} \\
\hline
GCC & 11.2 & Fast & Medium & Medium \\
Clang & 14.0 & Fastest & Small & Low \\
MSVC & 19.3 & Medium & Large & High \\
ICC & 2021.1 & Fast & Medium & Medium \\
\hline
\end{tabular}
\caption{Perbandingan Compiler Populer}
\end{table}

\subsection{Optimization Levels}

\begin{lstlisting}[language=sh]
# Benchmark different optimization levels
for level in O0 O1 O2 O3 Os; do
    echo "Testing with -$level optimization"
    time gcc -$level test.c -o test_$level
    ./test_$level
    echo "Size: $(stat -c%s test_$level)"
done
\end{lstlisting}

\section{Performance Analysis}

\subsection{Profiling Results}

\begin{lstlisting}[language=sh]
# Profile with gprof
gcc -pg -O2 test.c -o test_profile
./test_profile
gprof test_profile gmon.out > profile_report.txt

# Profile with perf
perf stat -e cycles,instructions,cache-misses ./test_program

# Profile with valgrind
valgrind --tool=callgrind ./test_program
\end{lstlisting}

\subsection{Statistical Analysis}

\begin{lstlisting}[language=python]
import statistics
import numpy as np

def analyze_benchmark_results(times):
    """Analyze benchmark results statistically"""
    mean = np.mean(times)
    std_dev = np.std(times)
    median = np.median(times)
    
    # Remove outliers (beyond 2 standard deviations)
    filtered = [t for t in times if abs(t - mean) <= 2 * std_dev]
    
    filtered_mean = np.mean(filtered)
    filtered_std = np.std(filtered)
    
    return {
        'raw_mean': mean,
        'raw_std': std_dev,
        'filtered_mean': filtered_mean,
        'filtered_std': filtered_std,
        'median': median,
        'sample_size': len(times),
        'outliers_removed': len(times) - len(filtered)
    }
\end{lstlisting}

\section{Analisis Dampak Optimasi}

Optimasi tingkat lanjut tidak hanya bergantung pada logika kode, tetapi juga pada bagaimana kode tersebut dieksekusi di dunia nyata.

\subsection{Profile Guided Optimization (PGO)}
\compiler{PGO} (disebut juga \textit{Feedback-Directed Optimization}) menggunakan data performa dari ekosistem nyata untuk memandu kompilator \cite{fedoraproject2024compilers}.
Siklus PGO:
\begin{enumerate}
    \item \textbf{Instrumentation}: Kompilasi program dengan penanda (\textit{probes}) untuk mencatat jalur mana yang sering dilewati (\textit{hot paths}).
    \item \textbf{Profiling}: Menjalankan program yang telah ditandai dengan beban kerja representatif.
    \item \textbf{Optimized Build}: Kompilasi ulang menggunakan data profil tersebut untuk memprioritaskan optimasi pada fungsi yang paling sering digunakan.
\end{enumerate}

\subsection{BOLT (Binary Optimization and Layout Tool)}
Setelah kode ditautkan (\textit{post-link}), alat seperti \compiler{BOLT} dapat mengoptimalkan biner yang sudah jadi.
\begin{itemize}
    \item \textbf{Reorganisasi Biner}: BOLT mengatur ulang tata letak instruksi fisik di dalam berkas eksekusi agar fungsi-fungsi yang sering memanggil satu sama lain diletakkan berdekatan.
    \item \textbf{Manfaat}: Mengurangi \textit{Instruction Cache Misses} dan meningkatkan akurasi prediksi cabang secara signifikan melampaui kemampuan kompilator standar.
\end{itemize}

\begin{figure}[!htbp]
    \centering
    \adjustbox{max width=0.8\textwidth,center}{%
    \begin{tikzpicture}[
        node/.style={rectangle, draw=red!50, fill=red!10, text width=6cm, font=\tiny, align=center}
    ]
    \node[node] (pgo) {PGO: Higher-level code decisions (Inlining, Branching)};
    \node[node, below=0.2cm of pgo] (bolt) {BOLT: Lower-level binary layout (Cache locality)};
    \draw[dashed] (pgo) -- (bolt) node[midway, right] {Combined Benefit (up to 20\%+)};
    \end{tikzpicture}%
    }
    \caption{Optimasi Berbasis Umpan Balik (Feedback-Driven)}
\end{figure}

\section{Otomasi Benchmark}

Dalam pengembangan kompilator modern, performa harus dipantau secara otomatis (CI/CD) untuk mencegah degradasi kinerja secara tidak sengaja.

\subsection{Pelacakan Regresi Performa}
Setiap kali ada perubahan kode sumber (\textit{commit}), sistem otomatis harus menjalankan \textit{benchmark suite}. Jika performa turun di bawah ambang batas (\textit{threshold}) tertentu, sistem akan menandainya sebagai \compiler{Performance Regression}.

\subsection{Teknik Bisection untuk Performa}
Jika terjadi penurunan kinerja yang signifikan, kita menggunakan teknik \textbf{Bisection} \cite{ashermancinelli2024bisecting}.
\begin{itemize}
    \item \textbf{Prinsip}: Melakukan pencarian biner (\textit{binary search}) pada riwayat perubahan kode untuk menemukan \textit{commit} tunggal mana yang menyebabkan perlambatan.
    \item \textbf{Automated Bisect}: Kompilator seperti LLVM menggunakan alat otomatis yang membangun versi kompilator di tengah rentang waktu tertentu, menjalankan benchmark, dan mengulangi proses hingga penyebab utama ditemukan.
\end{itemize}

\begin{figure}[!htbp]
    \centering
    \adjustbox{max width=0.8\textwidth,center}{%
    \begin{tikzpicture}[
        compilation_step/.style={circle, draw, fill=gray!20, minimum size=0.6cm, font=\tiny}
    ]
    \node[compilation_step] (s1) {Good};
    \node[compilation_step, right=0.5cm of s1] (s2) {Good};
    \node[compilation_step, right=0.5cm of s2, fill=red!30] (s3) {Bad?};
    \node[compilation_step, right=0.5cm of s3, fill=red!50] (s4) {Slow};
    \node[compilation_step, right=0.5cm of s4, fill=red!50] (s5) {Slow};
    
    \draw[->] (s1) -- (s2);
    \draw[->] (s2) -- (s3);
    \draw[->] (s3) -- (s4);
    \node[below=0.1cm of s3, font=\tiny] {\textbf{Regresi ditemukan di sini}};
    \end{tikzpicture}%
    }
    \caption{Proses Bisection untuk Mencari Penurunan Performa}
\end{figure}

\section{Real-World Performance}

\subsection{Industry Benchmarks}

\begin{itemize}
  \item \textbf{SPEC CPU2006}: CPU benchmark suite
  \item \textbf{SPEC CINT2006}: C integer benchmark
  \item \textbf{SPEC CFP2006}: C floating point benchmark
  \item \textbf{Compile-time benchmarks}: Compiler compilation speed
\end{itemize}

\subsection{Cloud Compilation}

\begin{lstlisting}[language=sh]
# Cloud compilation benchmark
for cloud_provider in "aws" "gcp" "azure"; do
    echo "Testing $cloud_provider cloud compilation"
    
    # Measure compilation time
    start=$(date +%s%N)
    $cloud_provider compile build_project
    end=$(date +%s%N)
    
    echo "Compilation time: $((end - start)) seconds"
    
    # Measure cost
    cost=$($cloud_provider billing get-cost --project build_project)
    echo "Cost: $cost"
done
\end{lstlisting}


% ============================================================
% AKTIVITAS PEMBELAJARAN
% ============================================================
\begin{aktivitas}
  \item \textbf{Benchmark Setup}: Buat benchmark suite untuk compiler testing.
  \item \textbf{Performance Analysis}: Analisis hasil benchmark dengan statistik.
  \item \textbf{Optimization Testing}: Uji impact berbagai optimasi levels.
  \item \textbf{Tool Comparison}: Bandingkan berbagai compiler tools.
  \item \textbf{Automation}: Buat automated benchmarking pipeline.
\end{aktivitas}

% ============================================================
% LATIHAN DAN REFLEKSI
% ============================================================
\begin{latihan}
  \item Desain benchmark suite untuk compiler dengan multiple test cases!
  \item Implementasikan high-resolution timer untuk akurasi pengukuran!
  \item Analisis hasil benchmark dengan statistical methods!
  \item Bandingkan performance GCC vs Clang untuk berbagai optimasi levels!
  \item Identifikasi bottleneck dalam compilation pipeline!
  \item \textbf{Refleksi}: Bagaimana performance evaluation mempengaruhi pengembangan compiler?
\end{latihan}

% ============================================================
% ASESMEN
% ============================================================
\begin{asesmen}
\textbf{Instrumen Penilaian untuk Sub-CPMK 6.2}

\textbf{A. Pilihan Ganda}

\begin{enumerate}
  \item Metrik yang TIDAK diukur dalam performance evaluation:
  \begin{enumerate}
    \item Code quality
    \item Developer productivity
    \item User satisfaction
    \item Compilation speed
  \end{enumerate}
  
  \item Tool untuk profiling memory usage:
  \begin{enumerate}
    \item GDB
    \item Valgrind
    \item /proc filesystem
    \item System monitor
  \end{enumerate}
  
  \item Optimasi yang memberikan speedup terbesar:
  \begin{enumerate}
    \item Loop unrolling
    \item Vectorization
    \item Inlining
    \item Constant folding
  \end{enumerate}
\end{enumerate}

\textbf{B. Essay}

\begin{enumerate}
  \item Jelaskan metodologi benchmarking yang komprehensif untuk compiler evaluation!
  \item Implementasikan automated benchmarking system dengan statistical analysis!
\end{enumerate}

\textbf{Rubrik Penilaian}: Lihat Lampiran A
\end{asesmen}

% ============================================================
% CHECKLIST KOMPETENSI
% ============================================================
\begin{checklist}
  \item Saya dapat mengevaluasi kinerja compiler dan optimasi
  \item Saya dapat merancang benchmark suite yang efektif
  \item Saya dapat menggunakan profiling tools untuk analisis
  \item Saya dapat melakukan statistical analysis pada hasil benchmark
  \item Saya dapat membandingkan performance berbagai compiler
  \item Saya dapat mengidentifikasi optimization opportunities
\end{checklist}

% ============================================================
% RANGKUMAN
% ============================================================
\begin{rangkuman}
Bab ini membahas performance evaluation dan benchmarking, termasuk metodologi, measurement tools, compiler comparison, optimization analysis, dan automated testing. Mahasiswa belajar mengevaluasi dan mengoptimasi kinerja compiler.

\textbf{Poin Kunci:}
\begin{itemize}
  \item Performance evaluation mengukur efektivitas compiler
  \item Benchmarking menyediakan pengukuran yang sistematis
  \item Profiling tools membantu identifikasi bottleneck
  \item Statistical analysis memastikan validitas hasil
  \item Optimizations memiliki trade-off yang perlu dipertimbangkan
  \item Automation mempermudah testing berulang
\end{itemize}

\textbf{Kata Kunci}: \compiler{Performance Evaluation}, \compiler{Benchmarking}, \compiler{Profiling}, \compiler{Optimization}, \compiler{Statistical Analysis}, \compiler{Compiler Comparison}
\end{rangkuman}

\ifSubfilesClassLoaded{%
    \clearpage
    \printbibliography[title={Daftar Pustaka}]
    \end{refsection}
}{}

\end{document}
