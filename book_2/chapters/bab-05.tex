\documentclass[../main.tex]{subfiles}

\addbibresource{\subfix{../references.bib}}

\begin{document}

\ifSubfilesClassLoaded{%
    \setcounter{chapter}{4}%
    \begin{refsection}
}{}

\chapter{Symbol Table dan Scope Management}
\label{chap:symbol-table}

\begin{subcpmk}
  \item \textbf{Sub-CPMK 3.1:} Mengimplementasikan symbol table dengan nested scopes
  \item \textbf{Sub-CPMK 3.2:} Melakukan type checking untuk ekspresi kompleks
\end{subcpmk}

% ============================================================
% MATERI POKOK
% ============================================================
\section{Dasar Symbol Table dan Perannya}

\compiler{Symbol Table} adalah struktur data fundamental yang menyimpan informasi tentang identifiers (variabel, fungsi, tipe) yang muncul dalam kode sumber.

\subsection{Kebutuhan Symbol Table}
Dalam fase analisis semantik, kompilator perlu menjawab pertanyaan:
\begin{itemize}
    \item Apakah variabel $x$ sudah dideklarasikan?
    \item Apa tipe data dari $y$?
    \item Apakah $z$ merupakan variabel lokal atau global?
\end{itemize}

\subsection{Komponen Entry}
Setiap entri dalam symbol table minimal menyimpan:
\begin{itemize}
    \item \textbf{Name}: Nama identifier.
    \item \textbf{Type}: Tipe data (misal: \texttt{int}, \texttt{float}).
    \item \textbf{Scope}: Level nesting tempat identifier berada.
    \item \textbf{Location}: Alamat memori atau offset (untuk fase \textit{code generation}).
\end{itemize}

\section{Struktur Data: Hash Table dan Scope Stack}

\subsection{Pilihan Struktur Data}
Kinerja kompilator sangat bergantung pada kecepatan Symbol Table. Mengapa? Karena setiap kali parser menemukan identifier, ia harus melakukan \textit{lookup}.
\begin{itemize}
    \item \textbf{Linear List}: $O(N)$. Sangat lambat, hanya cocok untuk bahasa mainan.
    \item \textbf{Binary Search Tree (BST)}: $O(\log N)$. Cukup cepat, tapi butuh penyeimbangan (AVL/Red-Black) agar tidak terdegradasi menjadi linked list.
    \item \textbf{Hash Table}: $O(1)$ rata-rata. Ini adalah standar industri. Dengan fungsi hash yang baik, akses ke ribuan variabel tetap instan.
\end{itemize}

\subsection{Manajemen Scope: The Cactus Stack}
Bahasa modern mendukung \textit{Nested Scopes} (blok di dalam blok). Struktur data yang paling tepat untuk ini adalah \textbf{Cactus Stack} (atau \textit{Chained Symbol Tables}).
\begin{itemize}
    \item Setiap scope memiliki Hash Table sendiri.
    \item Hash Table scope anak memiliki pointer \texttt{parent} ke Hash Table scope luar.
    \item Pencarian dimulai dari tabel saat ini. Jika tidak ketemu, lanjut ke \texttt{parent}, terus hingga \texttt{Global Scope} (yang parent-nya \texttt{NULL}).
\end{itemize}

\begin{figure}[!htbp]
    \centering
    \adjustbox{max width=0.8\textwidth,center}{%
    \begin{tikzpicture}[
        table/.style={rectangle, draw=blue!50, fill=blue!10, text width=2.5cm, minimum height=1.5cm, font=\tiny, align=center},
        arrow/.style={->, >=stealth, thick}
    ]
    \node[table] (global) {Global Scope\\ \texttt{int x}};
    \node[table, below left=1cm of global] (func) {Function Scope\\ \texttt{int y}};
    \node[table, below right=1cm of func] (block) {Block Scope\\ \texttt{int z}};
    
    \draw[arrow] (func) -- node[right, font=\tiny] {parent} (global);
    \draw[arrow] (block) -- node[right, font=\tiny] {parent} (func);
    
    \node[below=0.2cm of block, font=\itshape\footnotesize] {Lookup z: Found in Block};
    \node[right=0.2cm of block, font=\itshape\footnotesize] {Lookup x: Block $\to$ Func $\to$ Global (Found)};
    \end{tikzpicture}%
    }
    \caption{Ilustrasi Chained Symbol Tables (Cactus Stack)}
\end{figure}

\section{Shadowing dan Resolusi Nama}

\subsection{Konsep Shadowing}
\textit{Shadowing} terjadi ketika deklarasi variabel di \textit{nested scope} "menutupi" variabel dengan nama yang sama di \textit{outer scope}. Kompilator perlu memberi peringatan (\textit{warning}) jika shadowing tidak disengaja, namun harus mengizinkannya secara legal.
\begin{lstlisting}[language=C]
int x = 10; // Global x
void foo() {
    int x = 20; // x ini 'melindungi' foo dari akses ke global x
    {
        int x = 30; // x ini melindung blok ini
        print(x);   // Harus 30
    }
    print(x);       // Harus 20
}
\end{lstlisting}

\subsection{Algoritma Resolusi Nama (Lookup)}
Proses pencarian identifier dilakukan secara hierarkis:
\begin{enumerate}
    \item Mulai dari \texttt{current\_scope}. Jika ditemukan, kembalikan object Symbol tersebut.
    \item Jika tidak, pindah ke \texttt{current\_scope->parent}.
    \item Ulangi langkah 2 sampai menemukan \texttt{Global Scope}.
    \item Jika sudah sampai puncak (Global) dan masih tidak ketemu, lemparkan error: \texttt{Undefined Variable 'x'}.
\end{enumerate}

\begin{figure}[!htbp]
    \centering
    \adjustbox{max width=0.8\textwidth,center}{%
    \begin{tikzpicture}[
        scope/.style={draw, rectangle, rounded corners, minimum width=4cm, minimum height=1cm, align=left, fill=white, drop shadow},
        arrow/.style={->, >=stealth, thick, dashed}
    ]
    \node[scope] (s0) {Global: \texttt{int x}};
    \node[scope, below=0.5cm of s0] (s1) {Function foo: \texttt{int x}};
    \node[scope, below=0.5cm of s1] (s2) {Inner Block: \texttt{int x}};
    
    \draw[arrow] (s2.east) to[bend right=45] node[right, font=\tiny] {Ref x (starts here)} (s2.south east);
    \draw[arrow] (s2) -- node[left, font=\tiny] {Visible?} (s1);
    \draw[arrow] (s1) -- node[left, font=\tiny] {Visible?} (s0);
    
    \node[right=2cm of s1, align=left, font=\small, text width=4cm] {
        \textbf{Lookup logic}:\\
        Search(z) $\to$ not in Block\\
        $\to$ Parent (foo)\\
        $\to$ Parent (Global)
    };
    \end{tikzpicture}%
    }
    \caption{Visualisasi Hierarki Scope untuk Resolusi Variabel}
\end{figure}

\section{Penanganan Scope Entry dan Exit}

\subsection{Function Scope}
Saat mendeklarasikan fungsi, parser harus membuat scope baru untuk menampung parameter fungsi dan variabel lokal di dalamnya.

\subsection{Block Scope}
Setiap blok kode yang dibatasi kurung kurawal menciptakan scope sementara. Kompilator memanggil \texttt{beginScope()} saat menemukan \code{\{} dan \texttt{endScope()} saat menemukan \code{\}}.

\subsection{Contoh Kasus}
\begin{lstlisting}[language=C]
int x = 1; // Global
void f() {
    int x = 2; // Shadows global x
    {
        int x = 3; // Shadows f's x
    }
}
\end{lstlisting}
Symbol table akan mengelola tiga versi variabel \texttt{x} tersebut pada level nesting yang berbeda.

\section{Implementasi Symbol Table yang Lengkap}

Implementasi yang robust memerlukan manajemen memori yang baik untuk setiap entri simbol.

\begin{lstlisting}[language=C++]
class SymbolTable {
    Scope* current;
public:
    void beginScope() { current = new Scope(current); }
    void endScope() {
        Scope* old = current;
        current = current->parent;
        delete old;
    }
    bool insert(string name, Symbol* s) {
        return current->add(name, s);
    }
    Symbol* lookup(string name) {
        return current->find(name);
    }
};
\end{lstlisting}

\begin{figure}[!htbp]
    \centering
    \adjustbox{max width=0.8\textwidth,center}{%
    \begin{tikzpicture}[
        table/.style={rectangle, draw=green!50, fill=green!10, font=\tiny, align=center},
        arrow/.style={->, >=stealth, thick}
    ]
    \node[table] (parser) {Parser};
    \node[table, right=2cm of parser] (st) {Symbol Table};
    \node[table, below=1cm of st] (checker) {Type Checker};
    \draw[arrow] (parser) -- node[above, font=\tiny] {Insert/Lookup} (st);
    \draw[arrow] (st) -- node[right, font=\tiny] {Verify Types} (checker);
    \end{tikzpicture}%
    }
    \caption{Interaksi Symbol Table dalam integrasi sistem}
\end{figure}


% ============================================================
% AKTIVITAS PEMBELAJARAN
% ============================================================
\begin{aktivitas}
  \item \textbf{Hash Table Implementation}: Implementasikan symbol table dengan hash table dan chaining.
  \item \textbf{Scope Testing}: Buat program dengan nested scopes dan test symbol table operations.
  \item \textbf{Type Checker}: Implementasikan type checker untuk ekspresi aritmatika dan assignment.
  \item \textbf{Debug Visualization}: Buat tool untuk visualisasi symbol table dan scope hierarchy.
  \item \textbf{Performance Analysis}: Bandingkan berbagai hash functions untuk symbol table.
\end{aktivitas}

% ============================================================
% LATIHAN DAN REFLEKSI
% ============================================================
\begin{latihan}
  \item Implementasikan symbol table with support for nested scopes menggunakan hash table!
  \item Buat type checker untuk bahasa dengan int, float, dan boolean types!
  \item Analisis complexity dari symbol table operations (insert, lookup, delete)!
  \item Implementasikan scope stack untuk bahasa dengan function definitions!
  \item Desain error reporting system untuk type errors!
  \item \textbf{Refleksi}: Bagian mana dari symbol table implementation yang paling menantang?
\end{latihan}

% ============================================================
% ASESMEN
% ============================================================
\begin{asesmen}
\textbf{Instrumen Penilaian untuk Sub-CPMK 3.1-3.2}

\textbf{A. Pilihan Ganda}

\begin{enumerate}
  \item Data structure yang PALING cocok untuk symbol table adalah:
  \begin{enumerate}
    \item Linked list
    \item Array
    \item Hash table
    \item Binary tree
  \end{enumerate}
  
  \item Scope management menggunakan:
  \begin{enumerate}
    \item Queue
    \item Stack
    \item Priority queue
    \item Heap
  \end{enumerate}
  
  \item Type checking memastikan:
  \begin{enumerate}
    \item Syntax correctness
    \item Type compatibility
    \item Runtime efficiency
    \item Code optimization
  \end{enumerate}
\end{enumerate}

\textbf{B. Essay}

\begin{enumerate}
  \item Jelaskan implementasi symbol table dengan nested scopes dan berikan contoh kode!
  \item Desain type system untuk bahasa dengan arrays, functions, dan pointers!
\end{enumerate}

\textbf{Rubrik Penilaian}: Lihat Lampiran A
\end{asesmen}

% ============================================================
% CHECKLIST KOMPETENSI
% ============================================================
\begin{checklist}
  \item Saya dapat mengimplementasikan symbol table dengan nested scopes
  \item Saya dapat melakukan type checking untuk ekspresi kompleks
  \item Saya memahami berbagai implementasi symbol table
  \item Saya dapat mengelola scope dengan stack structure
  \item Saya dapat mendesain type system yang sederhana
  \item Saya dapat mengimplementasikan type inference algorithm
\end{checklist}

% ============================================================
% RANGKUMAN
% ============================================================
\begin{rangkuman}
Bab ini membahas symbol table dan scope management, termasuk implementasi hash table, nested scopes, type system, dan type checking. Mahasiswa belajar membangun data structure fundamental untuk semantic analysis.

\textbf{Poin Kunci:}
\begin{itemize}
  \item Symbol table menyimpan informasi identifiers dengan akses cepat
  \item Nested scopes dikelola menggunakan stack structure
  \item Type checking memastikan type compatibility dalam expressions
  \item Hash table adalah implementasi efisien untuk symbol table
  \item Type system adalah fondasi untuk semantic analysis
\end{itemize}

\textbf{Kata Kunci}: \compiler{Symbol Table}, \compiler{Scope Management}, \compiler{Type System}, \compiler{Type Checking}, \compiler{Hash Table}, \compiler{Nested Scopes}, \compiler{Type Inference}
\end{rangkuman}

\ifSubfilesClassLoaded{%
    \clearpage
    \printbibliography[title={Daftar Pustaka}]
    \end{refsection}
}{}

\end{document}
