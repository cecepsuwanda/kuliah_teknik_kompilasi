\documentclass[../main.tex]{subfiles}

\addbibresource{\subfix{../references.bib}}

\begin{document}

\ifSubfilesClassLoaded{%
    \setcounter{chapter}{5}%
    \begin{refsection}
}{}

\chapter{Semantic Analysis dan Error Handling}
\label{chap:semantic-analysis}

\begin{subcpmk}
  \item \textbf{Sub-CPMK 3.3:} Menangani semantic error dengan pesan yang informatif
  \item \textbf{Sub-CPMK 3.4:} Menerapkan attribute grammar (synthesized/inherited) dan syntax-directed definition (SDD) atau translation schemes
\end{subcpmk}

% ============================================================
% MATERI POKOK
% ============================================================
\section{Abstract Syntax Tree (AST) Deep Dive}

\compiler{Abstract Syntax Tree (AST)} adalah representasi internal program yang telah disederhanakan. Analisis semantik bertugas memastikan bahwa program memiliki makna yang valid sesuai dengan aturan bahasa pemrogramannya, di luar sekadar kebenaran struktur sintaksisnya \cite{nguyen2024semantic}.

\subsection{Struktur Node AST}
Setiap node dalam AST mewakili konstruk bahasa (misal: \code{IfStmt}, \code{BinaryExpr}).
\begin{enumerate}
    \item \textbf{Expression Nodes}: Literal, Identifier, Unary/Binary operations.
    \item \textbf{Statement Nodes}: Assignment, Loop, Conditional branches.
    \item \textbf{Declaration Nodes}: Variabel, Fungsi, Tipe.
\end{enumerate}

\begin{figure}[!htbp]
    \centering
    \adjustbox{max width=0.8\textwidth,center}{%
    \begin{tikzpicture}[
        node/.style={circle, draw=blue!50, fill=blue!10, minimum size=0.6cm, font=\tiny}
    ]
    \node[node] (plus) at (0,0) {+};
    \node[node, below left=of plus] (a) {a};
    \node[node, below right=of plus] (mul) {*};
    \node[node, below left=of mul] (b) {b};
    \node[node, below right=of mul] (c) {c};
    \draw (plus) -- (a); \draw (plus) -- (mul);
    \draw (mul) -- (b); \draw (mul) -- (c);
    \end{tikzpicture}%
    }
    \caption{Representasi AST untuk ekspresi \code{a + b * c}}
\end{figure}

\section{Sistem Tipe dan Type Checking}

\subsection{Tipe Data dalam Kompilator}
Sistem tipe mendefinisikan aturan bagaimana tipe data diasosiasikan dengan ekspresi. Tipe dasar meliputi \texttt{int}, \texttt{float}, \texttt{bool}, dan \texttt{void}.

\subsection{Algoritma Type Checking}
Type checking dilakukan dengan menelusuri (\textit{traversing}) AST secara \textit{post-order}.
\begin{enumerate}
    \item Untuk leaf node: ambil tipe dari symbol table atau literal.
    \item Untuk node internal: verifikasi kompatibilitas tipe operan (misal: \texttt{int} + \texttt{float} mungkin memerlukan promosi tipe).
\end{enumerate}

\begin{lstlisting}[language=C++]
Type checkBinaryExpr(BinaryExpr* node) {
    Type left = check(node->left);
    Type right = check(node->right);
    if (!isCompatible(left, right)) {
        error("Type mismatch: " + left.str() + " and " + right.str());
    }
    return resultType(left, right);
}
\end{lstlisting}

\section{Analisis Kontekstual dan Validasi}

Selain tipe yang benar, kompilator harus memastikan validitas kontekstual lainnya:
\begin{enumerate}
    \item \textbf{Control Flow}: \texttt{break} dan \texttt{continue} hanya boleh di dalam \textit{loop}.
    \item \textbf{Scope}: Identifier harus dideklarasikan sebelum digunakan.
    \item \textbf{Const Correctness}: Variabel \texttt{const} tidak boleh diubah (assignment l-value check).
\end{enumerate}

\subsection{Implicit Type Conversion (Coercion)}
Apakah \texttt{3 + 4.5} valid? Dalam banyak bahasa, ya. Kompilator secara otomatis memasukkan operasi konversi tipe (\textit{cast}) pada AST:
\[ \text{3 (int) + 4.5 (float)} \longrightarrow \text{to\_float(3) + 4.5} \]
Aturan ini disebut \textit{Promotion}: tipe yang lebih ''kecil'' (int) dikonversi ke tipe yang lebih ''besar''/presisi (float) untuk menghindari kehilangan data.

\subsection{Error Recovery: The Poison Pili}
Bagaimana jika \texttt{x} tidak dideklarasikan dalam \texttt{x + 5}?
\begin{enumerate}
    \item Laporkan error: ''Undeclared identifier 'x'''.
    \item Tandai node \texttt{x} sebagai \textbf{Poison} (atau \texttt{ErrorType}).
    \item Saat mengecek \texttt{Poison + 5}, hasilnya juga \textbf{Poison}.
    \item Jangan laporkan error lagi untuk operasi yang melibatkan Poison. Ini mencegah \textit{cascade error messages} yang membingungkan programmer.
\end{enumerate}

\section{Attribute Grammar}

Attribute Grammar memperluas CFG dengan menambahkan atribut dan aturan semantik pada setiap produksi. Dengan cara ini, struktur sintaksis dapat membawa informasi semantik seperti nilai, tipe, atau lokasi memori.

\subsection{Synthesized Attributes}
Synthesized attribute dihitung dari anak (child) ke parent dalam pohon sintaks. Contoh umum: menghitung nilai ekspresi aritmatika atau menentukan tipe hasil operasi berdasarkan tipe operand.

\subsection{Inherited Attributes}
Inherited attribute diteruskan dari parent atau saudara (sibling) untuk memberikan konteks. Contoh umum: memberi informasi tipe variabel pada deklarasi atau memvalidasi scope.

\subsection{Contoh Sederhana}
Misalkan grammar: $E \rightarrow E + T \mid T$. Kita dapat mendefinisikan atribut \texttt{val} untuk menghitung nilai ekspresi:
\[
E.val = E_1.val + T.val
\]
Aturan seperti ini memungkinkan parser sekaligus membangun struktur dan menghitung makna program.

\begin{figure}[!htbp]
\centering
\begin{forest}
for tree={align=center, parent anchor=south, child anchor=north, l=1.2cm, s sep=0.6cm}
[E\\(val=8)
  [E\\(val=3)
    [num\\(3)]
  ]
  [+]
  [T\\(val=5)
    [num\\(5)]
  ]
]
\end{forest}
\caption{Contoh atribut \texttt{val} pada pohon ekspresi}
\end{figure}

\subsection{Aturan Atribut Ringkas}
\begin{lstlisting}[language=C++]
// Synthesized attribute
E.val = E1.val + T.val;

// Inherited attribute (contoh konteks)
T.inh = E.inh;
\end{lstlisting}

\section{Syntax-Directed Definition dan Translation Schemes}

\subsection{Syntax-Directed Definition (SDD)}
SDD adalah kombinasi grammar dengan atribut dan aturan semantik. Setiap produksi memiliki aturan yang menjelaskan bagaimana atribut dihitung. SDD menjembatani struktur sintaks dengan makna yang ingin dihasilkan, seperti tipe ekspresi atau kode intermediate.

\subsection{Translation Schemes}
Translation scheme menempatkan aksi semantik langsung di dalam produksi grammar sebagai \textit{semantic actions}. Ini memudahkan implementasi karena aksi dapat dijalankan saat parsing berlangsung.

\begin{lstlisting}[language=C++]
// Contoh skema sederhana untuk ekspresi penjumlahan
E -> E1 + T   { E.val = E1.val + T.val; }
T -> num      { T.val = num.lexval; }
\end{lstlisting}

SDD lebih bersifat deklaratif, sedangkan translation scheme lebih operasional. Keduanya digunakan untuk menghubungkan parsing dengan analisis semantik atau generasi kode.

\begin{table}[!htbp]
\centering
\begin{tabularx}{\textwidth}{|l|X|X|}
\hline
\textbf{Aspek} & \textbf{SDD} & \textbf{Translation Scheme} \\
\hline
Sifat & Deklaratif, fokus pada aturan atribut & Operasional, aksi disisipkan di grammar \\
\hline
Waktu eksekusi & Dihitung berdasarkan dependensi atribut & Dieksekusi saat parsing berlangsung \\
\hline
Kegunaan & Spesifikasi semantik & Implementasi semantik praktis \\
\hline
\end{tabularx}
\caption{Perbandingan SDD dan Translation Schemes}
\end{table}

\subsection{Contoh SDD untuk Type Checking}
\begin{lstlisting}[language=C++]
// Contoh SDD untuk pengecekan tipe
E -> E1 + T {
  if (E1.type == T.type) E.type = E1.type;
  else E.type = error;
}
\end{lstlisting}

\section{Praktikum: Implementasi Type Checker}

Kita akan menggunakan \textbf{Visitor Pattern} untuk memisahkan logika pengecekan tipe dari struktur AST.

\subsection{TypeChecker Visitor}
\begin{lstlisting}[language=C++]
class TypeChecker : public Visitor {
    Scope* currentScope;

public:
    void visitBinary(BinaryExpr* node) override {
        node->left->accept(this);  // Cek anak kiri
        node->right->accept(this); // Cek anak kanan
        
        Type leftType = node->left->type;
        Type rightType = node->right->type;
        
        if (leftType != rightType) {
            // Coba coercion
            if (canPromote(leftType, rightType)) {
                insertCast(node->left, rightType);
            } else {
                error(node->op.line, "Tipe tidak kompatibel");
                node->type = Type::Error; // Poisoning
                return;
            }
        }
        node->type = leftType; // Propagasi tipe ke atas
    }
};
\end{lstlisting}

\subsection{Integrasi Symbol Table}
Saat traversal, TypeChecker Visitor juga harus memelihara \texttt{SymbolTable} persis seperti Parser, agar bisa memvalidasi variabel:
\begin{itemize}
    \item \texttt{visitBlock}: Panggil \texttt{enterScope()}, traverse anak-anak, lalu \texttt{exitScope()}.
    \item \texttt{visitVarDecl}: Masukkan variabel ke scope saat ini.
    \item \texttt{visitVariable}: Lookup nama variabel di scope. Jika tidak ketemu, set tipe ke \texttt{Error}.
\end{itemize}

\section{Kesimpulan Analisis Semantik}

Setelah fase semantik, kompilator memiliki dua aset utama:
\begin{enumerate}
    \item \textbf{Annotated AST}: Pohon sintaks yang dilengkapi \textit{tipe} dan konversi yang diperlukan (seperti \texttt{int-to-float}).
    \item \textbf{Validated Symbol Table}: Tabel yang terisi penuh dengan informasi tentang fungsi, variabel, dan scope, yang siap digunakan oleh Code Generator untuk menghitung offset memori.
\end{enumerate}

\subsection{Menuju Intermediate Representation}
Langkah selanjutnya (Chapter 7) adalah mengubah AST yang kaya makna ini menjadi bentuk yang lebih rendah tingkatannya (\textit{Intermediate Representation}), seperti \textit{Three-Address Code} (TAC), yang lebih dekat dengan mesin namun masih independen dari arsitektur CPU target.


% ============================================================
% AKTIVITAS PEMBELAJARAN
% ============================================================
\begin{aktivitas}
  \item \textbf{Semantic Rules}: Definisikan semantic rules untuk bahasa sederhana.
  \item \textbf{Attribute Grammar}: Rancang attribute grammar untuk ekspresi aritmatika sederhana.
  \item \textbf{SDD/Translation Scheme}: Susun aturan SDD untuk menghasilkan kode intermediate dasar.
  \item \textbf{AST Builder}: Implementasikan AST construction dari parse tree.
  \item \textbf{Type Checker}: Bangun type checker dengan error reporting informatif.
  \item \textbf{Error Recovery}: Implementasikan error recovery untuk semantic errors.
  \item \textbf{Symbol Table Integration}: Integrasikan semantic analyzer dengan symbol table.
\end{aktivitas}

% ============================================================
% LATIHAN DAN REFLEKSI
% ============================================================
\begin{latihan}
  \item Identifikasi semantic errors dalam potongan kode yang diberikan!
  \item Buat semantic rules untuk function calls dan parameter passing!
  \item Rancang attribute grammar untuk ekspresi \code{E -> E + T | T}!
  \item Tuliskan skema SDD untuk menghitung nilai ekspresi aritmatika!
  \item Implementasikan type checker untuk expressions dengan multiple types!
  \item Desain error messages yang informatif untuk berbagai semantic errors!
  \item Analisis trade-off antara strict vs lenient semantic checking!
  \item \textbf{Refleksi}: Bagaimana semantic analysis mempengaruhi kualitas compiler?
\end{latihan}

% ============================================================
% ASESMEN
% ============================================================
\begin{asesmen}
\textbf{Instrumen Penilaian untuk Sub-CPMK 3.3}

\textbf{A. Pilihan Ganda}

\begin{enumerate}
  \item Semantic analyzer bertugas untuk:
  \begin{enumerate}
    \item Memverifikasi syntax correctness
    \item Memverifikasi semantic correctness
    \item Mengoptimasi kode
    \item Mengenerate kode target
  \end{enumerate}
  
  \item Error reporting yang baik harus mencakup:
  \begin{enumerate}
    \item Error message saja
    \item Line number saja
    \item Line number, column, dan context
    \item Hanya error code
  \end{enumerate}
  
  \item Type coercion adalah:
  \begin{enumerate}
    \item Error handling mechanism
    \item Automatic type conversion
    \item Type checking algorithm
    \item Optimization technique
  \end{enumerate}
\end{enumerate}

\textbf{B. Essay}

\begin{enumerate}
  \item Jelaskan strategi error recovery dalam semantic analysis dan berikan contoh!
  \item Desain dan implementasikan semantic analyzer untuk bahasa dengan variabel assignment dan arithmetic expressions!
\end{enumerate}

\textbf{Rubrik Penilaian}: Lihat Lampiran A
\end{asesmen}

% ============================================================
% CHECKLIST KOMPETENSI
% ============================================================
\begin{checklist}
  \item Saya dapat menangani semantic error dengan pesan yang informatif
  \item Saya dapat merancang attribute grammar (synthesized/inherited)
  \item Saya dapat menyusun SDD atau translation schemes
  \item Saya dapat mengimplementasikan syntax-directed translation
  \item Saya dapat membangun AST dari parse tree
  \item Saya dapat mengintegrasikan semantic analyzer dengan symbol table
  \item Saya dapat mendesain error recovery strategies
  \item Saya dapat mengimplementasikan type checking dengan coercion
\end{checklist}

% ============================================================
% RANGKUMAN
% ============================================================
\begin{rangkuman}
Bab ini membahas semantic analysis dan error handling, termasuk attribute grammar, syntax-directed translation, type system implementation, error detection, dan recovery strategies. Mahasiswa belajar membangun semantic analyzer yang robust.

\textbf{Poin Kunci:}
\begin{itemize}
  \item Semantic analysis memverifikasi meaning dan correctness program
  \item Attribute grammar memperkaya CFG dengan aturan semantik
  \item Syntax-directed translation menggabungkan parsing dengan semantic analysis
  \item Type checking memastikan type compatibility dan consistency
  \item Error reporting yang baik memberikan informasi yang jelas dan berguna
  \item Error recovery memungkinkan compiler melanjutkan proses compilation
\end{itemize}

\textbf{Kata Kunci}: \compiler{Semantic Analysis}, \compiler{Syntax-Directed Translation}, \compiler{Type Checking}, \compiler{Error Handling}, \compiler{AST}, \compiler{Type Coercion}, \compiler{Error Recovery}
\end{rangkuman}

\ifSubfilesClassLoaded{%
    \clearpage
    \printbibliography[title={Daftar Pustaka}]
    \end{refsection}
}{}

\end{document}
