\documentclass[../main.tex]{subfiles}

\addbibresource{\subfix{../references.bib}}

\begin{document}

\ifSubfilesClassLoaded{%
    \setcounter{chapter}{11}%
    \begin{refsection}
}{}

\chapter{Code Generation dan Target Machine}
\label{chap:code-generation}

\begin{subcpmk}
  \item \textbf{Sub-CPMK 5.3:} Mengimplementasikan code generator untuk arsitektur target
\end{subcpmk}

% ============================================================
% MATERI POKOK
% ============================================================
\section{Pengenalan Target Machine (RISC vs CISC)}

\compiler{Target Machine} adalah arsitektur komputer tujuan di mana kode hasil kompilasi akan dijalankan. Memahami karakteristik perangkat keras sangat krusial bagi \textit{code generator} untuk menghasilkan instruksi yang optimal \cite{jhu2024compilers}.

\subsection{Arsitektur RISC (Reduced Instruction Set Computer)}
Contoh: RISC-V, ARM.
\begin{itemize}
    \item \textbf{Load-Store Architecture}: Hanya instruksi \code{Load} dan \code{Store} yang bisa mengakses memori. Instruksi aritmatika hanya bekerja pada register.
    \item \textbf{Register Pressure}: Karena semua operan harus berada di register, RISC membutuhkan lebih banyak register fisik dan temporer, yang meningkatkan "tekanan" pada pengalokasi register.
    \item \textbf{Fixed-Length}: Instruksi selalu berukuran tetap (32-bit), memudahkan \textit{pipelining}.
\end{itemize}

\subsection{Arsitektur CISC (Complex Instruction Set Computer)}
Contoh: x86 (Intel/AMD).
\begin{itemize}
    \item \textbf{Orthogonality}: Kemampuan instruksi untuk menggunakan berbagai mode pengalamatan secara bebas. Misalnya, instruksi \code{ADD} pada x86 bisa menjumlahkan register dengan memori secara langsung.
    \item \textbf{Variable-Length}: Instruksi berukuran 1-15 byte, menghemat ruang memori tapi mempersulit pendekodean instruksi (\textit{decoding}).
\end{itemize}

\subsection{Dampak pada Code Generation}
Code generator harus memilih strategi yang sesuai dengan arsitektur:
\begin{enumerate}
    \item Pada \textbf{RISC}, fokus pada jadwal instruksi (\textit{scheduling}) untuk menghindari \textit{pipeline stall} dan manajemen register yang agresif.
    \item Pada \textbf{CISC}, fokus pada pemilihan instruksi kompleks yang dapat menggabungkan beberapa operasi TAC menjadi satu instruksi mesin untuk mengurangi ukuran kode (\textit{code density}).
\end{enumerate}

\begin{figure}[!htbp]
    \centering
    \adjustbox{max width=0.8\textwidth,center}{%
    \begin{tikzpicture}[
        node/.style={rectangle, draw=blue!50, fill=blue!10, text width=6cm, font=\tiny, align=center}
    ]
    \node[node] (risc) {RISC: Banyak Temporary $\rightarrow$ Register Allocator harus cerdas.};
    \node[node, below=0.5cm of risc] (cisc) {CISC: Instruksi Kompleks $\rightarrow$ Instruction Selector harus cerdas.};
    \end{tikzpicture}%
    }
    \caption{Prioritas Optimasi berdasarkan Arsitektur Target}
\end{figure}

\section{Pemilihan Instruksi (Instruction Selection)}

\compiler{Instruction Selection} adalah proses memetakan instruksi tingkat menengah (\textit{TAC}) ke instruksi spesifik mesin target yang memberikan performa terbaik.

\subsection{Tiling (Pengubinan)}
Proses pemilihan instruksi sering divisualisasikan sebagai "menutupi" pohon ekspresi (\textit{Expression Tree}) dengan "ubin" (\textit{tiles}). Setiap ubin mewakili satu instruksi mesin yang dapat menggantikan satu atau lebih simpul pada pohon tersebut.

\subsection{Algoritma Pilihan}
Ada dua strategi populer untuk melakukan \textit{tiling}:
\begin{enumerate}
    \item \textbf{Maximal Munch}: Strategi \textit{greedy} (rakus). Dimulai dari akar pohon, pilih ubin terbesar yang cocok (\textit{match}). Jika ada ubin berukuran 3 simpul dan 1 simpul, ubin 3 simpul akan dipilih. Sangat efektif untuk arsitektur RISC.
    \item \textbf{Dynamic Programming}: Strategi optimal. Menghitung biaya (\textit{cost}) minimum untuk menutupi setiap sub-pohon. Algoritma ini memastikan total biaya seluruh pohon adalah yang terendah. Sangat berguna jika CPU memiliki banyak instruksi kompleks dengan biaya yang bervariasi.
\end{enumerate}

\begin{figure}[!htbp]
    \centering
    \adjustbox{max width=0.8\textwidth,center}{%
    \begin{tikzpicture}[
        node/.style={circle, draw, minimum size=0.6cm, font=\tiny}
    ]
    \node[node] (add) {+};
    \node[node, below left=0.5cm and 0.3cm of add] (mul) {*};
    \node[node, below right=0.5cm and 0.3cm of add] (d) {d};
    \node[node, below left=0.5cm and 0.2cm of mul] (b) {b};
    \node[node, below right=0.5cm and 0.2cm of mul] (c) {c};
    
    \draw[blue, thick, dashed] (-1, -1.5) rectangle (1.2, 0.5);
    \node[blue, font=\tiny] at (1.5, 0) {Tile: MADD};
    \end{tikzpicture}%
    }
    \caption{Representasi Tiling: Satu instruksi MADD menutupi operasi Multiply dan Add}
\end{figure}

\section{Register Allocation}

\subsection{Register Allocation Problem}

Masalah alokasi register:

\begin{itemize}
  \item Variabel terbatas vs unlimited temporaries
  \item Register interference
  \item Spilling ke memory
  \item Calling convention constraints
\end{itemize}

\subsection{Linear Scan Allocation}

\begin{lstlisting}[language=C]
typedef struct {
    char *var_name;
    int start_point;    // First use
    int end_point;      // Last use
    int register_num;   // Assigned register (-1 if spilled)
} LiveRange;

void linear_scan_allocation(LiveRange *ranges, int count, 
                           int num_registers) {
    // Sort ranges by start point
    sort_ranges_by_start(ranges, count);
    
    bool *registers_used = calloc(num_registers, sizeof(bool));
    
    for (int i = 0; i < count; i++) {
        // Free registers whose ranges have ended
        free_expired_registers(ranges[i].start_point, 
                             registers_used, num_registers);
        
        // Find free register
        int reg = find_free_register(registers_used, num_registers);
        if (reg != -1) {
            ranges[i].register_num = reg;
            registers_used[reg] = true;
        } else {
            // Spill to memory
            ranges[i].register_num = -1;
        }
    }
}
\end{lstlisting}

\section{Target Architecture}

\subsection{x86 Architecture}

Karakteristik x86:

\begin{itemize}
  \item CISC (Complex Instruction Set Computer)
  \item Variable-length instructions
  \item Rich addressing modes
  \item Backward compatibility
\end{itemize}

\begin{lstlisting}[language=C]
// x86 instruction examples
MOV EAX, EBX          ; Register to register
MOV EAX, [EBX+4]     ; Base + offset addressing
MOV EAX, [EBX+ECX*4] ; Base + index*scale
LEA EAX, [EBX+ECX*2] ; Load effective address
PUSH EAX             ; Stack operation
POP EBX              ; Stack operation
\end{lstlisting}

\subsection{RISC Architecture}

Karakteristik RISC (MIPS/ARM):

\begin{itemize}
  \item Fixed-length instructions
  \item Load/store architecture
  \item Simple addressing modes
  \item Large register file
\end{itemize}

\begin{lstlisting}[language=C]
// MIPS instruction examples
ADD $t0, $t1, $t2    ; $t0 = $t1 + $t2
LW $t0, 4($t1)       ; Load word from memory
SW $t0, 8($t1)       ; Store word to memory
ADDI $t0, $t1, 10    ; $t0 = $t1 + 10 (immediate)
\end{lstlisting}

\section{Code Generation Algorithm}

\subsection{Basic Block Code Generation}

\begin{lstlisting}[language=C]
void generate_block_code(BasicBlock *block, 
                         RegisterAllocator *alloc) {
    for (int i = 0; i < block->instruction_count; i++) {
        TACInstruction *inst = &block->instructions[i];
        
        // Allocate registers for operands
        int reg1 = allocate_register(inst->arg1, alloc);
        int reg2 = allocate_register(inst->arg2, alloc);
        int reg3 = allocate_register(inst->result, alloc);
        
        // Generate target instruction
        switch (inst->op) {
            case OP_ADD:
                emit_add(reg3, reg1, reg2);
                break;
            case OP_MUL:
                emit_mul(reg3, reg1, reg2);
                break;
            case OP_ASSIGN:
                emit_mov(reg3, reg1);
                break;
            // ... other operations
        }
        
        // Release temporary registers
        release_register(reg1, alloc);
        release_register(reg2, alloc);
    }
}
\end{lstlisting}

\subsection{Function Call Generation}

\begin{lstlisting}[language=C]
void generate_function_call(FunctionCall *call) {
    // Save caller-saved registers
    save_caller_saved_registers();
    
    // Push parameters (right-to-left for cdecl)
    for (int i = call->param_count - 1; i >= 0; i--) {
        push_parameter(call->parameters[i]);
    }
    
    // Call function
    emit_call(call->function_name);
    
    // Clean up stack (callee or caller depending on convention)
    cleanup_stack(call->param_count * sizeof(int));
    
    // Restore caller-saved registers
    restore_caller_saved_registers();
    
    // Move return value to target register
    if (call->has_return_value) {
        emit_mov(call->return_reg, EAX);
    }
}
\end{lstlisting}

\section{Optimization in Code Generation}

\subsection{Peephole Optimization}

\begin{lstlisting}[language=C]
void peephole_optimization(Instruction *instructions, int count) {
    for (int i = 0; i < count - 1; i++) {
        // MOV reg, reg -> NOP
        if (is_mov_reg_to_reg(&instructions[i])) {
            instructions[i].opcode = NOP;
        }
        
        // PUSH reg; POP reg -> NOP
        if (is_push_pop_same_reg(&instructions[i], &instructions[i+1])) {
            instructions[i].opcode = NOP;
            instructions[i+1].opcode = NOP;
        }
        
        // MOV reg, imm; ADD reg, imm -> LEA reg, [imm]
        if (can_convert_to_lea(&instructions[i], &instructions[i+1])) {
            convert_to_lea(&instructions[i], &instructions[i+1]);
        }
    }
}
\end{lstlisting}

\subsection{Instruction Scheduling}

\begin{lstlisting}[language=C]
void schedule_instructions(Instruction *instructions, int count) {
    // Simple list scheduling for basic blocks
    Instruction *scheduled[count];
    int scheduled_count = 0;
    
    while (scheduled_count < count) {
        // Find ready instructions (no dependencies)
        for (int i = 0; i < count; i++) {
            if (!is_scheduled(&instructions[i]) && 
                is_ready(&instructions[i], scheduled, scheduled_count)) {
                scheduled[scheduled_count++] = instructions[i];
                break;
            }
        }
    }
    
    // Copy scheduled instructions back
    memcpy(instructions, scheduled, count * sizeof(Instruction));
}
\end{lstlisting}


% ============================================================
% AKTIVITAS PEMBELAJARAN
% ============================================================
\begin{aktivitas}
  \item \textbf{Instruction Selection}: Implementasikan instruction selector untuk subset x86.
  \item \textbf{Register Allocation}: Bangun linear scan register allocator.
  \item \textbf{Code Generation}: Implementasikan code generator untuk simple expressions.
  \item \textbf{Peephole Optimization}: Buat peephole optimizer untuk assembly code.
  \item \textbf{Function Calls}: Generate code untuk function calls dengan calling conventions.
\end{aktivitas}

% ============================================================
% LATIHAN DAN REFLEKSI
% ============================================================
\begin{latihan}
  \item Generate assembly code untuk expression tree kompleks!
  \item Implementasikan register allocator dengan spilling strategy!
  \item Analisis instruction selection for different target architectures!
  \item Optimasi generated code dengan peephole optimizations!
  \item Generate code untuk recursive functions dengan proper stack management!
  \item \textbf{Refleksi}: Bagaimana target architecture mempengaruhi code generation strategy?
\end{latihan}

% ============================================================
% ASESMEN
% ============================================================
\begin{asesmen}
\textbf{Instrumen Penilaian untuk Sub-CPMK 5.3}

\textbf{A. Pilihan Ganda}

\begin{enumerate}
  \item Register allocation problem terjadi karena:
  \begin{enumerate}
    \item Terlalu banyak variabel
    \item Terbatasnya jumlah register
    \item Memory terlalu kecil
    \item Instruksi terlalu kompleks
  \end{enumerate}
  
  \item CISC architecture memiliki:
  \begin{enumerate}
    \item Fixed-length instructions
    \item Variable-length instructions
    \item Load/store only
    \item Large register file
  \end{enumerate}
  
  \item Peephole optimization bekerja pada:
  \begin{enumerate}
    \item Single instruction
    \item Small window of instructions
    \item Entire program
    \item Basic blocks
  \end{enumerate}
\end{enumerate}

\textbf{B. Essay}

\begin{enumerate}
  \item Jelaskan complete code generation pipeline dari three-address code ke assembly!
  \item Implementasikan code generator untuk bahasa sederhana dengan arithmetic expressions dan function calls!
\end{enumerate}

\textbf{Rubrik Penilaian}: Lihat Lampiran A
\end{asesmen}

% ============================================================
% CHECKLIST KOMPETENSI
% ============================================================
\begin{checklist}
  \item Saya dapat mengimplementasikan code generator untuk arsitektur target
  \item Saya dapat melakukan instruction selection yang efisien
  \item Saya dapat mengimplementasikan register allocation algorithms
  \item Saya dapat mengenerate code untuk function calls
  \item Saya dapat melakukan peephole optimizations
  \item Saya memahami perbedaan CISC dan RISC architectures
\end{checklist}

% ============================================================
% RANGKUMAN
% ============================================================
\begin{rangkuman}
Bab ini membahas code generation dan target machine, termasuk instruction selection, register allocation, target architectures, dan optimization techniques. Mahasiswa belajar membangun code generator yang efisien.

\textbf{Poin Kunci:}
\begin{itemize}
  \item Code generation mengkonversi intermediate code ke target code
  \item Instruction selection memilih optimal target instructions
  \item Register allocation mengelola limited register resources
  \item Target architecture mempengaruhi generation strategy
  \item Peephole optimization mengoptimasi local instruction patterns
  \item Function calls memerlukan proper calling convention handling
\end{itemize}

\textbf{Kata Kunci}: \compiler{Code Generation}, \compiler{Instruction Selection}, \compiler{Register Allocation}, \compiler{Target Architecture}, \compiler{x86}, \compiler{RISC}, \compiler{Peephole Optimization}, \compiler{Calling Convention}
\end{rangkuman}

\ifSubfilesClassLoaded{%
    \clearpage
    \printbibliography[title={Daftar Pustaka}]
    \end{refsection}
}{}

\end{document}
