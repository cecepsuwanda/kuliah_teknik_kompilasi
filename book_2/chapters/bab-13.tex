\documentclass[../main.tex]{subfiles}

\addbibresource{\subfix{../references.bib}}

\begin{document}

\ifSubfilesClassLoaded{%
    \setcounter{chapter}{12}%
    \begin{refsection}
}{}

\chapter{Register Allocation dan Optimization}
\label{chap:register-allocation}

\begin{subcpmk}
  \item \textbf{Sub-CPMK 5.3:} Mengoptimasi penggunaan register dan meminimalkan memory access
\end{subcpmk}

% ============================================================
% MATERI POKOK
% ============================================================
\section{Alokasi Register: Strategi dan Kompleksitas}

Karena jumlah register fisik dalam CPU sangat terbatas, kompilator harus memutuskan variabel mana yang layak menempati register dan mana yang harus dipindahkan (\textit{spilled}) ke RAM.

\subsection{Graph Coloring}
Masalah alokasi register dapat dimodelkan sebagai pewarnaan graf (\textit{Graph Coloring}).
\begin{itemize}
    \item Node: Variabel yang aktif (\textit{live variables}).
    \item Edge: Saling berpotongan (\textit{interfere}), artinya dua variabel hidup di waktu yang sama.
    \item Warna: Jumlah register fisik yang tersedia.
\end{itemize}

\subsection{Linear Scan}
Untuk kompilasi yang cepat (\textit{JIT compilers}), algoritma \textit{Linear Scan} digunakan karena lebih sederhana dibanding \textit{graph coloring} namun tetap memberikan hasil yang memadai.

\section{Interference Graph}

\subsection{Graph Coloring}

Register allocation sebagai graph coloring:

\begin{itemize}
  \item \textbf{Nodes}: Variabel/temporaries
  \item \textbf{Edges}: Interference (variables live simultaneously)
  \item \textbf{Colors}: Register assignments
  \item \textbf{Spilling}: Variables yang tidak dapat diwarnai
\end{itemize}

\subsection{Interference Graph Construction}

\begin{lstlisting}[language=C]
typedef struct {
    int var_id;
    char *var_name;
    int start_point;
    int end_point;
} LiveRange;

typedef struct {
    int num_vars;
    bool **adjacency;  // Interference matrix
    LiveRange *ranges;
} InterferenceGraph;

InterferenceGraph* build_interference_graph(BasicBlock *block) {
    // Calculate live ranges
    LiveRange *ranges = calculate_live_ranges(block);
    
    // Build interference graph
    InterferenceGraph *graph = create_graph(num_vars);
    
    for (int i = 0; i < num_vars; i++) {
        for (int j = i + 1; j < num_vars; j++) {
            if (ranges_interfere(ranges[i], ranges[j])) {
                add_interference_edge(graph, i, j);
            }
        }
    }
    
    return graph;
}
\end{lstlisting}

\section{Graph Coloring Algorithm}

\subsection{Simplified Graph Coloring}

\begin{lstlisting}[language=C]
typedef struct {
    int var_id;
    int color;        // -1 = spilled, 0+ = register number
    int degree;       // Number of neighbors
    bool removed;
} GraphNode;

int graph_coloring(InterferenceGraph *graph, int num_registers) {
    GraphNode *nodes = initialize_nodes(graph);
    int spilled_count = 0;
    
    // Simplify phase
    while (has_uncolored_nodes(nodes)) {
        // Find node with degree < num_registers
        int node = find_low_degree_node(nodes, num_registers);
        
        if (node != -1) {
            // Remove from graph
            remove_node(nodes, node);
        } else {
            // Spill a node
            int spill_node = select_spill_node(nodes);
            nodes[spill_node].color = -1;  // Mark as spilled
            remove_node(nodes, spill_node);
            spilled_count++;
        }
    }
    
    // Select phase - assign colors
    assign_colors(nodes, num_registers);
    
    return spilled_count;
}
\end{lstlisting}

\section{Linear Scan Allocation}

\subsection{Linear Scan Algorithm}

Alokasi register yang lebih efisien:

\begin{lstlisting}[language=C]
typedef struct {
    int var_id;
    int start;
    int end;
    int reg;          // -1 if not allocated
    bool active;
} Interval;

void linear_scan(Interval *intervals, int count, int num_registers) {
    // Sort intervals by start point
    sort_intervals_by_start(intervals, count);
    
    Interval *active[num_registers];
    int active_count = 0;
    
    for (int i = 0; i < count; i++) {
        // Expire old intervals
        expire_old_intervals(intervals[i].start, active, &active_count);
        
        if (active_count < num_registers) {
            // Allocate register
            intervals[i].reg = find_free_register(active, active_count);
            add_to_active(&intervals[i], active, &active_count);
        } else {
            // Spill
            int spill_index = select_spill_candidate(active, active_count);
            spill_interval(active[spill_index]);
            intervals[i].reg = active[spill_index]->reg;
            active[spill_index] = &intervals[i];
        }
    }
}
\end{lstlisting}

\section{Spilling Strategies}

\subsection{Spill Cost Analysis}

Menentukan variabel yang akan di-spill:

\begin{lstlisting}[language=C]
typedef struct {
    int var_id;
    float spill_cost;  // Lower = better to spill
    int memory_accesses;
    int loop_depth;
    int use_count;
} SpillCandidate;

float calculate_spill_cost(SpillCandidate *candidate) {
    // Consider multiple factors
    float cost = 0.0;
    
    // More memory accesses = higher cost (don't want to spill)
    cost += candidate->memory_accesses * 10.0;
    
    // Deeper in loops = higher cost
    cost += candidate->loop_depth * 5.0;
    
    // More uses = higher cost  
    cost += candidate->use_count * 2.0;
    
    // Normalize by live range length
    int range_length = candidate->end - candidate->start;
    return cost / range_length;
}

\subsection{Spill Code Generation}

\begin{lstlisting}[language=C]
void generate_spill_code(Instruction *instructions, int *count, 
                        SpillCandidate *spilled_vars, int num_spilled) {
    for (int i = 0; i < num_spilled; i++) {
        int var_id = spilled_vars[i].var_id;
        
        // Insert load before each use
        for (int j = 0; j < *count; j++) {
            if (uses_variable(&instructions[j], var_id)) {
                insert_load_before(&instructions[j], var_id, count);
                j++;  // Skip inserted instruction
            }
        }
        
        // Insert store after each definition
        for (int j = 0; j < *count; j++) {
            if (defines_variable(&instructions[j], var_id)) {
                insert_store_after(&instructions[j], var_id, count);
                j++;  // Skip inserted instruction
            }
        }
    }
}
\end{lstlisting}

\section{Penggabungan (Coalescing)}

\compiler{Coalescing} adalah teknik untuk menghilangkan instruksi salin (\code{MOVE}) yang tidak perlu dengan memberikan register yang sama kepada sumber dan target salinan.

\subsection{Heuristik Penggabungan yang Aman}
Penggabungan agresif dapat membuat graf yang awalnya bisa diwarnai menjadi tidak bisa diwarnai (\textit{uncolorable}). Oleh karena itu, kompilator menggunakan dua kriteria utama:
\begin{enumerate}
    \item \textbf{Kriteria Briggs}: Dua simpul dapat digabungkan jika simpul hasil penggabungan memiliki kurang dari $k$ tetangga yang memiliki derajat $\ge k$. Ini menjamin fase \textit{Simplify} masih bisa berjalan lancar.
    \item \textbf{Kriteria George}: Simpul $U$ dan $V$ dapat digabungkan jika untuk setiap tetangga $T$ dari $V$, $T$ sudah bertetangga dengan $U$ atau $T$ memiliki derajat rendah ($< k$).
\end{enumerate}

\subsection{Manfaat}
Selain mengurangi jumlah instruksi, \textit{coalescing} juga mengurangi kebutuhan register secara keseluruhan jika banyak variabel temporer yang sebenarnya hanya merupakan salinan dari variabel lain.

\begin{figure}[!htbp]
    \centering
    \adjustbox{max width=0.8\textwidth,center}{%
    \begin{tikzpicture}[
        node/.style={circle, draw, minimum size=0.6cm, font=\tiny}
    ]
    \node[node] (x) {x};
    \node[node, right=1.5cm of x] (t) {t1};
    \draw[->, double] (x) -- node[above, font=\tiny] {MOV} (t);
    \node[node, right=3.5cm of x] (combined) {x/t1};
    \node[font=\tiny] at (2.5, 0) {$\Rightarrow$};
    \end{tikzpicture}%
    }
    \caption{Coalescing: Menghapus MOVE dengan menyatukan node dalam graf interferensi}
\end{figure}

\section{Advanced Optimizations}

\subsection{Register Renaming}

Menghilangkan false dependencies:

\begin{lstlisting}[language=C]
// Before renaming:
t1 = a + b
t2 = t1 * c
t1 = d + e    // False dependency on previous t1
t3 = t1 * f

// After renaming:
t1 = a + b
t2 = t1 * c
t3 = d + e    // No false dependency
t4 = t3 * f

void rename_registers(BasicBlock *block) {
    int next_temp = 0;
    int register_map[MAX_VARIABLES];
    
    for (int i = 0; i < block->instruction_count; i++) {
        Instruction *inst = &block->instructions[i];
        
        // Rename destination
        if (inst->result) {
            int new_reg = next_temp++;
            register_map[inst->result] = new_reg;
            inst->result = new_reg;
        }
        
        // Update source operands
        if (inst->arg1 && register_map[inst->arg1]) {
            inst->arg1 = register_map[inst->arg1];
        }
        if (inst->arg2 && register_map[inst->arg2]) {
            inst->arg2 = register_map[inst->arg2];
        }
    }
}

\subsection{Loop Optimization}

\begin{lstlisting}[language=C]
void optimize_loop_registers(Loop *loop) {
    // Allocate loop invariants to registers
    identify_loop_invariants(loop);
    
    // Keep frequently used loop variables in registers
    analyze_loop_variable_usage(loop);
    
    // Minimize register pressure in loop body
    reduce_loop_register_pressure(loop);
    
    // Prefer registers for induction variables
    allocate_induction_variables(loop);
}
\end{lstlisting}


% ============================================================
% AKTIVITAS PEMBELAJARAN
% ============================================================
\begin{aktivitas}
  \item \textbf{Interference Graph}: Implementasikan interference graph construction.
  \item \textbf{Graph Coloring}: Bangun graph coloring register allocator.
  \item \textbf{Linear Scan}: Implementasikan linear scan allocation algorithm.
  \item \textbf{Spilling}: Desain spill cost analysis dan spill code generation.
  \item \textbf{Coalescing}: Implementasikan copy coalescing optimization.
\end{aktivitas}

% ============================================================
% LATIHAN DAN REFLEKSI
% ============================================================
\begin{latihan}
  \item Bangun interference graph untuk potongan kode dengan multiple variables!
  \item Implementasikan graph coloring dengan backtracking untuk optimal solution!
  \item Analisis spill cost untuk berbagai variabel dalam nested loops!
  \item Implementasikan linear scan dengan heuristic improvements!
  \item Optimasi register allocation untuk loop-intensive code!
  \item \textbf{Refleksi}: Bagaimana register allocation mempengaruhi performance generated code?
\end{latihan}

% ============================================================
% ASESMEN
% ============================================================
\begin{asesmen}
\textbf{Instrumen Penilaian untuk Sub-CPMK 5.3}

\textbf{A. Pilihan Ganda}

\begin{enumerate}
  \item Interference graph edge menunjukkan:
  \begin{enumerate}
    \item Variables yang sama
    \item Variables yang hidup bersamaan
    \item Variables yang di-copy
    \item Variables yang di-spill
  \end{enumerate}
  
  \item Linear scan allocation memiliki complexity:
  \begin{enumerate}
    \item O(n)
    \item O(n log n)
    \item O(n²)
    \item O(n³)
  \end{enumerate}
  
  \item Spilling dilakukan ketika:
  \begin{enumerate}
    \item Register penuh
    \item Memory penuh
    \item Graph tidak bisa diwarnai
    \item Variabel tidak digunakan
  \end{enumerate}
\end{enumerate}

\textbf{B. Essay}

\begin{enumerate}
  \item Jelaskan complete register allocation pipeline dengan interference graph and graph coloring!
  \item Implementasikan register allocator dengan linear scan algorithm and spill handling!
\end{enumerate}

\textbf{Rubrik Penilaian}: Lihat Lampiran A
\end{asesmen}

% ============================================================
% CHECKLIST KOMPETENSI
% ============================================================
\begin{checklist}
  \item Saya dapat mengoptimasi penggunaan register dan meminimalkan memory access
  \item Saya dapat membangun interference graph untuk register allocation
  \item Saya dapat mengimplementasikan graph coloring algorithm
  \item Saya dapat melakukan linear scan register allocation
  \item Saya dapat mengimplementasikan spill strategies
  \item Saya dapat melakukan register coalescing dan renaming
\end{checklist}

% ============================================================
% RANGKUMAN
% ============================================================
\begin{rangkuman}
Bab ini membahas register allocation dan optimization, termasuk interference graph, graph coloring, linear scan allocation, spilling strategies, dan advanced optimizations. Mahasiswa belajar mengoptimasi penggunaan register untuk performance maksimal.

\textbf{Poin Kunci:}
\begin{itemize}
  \item Register allocation memetakan variabel unlimited ke register terbatas
  \item Interference graph modeling conflicts antar variabel
  \item Graph coloring adalah NP-complete problem
  \item Linear scan memberikan heuristic yang efisien
  \item Spilling menangani kasus ketika register tidak cukup
  \item Coalescing dan renaming mengoptimasi register usage
\end{itemize}

\textbf{Kata Kunci}: \compiler{Register Allocation}, \compiler{Interference Graph}, \compiler{Graph Coloring}, \compiler{Linear Scan}, \compiler{Spilling}, \compiler{Coalescing}, \compiler{Register Renaming}
\end{rangkuman}

\ifSubfilesClassLoaded{%
    \clearpage
    \printbibliography[title={Daftar Pustaka}]
    \end{refsection}
}{}

\end{document}
