\documentclass[../main.tex]{subfiles}

\addbibresource{\subfix{../references.bib}}

\begin{document}

\ifSubfilesClassLoaded{%
    \setcounter{chapter}{1}%
    \begin{refsection}
}{}

\chapter{Landasan Teori dan Konsep Dasar Kompilasi}
\label{chap:landasan-teori}

\begin{subcpmk}
  \item \textbf{Sub-CPMK 1.1:} Menjelaskan perbedaan antara interpreter dan compiler
  \item \textbf{Sub-CPMK 1.2:} Mengidentifikasi fase-fase kompilator dalam arsitektur kompilator nyata
  \item \textbf{Sub-CPMK 1.3:} Menganalisis trade-off antara one-pass vs multi-pass compiler
\end{subcpmk}

% ============================================================
% MATERI POKOK
% ============================================================
\section{Konsep dan Definisi Kunci}

\subsection{Definisi Kompilator}

Secara tradisional, kompilator dipandang sebagai "kotak hitam" yang mengubah kode sumber menjadi kode target executable. Menurut \cite{aho2006compilers}, proses ini sebenarnya terdiri dari serangkaian fase yang saling terkait \cite{diznr2024phases}.
Secara formal, \compiler{compiler} adalah program yang melakukan translasi dari bahasa sumber (source language) ke bahasa target (target language), dengan mempertahankan makna semantik dari program sumber.

Menurut Aho, Lam, Sethi, dan Ullman dalam buku klasik "Compilers: Principles, Techniques, dan Tools"\cite{aho2006compilers}:

\begin{quote}
``A compiler is a program that can read a program in one language (the source language) and translate it into an equivalent program in another language (the target language).''
\end{quote}

\subsection{Karakteristik Kompilator}

Kompilator memiliki beberapa karakteristik penting yang membedakannya dari pemroses bahasa lainnya:

\begin{itemize}
    \item \textbf{Translasi Lengkap}: Kompilator membaca dan menganalisis seluruh program sumber sebelum menghasilkan output.
    \item \textbf{Analisis Mendalam}: Melakukan pengecekan struktur (\textit{syntax}) dan makna (\textit{semantic}) secara menyeluruh.
    \item \textbf{Output Terpisah}: Menghasilkan file terpisah (seperti \textit{object file} atau \textit{executable}) yang dapat berjalan tanpa kode sumber asli.
    \item \textbf{Optimasi}: Menggunakan teknik matematis dan heuristik untuk meningkatkan efisiensi kode hasil translasi.
\end{itemize}

\subsection{Interpreter vs Compiler}

Perbedaan fundamental antara \compiler{interpreter} dan \compiler{compiler}:

\begin{table}[h]
\centering
\begin{tabular}{|l|l|l|}
\hline
\textbf{Aspek} & \textbf{Compiler} & \textbf{Interpreter} \\
\hline
Eksekusi & Compile lalu run & Run langsung \\
Kecepatan & Lebih cepat & Lebih lambat \\
Debugging & Lebih sulit & Lebih mudah \\
Platform & Platform-dependent & Platform-independent \\
Memory usage & Lebih besar & Lebih kecil \\
\hline
\end{tabular}
\caption{Perbandingan Compiler dan Interpreter}
\end{table}

\subsection{Arsitektur Kompilator}

Kompilator modern memiliki arsitektur berlapis yang terdiri dari beberapa fase:

\begin{enumerate}
  \item \textbf{Analysis Phase}
  \begin{itemize}
    \item Lexical Analysis (Scanner)
    \item Syntax Analysis (Parser)
    \item Semantic Analysis
  \end{itemize}
  \item \textbf{Synthesis Phase}
  \begin{itemize}
    \item Intermediate Code Generation
    \item Code Optimization
    \item Code Generation
  \end{itemize}
\end{enumerate}

\section{Tingkat Bahasa Pemrograman}

Bahasa pemrograman dapat diklasifikasikan berdasarkan tingkat abstraksinya terhadap mesin. Semakin tinggi tingkat abstraksi, semakin besar peran translator untuk menjembatani kode sumber menjadi instruksi mesin yang dapat dieksekusi.

\begin{table}[!htbp]
\centering
\begin{tabularx}{\textwidth}{|l|X|l|}
\hline
\textbf{Level} & \textbf{Contoh} & \textbf{Translator} \\
\hline
Bahasa Mesin & Kode biner spesifik arsitektur (x86-64, ARM) & -- (langsung CPU) \\
\hline
Tingkat Rendah & Assembly, instruksi dekat hardware & Assembler \\
\hline
Tingkat Menengah & C dengan pointer dan kontrol memori & Compiler \\
\hline
Tingkat Tinggi & Python, Java, C\# dengan abstraksi tinggi & Compiler/Interpreter \\
\hline
\end{tabularx}
\caption{Level bahasa pemrograman dan kebutuhan translator}
\end{table}

\subsection{Contoh Ringkas}
\begin{itemize}
  \item \textbf{Bahasa Mesin (konseptual)}: \texttt{10110000 00000001}
  \item \textbf{Assembly}: \texttt{mov eax, 1}
  \item \textbf{C}: \texttt{int x = 1;}
  \item \textbf{Python}: \texttt{print(1)}
\end{itemize}

\subsection{Bahasa Mesin}
Bahasa mesin adalah representasi biner yang dipahami langsung oleh CPU. Setiap instruksi dan data direpresentasikan sebagai bit (0 dan 1) yang spesifik terhadap arsitektur prosesor tertentu (misalnya x86-64 atau ARM).

\subsection{Bahasa Tingkat Rendah}
Bahasa tingkat rendah umumnya merujuk pada bahasa \textit{assembly}. Instruksi assembly memiliki korespondensi yang sangat dekat dengan instruksi mesin, sehingga efisien tetapi sulit dipahami dan tidak portabel.

\subsection{Bahasa Tingkat Menengah}
Bahasa tingkat menengah (contoh: C) berada di antara kemudahan bahasa tingkat tinggi dan kendali bahasa tingkat rendah. Bahasa ini menyediakan abstraksi seperti fungsi dan tipe data, namun tetap memungkinkan akses memori langsung melalui pointer.

\subsection{Bahasa Tingkat Tinggi}
Bahasa tingkat tinggi (contoh: Python, Java, C\#) lebih dekat dengan cara berpikir manusia. Bahasa ini memiliki abstraksi yang kaya (object, module, library) sehingga produktif, tetapi membutuhkan compiler atau interpreter untuk dieksekusi oleh mesin.

\section{Translator: Assembler, Interpreter, dan Compiler}

Translator adalah program yang menerjemahkan kode sumber dari satu bahasa ke bahasa lain. Dalam ekosistem pemrograman, terdapat tiga jenis translator utama yang bekerja pada level berbeda.

\subsection{Assembler}
Assembler menerjemahkan kode \textit{assembly} ke bahasa mesin. Proses ini umumnya satu-ke-satu (setiap instruksi assembly dipetakan ke instruksi mesin yang setara) dan menghasilkan \textit{object file} yang siap untuk di-link.

\subsection{Interpreter}
Interpreter menjalankan program secara langsung dengan membaca dan mengeksekusi instruksi baris demi baris. Interpreter tidak menghasilkan file biner permanen; kinerja biasanya lebih lambat, tetapi fleksibel dan cocok untuk prototyping cepat.

\subsection{Compiler}
Compiler menerjemahkan seluruh program ke bentuk target (assembly atau kode mesin) sebelum dijalankan. Hasil kompilasi berupa artefak biner yang dapat dieksekusi berulang kali tanpa kode sumber. Compiler memungkinkan optimasi yang lebih agresif.

\subsection{Perbandingan Singkat}
\begin{itemize}
  \item \textbf{Assembler}: input assembly $\rightarrow$ output machine code (sangat dekat dengan hardware).
  \item \textbf{Interpreter}: input source $\rightarrow$ eksekusi langsung (tanpa output biner permanen).
  \item \textbf{Compiler}: input source $\rightarrow$ output biner/assembly (dengan fase optimasi).
\end{itemize}

\begin{table}[!htbp]
\centering
\begin{tabularx}{\textwidth}{|l|X|X|l|l|}
\hline
\textbf{Translator} & \textbf{Input} & \textbf{Output} & \textbf{Eksekusi} & \textbf{Contoh} \\
\hline
Assembler & Assembly & Machine code / object file & Setelah link & NASM, GAS \\
\hline
Interpreter & Source code & Eksekusi langsung & Saat run & Python, Ruby \\
\hline
Compiler & Source code & Assembly / biner & Setelah compile & GCC, Clang \\
\hline
\end{tabularx}
\caption{Perbandingan assembler, interpreter, dan compiler}
\end{table}

\begin{figure}[!htbp]
\centering
\begin{tikzpicture}[
  box/.style={rectangle, draw=blue!50, fill=blue!10, text width=2.2cm, text centered, minimum height=0.8cm, rounded corners, font=\footnotesize},
  arrow/.style={->, >=stealth, thick},
  node distance=1.6cm
]
  \node[box] (src) {Source Code};
  \node[box, right=of src] (comp) {Compiler};
  \node[box, right=of comp] (bin) {Binary};
  \node[box, below=1.0cm of comp] (interp) {Interpreter};
  \node[box, right=of interp] (run) {Execution};

  \draw[arrow] (src) -- (comp);
  \draw[arrow] (comp) -- (bin);
  \draw[arrow] (src) |- (interp);
  \draw[arrow] (interp) -- (run);
\end{tikzpicture}
\caption{Alur eksekusi dengan compiler vs interpreter}
\end{figure}

\section{Teori Utama Compiler}

\subsection{Formal Language Theory}

Teori bahasa formal menjadi dasar bagi compiler design:

\begin{itemize}
  \item \textbf{Regular Expressions}: Untuk lexical analysis
  \item \textbf{Context-Free Grammars}: Untuk syntax analysis
  \item \textbf{Finite Automata}: Model recognizer untuk tokens
  \item \textbf{Pushdown Automata}: Model recognizer untuk parsing
\end{itemize}

\subsection{One-Pass vs Multi-Pass Compiler}

\subsubsection{One-Pass Compiler}

Kompilator \textit{single-pass} mencoba menyelesaikan semua fase dalam satu kali pembacaan kode (\textit{pass}).
\begin{itemize}
  \item \textbf{Kelebihan}: Proses kompilasi sangat cepat dan hemat memori.
  \item \textbf{Kekurangan}: Lebih sulit dikembangkan, optimasi sangat terbatas, dan tidak fleksibel terhadap struktur bahasa yang kompleks.
  \item \textbf{Contoh}: Pascal, implementasi awal bahasa C.
\end{itemize}

\subsubsection{Multi-Pass Compiler}

Kompilator modern umumnya menggunakan pendekatan \textit{multi-pass}, di mana setiap fase (atau kelompok fase) dijalankan dalam \textit{pass} terpisah.
\begin{itemize}
  \item \textbf{Kelebihan}: Modularitas tinggi, pemisahan perhatian tiap fase, dan memungkinkan optimasi global yang mendalam.
  \item \textbf{Kekurangan}: Membutuhkan lebih banyak memori dan waktu kompilasi dibandingkan \textit{single-pass}.
  \item \textbf{Contoh}: GCC, LLVM/Clang, modern C++, Java compiler.
\end{itemize}

\begin{figure}[!htbp]
    \centering
    \adjustbox{max width=0.85\textwidth,center}{%
    \begin{tikzpicture}[
        box/.style={rectangle, draw=blue!50, fill=blue!10, text width=2.5cm, text centered, minimum height=0.7cm, rounded corners, font=\footnotesize, inner sep=4pt},
        bigbox/.style={rectangle, draw=blue!50, fill=blue!10, text width=2.8cm, text centered, minimum height=3.2cm, rounded corners, font=\footnotesize, inner sep=6pt},
        arrow/.style={->, >=stealth, thick},
        title/.style={font=\bfseries\small},
        node distance=0.4cm and 3.0cm
    ]
    
    \node[title] (mp-title) {MULTI-PASS};
    \node[box, below=of mp-title] (mp1) {Pass 1: Lexical};
    \node[box, below=of mp1] (mp2) {Pass 2: Syntax};
    \node[box, below=of mp2] (mp3) {Pass 3: Semantic};
    \node[box, below=of mp3] (mpn) {... Code Gen};
    \draw[arrow] (mp1) -- (mp2);
    \draw[arrow] (mp2) -- (mp3);
    \draw[arrow] (mp3) -- (mpn);
    
    \node[title, right=of mp-title] (sp-title) {SINGLE-PASS};
    \node[bigbox, below=of sp-title] (sp-all) {Semua fase\\dalam satu pass};
    
    \end{tikzpicture}%
    }
    \caption{Perbandingan arsitektur Multi-Pass dan Single-Pass Compiler}
    \label{fig:multipass-vs-singlepass}
\end{figure}

\section{Grammar dan Hierarki Chomsky}

Hierarki Chomsky mengklasifikasikan bahasa formal berdasarkan kekuatan ekspresinya dan jenis automata yang dapat mengenalinya. Klasifikasi ini penting karena fase kompilasi memanfaatkan kelas bahasa yang berbeda.

\begin{itemize}
  \item \textbf{Type 0 (Unrestricted Grammar)}: Bahasa paling umum, setara dengan \textit{Turing Machine}. Semua bahasa yang dapat dihitung (recursively enumerable) berada pada level ini.
  \item \textbf{Type 1 (Context-Sensitive Grammar)}: Dikenali oleh \textit{Linear Bounded Automata}. Dapat memodelkan dependensi konteks yang tidak dapat ditangani oleh CFG.
  \item \textbf{Type 2 (Context-Free Grammar)}: Dikenali oleh \textit{Pushdown Automata}. Digunakan untuk mendefinisikan struktur sintaksis bahasa pemrograman.
  \item \textbf{Type 3 (Regular Grammar)}: Dikenali oleh \textit{Finite Automata}. Digunakan untuk mendefinisikan token pada analisis leksikal.
\end{itemize}

Dalam praktik kompilator, \textit{regular language} digunakan pada lexer dan \textit{context-free language} digunakan pada parser. Level di atasnya jarang dipakai secara langsung dalam konstruksi compiler dasar.

\begin{table}[!htbp]
\centering
\begin{tabularx}{\textwidth}{|l|l|X|}
\hline
\textbf{Tipe} & \textbf{Automata Pengenal} & \textbf{Penggunaan di Compiler} \\
\hline
Type 0 & Turing Machine & Teori komputasi umum, tidak dipakai langsung \\
\hline
Type 1 & Linear Bounded Automata & Analisis kontekstual khusus (jarang) \\
\hline
Type 2 & Pushdown Automata & Parsing dengan CFG (struktur sintaks) \\
\hline
Type 3 & Finite Automata & Tokenization di lexer (regex) \\
\hline
\end{tabularx}
\caption{Ringkasan hierarki Chomsky dan peran di kompilator}
\end{table}

\subsection{Contoh Mini Grammar}
\textbf{Regular Grammar} (Type 3):
\begin{verbatim}
R -> aR | bR | a | b
\end{verbatim}
\textbf{Context-Free Grammar} (Type 2):
\begin{verbatim}
S -> a S b | ab
\end{verbatim}

\section{Alur Kerja dan Arsitektur Kompilator}

\subsection{Alur Kerja Kompilator: Dari Source ke Executable}

Sebelum membahas detail arsitektur, mari kita lihat gambaran umum alur kerja kompilator secara utuh. Gambar \ref{fig:compiler-flow} menunjukkan transformasi kode dari teks mentah hingga menjadi file yang dapat dieksekusi oleh mesin, sesuai dengan model sistem kompilasi standar \cite{uw2024compiler}.

\begin{figure}[!htbp]
\centering
\adjustbox{max width=0.85\textwidth,center}{%
\begin{tikzpicture}[
    process/.style={rectangle, draw=blue!50, fill=blue!10, text width=2.5cm, text centered, minimum height=0.6cm, rounded corners, font=\footnotesize},
    output/.style={rectangle, draw=green!50, fill=green!10, text width=2cm, text centered, minimum height=0.5cm, rounded corners, font=\tiny},
    arrow/.style={->, >=stealth, thick}
]
    \node[process] (source) {\textbf{Source Code}\\\footnotesize(C, C++, dll.)};
    \node[process, below=0.6cm of source] (preproc) {\textbf{Preprocessing}};
    \node[process, below=0.6cm of preproc] (compiler) {\textbf{Compiler}};
    \node[output, right=1cm of compiler] (asm) {Assembly Code};
    \node[process, below=0.6cm of compiler] (assemble) {\textbf{Assembler}};
    \node[output, right=1cm of assemble] (obj) {Object Code};
    \node[process, below=0.6cm of assemble] (link) {\textbf{Linker}};
    \node[output, right=1cm of link] (exe) {\textbf{Executable}};
    
    \draw[arrow] (source) -- (preproc);
    \draw[arrow] (preproc) -- (compiler);
    \draw[arrow] (compiler) -- (asm);
    \draw[arrow] (compiler) -- (assemble);
    \draw[arrow] (assemble) -- (obj);
    \draw[arrow] (assemble) -- (link);
    \draw[arrow] (link) -- (exe);
\end{tikzpicture}%
}
\caption{Alur kerja sistem kompilasi secara keseluruhan}
\label{fig:compiler-flow}
\end{figure}

\subsection{Dua Sisi Kompilator: Front-End dan Back-End}

Kompilator modern umumnya dibagi menjadi dua bagian utama: \textbf{front-end} (analisis) dan \textbf{back-end} (sintesis). Pemisahan ini memungkinkan satu front-end (misal untuk bahasa C) digunakan untuk berbagai back-end (misal untuk mesin Intel x86 dan ARM).

\begin{figure}[!htbp]
    \centering
    \adjustbox{max width=0.9\textwidth,center}{%
    \begin{tikzpicture}[
        box/.style={rectangle, draw=blue!50, fill=blue!10, text width=2.2cm, text centered, minimum height=0.9cm, rounded corners, font=\footnotesize, inner sep=4pt},
        arrow/.style={->, >=stealth, thick},
        section/.style={rectangle, draw=black!50, fill=gray!20, text width=3.0cm, text centered, minimum height=1.0cm, rounded corners, font=\bfseries\small, inner sep=5pt},
        irbox/.style={rectangle, draw=purple!50, fill=purple!10, text width=2.4cm, text centered, minimum height=1.0cm, rounded corners, font=\footnotesize, inner sep=4pt},
        node distance=0.8cm and 1.5cm
    ]
    
    \node[box] (source) {Source Code};
    \node[section, right=of source] (frontend) {FRONT-END\\(Analisis)};
    \node[irbox, right=of frontend] (ir) {Representasi\\Intermediate (IR)};
    \node[section, right=of ir] (backend) {BACK-END\\(Sintesis)};
    \node[box, right=of backend] (target) {Target Code};
    
    \draw[arrow] (source) -- (frontend);
    \draw[arrow] (frontend) -- (ir);
    \draw[arrow] (ir) -- (backend);
    \draw[arrow] (backend) -- (target);
    \end{tikzpicture}%
    }
    \caption{Struktur logis kompilator: Pemisahan Analisis dan Sintesis}
    \label{fig:compiler-architecture-simple}
\end{figure}

\section{Fase-Fase Kompilasi Secara Detail}

Di balik pembagian besar front-end dan back-end, terdapat enam fase utama yang bekerja secara sekuensial untuk melakukan transformasi kode.

\subsection{Analisis Leksikal (Scanner)}
Fase ini memecah karakter-karakter dalam \textit{source code} menjadi unit-unit atomik bermakna yang disebut \textbf{token}.
\begin{itemize}
    \item \textbf{Input}: String karakter kode sumber.
    \item \textbf{Output}: Stream token (\texttt{keyword}, \texttt{id}, \texttt{literal}, dll.).
\end{itemize}

\subsection{Analisis Sintaksis (Parser)}
Mengambil token dari scanner dan memeriksa apakah urutannya membentuk struktur yang valid sesuai \textit{grammar} bahasa.
\begin{itemize}
    \item \textbf{Output}: \textit{Abstract Syntax Tree} (AST).
\end{itemize}

\subsection{Analisis Semantik}
Memeriksa makna dari kode, seperti kecocokan tipe data (\textit{type checking}) dan keberadaan deklarasi variabel (\textit{scope resolution}).

\subsection{Generasi Intermediate Code (IR)}
Translasi AST ke bentuk yang lebih dekat dengan instruksi mesin tetapi tetap independen terhadap jenis prosesor tertentu.

\subsection{Optimasi Kode}
Transformasi IR untuk menghasilkan kode yang lebih efisien (cepat dijalankan atau hemat memori) tanpa mengubah maksud program aslinya.

\subsection{Generasi Kode Target}
Tahap akhir yang mengubah IR menjadi instruksi spesifik untuk arsitektur mesin target (misalnya assembly x86 atau ARM).

\begin{figure}[!htbp]
    \centering
    \adjustbox{max width=0.88\textwidth,center}{%
    \begin{tikzpicture}[
        node distance=0.8cm,
        astnode/.style={circle, draw=blue!50, fill=blue!10, minimum size=0.6cm, font=\footnotesize},
        tacnode/.style={rectangle, draw=green!50, fill=green!10, text width=1.8cm, text centered, minimum height=0.6cm, rounded corners, font=\footnotesize},
        arrow/.style={->, >=stealth, thick}
    ]
        \node[astnode] (plus) at (0,1.5) {+};
        \node[astnode, below left=0.6cm of plus] (a) {a};
        \node[astnode, below right=0.6cm of plus] (b) {b};
        \draw[arrow] (plus) -- (a);
        \draw[arrow] (plus) -- (b);
        \node[left=1.5cm of plus] {\textbf{AST}};
        
        \node[right=2.2cm of plus] (arrow) {\Large $\Rightarrow$};
        
        \node[tacnode, right=1.6cm of arrow] (tac) {\texttt{t1 = a + b}};
        \node[above=0.2cm of tac, font=\small] {\textbf{Intermediate Code}};
    \end{tikzpicture}%
    }
    \caption{Visualisasi transformasi dari AST ke Intermediate Code (TAC)}
\end{figure}

\section{Contoh Praktis: Alur Kompilasi Program}

Untuk mengonkretkan teori fase-fase di atas, mari kita telusuri perjalanan satu baris kode C sederhana. Kita akan melihat bagaimana kode ini bermetamorfosis di setiap tahap.

\textbf{Kode Sumber:}
\begin{lstlisting}[language=C]
int sum = old_val + 10;
\end{lstlisting}

\begin{enumerate}
    \item \textbf{Scanner (Lexical Analysis)} \\
    Scanner membaca aliran karakter dan memecahnya menjadi token:
    \begin{itemize}
        \item \texttt{<KEYWORD, "int">}
        \item \texttt{<ID, "sum">}
        \item \texttt{<OP, "=">}
        \item \texttt{<ID, "old\_val">}
        \item \texttt{<OP, "+">}
        \item \texttt{<LITERAL, "10">}
        \item \texttt{<PUNCT, ";">}
    \end{itemize}

    \item \textbf{Parser (Syntax Analysis)} \\
    Parser menyusun token menjadi pohon sintaks. Ia mengenali pola deklarasi variabel:
    \begin{verbatim}
    Declaration
    |-- Type: int
    |-- Var: sum
    `-- Init: Assignment
        |-- LHS: sum
        `-- RHS: BinaryOp (+)
            |-- Left: old_val
            `-- Right: 10
    \end{verbatim}
    Jika kita lupa menulis titik-koma (\texttt{;}), parser akan melaporkan \textit{Syntax Error}.

    \item \textbf{Semantic Analyzer} \\
    Di sini kompilator memeriksa konteks.
    \begin{itemize}
        \item Apakah variabel \texttt{old\_val} sudah dideklarasikan sebelumnya?
        \item Apakah tipe data \texttt{old\_val} kompatibel untuk dijumlahkan dengan integer \texttt{10}?
    \end{itemize}
    \textit{Skenario Error}: Jika \texttt{old\_val} ternyata adalah string (\texttt{char*}), fase ini akan menghentikan proses dengan \textit{Type Mismatch Error}, meskipun secara sintaksis kalimat tersebut benar.

    \item \textbf{IR Generator \& Optimizer} \\
    Pohon di atas diterjemahkan menjadi kode sementara (misal: Quadruples):
    \begin{verbatim}
    LOAD  t1, old_val  ; Muat nilai variabel ke temp
    ADD   t2, t1, 10   ; Lakukan penjumlahan
    STORE sum, t2      ; Simpan hasil ke sum
    \end{verbatim}
    \textit{Optimasi}: Jika \texttt{old\_val} diketahui bernilai konstan 5, optimizer dapat langsung mengubahnya menjadi \texttt{STORE sum, 15}.

    \item \textbf{Code Generator (x86-64)} \\
    IR akhirnya dipetakan ke register dan instruksi mesin nyata:
    \begin{verbatim}
    mov eax, [rbp-4]           ; Ambil old_val dari stack
    add eax, 10                 ; Tambahkan 10
    mov DWORD PTR [rbp-8], eax  ; Simpan ke lokasi sum
    \end{verbatim}
\end{enumerate}


% ============================================================
% AKTIVITAS PEMBELAJARAN
% ============================================================
\begin{aktivitas}
  \item \textbf{Analisis Compiler}: Identifikasi compiler yang Anda gunakan sehari-hari dan klasifikasikan sebagai one-pass atau multi-pass.
  \item \textbf{Studi Kasus}: Bandingkan GCC dan Clang dari segi arsitektur dan fase kompilasi.
  \item \textbf{Eksperimen}: Implementasikan interpreter sederhana untuk kalkulator aritmatika.
  \item \textbf{Research}: Pelajari compiler untuk bahasa modern (Rust, Go) dan identifikasi fitur inovatifnya.
  \item \textbf{Debat}: Diskusikan keuntungan dan kerugian JIT compilation vs AOT compilation.
\end{aktivitas}

% ============================================================
% LATIHAN DAN REFLEKSI
% ============================================================
\begin{latihan}
  \item Jelaskan perbedaan mendasar antara compiler dan interpreter dengan contoh nyata!
  \item Gambarkan flowchart lengkap dari source code hingga executable file!
  \item Analisis trade-off antara one-pass dan multi-pass compiler untuk embedded system!
  \item Mengapa semantic analysis diperlukan setelah syntax analysis?
  \item Buat contoh regular expression untuk mengenali identifier dalam bahasa C!
  \item \textbf{Refleksi}: Konsep mana yang paling menantang dalam bab ini dan bagaimana cara mengatasinya?
\end{latihan}

% ============================================================
% ASESMEN
% ============================================================
\begin{asesmen}
\textbf{Instrumen Penilaian untuk Sub-CPMK 1.1-1.3}

\textbf{A. Pilihan Ganda}

\begin{enumerate}
  \item Manakah yang BUKAN termasuk fase analysis phase?
  \begin{enumerate}
    \item Lexical Analysis
    \item Syntax Analysis
    \item Code Generation
    \item Semantic Analysis
  \end{enumerate}
  
  \item Keuntungan utama multi-pass compiler adalah:
  \begin{enumerate}
    \item Kecepatan kompilasi lebih tinggi
    \item Memory usage lebih rendah
    \item Optimasi yang lebih baik
    \item Debugging lebih mudah
  \end{enumerate}
  
  \item Regular expression digunakan dalam:
  \begin{enumerate}
    \item Semantic Analysis
    \item Code Generation
    \item Lexical Analysis
    \item Syntax Analysis
  \end{enumerate}
\end{enumerate}

\textbf{B. Essay}

\begin{enumerate}
  \item Jelaskan perbedaan antara one-pass dan multi-pass compiler beserta contoh implementasinya!
  \item Analisis arsitektur compiler favorit Anda dan jelaskan mengapa arsitektur tersebut efektif!
\end{enumerate}

\textbf{Rubrik Penilaian}: Lihat Lampiran A
\end{asesmen}

% ============================================================
% CHECKLIST KOMPETENSI
% ============================================================
\begin{checklist}
  \item Saya dapat menjelaskan perbedaan antara interpreter dan compiler
  \item Saya dapat mengidentifikasi fase-fase kompilator dalam arsitektur nyata
  \item Saya dapat menganalisis trade-off one-pass vs multi-pass compiler
  \item Saya memahami peran formal language theory dalam compiler design
  \item Saya dapat menggambar arsitektur kompilator lengkap
  \item Saya dapat menjelaskan fungsi setiap fase kompilasi
\end{checklist}

% ============================================================
% RANGKUMAN
% ============================================================
\begin{rangkuman}
Bab ini membahas landasan teori dan konsep dasar kompilator, termasuk perbedaan interpreter vs compiler, arsitektur kompilator, teori bahasa formal, dan perbandingan one-pass vs multi-pass compiler.

\textbf{Poin Kunci:}
\begin{itemize}
  \item Compiler menerjemahkan source code ke target code melalui beberapa fase
  \item Interpreter mengeksekusi code langsung tanpa kompilasi terpisah
  \item Arsitektur kompilator terdiri dari analysis dan synthesis phase
  \item One-pass compiler cepat tapi terbatas, multi-pass fleksibel tapi kompleks
  \item Formal language theory adalah fondasi matematis untuk compiler design
\end{itemize}

\textbf{Kata Kunci}: \compiler{Compiler}, \compiler{Interpreter}, \compiler{Lexical Analysis}, \compiler{Syntax Analysis}, \compiler{Semantic Analysis}, \compiler{One-Pass}, \compiler{Multi-Pass}, \compiler{Regular Expression}, \compiler{Context-Free Grammar}
\end{rangkuman}

\ifSubfilesClassLoaded{%
    \clearpage
    \printbibliography[title={Daftar Pustaka}]
    \end{refsection}
}{}

\end{document}
