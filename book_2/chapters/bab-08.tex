\documentclass[../main.tex]{subfiles}

% Untuk kompilasi standalone, tambahkan \addbibresource di preamble
% Menggunakan \subfix untuk menangani path relatif dengan benar
% biblatex akan mengabaikan duplikasi jika sudah ada di main.tex
\addbibresource{\subfix{../references.bib}}

\begin{document}

% Set chapter counter untuk kompilasi standalone
% \chapter akan mengincrement counter, jadi untuk bab-08 (Bab 8), set ke 7
% Saat dikompilasi dari main.tex, counter akan diatur oleh subfiles package
\ifSubfilesClassLoaded{%
    \setcounter{chapter}{7}%
    % Untuk kompilasi standalone, gunakan refsection manual
    \begin{refsection}
}{}

\chapter{Parser Generator (Bison/Yacc) dan Praktikum Parser}
\label{chap:parser-generator}

\section{Tujuan Pembelajaran}

Setelah mempelajari bab ini, mahasiswa diharapkan mampu:
\begin{enumerate}
    \item Memahami konsep dan keuntungan menggunakan parser generator
    \item Menulis grammar specification menggunakan Bison/Yacc
    \item Mengintegrasikan Flex lexer dengan Bison parser
    \item Mengimplementasikan semantic actions untuk membangun AST
    \item Menangani error dalam parser yang dihasilkan Bison
    \item Membangun parser lengkap untuk subset bahasa sederhana
\end{enumerate}
\section{Pendahuluan}

Pada bab ini kita menyelesaikan \textbf{parser proyek compiler subset C} dengan Bison. Lexer proyek sudah ada di Bab 4 (file \texttt{simplec.l}); grammar proyek telah didefinisikan di Bab 1 dan Bab 5 (Bagian~\ref{sec:spec-subset-c}, \ref{sec:grammar-proyek-subset-c}). File \texttt{simplec.y} yang dibangun di bab ini mengimplementasikan grammar tersebut dan bersama \texttt{simplec.l} menjadi tulang punggung codebase proyek.

Setelah mempelajari implementasi parser secara manual (recursive descent) dan memahami konsep LR parsing, kita akan mempelajari cara menggunakan \textbf{parser generator} untuk menghasilkan parser secara otomatis dari grammar specification. Pendekatan ini memiliki beberapa keuntungan:

\begin{itemize}
    \item \textbf{Produktivitas}: Menghemat waktu dan mengurangi kesalahan dalam implementasi parser
    \item \textbf{Maintainability}: Grammar dapat diubah dengan mudah tanpa menulis ulang parser secara manual
    \item \textbf{Konsistensi}: Parser yang dihasilkan konsisten dengan grammar specification
    \item \textbf{Error Handling}: Parser generator menyediakan mekanisme error handling yang lebih baik
\end{itemize}

Menurut sumber dari IT Trip:

\begin{quote}
``Bison / YACC: define grammar in a .y file, specify \%token s, grammar rules, actions, etc. Generates C parser (or C++ variants). Flex + Bison: use Flex to build the lexer (.l file), Bison for parser, integrate them via tokens.''\cite{ittrip2024bison}
\end{quote}

\subsection{Bison vs Yacc}

\textbf{Yacc} (Yet Another Compiler Compiler) adalah parser generator yang dikembangkan di Bell Labs pada tahun 1970-an. \textbf{Bison} adalah implementasi GNU dari Yacc dengan beberapa peningkatan:

\begin{itemize}
    \item Dukungan untuk C++ output
    \item Fitur tambahan untuk error recovery
    \item Dukungan untuk GLR (Generalized LR) parsing
    \item Dokumentasi yang lebih lengkap
    \item Open source dan aktif dikembangkan
\end{itemize}

Dalam buku ini, kita akan menggunakan \textbf{Bison} karena lebih modern dan tersedia secara luas. Namun, konsep yang dipelajari juga berlaku untuk Yacc.

Gambar \ref{fig:bison-overview} menunjukkan alur kerja Bison parser generator.

\begin{figure}[!htbp]
    \centering
    \adjustbox{max width=0.9\textwidth,center}{%
    \begin{tikzpicture}[
        file/.style={rectangle, draw=blue!50, fill=blue!10, text width=2.5cm, text centered, minimum height=0.7cm, rounded corners, font=\footnotesize, inner sep=4pt, align=center},
        process/.style={rectangle, draw=green!50, fill=green!10, text width=2.5cm, text centered, minimum height=0.7cm, rounded corners, font=\footnotesize, inner sep=4pt, align=center},
        arrow/.style={->, >=stealth, thick},
        node distance=1.2cm
    ]
    
    \node[file] (spec) {parser.y\\Grammar};
    \node[process, right=of spec] (bison) {bison\\Generator};
    \node[file, right=of bison] (code) {parser.tab.c\\Generated};
    \node[process, below=of code] (gcc) {gcc\\Compiler};
    \node[file, left=of gcc] (exe) {parser\\Executable};
    
    \draw[arrow] (spec) -- node[above, font=\tiny] {Input} (bison);
    \draw[arrow] (bison) -- node[above, font=\tiny] {Generate} (code);
    \draw[arrow] (code) -- node[right, font=\tiny] {Compile} (gcc);
    \draw[arrow] (gcc) -- node[below, font=\tiny] {Link} (exe);
    
    \end{tikzpicture}%
    }
    \caption{Workflow Bison parser generator}
    \label{fig:bison-overview}
\end{figure}
\section{Struktur File Grammar Bison}

File grammar Bison memiliki ekstensi \texttt{.y} dan terdiri dari tiga bagian utama yang dipisahkan oleh \texttt{\%\%}:

\subsection{Bagian 1: Definitions (Prologue)}

Bagian ini berisi:
\begin{itemize}
    \item Deklarasi token (\texttt{\%token})
    \item Deklarasi tipe semantic value (\texttt{\%union}, \texttt{\%type})
    \item Precedence dan associativity (\texttt{\%left}, \texttt{\%right}, \texttt{\%nonassoc})
    \item Kode C yang akan disisipkan (\texttt{\%\{ ... \%\}})
    \item Deklarasi start symbol (\texttt{\%start})
\end{itemize}

\subsection{Bagian 2: Grammar Rules}

Bagian ini berisi aturan-aturan grammar dengan semantic actions. Formatnya:

\begin{verbatim}
nonterminal: production1 { action1 }
           | production2 { action2 }
           | ...
           ;
\end{verbatim}

Gambar \ref{fig:bison-structure} menunjukkan struktur file grammar Bison.

\begin{figure}[!htbp]
    \centering
    \adjustbox{max width=0.85\textwidth,center}{%
    \begin{tikzpicture}[
        section/.style={rectangle, draw=blue!50, fill=blue!10, text width=6cm, minimum height=1cm, font=\footnotesize, align=left, inner sep=8pt, rounded corners},
        separator/.style={rectangle, draw=red!50, fill=red!10, text width=6cm, minimum height=0.3cm, font=\small\bfseries, align=center},
        arrow/.style={->, >=stealth, thick},
        node distance=0.4cm
    ]
    
    \node[section] (def) {Definitions Section\\\texttt{\%token}, \texttt{\%union}\\\texttt{\%left}, \texttt{\%right}\\\texttt{\%\{ \%\}} C code};
    \node[separator, below=of def] (sep1) {\%\%};
    \node[section, below=of sep1] (rules) {Grammar Rules Section\\Production rules\\Semantic actions};
    \node[separator, below=of rules] (sep2) {\%\%};
    \node[section, below=of sep2] (user) {Auxiliary Code Section\\\texttt{main()}, \texttt{yyerror()}\\Helper functions};
    
    \draw[arrow] (def) -- (sep1);
    \draw[arrow] (sep1) -- (rules);
    \draw[arrow] (rules) -- (sep2);
    \draw[arrow] (sep2) -- (user);
    
    \end{tikzpicture}%
    }
    \caption{Struktur file grammar Bison}
    \label{fig:bison-structure}
\end{figure}

\subsection{Bagian 3: Auxiliary Code}

Bagian ini berisi kode C tambahan seperti:
\begin{itemize}
    \item Fungsi \texttt{main()}
    \item Fungsi \texttt{yyerror()} untuk error handling
    \item Fungsi pendukung lainnya
\end{itemize}

Gambar \ref{fig:bison-workflow-complete} menunjukkan workflow lengkap dari specification hingga executable.

\begin{figure}[!htbp]
    \centering
    \adjustbox{max width=0.9\textwidth,center}{%
    \begin{tikzpicture}[
        file/.style={rectangle, draw=blue!50, fill=blue!10, text width=2.5cm, text centered, minimum height=0.7cm, rounded corners, font=\footnotesize, inner sep=4pt, align=center},
        process/.style={rectangle, draw=green!50, fill=green!10, text width=2.5cm, text centered, minimum height=0.7cm, rounded corners, font=\footnotesize, inner sep=4pt, align=center},
        arrow/.style={->, >=stealth, thick},
        node distance=1cm
    ]
    
    \node[file] (y) {parser.y};
    \node[file, below=of y] (l) {lexer.l};
    \node[process, right=of y] (bison) {bison};
    \node[process, right=of l] (flex) {flex};
    \node[file, right=of bison] (tab-c) {parser.tab.c};
    \node[file, right=of flex] (yy-c) {lex.yy.c};
    \node[process, below=of tab-c] (gcc) {gcc};
    \node[file, below=of gcc] (exe) {program};
    
    \draw[arrow] (y) -- (bison);
    \draw[arrow] (l) -- (flex);
    \draw[arrow] (bison) -- (tab-c);
    \draw[arrow] (flex) -- (yy-c);
    \draw[arrow] (tab-c) -- (gcc.north west);
    \draw[arrow] (yy-c) -- (gcc.north east);
    \draw[arrow] (gcc) -- (exe);
    
    \end{tikzpicture}%
    }
    \caption{Workflow lengkap Flex + Bison}
    \label{fig:bison-workflow-complete}
\end{figure}
\section{Semantic Values dan Semantic Actions}

\subsection{Konsep Semantic Values}

Setiap token (terminal) dan nonterminal dalam grammar dapat membawa \textbf{semantic value}—sebuah nilai data yang diasosiasikan dengan simbol tersebut. Semantic value dapat berupa:
\begin{itemize}
    \item Integer (untuk literal angka)
    \item String (untuk identifier)
    \item Pointer ke AST node
    \item Tipe data kompleks lainnya
\end{itemize}

Dalam Bison:
\begin{itemize}
    \item \texttt{\$\$} merujuk pada semantic value dari \textbf{left-hand side (LHS)} (hasil produksi)
    \item \texttt{\$1}, \texttt{\$2}, ..., \texttt{\$n} merujuk pada semantic value dari simbol ke-1, ke-2, ..., ke-n di \textbf{right-hand side (RHS)}
\end{itemize}

\subsection{Semantic Actions}

\textbf{Semantic actions} adalah blok kode C yang dieksekusi ketika sebuah produksi di-reduce. Actions dapat digunakan untuk:
\begin{itemize}
    \item Membangun AST
    \item Mengevaluasi ekspresi
    \item Memvalidasi semantik
    \item Menghasilkan output
\end{itemize}

Contoh sederhana:

\begin{lstlisting}[language=C, caption={Contoh semantic action sederhana}]
expr: expr '+' expr { $$ = $1 + $3; }
    | expr '-' expr { $$ = $1 - $3; }
    | NUMBER        { $$ = $1; }
    ;
\end{lstlisting}

Pada contoh di atas, ketika parser menemukan \texttt{expr '+' expr}, action \texttt{\$\$ = \$1 + \$3} akan dieksekusi untuk menghitung jumlah dari kedua ekspresi.

Gambar \ref{fig:semantic-actions} menunjukkan bagaimana semantic actions bekerja dalam Bison.

\begin{figure}[H]
    \centering
    \adjustbox{max width=0.85\textwidth,center}{%
    \begin{tikzpicture}[
        rule/.style={rectangle, draw=blue!50, fill=blue!10, text width=4cm, minimum height=0.6cm, font=\tiny\ttfamily, align=left, inner sep=4pt, rounded corners},
        action/.style={rectangle, draw=green!50, fill=green!10, text width=4cm, minimum height=0.6cm, font=\tiny\ttfamily, align=left, inner sep=4pt, rounded corners},
        arrow/.style={->, >=stealth, thick},
        node distance=0.4cm
    ]
    
    \node[rule] (r1) {expr: expr '+' expr};
    \node[action, below=of r1] (a1) {\{ \$\$ = \$1 + \$3; \}};
    
    \draw[arrow] (r1) -- node[right, font=\tiny] {Action} (a1);
    
    \node[below=0.3cm of a1, font=\tiny, align=center] {\$\$ = LHS, \$1, \$2, \$3 = RHS};
    
    \end{tikzpicture}%
    }
    \caption{Semantic actions dalam Bison}
    \label{fig:semantic-actions}
\end{figure}
\section{Typing Semantic Values dengan \%union}

Untuk mendukung lebih dari satu tipe semantic value, kita menggunakan \texttt{\%union}:

\begin{lstlisting}[language=C, caption={Deklarasi \%union untuk berbagai tipe}]
%union {
    int intval;           // Untuk integer literals
    double dval;          // Untuk floating point
    char *strval;         // Untuk string/identifier
    ASTNode *astnode;     // Untuk AST nodes
}
\end{lstlisting}

Kemudian kita mendeklarasikan token dan nonterminal dengan tipe tertentu:

\begin{verbatim}
%token <intval> NUMBER
%token <strval> IDENTIFIER
%type <astnode> expr statement
\end{verbatim}
\section{Contoh Lengkap: Calculator dengan Bison}

Mari kita buat contoh lengkap parser untuk calculator sederhana menggunakan Flex dan Bison.

\subsection{File Lexer (calc.l)}

\begin{lstlisting}[language=C, caption={File lexer untuk calculator}]
%{
#include "calc.tab.h"  // Header yang dihasilkan Bison
#include <stdlib.h>
%}

%%
[0-9]+          { yylval.intval = atoi(yytext); return NUMBER; }
[0-9]+\.[0-9]+  { yylval.dval = atof(yytext); return FLOAT; }
[ \t\n]         { /* skip whitespace */ }
"+"             { return PLUS; }
"-"             { return MINUS; }
"*"             { return MULTIPLY; }
"/"             { return DIVIDE; }
"("             { return LPAREN; }
")"             { return RPAREN; }
.               { return yytext[0]; }
%%

int yywrap() { return 1; }
\end{lstlisting}

\subsection{File Parser (calc.y)}

\begin{lstlisting}[language=C, caption={File parser untuk calculator}]
%{
#include <stdio.h>
#include <stdlib.h>

extern int yylex();
extern int yyparse();
void yyerror(const char *s);
%}

%union {
    int intval;
    double dval;
}

%token <intval> NUMBER
%token <dval> FLOAT
%token PLUS MINUS MULTIPLY DIVIDE LPAREN RPAREN

%left PLUS MINUS
%left MULTIPLY DIVIDE

%type <dval> expr

%%
input:
    /* empty */
  | input expr '\n' { printf("Result = %g\n", $2); }
  ;

expr:
    NUMBER        { $$ = (double)$1; }
  | FLOAT         { $$ = $1; }
  | expr '+' expr { $$ = $1 + $3; }
  | expr '-' expr { $$ = $1 - $3; }
  | expr '*' expr { $$ = $1 * $3; }
  | expr '/' expr {
        if ($3 == 0.0) {
          yyerror("Division by zero");
          $$ = 0.0;
        } else {
          $$ = $1 / $3;
        }
      }
  | '(' expr ')'  { $$ = $2; }
  ;

%%

void yyerror(const char *s) {
    fprintf(stderr, "Error: %s\n", s);
}

int main() {
    printf("Calculator - Enter expressions (Ctrl+D to exit)\n");
    yyparse();
    return 0;
}
\end{lstlisting}

\subsection{Kompilasi dan Eksekusi}

Untuk mengompilasi program:

\begin{verbatim}
flex calc.l          # Generate lexer
bison -d calc.y      # Generate parser (d flag untuk header)
gcc lex.yy.c calc.tab.c -o calc -lfl
./calc
\end{verbatim}
\section{Integrasi Flex dan Bison}

\subsection{Interface antara Lexer dan Parser}

Integrasi antara Flex lexer dan Bison parser dilakukan melalui:

\begin{enumerate}
    \item \textbf{Token Definitions}: Bison menghasilkan file header (misalnya \texttt{calc.tab.h}) yang berisi definisi token. Lexer meng-include header ini.
    
    \item \textbf{Token Values}: Lexer mengisi \texttt{yylval} dengan semantic value sebelum mengembalikan token.
    
    \item \textbf{Function Calls}: Parser memanggil \texttt{yylex()} untuk mendapatkan token berikutnya.
\end{enumerate}

\subsection{Contoh Integrasi}

\begin{lstlisting}[language=C, caption={Contoh integrasi Flex dan Bison}]
// Dalam calc.l
%{
#include "calc.tab.h"  // Include header dari Bison
%}

// Dalam calc.y
%{
extern int yylex();    // Deklarasi fungsi lexer
%}
\end{lstlisting}
\section{Semantic Actions untuk Membangun AST}

Salah satu penggunaan utama semantic actions adalah untuk membangun \textbf{Abstract Syntax Tree (AST)}. Mari kita lihat contoh yang lebih kompleks.

\subsection{Struktur AST Node}

Pertama, kita definisikan struktur untuk AST node:

\begin{lstlisting}[language=C, caption={Definisi AST node}]
typedef enum {
    NODE_NUMBER,
    NODE_ADD,
    NODE_SUB,
    NODE_MUL,
    NODE_DIV
} NodeType;

typedef struct ASTNode {
    NodeType type;
    double value;  // Untuk NODE_NUMBER
    struct ASTNode *left;   // Untuk operasi binary
    struct ASTNode *right;
} ASTNode;

ASTNode* create_number_node(double value) {
    ASTNode *node = malloc(sizeof(ASTNode));
    node->type = NODE_NUMBER;
    node->value = value;
    node->left = node->right = NULL;
    return node;
}

ASTNode* create_binary_node(NodeType type, ASTNode *left, ASTNode *right) {
    ASTNode *node = malloc(sizeof(ASTNode));
    node->type = type;
    node->left = left;
    node->right = right;
    return node;
}
\end{lstlisting}

\subsection{Grammar dengan AST Building}

\begin{lstlisting}[language=C, caption={Grammar dengan semantic actions untuk AST}]
%union {
    double dval;
    ASTNode *astnode;
}

%token <dval> NUMBER
%type <astnode> expr

%%
expr:
    NUMBER { $$ = create_number_node($1); }
  | expr '+' expr { $$ = create_binary_node(NODE_ADD, $1, $3); }
  | expr '-' expr { $$ = create_binary_node(NODE_SUB, $1, $3); }
  | expr '*' expr { $$ = create_binary_node(NODE_MUL, $1, $3); }
  | expr '/' expr { $$ = create_binary_node(NODE_DIV, $1, $3); }
  | '(' expr ')' { $$ = $2; }
  ;
\end{lstlisting}
\section{Precedence dan Associativity}

Untuk menangani precedence dan associativity tanpa membuat grammar yang ambigu, Bison menyediakan deklarasi khusus:

\subsection{Deklarasi Precedence}

\begin{verbatim}
%left '+' '-'      // Left-associative, precedence rendah
%left '*' '/'      // Left-associative, precedence tinggi
%right '^'         // Right-associative (exponentiation)
%nonassoc '<' '>'  // Non-associative (comparison)
\end{verbatim}

Urutan deklarasi menentukan precedence: yang dideklarasikan terakhir memiliki precedence tertinggi.

\subsection{Contoh Penggunaan}

\begin{lstlisting}[language=C, caption={Grammar dengan precedence}]
%left PLUS MINUS
%left MULTIPLY DIVIDE
%right POWER

%%
expr:
    NUMBER
  | expr PLUS expr    // Precedence rendah
  | expr MINUS expr
  | expr MULTIPLY expr // Precedence tinggi
  | expr DIVIDE expr
  | expr POWER expr   // Precedence tertinggi, right-associative
  ;
\end{lstlisting}

Dengan deklarasi ini, ekspresi \texttt{2 + 3 * 4} akan di-parse sebagai \texttt{2 + (3 * 4)}, dan ekspresi \verb|2^3^4| akan di-parse sebagai \verb|2^(3^4)|.
\section{Error Handling dalam Bison}

\subsection{Error Token}

Bison menyediakan token khusus \texttt{error} untuk error recovery. Ketika parser menemukan error, ia dapat mencoba recovery dengan menggunakan produksi yang mengandung \texttt{error}.

\subsection{Contoh Error Recovery}

\begin{lstlisting}[language=C, caption={Error recovery dalam Bison}]
%%
input:
    /* empty */
  | input line
  ;

line:
    expr '\n' { printf("Result = %g\n", $1); }
  | error '\n' {
        yyerrok;  // Reset error state
        yyclearin; // Clear lookahead token
        printf("Syntax error, please try again.\n");
      }
  ;
\end{lstlisting}

\subsection{Fungsi yyerror}

Fungsi \texttt{yyerror()} dipanggil ketika terjadi syntax error. Kita dapat mengimplementasikannya untuk memberikan pesan error yang informatif:

\begin{lstlisting}[language=C, caption={Implementasi yyerror yang lebih informatif}]
void yyerror(const char *s) {
    extern int yylineno;
    fprintf(stderr, "Error at line %d: %s\n", yylineno, s);
}
\end{lstlisting}

\subsection{Location Tracking}

Untuk memberikan informasi lokasi yang lebih akurat, kita dapat menggunakan location tracking:

\begin{lstlisting}[language=C, caption={Location tracking dalam Bison}]
%locations

%%
expr: expr '+' expr {
    @$.first_line = @1.first_line;
    @$.first_column = @1.first_column;
    @$.last_line = @3.last_line;
    @$.last_column = @3.last_column;
    if (/* some error condition */) {
        yyerror("Error in expression");
    }
    $$ = $1 + $3;
}
\end{lstlisting}
\section{Mid-Rule Actions}

Bison mendukung \textbf{mid-rule actions}—actions yang muncul di tengah-tengah produksi, sebelum seluruh RHS di-match. Ini berguna untuk:

\begin{itemize}
    \item Validasi intermediate state
    \item Inisialisasi variabel
    \item Side effects tertentu
\end{itemize}

\begin{lstlisting}[language=C, caption={Contoh mid-rule action}]
stmt:
    IDENTIFIER { printf("Variable: %s\n", $1); }
    '=' 
    { /* mid-rule action */ 
      check_variable($1); 
    }
    expr { assign($1, $4); }
  ;
\end{lstlisting}

Catatan penting: Hanya action terakhir yang menentukan \texttt{\$\$} untuk LHS.
\section{Contoh Praktis: Parser untuk Subset C}

Mari kita buat parser yang lebih kompleks untuk subset bahasa C yang mendukung:
\begin{itemize}
    \item Deklarasi variabel (int, float)
    \item Assignment statements
    \item Ekspresi aritmatika
    \item Print statements
\end{itemize}

\subsection{File Lexer (simplec.l)}

\begin{lstlisting}[language=C, caption={Lexer untuk subset C}]
%{
#include "simplec.tab.h"
#include <string.h>
%}

%%
"int"       { return INT; }
"float"     { return FLOAT; }
"print"     { return PRINT; }
[a-zA-Z_][a-zA-Z0-9_]* { 
    yylval.strval = strdup(yytext); 
    return IDENTIFIER; 
}
[0-9]+      { yylval.intval = atoi(yytext); return NUMBER; }
[0-9]+\.[0-9]+ { yylval.dval = atof(yytext); return FLOAT_LITERAL; }
[ \t\n]     { /* skip */ }
"="         { return ASSIGN; }
";"         { return SEMICOLON; }
"+"         { return PLUS; }
"-"         { return MINUS; }
"*"         { return MULTIPLY; }
"/"         { return DIVIDE; }
"("         { return LPAREN; }
")"         { return RPAREN; }
.           { return yytext[0]; }
%%

int yywrap() { return 1; }
\end{lstlisting}

\subsection{File Parser (simplec.y)}

\begin{lstlisting}[language=C, caption={Parser untuk subset C}]
%{
#include <stdio.h>
#include <stdlib.h>
#include <string.h>

extern int yylex();
void yyerror(const char *s);

// Symbol table sederhana
typedef struct {
    char *name;
    int type;  // 0 = int, 1 = float
    union {
        int intval;
        double dval;
    } value;
} Symbol;

Symbol symbols[100];
int sym_count = 0;
%}

%union {
    int intval;
    double dval;
    char *strval;
}

%token <intval> NUMBER
%token <dval> FLOAT_LITERAL
%token <strval> IDENTIFIER
%token INT FLOAT PRINT ASSIGN SEMICOLON
%token PLUS MINUS MULTIPLY DIVIDE LPAREN RPAREN

%left PLUS MINUS
%left MULTIPLY DIVIDE

%type <dval> expr

%%
program:
    /* empty */
  | program statement
  ;

statement:
    declaration SEMICOLON
  | assignment SEMICOLON
  | print_stmt SEMICOLON
  ;

declaration:
    INT IDENTIFIER { 
        // Add to symbol table
        symbols[sym_count].name = strdup($2);
        symbols[sym_count].type = 0;
        sym_count++;
        printf("Declared int variable: %s\n", $2);
    }
  | FLOAT IDENTIFIER {
        symbols[sym_count].name = strdup($2);
        symbols[sym_count].type = 1;
        sym_count++;
        printf("Declared float variable: %s\n", $2);
    }
  ;

assignment:
    IDENTIFIER ASSIGN expr {
        // Find variable in symbol table and assign
        printf("Assigning %g to %s\n", $3, $1);
    }
  ;

print_stmt:
    PRINT expr {
        printf("Print: %g\n", $2);
    }
  ;

expr:
    NUMBER { $$ = (double)$1; }
  | FLOAT_LITERAL { $$ = $1; }
  | IDENTIFIER { 
        // Lookup in symbol table
        $$ = 0.0; // Simplified
    }
  | expr PLUS expr { $$ = $1 + $3; }
  | expr MINUS expr { $$ = $1 - $3; }
  | expr MULTIPLY expr { $$ = $1 * $3; }
  | expr DIVIDE expr { 
        if ($3 == 0.0) {
            yyerror("Division by zero");
            $$ = 0.0;
        } else {
            $$ = $1 / $3;
        }
    }
  | LPAREN expr RPAREN { $$ = $2; }
  ;

%%

void yyerror(const char *s) {
    fprintf(stderr, "Syntax error: %s\n", s);
}

int main() {
    printf("Simple C Parser - Enter statements (Ctrl+D to exit)\n");
    yyparse();
    return 0;
}
\end{lstlisting}
\section{Best Practices dan Tips}

\subsection{Organisasi File}

\begin{itemize}
    \item Pisahkan definisi AST node ke file header terpisah
    \item Gunakan Makefile untuk mengotomatisasi kompilasi
    \item Dokumentasikan grammar dengan baik
\end{itemize}

\subsection{Debugging}

\begin{itemize}
    \item Gunakan flag \texttt{-v} pada Bison untuk menghasilkan file \texttt{.output} yang berisi informasi tentang parsing table
    \item Gunakan \texttt{YYDEBUG} untuk debugging parser
    \item Tambahkan print statements dalam semantic actions untuk tracing
\end{itemize}

\subsection{Performance}

\begin{itemize}
    \item Hindari semantic actions yang terlalu kompleks
    \item Gunakan \texttt{\%destructor} untuk membersihkan memory jika menggunakan pointer
    \item Pertimbangkan menggunakan GLR parsing untuk grammar yang ambigu
\end{itemize}
\section{Kesimpulan}

Dalam bab ini, kita telah mempelajari:

\begin{enumerate}
    \item Konsep parser generator dan keuntungannya
    \item Struktur file grammar Bison dengan tiga bagian utama
    \item Semantic values dan semantic actions untuk membangun AST
    \item Integrasi Flex lexer dengan Bison parser
    \item Precedence dan associativity dalam grammar
    \item Error handling dan recovery dalam Bison
    \item Contoh praktis parser untuk calculator dan subset C
\end{enumerate}

Parser generator seperti Bison sangat powerful untuk membangun parser yang robust dan maintainable. Dengan memahami konsep-konsep ini, kita dapat membangun parser untuk bahasa yang lebih kompleks.
\section{Referensi dan Bahan Bacaan Lanjutan}

Untuk memperdalam pemahaman tentang parser generator, mahasiswa disarankan membaca:

\begin{itemize}
    \item \textbf{flex \& bison}: Levine, J. R. (2009). \textit{flex \& bison: Text Processing Tools} \cite{levine2009flex} - Buku lengkap tentang Flex dan Bison
    
    \item \textbf{Dragon Book}: Aho, Lam, Sethi, \& Ullman (2006). \textit{Compilers: Principles, Techniques, and Tools} \cite{aho2006compilers} - Bab 4: Syntax-Directed Translation
    
    \item \textbf{GNU Bison Manual}: \url{https://www.gnu.org/software/bison/manual/} - Dokumentasi resmi Bison dengan contoh lengkap
    
    \item \textbf{IT Trip Tutorial}: Tutorial tentang C Parser dengan Flex dan Bison \cite{ittrip2024bison}
    
    \item \textbf{Engineering a Compiler}: Cooper \& Torczon (2011) \cite{cooper2011engineering} - Bab tentang parser generators
\end{itemize}

% Daftar pustaka (hanya muncul saat kompilasi standalone dan hanya jika ada citation)
\ifSubfilesClassLoaded{%
    \clearpage
    \printbibliography[title={Daftar Pustaka}]
    \end{refsection}
}{}

\end{document}
