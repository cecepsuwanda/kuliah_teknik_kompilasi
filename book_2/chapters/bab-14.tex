\documentclass[../main.tex]{subfiles}

\addbibresource{\subfix{../references.bib}}

\begin{document}

\ifSubfilesClassLoaded{%
    \setcounter{chapter}{13}%
    \begin{refsection}
}{}

\chapter{Activation Records dan Stack Management}
\label{chap:activation-records}

\begin{subcpmk}
  \item \textbf{Sub-CPMK 5.3:} Mengimplementasikan activation records untuk procedure calls
\end{subcpmk}

% ============================================================
% MATERI POKOK
% ============================================================
\section{Pengenalan Activation Records}

\subsection{Definisi Activation Record}

\compiler{Activation Record} (stack frame) adalah struktur data yang menyimpan:

\begin{itemize}
  \item Local variables
  \item Parameters
  \item Return address
  \item Saved registers
  \item Dynamic link (pointer ke caller frame)
  \item Static link (pointer ke enclosing scope)
\end{itemize}

\subsection{Stack Frame Layout}

\begin{lstlisting}[language=C]
// Typical stack frame layout (grows downward)
+-------------------+  // Higher addresses
| Parameters        |  (from caller)
+-------------------+
| Return Address     |  (pushed by CALL)
+-------------------+
| Dynamic Link       |  (old frame pointer)
+-------------------+
| Saved Registers    |  (caller-saved)
+-------------------+
| Local Variables    |  (allocated by callee)
+-------------------+
| Temporaries        |  (temporary storage)
+-------------------+  // Lower addresses
\end{lstlisting}

\section{Procedure Call Mechanism}

\subsection{Calling Convention}

Prosedur call untuk function \code{func(a, b, c)}:

\begin{lstlisting}[language=C]
// Caller side:
push c              // Push parameters (right-to-left)
push b
push a
call func           // Push return address and jump

// Callee side (func entry):
push ebp            // Save old frame pointer
mov ebp, esp         // Set new frame pointer
sub esp, local_size // Allocate space for locals

// Function body here...

// Callee side (func exit):
mov esp, ebp         // Deallocate locals
pop ebp             // Restore old frame pointer
ret                 // Pop return address and jump
\end{lstlisting}

\subsection{Parameter Passing}

\begin{table}[h]
\centering
\begin{tabular}{|l|l|l|}
\hline
\textbf{Method} & \textbf{Pros} & \textbf{Cons} \\
\hline
Stack & Simple, unlimited parameters & Slow, memory access \\
Register & Fast, no memory access & Limited registers \\
Mixed & Balance of speed & Complex \\
\hline
\end{tabular}
\caption{Parameter Passing Methods}
\end{table}

\section{Stack Management}

\subsection{Stack Operations}

\begin{lstlisting}[language=C]
typedef struct {
    void *base;        // Stack base
    void *top;         // Stack top
    size_t size;       // Current size
    size_t capacity;   // Maximum size
} Stack;

void push(Stack *stack, void *data, size_t data_size) {
    if (stack->size + data_size > stack->capacity) {
        stack_overflow_error();
    }
    
    memcpy(stack->top, data, data_size);
    stack->top = (char*)stack->top + data_size;
    stack->size += data_size;
}

void* pop(Stack *stack, size_t data_size) {
    if (stack->size < data_size) {
        stack_underflow_error();
    }
    
    stack->top = (char*)stack->top - data_size;
    stack->size -= data_size;
    return stack->top;
}
\end{lstlisting}

\subsection{Frame Pointer vs Stack Pointer}

\begin{lstlisting}[language=C]
// Using frame pointer (EBP/RBP)
int access_local(int offset) {
    // Access local at [EBP - offset]
    return *(int*)((char*)EBP - offset);
}

// Using stack pointer only (ESP/RSP)
int access_local_no_fp(int offset_from_sp) {
    // Access local at [ESP + offset]
    return *(int*)((char*)ESP + offset_from_sp);
}
\end{lstlisting}

\section{Fungsi Bersarang (Nested Functions)}

Untuk bahasa seperti Pascal, Ada, atau ekstensi GCC pada C, fungsi dapat dideklarasikan di dalam fungsi lain. Hal ini memerlukan mekanisme untuk mengakses variabel dari lingkup (\textit{scope}) luar.

\subsection{Static Links (Access Links)}
Setiap \textit{activation record} menyimpan pointer ke \textit{frame} fungsi yang secara leksikal membungkusnya.
\begin{itemize}
    \item \textbf{Mekanisme}: Untuk mengakses variabel di level $N$ tingkat ke atas, kompilator menelusuri rantai \textit{static link} sebanyak $N$ kali.
    \item \textbf{Performa}: Semakin dalam penyarangan fungsi, semakin banyak dereferensi memori yang diperlukan.
\end{itemize}

\subsection{Display (Tabel Array)}
Alternatif performa tinggi menggunakan array global atau register khusus untuk menyimpan pointer ke \textit{activation record} aktif di setiap level penyarangan.
\begin{itemize}
    \item \textbf{Mekanisme}: Akses variabel di level $L$ dilakukan dengan indeks langsung: \code{Display[L] + offset}.
    \item \textbf{Performa}: Akses sangat cepat (O(1)), namun memiliki overhead lebih besar saat masuk/keluar fungsi karena harus memperbarui entri tabel \textit{Display}.
\end{itemize}

\subsection{Lambda Lifting}
Banyak kompilator modern (terutama untuk bahasa fungsional) menghindari penyarangan fisik dengan teknik \compiler{Lambda Lifting}: variabel lingkup luar yang digunakan oleh fungsi dalam diubah menjadi parameter eksplisit tambahan.

\begin{table}[h]
\centering
\begin{tabular}{|l|l|l|}
\hline
\textbf{Fitur} & \textbf{Static Links} & \textbf{Display} \\
\hline
Akses Non-Lokal & O(N) traversal & O(1) direct index \\
Overhead Call & Rendah (set 1 pointer) & Tinggi (update tabel) \\
Implementasi & Sederhana & Kompleks \\
\hline
\end{tabular}
\caption{Perbandingan Mekanisme Akses Lingkup Luar}
\end{table}

\section{Penanganan Eksepsi (Exception Handling)}

Saat terjadi kesalahan (\textit{error/exception}), program harus dapat "melompat" keluar dari banyak fungsi ke \textit{handler} yang sesuai tanpa merusak struktur data stack.

\subsection{Stack Unwinding}
Proses membersihkan \textit{activation records} satu per satu dari atas ke bawah untuk mencari penangkap (\textit{catch block}) yang cocok disebut \compiler{Stack Unwinding}.

\subsection{Metadata DWARF dan CFI}
Kompilator modern seringkali tidak menggunakan \textit{Frame Pointer} (\code{EBP}) untuk menghemat satu register. Tanpa \code{EBP}, bagaimana kita bisa menelusuri stack?
\begin{itemize}
    \item \textbf{CFI (Call Frame Information)}: Kompilator menyisipkan tabel metadata di bagian khusus berkas biner (\code{.eh\_frame} atau \code{.debug\_frame}).
    \item \textbf{Aturan Unwinding}: Metadata ini memberikan instruksi kepada \textit{runtime library} (seperti \code{libgcc}) tentang bagaimana cara menghitung alamat kembalian dan mengembalikan register hanya dengan menggunakan informasi lokasi kode (\code{Program Counter}).
\end{itemize}

\begin{figure}[!htbp]
    \centering
    \adjustbox{max width=0.8\textwidth,center}{%
    \begin{tikzpicture}[
        node/.style={rectangle, draw=purple!50, fill=purple!10, text width=6cm, font=\tiny, align=center}
    ]
    \node[node] (bin) {Berkas Biner (.exe / .elf)};
    \node[node, below=0.2cm of bin] (eh) {Seksi .eh\_frame (Tabel Aturan DWARF)};
    \node[node, below=0.2cm of eh] (unw) {Unwinder: Membaca tabel saat EXCEPTION terjadi.};
    \draw[->] (bin) -- (eh);
    \draw[->] (eh) -- (unw);
    \end{tikzpicture}%
    }
    \caption{Penanganan Eksepsi Tanpa Frame Pointer menggunakan DWARF}
\end{figure}

\section{Teknik Optimasi Stack}

Optimasi pada tingkat manajemen stack sangat krusial karena setiap panggilan fungsi memiliki biaya overhead (\textit{prologue} dan \textit{epilogue}).

\subsection{Optimasi Fungsi Daun (Leaf Function)}
Fungsi daun (\textit{leaf function}) adalah fungsi yang tidak memanggil fungsi lain.
\begin{itemize}
    \item \textbf{Optimasi}: Kompilator seringkali tidak membuat \textit{activation record} sama sekali jika variabel lokal dapat ditampung sepenuhnya dalam register. Instruksi \code{push} dan \code{pop} dihilangkan sepenuhnya.
\end{itemize}

\subsection{Tail Call vs Tail Recursion}
Sebuah \textit{panggilan ekor} (\cite{jhu2024compilers}) terjadi jika sebuah fungsi memanggil fungsi lain sebagai tindakan terakhirnya.
\begin{itemize}
    \item \textbf{Tail Recursion}: Fungsi memanggil dirinya sendiri di akhir. Kompilator mengubah rekursi menjadi \textit{loop} biasa (melompat ke awal fungsi).
    \item \textbf{Tail Call}: Fungsi memanggil fungsi BERBEDA di akhir. Kompilator dapat menghapus \textit{frame} saat ini sebelum melakukan \code{JUMP} (bukan \code{CALL}) ke fungsi target. Hal ini mencegah pertumbuhan stack yang tidak perlu.
\end{itemize}

\begin{figure}[!htbp]
    \centering
    \adjustbox{max width=0.8\textwidth,center}{%
    \begin{tikzpicture}[
        node/.style={rectangle, draw=red!50, fill=red!10, text width=6cm, font=\tiny, align=center}
    ]
    \node[node] (std) {Standard Call: PUSH Args $\rightarrow$ CALL f $\rightarrow$ RET};
    \node[node, below=0.2cm of std] (opt) {Tail Call Opt: POP Frame $\rightarrow$ JUMP f};
    \end{tikzpicture}%
    }
    \caption{Efisiensi Tail Call Optimization}
\end{figure}

\section{Variable Length Arrays}

\subsection{VLA on Stack}

\begin{lstlisting}[language=C]
void function_with_vla(int n) {
    int vla[n];  // Variable length array
    
    // Stack layout after VLA allocation:
    // +-------------------+
    // | VLA (n * sizeof(int)) |
    // +-------------------+
    // | Other locals       |
    // +-------------------+
    // | Frame pointer      |
    // +-------------------+
    
    // Access VLA
    for (int i = 0; i < n; i++) {
        vla[i] = i * 2;
    }
}
\end{lstlisting}

\subsection{Alloca Implementation}

\begin{lstlisting}[language=C]
void* alloca(size_t size) {
    void *result = (char*)current_frame_pointer - size;
    
    // Check for stack overflow
    if (result < stack_limit) {
        stack_overflow_error();
    }
    
    // Adjust stack pointer
    current_stack_pointer = result;
    return result;
}
\end{lstlisting}


% ============================================================
% AKTIVITAS PEMBELAJARAN
% ============================================================
\begin{aktivitas}
  \item \textbf{Stack Frame}: Implementasikan stack frame management system.
  \item \textbf{Calling Convention}: Bangun procedure call mechanism dengan berbagai conventions.
  \item \textbf{Nested Functions}: Implementasikan static links untuk nested functions.
  \item \textbf{Exception Handling}: Buat exception handling dengan stack unwinding.
  \item \textbf{Optimization}: Implementasikan leaf function dan tail call optimizations.
\end{aktivitas}

% ============================================================
% LATIHAN DAN REFLEKSI
% ============================================================
\begin{latihan}
  \item Gambarkan stack frame layout untuk function dengan multiple parameters dan locals!
  \item Implementasikan calling convention untuk variadic functions!
  \item Analisis overhead dari frame pointer vs stack pointer only!
  \item Implementasikan exception handling dengan proper cleanup!
  \item Optimasi function calls dengan tail recursion elimination!
  \item \textbf{Refleksi}: Bagaimana activation records mempengaruhi performance function calls?
\end{latihan}

% ============================================================
% ASESMEN
% ============================================================
\begin{asesmen}
\textbf{Instrumen Penilaian untuk Sub-CPMK 5.3}

\textbf{A. Pilihan Ganda}

\begin{enumerate}
  \item Dynamic link menunjuk ke:
  \begin{enumerate}
    \item Enclosing function frame
    \item Caller frame
    \item Global frame
    \item Stack base
  \end{enumerate}
  
  \item Tail call optimization menghilangkan:
  \begin{enumerate}
    \item Parameter passing
    \item Frame setup overhead
    \item Return value
    \item Function call
  \end{enumerate}
  
  \item Static link digunakan untuk:
  \begin{enumerate}
    \item Exception handling
    \item Nested functions
    \item Recursion
    \item Optimization
  \end{enumerate}
\end{enumerate}

\textbf{B. Essay}

\begin{enumerate}
  \item Jelaskan complete procedure call mechanism dengan activation records!
  \item Implementasikan stack management system dengan support untuk nested functions dan exception handling!
\end{enumerate}

\textbf{Rubrik Penilaian}: Lihat Lampiran A
\end{asesmen}

% ============================================================
% CHECKLIST KOMPETENSI
% ============================================================
\begin{checklist}
  \item Saya dapat mengimplementasikan activation records untuk procedure calls
  \item Saya dapat mengelola stack operations dan frame management
  \item Saya dapat mengimplementasikan berbagai calling conventions
  \item Saya dapat menangani nested functions dengan static links
  \item Saya dapat mengimplementasikan exception handling
  \item Saya dapat melakukan stack-related optimizations
\end{checklist}

% ============================================================
% RANGKUMAN
% ============================================================
\begin{rangkuman}
Bab ini membahas activation records dan stack management, termasuk stack frame layout, procedure call mechanisms, nested functions, exception handling, and optimization techniques. Mahasiswa belajar mengimplementasikan efficient function call infrastructure.

\textbf{Poin Kunci:}
\begin{itemize}
  \item Activation records menyimpan state untuk function execution
  \item Stack management mengelola dynamic allocation/deallocation
  \item Calling conventions menentukan parameter passing dan register saving
  \item Static links enable access to enclosing function variables
  \item Exception handling requires proper stack unwinding
  \item Optimizations reduce overhead of function calls
\end{itemize}

\textbf{Kata Kunci}: \compiler{Activation Record}, \compiler{Stack Frame}, \compiler{Calling Convention}, \compiler{Dynamic Link}, \compiler{Static Link}, \compiler{Tail Call Optimization}, \compiler{Exception Handling}
\end{rangkuman}

\ifSubfilesClassLoaded{%
    \clearpage
    \printbibliography[title={Daftar Pustaka}]
    \end{refsection}
}{}

\end{document}
