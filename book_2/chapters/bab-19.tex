\documentclass[../main.tex]{subfiles}

\addbibresource{\subfix{../references.bib}}

\begin{document}

\ifSubfilesClassLoaded{%
    \setcounter{chapter}{18}%
    \begin{refsection}
}{}

\chapter{Daftar Referensi}
\label{chap:referensi}

\begin{subcpmk}
  \item \textbf{Sub-CPMK 1.1-6.2:} Menunjukkan kemampuan literasi informasi dan penggunaan referensi ilmiah dalam bidang teknik kompilasi
\end{subcpmk}

% ============================================================
% MATERI POKOK
% ============================================================
\section{Buku Referensi Utama}

\subsection{Compiler Design Fundamentals}

\begin{itemize}
  \item \textbf{Compilers: Principles, Techniques, and Tools} (2nd Edition)\\
  Aho, Alfred V., Monica S. Lam, Ravi Sethi, and Jeffrey D. Ullman\\
  Pearson/Addison-Wesley, 2007\\
  ISBN: 978-0321486813

  \item \textbf{Modern Compiler Implementation in C}\\
  Appel, Andrew W.\\
  Cambridge University Press, 1998\\
  ISBN: 978-0521583909

  \item \textbf{Engineering a Compiler} (2nd Edition)\\
  Cooper, Keith D. and Linda Torczon\\
  Morgan Kaufmann, 2011\\
  ISBN: 978-0120884788

  \item \textbf{Programming Language Pragmatics} (4th Edition)\\
  Scott, Michael L.\\
  Morgan Kaufmann, 2015\\
  ISBN: 978-0124104099

  \item \textbf{Automatic Vectorization for Free}\\
  Nuzman, Dorit, et al.\\
  ACM SIGPLAN Notices, 2006

  \item \textbf{MLIR: A Compiler Infrastructure for the End of Moore's Law}\\
  Lattner, Chris, et al.\\
  arXiv preprint arXiv:2002.11054, 2020

  \item \textbf{Large Language Models for Compiler Optimization}\\
  Cummins, Chris, et al.\\
  arXiv preprint arXiv:2309.07062, 2023

  \item \textbf{Formal Verification of a Realistic Compiler}\\
  Leroy, Xavier\\
  Communications of the ACM, 2009

  \item \textbf{Otomata Bahasa dan Teknik Kompilasi}\\
  Vulandari, Retno Tri\\
  Penerbit Informatika, 2017\\
  ISBN: 978-6026232410
\end{itemize}

\subsection{Advanced Compiler Topics}

\begin{itemize}
  \item \textbf{Advanced Compiler Design and Implementation}\\
  Muchnick, Steven S.\\
  Morgan Kaufmann, 1997\\
  ISBN: 978-1558603204

  \item \textbf{Optimizing Compilers for Modern Architectures}\\
  Allen, Randy and Ken Kennedy\\
  Morgan Kaufmann, 2001\\
  ISBN: 978-1558602863

  \item \textbf{The Dragon Book: Compilers}\\
  Aho, Alfred V., Ravi Sethi, and Jeffrey D. Ullman\\
  Addison-Wesley, 1986\\
  ISBN: 978-0201100884
\end{itemize}

\section{Buku Praktis dan Implementasi}

\subsection{Lexical Analysis and Parsing}

\begin{itemize}
  \item \textbf{flex \& bison: Text Processing Tools}\\
  Levine, John R.\\
  O'Reilly Media, 2009\\
  ISBN: 978-0596155971

  \item \textbf{The Definitive ANTLR 4 Reference}\\
  Parr, Terence\\
  Pragmatic Bookshelf, 2013\\
  ISBN: 978-1934356599

  \item \textbf{Building a Parser with JavaCC}\\
  Grune, Dick and Ceriel J.H. Jacobs\\
  Addison-Wesley, 2008\\
  ISBN: 978-0321316324
\end{itemize}

\subsection{Code Generation and Optimization}

\begin{itemize}
  \item \textbf{The LLVM Compiler Infrastructure}\\
  Lattner, Chris and Vikram Adve\\
  ACM SIGPLAN Notices, 2004

  \item \textbf{Linkers and Loaders}\\
  Levine, John R.\\
  Morgan Kaufmann, 2000\\
  ISBN: 978-1558604964

  \item \textbf{Computer Systems: A Programmer's Perspective}\\
  Bryant, Randal E. and David R. O'Hallaron\\
  Pearson, 2015\\
  ISBN: 978-0134092669
\end{itemize}

\section{Jurnal dan Paper Akademik}

\subsection{Seminal Papers}

\begin{itemize}
  \item \textbf{A Fast Algorithm for Finding Dominators in a Flowgraph}\\
  Lengauer, Thomas and Robert E. Tarjan\\
  ACM Transactions on Programming Languages and Systems, 1979

  \item \textbf{Register Allocation via Graph Coloring}\\
  Chaitin, Gregory J.\\
  ACM SIGPLAN Notices, 1982

  \item \textbf{SSA-Based Optimizations}\\
  Cytron, Ron, et al.\\
  ACM Transactions on Programming Languages and Systems, 1991

  \item \textbf{Linear Scan Register Allocation}\\
  Poletto, Massimiliano and Vivek Sarkar\\
  ACM SIGPLAN Notices, 1999
\end{itemize}

\subsection{Recent Research}

\begin{itemize}
  \item \textbf{LLVM: A Compilation Framework for Lifelong Program Analysis \& Transformation}\\
  Lattner, Chris and Vikram Adve\\
  International Symposium on Code Generation and Optimization, 2004

  \item \textbf{The LLVM Instruction Set and Compilation Strategy}\\
  Lattner, Chris\\
  University of Illinois at Urbana-Champaign, 2002

  \item \textbf{Automatic Vectorization for Free}\\
  Nuzman, Dorit, et al.\\
  ACM SIGPLAN Notices, 2006
\end{itemize}

\section{Dokumentasi Teknis}

\subsection{Compiler Tools Documentation}

\begin{itemize}
  \item \textbf{GNU Compiler Collection Internals}\\
  Free Software Foundation\\
  \url{https://gcc.gnu.org/onlinedocs/gccint/}

  \item \textbf{LLVM Language Reference Manual}\\
  LLVM Project\\
  \url{https://llvm.org/docs/LangRef.html}

  \item \textbf{Flex Manual}\\
  Free Software Foundation\\
  \url{https://westes.github.io/flex/manual/}

  \item \textbf{Bison Manual}\\
  Free Software Foundation\\
  \url{https://www.gnu.org/software/bison/manual/}

  \item \textbf{Writing a Simple Compiler}\\
  Carl Burch\\
  \url{https://www.cburch.com/cs/330/}

  \item \textbf{Compiler Explorer}\\
  Matt Godbolt\\
  \url{https://godbolt.org/}

  \item \textbf{Wasmtime: A small and efficient runtime for WebAssembly}\\
  Bytecode Alliance\\
  \url{https://wasmtime.dev/}
\end{itemize}

\subsection{Architecture Specifications}

\begin{itemize}
  \item \textbf{x86-64 Architecture Programmer's Manual}\\
  Intel Corporation\\
  \url{https://software.intel.com/content/www/us/en/develop/articles/intel-sdm.html}

  \item \textbf{ARM Architecture Reference Manual}\\
  ARM Limited\\
  \url{https://developer.arm.com/documentation/ddi0487/latest}

  \item \textbf{RISC-V ISA Manual}\\
  RISC-V Foundation\\
  \url{https://riscv.org/technical/specifications/}

  \item \textbf{WebAssembly Core Specification}\\
  W3C Working Group - Living Standard\\
  \url{https://www.w3.org/TR/wasm-core/}

  \item \textbf{Language Server Protocol Specification}\\
  Microsoft\\
  \url{https://microsoft.github.io/language-server-protocol/specifications/lsp/3.17/specification/}
\end{itemize}

\section{Online Resources}

\subsection{Course Materials}

\begin{itemize}
  \item \textbf{CS4120/CS4121: Compilers}\\
  Cornell University\\
  \url{https://www.cs.cornell.edu/courses/cs4120/2018fa/}

  \item \textbf{CS143: Compilers}\\
  Stanford University\\
  \url{https://www.stanford.edu/class/cs143/}

  \item \textbf{15-213: Introduction to Computer Systems}\\
  Carnegie Mellon University\\
  \url{https://www.cs.cmu.edu/~213/}
\end{itemize}

\subsection{Tutorials and Examples}

\begin{itemize}
  \item \textbf{LLVM Tutorial: My First Language Frontend}\\
  LLVM Project\\
  \url{https://llvm.org/docs/tutorial/MyFirstLanguageFrontend/index.html}

  \item \textbf{Let's Build a Compiler}\\
  Jack Crenshaw\\
  \url{https://compilers.iecc.com/crenshaw/}

  \item \textbf{Writing a Simple Compiler}\\
  Carl Burch\\
  \url{https://www.cburch.com/cs/330/}
\end{itemize}

\section{Standar dan Spesifikasi}

\subsection{Programming Language Standards}

\begin{itemize}
  \item \textbf{ISO/IEC 9899:2018 - C Programming Language}\\
  International Organization for Standardization

  \item \textbf{ISO/IEC 14882:2020 - C++ Programming Language}\\
  International Organization for Standardization

  \item \textbf{ISO/IEC 23660:2020 - Programming Language Specifications}\\
  International Organization for Standardization
\end{itemize}

\subsection{Compiler Standards}

\begin{itemize}
  \item \textbf{DWARF Debugging Information Format}\\
  DWARF Standards Committee\\
  \url{http://dwarfstd.org/}

  \item \textbf{ELF: Executable and Linkable Format}\\
  UNIX System Laboratories\\
  \url{https://refspecs.linuxfoundation.org/elf/}

  \item \textbf{COFF: Common Object File Format}\\
  Microsoft Corporation
\end{itemize}

\section{Software dan Tools}

\subsection{Open Source Compilers}

\begin{itemize}
  \item \textbf{GCC (GNU Compiler Collection)}\\
  Free Software Foundation\\
  \url{https://gcc.gnu.org/}

  \item \textbf{Clang}\\
  LLVM Project\\
  \url{https://clang.llvm.org/}

  \item \textbf{TinyCC (Tiny C Compiler)}\\
  Fabrice Bellard\\
  \url{https://bellard.org/tcc/}
\end{itemize}

\subsection{Compiler Development Tools}

\begin{itemize}
  \item \textbf{ANTLR}\\
  University of San Francisco\\
  \url{https://www.antlr.org/}

  \item \textbf{Flex}\\
  Free Software Foundation\\
  \url{https://github.com/westes/flex}

  \item \textbf{Bison}\\
  Free Software Foundation\\
  \url{https://www.gnu.org/software/bison/}
\end{itemize}

\section{Historical References}

\subsection{Classic Papers}

\begin{itemize}
  \item \textbf{Recursive Functions of Symbolic Expressions and Their Computation by Machine}\\
  McCarthy, John\\
  Communications of the ACM, 1960

  \item \textbf{A Method for Translating Algol 60}\\
  Bauer, Friedrich L. and Klaus Samelson\\
  Communications of the ACM, 1960

  \item \textbf{A Formal System for Specification and Verification of Compilers}\\
  McCarthy, John and James Painter\\
  Communications of the ACM, 1967
\end{itemize}

\subsection{Historical Books}

\begin{itemize}
  \item \textbf{The Design of an Optimizing Compiler}\\
  Aho, Alfred V. and Jeffrey D. Ullman\\
  Elsevier, 1977

  \item \textbf{Compiler Construction: Theory and Practice}\\
  Gries, David\\
  Wiley, 1971

  \item \textbf{A Compiler Generator}\\
  Feldman, Joel and David Gries\\
  Prentice-Hall, 1970
\end{itemize}

\section{Catatan Penggunaan Referensi}

Referensi-referensi ini disusun untuk mendukung pembelajaran Teknik Kompilasi secara komprehensif:

\begin{itemize}
  \item \textbf{Buku Utama}: Digunakan sebagai referensi fundamental untuk konsep dan teori
  \item \textbf{Buku Praktis}: Memberikan panduan implementasi dan penggunaan tools
  \item \textbf{Paper Akademik}: Menyajikan penelitian terkini dan algoritma modern
  \item \textbf{Dokumentasi Teknis}: Referensi spesifik untuk tools dan arsitektur
  \item \textbf{Online Resources}: Materi tambahan dan tutorial interaktif
  \item \textbf{Standar}: Spesifikasi resmi untuk bahasa dan format
  \item \textbf{Software}: Tools yang dapat digunakan untuk praktikum
  \item \textbf{Historical}: Konteks perkembangan compiler construction
\end{itemize}

Semua referensi ini dapat diakses melalui perpustakaan universitas, online repositories, atau pembelian langsung. Mahasiswa disarankan untuk memilih referensi yang sesuai dengan kebutuhan pembelajaran dan proyek yang sedang dikerjakan.


% ============================================================
% AKTIVITAS PEMBELAJARAN
% ============================================================
\begin{aktivitas}
  \item \textbf{Reference Deep Dive}: Pilih satu buku referensi utama dan rangkum satu bab.
  \item \textbf{Paper Analysis}: Baca paper seminal tentang register allocation dan analisis algoritmanya.
  \item \textbf{Tool Investigation}: Eksplorasi dokumentasi LLVM atau GCC untuk fitur optimasi tertentu.
  \item \textbf{Bibliography Management}: Buat database referensi pribadi menggunakan BibTeX.
  \item \textbf{Resource Mapping}: Petakan referensi ke setiap Sub-CPMK yang ada dalam mata kuliah.
\end{aktivitas}

% ============================================================
% LATIHAN DAN REFLEKSI
% ============================================================
\begin{latihan}
  \item Temukan perbedaan pendekatan antara Dragon Book dan Engineering a Compiler!
  \item Cari paper terbaru (3 tahun terakhir) tentang compiler optimization di ACM Digital Library!
  \item Bandingkan spesifikasi x86 dan ARM dalam hal calling conventions dari manual teknis!
  \item Identifikasi kontribusi John McCarthy dalam sejarah teknik kompilasi!
  \item Susun rencana membaca referensi untuk mendukung proyek portofolio Anda!
  \item \textbf{Refleksi}: Bagaimana keragaman referensi membantu Anda memahami kompleksitas teknik kompilasi?
\end{latihan}

% ============================================================
% ASESMEN
% ============================================================
\begin{asesmen}
\textbf{Instrumen Evaluasi Kemampuan Literasi}

\textbf{A. Review Literatur}
\begin{enumerate}
  \item Kualitas sitasi dalam laporan proyek
  \item Kedalaman analisis perbandingan antar referensi
  \item Ketepatan penggunaan standar teknis
\end{enumerate}

\textbf{B. Presentasi Referensi}
\begin{enumerate}
  \item Kemampuan menjelaskan konsep dari paper akademik
  \item Relevansi referensi yang dipilih dengan masalah teknis yang dihadapi
\end{enumerate}

\textbf{Rubrik Penilaian}: Lihat Lampiran A
\end{asesmen}

% ============================================================
% CHECKLIST KOMPETENSI
% ============================================================
\begin{checklist}
  \item Saya mengenal buku-buku referensi utama dalam bidang teknik kompilasi
  \item Saya dapat menelusuri paper akademik untuk algoritma spesifik
  \item Saya mahir membaca dan mengikuti dokumentasi teknis compiler tools
  \item Saya memahami standar dan spesifikasi arsitektur serta format file
  \item Saya dapat menggunakan online resources dan tutorial secara efektif
  \item Saya memahami konteks sejarah dan perkembangan teknologi kompilator
\end{checklist}

% ============================================================
% RANGKUMAN
% ============================================================
\begin{rangkuman}
Bab ini menyajikan daftar referensi komprehensif yang mencakup buku teks, paper akademik, dokumentasi teknis, dan online resources. Mahasiswa belajar memanfaatkan berbagai sumber informasi untuk memperdalam pemahaman teknik kompilasi.

\textbf{Poin Kunci:}
\begin{itemize}
  \item Literatur klasik memberikan fondasi teori yang kuat
  \item Dokumentasi modern memberikan panduan praktis implementasi
  \item Research papers menunjukkan arah perkembangan teknologi masa depan
  \item Standar teknis memastikan interoperabilitas dan kepatuhan sistem
  \item Pembelajaran mandiri didorong melalui eksplorasi berbagai media belajar
\end{itemize}

\textbf{Kata Kunci}: \compiler{Referensi}, \compiler{Literatur}, \compiler{Paper Akademik}, \compiler{Dokumentasi}, \compiler{Standar}, \compiler{Historical Context}
\end{rangkuman}

\ifSubfilesClassLoaded{%
    \clearpage
    \printbibliography[title={Daftar Pustaka}]
    \end{refsection}
}{}

\end{document}
