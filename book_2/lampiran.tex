\chapter*{Lampiran}
\addcontentsline{toc}{chapter}{Lampiran}

% ============================================================
% Lampiran A: Rubrik Penilaian
% ============================================================
\section*{Lampiran A: Rubrik Penilaian Proyek Kompilator}

\begin{table}[h]
\centering
\small
\begin{tabular}{|>{\raggedright\arraybackslash}p{3cm}|>{\raggedright\arraybackslash}p{3.5cm}|>{\raggedright\arraybackslash}p{3.5cm}|>{\raggedright\arraybackslash}p{3.5cm}|}
\hline
\textbf{Kriteria} & \textbf{Sangat Baik} & \textbf{Baik} & \textbf{Perlu Perbaikan} \\
\hline
Analisis Leksikal & Tokenisasi akurat, penanganan error baik & Tokenisasi akurat, error handling terbatas & Tokenisasi kurang akurat \\
\hline
Analisis Sintaksis & Parser efisien, AST terbentuk benar & Parser berfungsi, AST ada sedikit error & Parser sering gagal \\
\hline
Manajemen Tabel Simbol & Implementasi hashtable/tree efisien & Implementasi linear search & Tidak ada tabel simbol \\
\hline
Kualitas Kode & Modular, efisien, naming konsisten & Cukup modular, ada duplikasi & Tidak modular, sulit dibaca \\
\hline
Dokumentasi & Lengkap dan membantu & Cukup lengkap & Minim dokumentasi \\
\hline
\end{tabular}
\caption{Rubrik Penilaian Proyek Kompilator}
\end{table}

% ============================================================
% Lampiran B: Glosarium
% ============================================================
\section*{Lampiran B: Glosarium Istilah Teknik Kompilasi}
\begin{itemize}
  \item \textbf{Compiler}: Program penerjemah bahasa tingkat tinggi ke bahasa mesin.
  \item \textbf{Interpreter}: Program pengeksekusi kode sumber secara langsung.
  \item \textbf{Token}: Unit leksikal terkecil dalam bahasa pemrograman.
  \item \textbf{Lexeme}: String aktual yang membentuk token.
  \item \textbf{AST (Abstract Syntax Tree)}: Representasi pohon dari struktur sintaksis kode.
  \item \textbf{Intermediate Representation}: Bentuk antara antara source dan target code.
\end{itemize}
