\documentclass[12pt,a4paper,twoside]{book}

% ============================================
% LOAD KONFIGURASI TERPUSAT
% ============================================
% Load semua package dan setting yang sama
% File konfigurasi terpusat untuk semua dokumen
% Digunakan oleh main.tex dan semua file subfiles
% File ini berisi semua package dan setting yang sama

% ============================================
% PACKAGES DASAR
% ============================================
\usepackage[utf8]{inputenc}
\usepackage[T1]{fontenc}
\usepackage[bahasa]{babel}

% Package csquotes untuk biblatex (recommended)
\usepackage{csquotes}

% Font standar buku ajar Indonesia (Times New Roman atau serif)
\usepackage{times}  % Times New Roman untuk teks utama
% Alternatif: \usepackage{mathptmx} untuk Times dengan math support

% ============================================
% PACKAGES UNTUK LAYOUT DAN FORMATTING
% ============================================
\usepackage{geometry}
\usepackage{enumitem}
\usepackage{setspace}

% ============================================
% PACKAGES UNTUK HYPERLINK DAN URL
% ============================================
\usepackage[pdfencoding=auto,unicode=true,breaklinks=true,draft=false]{hyperref}
\hypersetup{
    pdfstartview={FitH},
    pdfpagelayout=SinglePage,
    colorlinks=true,
    linkcolor=blue,
    urlcolor=blue,
    citecolor=blue,
    bookmarksopen=true,
    bookmarksnumbered=true,
    pdfduplex=Simplex,
    hypertexnames=false
}
\usepackage{url}
\usepackage{xurl}

% ============================================
% PACKAGES UNTUK TABEL
% ============================================
\usepackage{tabularx}
\usepackage{booktabs}
\usepackage{longtable}

% ============================================
% PACKAGES UNTUK GRAFIK DAN GAMBAR
% ============================================
\usepackage{graphicx}
\usepackage{adjustbox}
\usepackage{float}
\usepackage{placeins}

% Konfigurasi default posisi untuk figure dan table (dinamis)
% [!htbp] = here, top, bottom, atau separate page (dengan override LaTeX parameters)
% Ini memungkinkan LaTeX untuk memilih posisi terbaik secara otomatis
\floatplacement{figure}{!htbp}
\floatplacement{table}{!htbp}

% ============================================
% PACKAGES UNTUK MATEMATIKA
% ============================================
\usepackage{amsmath}
\usepackage{amssymb}
\usepackage{textcomp}
\usepackage{textgreek}
\usepackage{silence}
\WarningFilter{latexfont}{Font shape}
\WarningFilter{latexfont}{Some font shapes}
% Menekan warning citation undefined saat kompilasi per-bab
\WarningFilter{biblatex}{Citation}
\WarningFilter{biblatex}{Patching footnotes failed}
\WarningFilter{biblatex}{Language}

% ============================================
% PACKAGES UNTUK KODE PROGRAM
% ============================================
\usepackage{listings}
\usepackage{xcolor}
\usepackage{fancyvrb}

% ============================================
% PACKAGES UNTUK DIAGRAM DAN TREE
% ============================================
\usepackage{forest}
\usepackage{tikz}
\usetikzlibrary{shapes.geometric, arrows.meta, positioning, calc, matrix}

% ============================================
% PACKAGES UNTUK BIBLIOGRAFI
% ============================================
% Menggunakan biblatex dengan refsegment untuk bibliografi per chapter
% refsegment=chapter: setiap chapter memiliki bibliografi sendiri (untuk kompilasi dari main.tex)
% defernumbers=true: penomoran ditunda sampai akhir untuk konsistensi
% sorting=none: tidak mengurutkan referensi (menjaga urutan citation)
% style=numeric: menggunakan style numerik (seperti plainnat)
% backend=biber: menggunakan biber sebagai backend (lebih modern dari bibtex)
% Untuk kompilasi standalone, refsection akan dibuat manual di setiap file bab
\usepackage[style=numeric,backend=biber,refsegment=chapter,defernumbers=true,sorting=none]{biblatex}

% ============================================
% PACKAGE UNTUK SUBFILES (untuk kompilasi independen chapter)
% ============================================
\usepackage{subfiles}

% ============================================
% DUKUNGAN UNICODE KHUSUS
% ============================================
\DeclareUnicodeCharacter{2192}{$\rightarrow$}  % →
\DeclareUnicodeCharacter{2190}{$\leftarrow$}  % ←
\DeclareUnicodeCharacter{2713}{$\checkmark$}   % ✓
\DeclareUnicodeCharacter{2717}{$\times$}      % ✗
\DeclareUnicodeCharacter{8320}{$_0$}           % ₀
\DeclareUnicodeCharacter{8321}{$_1$}           % ₁
\DeclareUnicodeCharacter{8322}{$_2$}           % ₂

% ============================================
% KONFIGURASI GEOMETRY
% ============================================
% Konfigurasi geometry sesuai standar buku ajar Indonesia (Kementerian Pendidikan Tinggi)
% Untuk kertas A4 (29,7 x 21 cm):
% - Margin atas: 2,5-3 cm
% - Margin kiri: 3 cm (untuk penjilidan)
% - Margin kanan: 2 cm
% - Margin bawah: 2-2,5 cm
\geometry{
    a4paper,
    top=2.5cm,
    bottom=2.5cm,
    left=3cm,
    right=2cm,
    bindingoffset=0.5cm,
    headheight=25.2pt,
    headsep=0.5cm,
    footskip=1cm
}

% ============================================
% KONFIGURASI SPACING
% ============================================
% Spacing 1.5 sesuai standar Indonesia
\onehalfspacing

% ============================================
% KONFIGURASI UNTUK MENGURANGI WARNING
% ============================================
% Mengurangi overfull hbox warnings
\sloppy
\emergencystretch=3em
\hbadness=10000
\vbadness=10000
\tolerance=10000
\pretolerance=10000
% Memungkinkan breaking URL di mana saja
\Urlmuskip=0mu plus 1mu

% ============================================
% KONFIGURASI LISTINGS UNTUK KODE
% ============================================
\lstset{
    language=C++,
    basicstyle=\ttfamily\footnotesize,
    keywordstyle=\color{blue}\bfseries,
    commentstyle=\color{green!60!black},
    stringstyle=\color{red},
    numbers=left,
    numberstyle=\tiny,
    stepnumber=1,
    numbersep=4pt,
    frame=single,
    breaklines=true,
    breakatwhitespace=false,
    breakindent=0pt,
    postbreak=\mbox{\textcolor{red}{$\hookrightarrow$}\space},
    showstringspaces=false,
    tabsize=2,
    extendedchars=true,
    upquote=true,
    columns=flexible,
    keepspaces=true,
    aboveskip=3pt,
    belowskip=3pt,
    lineskip=1pt
}

% ============================================
% KONFIGURASI VERBATIM UNTUK UKURAN LEBIH KECIL
% ============================================
% Redefine verbatim environment untuk menggunakan font lebih kecil
\let\oldverbatim\verbatim
\let\endoldverbatim\endverbatim
\renewenvironment{verbatim}{%
    \footnotesize
    \oldverbatim
}{%
    \endoldverbatim
}

% ============================================
% PACKAGES UNTUK BOXES DAN HIGHLIGHT OBE
% ============================================
\usepackage{tcolorbox}
\tcbuselibrary{skins,breakable}

% Box untuk konsep penting
\newtcolorbox{konsep}{
  colback=blue!5!white,
  colframe=blue!75!black,
  fonttitle=\bfseries,
  title=Konsep Penting,
  breakable
}

% Box untuk catatan
\newtcolorbox{catatan}{
  colback=yellow!5!white,
  colframe=yellow!75!black,
  fonttitle=\bfseries,
  title=Catatan,
  breakable
}

% Box untuk contoh
\newtcolorbox{contoh}{
  colback=green!5!white,
  colframe=green!75!black,
  fonttitle=\bfseries,
  title=Contoh,
  breakable
}

% ============================================================
% CUSTOM ENVIRONMENTS UNTUK OBE
% ============================================================

% --- Environment untuk Sub-CPMK ---
\newenvironment{subcpmk}{
  \vspace{0.5cm}
  \noindent\colorbox{blue!10}{\parbox{\dimexpr\textwidth-2\fboxsep}{
    \textbf{\large Sub-CPMK yang Dicakup dalam Bab Ini:}
  }}
  \vspace{0.3cm}
  \begin{itemize}[leftmargin=*, itemsep=5pt]
}{
  \end{itemize}
  \vspace{0.5cm}
}

% --- Environment untuk Aktivitas Pembelajaran ---
\newenvironment{aktivitas}{
  \vspace{0.5cm}
  \noindent\colorbox{green!10}{\parbox{\dimexpr\textwidth-2\fboxsep}{
    \textbf{\large Aktivitas Pembelajaran}
  }}
  \vspace{0.3cm}
  \begin{enumerate}[leftmargin=*, itemsep=8pt]
}{
  \end{enumerate}
  \vspace{0.5cm}
}

% --- Environment untuk Latihan ---
\newenvironment{latihan}{
  \vspace{0.5cm}
  \noindent\colorbox{orange!10}{\parbox{\dimexpr\textwidth-2\fboxsep}{
    \textbf{\large Latihan dan Refleksi}
  }}
  \vspace{0.3cm}
  \begin{enumerate}[leftmargin=*, itemsep=8pt]
}{
  \end{enumerate}
  \vspace{0.5cm}
}

% --- Environment untuk Asesmen ---
\newenvironment{asesmen}{
  \vspace{0.5cm}
  \noindent\colorbox{red!10}{\parbox{\dimexpr\textwidth-2\fboxsep}{
    \textbf{\large Asesmen (Evaluasi Kinerja)}
  }}
  \vspace{0.3cm}
}{
  \vspace{0.5cm}
}

% --- Environment untuk Checklist Kompetensi ---
\newenvironment{checklist}{
  \vspace{0.5cm}
  \noindent\colorbox{purple!10}{\parbox{\dimexpr\textwidth-2\fboxsep}{
    \textbf{\large Checklist Pencapaian Kompetensi}
  }}
  \vspace{0.3cm}
  \noindent\textit{Centang item berikut setelah Anda yakin telah menguasainya:}
  \vspace{0.2cm}
  \begin{itemize}[leftmargin=*, itemsep=5pt, label=$\square$]
}{
  \end{itemize}
  \vspace{0.5cm}
}

% --- Environment untuk Rangkuman ---
\newenvironment{rangkuman}{
  \vspace{0.5cm}
  \noindent\colorbox{gray!10}{\parbox{\dimexpr\textwidth-2\fboxsep}{
    \textbf{\large Rangkuman}
  }}
  \vspace{0.3cm}
}{
  \vspace{0.5cm}
}

% ============================================================
% CUSTOM COMMANDS UNTUK KOMPILATOR
% ============================================================

% Command untuk highlight istilah compiler
\newcommand{\compiler}[1]{\textbf{\textit{#1}}}

% Command untuk code inline
\newcommand{\code}[1]{\texttt{#1}}

% Command untuk keyword
\newcommand{\keyword}[1]{\textcolor{blue}{\texttt{\textbf{#1}}}}

% Command untuk nama tool
\newcommand{\tool}[1]{\texttt{#1}}

% Command untuk nama file
\newcommand{\filename}[1]{\texttt{#1}}

% ============================================
% KONFIGURASI KHUSUS UNTUK BOOK CLASS
% ============================================
% Conditional loading: hanya load jika documentclass adalah 'book'
\makeatletter
\@ifclassloaded{book}{%
    % Packages khusus untuk book
    \usepackage[toc]{appendix}
    \usepackage{fancyhdr}
    
    % Konfigurasi ukuran font untuk judul bab dan subjudul
    % Judul bab: 16pt, Section: 14pt (sesuai standar Kementerian Pendidikan Tinggi)
    \renewcommand{\@makechapterhead}[1]{%
      \vspace*{50\p@}%
      {\parindent \z@ \raggedright \normalfont
        \ifnum \c@secnumdepth >\m@ne
            \huge\bfseries \@chapapp\space \thechapter
            \par\nobreak
            \vskip 20\p@
        \fi
        \interlinepenalty\@M
        \fontsize{16}{20}\selectfont\bfseries #1\par\nobreak
        \vskip 40\p@
      }}
    % Konfigurasi section (14pt sesuai standar)
    \renewcommand{\section}{\@startsection {section}{1}{\z@}%
                                       {-3.5ex \@plus -1ex \@minus -.2ex}%
                                       {2.3ex \@plus.2ex}%
                                       {\normalfont\fontsize{14}{18}\selectfont\bfseries}}
    
    % Konfigurasi header dan footer
    \pagestyle{fancy}
    \fancyhf{}
    \fancyhead[LE]{\small\leftmark}
    \fancyhead[RO]{\small\rightmark}
    \fancyfoot[C]{\small\thepage}
    \renewcommand{\headrulewidth}{0.4pt}
    \renewcommand{\footrulewidth}{0pt}
}{}
\makeatother

% ============================================
% KONFIGURASI KHUSUS UNTUK ARTICLE CLASS
% ============================================
% Conditional loading: hanya load jika documentclass adalah 'article'
\makeatletter
\@ifclassloaded{article}{%
    % Untuk simulasi chapter di article
    % Definisi ini akan di-override di beberapa file standalone yang menggunakan \clearpage
    % Tapi default-nya adalah \newpage dengan \section*
    \newcommand{\chapter}[1]{%
        \newpage
        \section*{#1}
    }
    
    % Load bibliografi untuk kompilasi standalone (dari folder chapters)
    \addbibresource{../references.bib}
}{}
\makeatother


% Load konfigurasi khusus untuk book
% Konfigurasi khusus untuk document class 'book' (main.tex)
% File ini di-include setelah preamble.tex di main.tex

% ============================================
% PACKAGES KHUSUS UNTUK BOOK
% ============================================
\usepackage[toc]{appendix}
\usepackage{fancyhdr}

% ============================================
% KONFIGURASI UKURAN FONT UNTUK JUDUL BAB DAN SUBJUDUL
% ============================================
% Judul bab: 16pt, Section: 14pt (sesuai standar Kementerian Pendidikan Tinggi)
\makeatletter
\renewcommand{\@makechapterhead}[1]{%
  \vspace*{50\p@}%
  {\parindent \z@ \raggedright \normalfont
    \ifnum \c@secnumdepth >\m@ne
        \huge\bfseries \@chapapp\space \thechapter
        \par\nobreak
        \vskip 20\p@
    \fi
    \interlinepenalty\@M
    \fontsize{16}{20}\selectfont\bfseries #1\par\nobreak
    \vskip 40\p@
  }}
% Konfigurasi section (14pt sesuai standar)
\renewcommand{\section}{\@startsection {section}{1}{\z@}%
                                   {-3.5ex \@plus -1ex \@minus -.2ex}%
                                   {2.3ex \@plus.2ex}%
                                   {\normalfont\fontsize{14}{18}\selectfont\bfseries}}
\makeatother

% ============================================
% KONFIGURASI HEADER DAN FOOTER
% ============================================
\pagestyle{fancy}
\fancyhf{}
\fancyhead[LE]{\small\leftmark}
\fancyhead[RO]{\small\rightmark}
\fancyfoot[C]{\small\thepage}
\renewcommand{\headrulewidth}{0.4pt}
\renewcommand{\footrulewidth}{0pt}


% Informasi buku
\title{\textbf{BUKU AJAR}\\
       \textbf{TEKNIK KOMPILASI}\\
       \large Berbasis Outcome-Based Education (OBE)\\
       \large Praktis dengan C/C++}
\author{Program Studi S1 Teknik Informatika}
\date{\today}

\begin{document}

% Halaman judul (format standar buku ajar Indonesia - Kementerian Pendidikan Tinggi)
\begin{titlepage}
\centering
\vspace*{1.5cm}

% Logo universitas (jika ada, uncomment dan sesuaikan path)
% \includegraphics[width=4cm]{logo_universitas}\\[1.5cm]

{\fontsize{18}{22}\selectfont \textbf{BUKU AJAR}}\\[0.8cm]
{\fontsize{20}{24}\selectfont \textbf{TEKNIK KOMPILASI}}\\[0.5cm]
{\fontsize{14}{18}\selectfont Berbasis Outcome-Based Education (OBE)}\\[0.3cm]
{\fontsize{14}{18}\selectfont Praktis dengan C/C++}\\[2.5cm]

{\fontsize{12}{16}\selectfont Oleh}\\[0.5cm]
{\fontsize{14}{18}\selectfont \textbf{[Nama Dosen Pengampu]}}\\[2cm]

{\fontsize{12}{16}\selectfont \textit{Digunakan di lingkungan sendiri, sebagai buku ajar mata kuliah\\
\textbf{Teknik Kompilasi} pada Program Studi S1 Teknik Informatika}}\\[2.5cm]

{\fontsize{14}{18}\selectfont \textbf{Program Studi S1 Teknik Informatika}}\\[0.3cm]
{\fontsize{14}{18}\selectfont Fakultas Teknik}\\[0.3cm]
{\fontsize{14}{18}\selectfont Universitas}\\[2cm]

{\fontsize{12}{16}\selectfont \textbf{\the\year}}
\end{titlepage}

% Halaman informasi buku (format standar buku ajar)
\newpage
\thispagestyle{empty}
\vspace*{1.5cm}

\begin{center}
{\fontsize{16}{20}\selectfont \textbf{INFORMASI BUKU}}\\[1.5cm]
\end{center}

\vspace{0.5cm}

\begin{tabularx}{\textwidth}{@{}>{\scriptsize\raggedright\arraybackslash}p{1.5cm}>{\raggedright\arraybackslash}X@{}}
\textbf{Judul} & : Buku Ajar Teknik Kompilasi\\
\textbf{Subjudul} & : Berbasis Outcome-Based Education (OBE), Praktis dengan C/C++\\
\textbf{Penulis} & : [Nama Dosen Pengampu]\\
\textbf{Program Studi} & : S1 Teknik Informatika\\
\textbf{Fakultas} & : Fakultas Teknik\\
\textbf{Universitas} & : Universitas\\
\textbf{Mata Kuliah} & : Teknik Kompilasi\\
\textbf{SKS} & : 3 SKS\\
\textbf{Pertemuan} & : 16 Pertemuan\\
\textbf{Tahun} & : \the\year\\
\textbf{ISBN} & : [ISBN jika ada]\\
\end{tabularx}\par

\vspace{1.5cm}

\begin{center}
\begin{minipage}{0.9\textwidth}
\centering
{\fontsize{11}{14}\selectfont \textit{Buku ajar ini disusun sebagai bahan pembelajaran untuk mata kuliah \textbf{Teknik Kompilasi} pada Program Studi S1 Teknik Informatika. Buku ini dirancang mengikuti pendekatan \textbf{Outcome-Based Education (OBE)} dengan fokus pada pembelajaran berbasis praktik.}}
\end{minipage}
\end{center}

\vspace*{\fill}

% Halaman kosong untuk penjilidan
\newpage
\thispagestyle{empty}
\mbox{}

% Penomoran halaman Romawi untuk bagian awal
\pagenumbering{roman}
\setcounter{page}{1}
\pagestyle{plain}  % Tidak ada header untuk bagian awal

% Prakata
\input{chapters/prakata}
\clearpage

% Daftar isi
\tableofcontents
\clearpage

% Daftar gambar (jika ada)
\listoffigures
\clearpage

% Daftar tabel (jika ada)
\listoftables
\FloatBarrier
\clearpage

% Mulai penomoran halaman Arab untuk isi buku
\cleardoublepage
\pagenumbering{arabic}
\setcounter{page}{1}
\pagestyle{fancy}  % Aktifkan header/footer untuk isi buku

% Bab-bab
\documentclass[../main.tex]{subfiles}

\addbibresource{\subfix{../references.bib}}

\begin{document}

\ifSubfilesClassLoaded{%
    \setcounter{chapter}{0}%
    \begin{refsection}
}{}

\chapter{Pendahuluan dan Orientasi OBE}
\label{chap:pendahuluan}

\begin{subcpmk}
  \item Memahami keterkaitan antara kurikulum \textit{Outcome-Based Education} (OBE) dengan desain sistem kompilator.
  \item Mengidentifikasi alur pembelajaran Teknik Kompilasi melalui peta konsep buku ajar.
  \item Merencanakan strategi belajar mandiri untuk menguasai setiap fase kompilasi secara sistematis.
\end{subcpmk}

% ============================================================
% MATERI POKOK
% ============================================================
\section{Tujuan Buku Ajar}

Buku ajar ini dirancang sebagai panduan komprehensif untuk menguasai \compiler{Teknik Kompilasi} secara sistematis dan terukur, selaras dengan standar \textit{Outcome-Based Education} (OBE) \cite{studylib2024obe}. Fokus utama buku ini adalah pada pembangunan fondasi teoretis dan keterampilan praktis dalam membangun arsitektur kompilator modern. Menurut \cite{aho2006compilers}, kompilasi adalah proses transformasi dari bahasa sumber ke bahasa sasaran.
\begin{enumerate}
  \item Memberikan pemahaman mendalam tentang setiap fase kompilasi, mulai dari analisis leksikal hingga generasi kode target.
  \item Mengembangkan kemampuan merancang dan mengimplementasikan komponen-komponen utama kompilator seperti \textit{lexer}, \textit{parser}, dan \textit{semantic analyzer}.
  \item Membangun keterampilan dalam optimasi kode dan manajemen memori pada \textit{runtime}.
  \item Memfasilitasi pencapaian \textit{Capaian Pembelajaran Lulusan} (CPL) dan \textit{Capaian Pembelajaran Mata Kuliah} (CPMK) yang telah ditetapkan dalam kurikulum.
\end{enumerate}

Setelah mempelajari buku ini secara menyeluruh, mahasiswa diharapkan mampu:
\begin{itemize}
  \item Menjelaskan arsitektur kompilator dan fungsi setiap fasenya.
  \item Membangun pemroses bahasa (\textit{language processor}) menggunakan teknik manual maupun generator (\textit{Flex/Bison}).
  \item Mengelola struktur data kompleks seperti \textit{Symbol Table} dan \textit{Abstract Syntax Tree} (AST).
  \item Menghasilkan kode target yang efisien untuk arsitektur mesin tertentu.
  \item Melakukan evaluasi performa dan optimasi pada tingkat \textit{intermediate code} dan \textit{target code}.
\end{itemize}

\section[Keterkaitan Buku Ajar dengan RPS Berbasis OBE]{Keterkaitan Buku Ajar dengan RPS \protect\\ Berbasis OBE}

Buku ajar ini disusun dengan penyelarasan yang ketat terhadap Rencana Pembelajaran Semester (RPS) mata kuliah Teknik Kompilasi yang menerapkan kerangka kerja \textit{Outcome-Based Education} (OBE). Keterkaitan ini menjamin bahwa setiap aktivitas kognitif yang dilakukan mahasiswa—mulai dari memahami definisi hingga merancang sistem—memiliki kontribusi langsung terhadap pencapaian Capaian Pembelajaran Lulusan (CPL). Integrasi yang kuat antara konten buku dan CPL memastikan bahwa keterampilan yang dikuasai mahasiswa relevan dengan kebutuhan dan standar kompetensi di industri teknologi informasi modern \cite{neu2024compiler}.

Secara spesifik, buku ini dirancang untuk memenuhi CPL-1 (Pengetahuan Rekayasa) dengan menyajikan teori otoritatif mengenai bahasa formal, automata, dan tata bahasa bebas konteks. Di sisi lain, buku ini juga secara intensif menargetkan CPL-3 (Perancangan dan Pengembangan Solusi) dengan membimbing mahasiswa melalui proses iteratif pembangunan kompilator yang fungsional. Keseimbangan ini memastikan mahasiswa tidak hanya tumbuh sebagai teoretisi yang memahami konsep abstrak, tetapi juga sebagai insinyur perangkat lunak yang mampu menghasilkan solusi konkret untuk masalah yang kompleks.

Aktivitas pembelajaran dalam buku ini disusun secara bertingkat mengikuti hierarki Taksonomi Bloom untuk memandu perkembangan kognitif mahasiswa. Dimulai dari level dasar seperti memahami sintaksis dan semantik bahasa, mahasiswa dibimbing menuju level analisis untuk memecahkan konflik \textit{parsing}, dan akhirnya mencapai level tertinggi yaitu menciptakan (\textit{creating}) sebuah pemroses bahasa yang utuh. Progresi bertahap ini sangat krusial dalam pendidikan teknik untuk membantu mahasiswa menguasai permasalahan rekayasa yang rumit secara terstruktur dan percaya diri.

\section{Arsitektur Kompilator Modern}

Arsitektur kompilator modern umumnya terbagi menjadi front-end dan back-end \cite{cooper2011engineering}.

\subsection{Alignment dengan CPL dan CPMK}

Struktur buku ini disusun untuk mendukung pencapaian indikator-indikator berikut:
\begin{itemize}
  \item \textbf{CPL-1 (Pengetahuan)}: Menguasai konsep teoretis analisis leksikal, sintaksis, semantik, dan generasi kode secara mendalam.
  \item \textbf{CPL-3 (Keterampilan Khusus)}: Mampu merancang, mengimplementasikan, dan mengevaluasi sistem kompilator lengkap.
  \item \textbf{CPMK-1 s.d CPMK-6}: Meliputi seluruh spektrum pengembangan kompilator dari arsitektur awal hingga evaluasi performa akhir.
\end{itemize}

Setiap bab dalam buku ini memuat daftar \textbf{Sub-CPMK} di bagian awal untuk memberikan fokus yang jelas bagi mahasiswa mengenai kompetensi spesifik yang akan dikuasai.

\subsection{Integrasi Metode Pembelajaran Aktif}

Sesuai dengan RPS berbasis OBE, buku ini mendukung berbagai metode pembelajaran:
\begin{itemize}
  \item \textbf{Problem-Based Learning}: Melalui studi kasus penanganan \textit{semantic errors} dan optimasi lokal.
  \item \textbf{Project-Based Learning}: Pengembangan kompilator secara bertahap dalam setiap bab.
  \item \textbf{Praktikum Terbimbing}: Implementasi komponen menggunakan \textit{tooling} industri seperti LLVM atau Clang.
\end{itemize}

\subsection{Sistem Evaluasi Berbasis Kompetensi}

Komponen asesmen yang disediakan di setiap akhir bab (latihan, asesmen, dan \textit{checklist}) dirancang untuk mengukur pencapaian \textit{Sub-CPMK} secara objektif, yang nantinya akan menjadi bobot penilaian utama dalam UTS dan UAS (total 35\% sesuai RPS).

\section{Petunjuk Penggunaan Buku Ajar}

\subsection{Untuk Mahasiswa}

Mengingat kompleksitas pengembangan kompilator, mahasiswa disarankan untuk menggunakan buku ini dengan langkah-langkah berikut:

\textbf{Tahap Persiapan:}
\begin{enumerate}
  \item Pahami target \textit{Sub-CPMK} di awal bab agar fokus belajar tetap terjaga.
  \item Tinjau kembali materi prasyarat (Struktur Data dan Algoritma) jika diperlukan.
\end{enumerate}

\textbf{Tahap Implementasi (Eksperimental):}
\begin{enumerate}
  \item Pelajari kode contoh yang disediakan dan jalankan menggunakan \textit{compiler} atau \textit{interpreter} yang sesuai.
  \item Lakukan modifikasi pada parameter \textit{lexer} atau aturan \textit{grammar} untuk melihat dampaknya terhadap proses \textit{parsing}.
  \item Gunakan perangkat lunak pendukung seperti \textit{Flex}, \textit{Bison}, atau \textit{Graphviz} untuk memvisualisasikan AST.
\end{enumerate}

\textbf{Tahap Evaluasi:}
\begin{enumerate}
  \item Kerjakan latihan refleksi untuk memperdalam pemahaman teoretis.
  \item Lakukan penilaian mandiri menggunakan \textit{checklist} kompetensi di akhir bab.
  \item Gabungkan komponen yang telah dibuat di setiap bab menjadi satu proyek kompilator utuh.
\end{enumerate}

\subsection{Untuk Dosen}

Dosen dapat memanfaatkan buku ini sebagai instrumen pembelajaran utama:
\begin{itemize}
  \item \textbf{Modul Praktikum}: Gunakan aktivitas pembelajaran di setiap bab sebagai panduan tugas mingguan.
  \item \textbf{Bank Soal}: Manfaatkan bagian asesmen sebagai referensi dalam menyusun soal UTS (CPMK-1, 2) dan UAS (CPMK 1 s.d 6).
  \item \textbf{Alat Ukur Capaian}: Gunakan rubrik penilaian dan indikator dalam buku untuk mengukur ketercapaian outcomes mahasiswa.
\end{itemize}

\section{Konteks Kurikulum OBE}

\subsection{Filosofi OBE dalam Teknik Kompilasi}

\compiler{Outcome-Based Education (OBE)} adalah pendekatan yang menekankan pada apa yang bisa dilakukan oleh mahasiswa di akhir masa studi, bukan sekadar apa yang diajarkan. Dalam Teknik Kompilasi, hal ini berarti mahasiswa tidak hanya menghafal algoritma \textit{parsing}, tetapi mampu membangun sebuah program yang secara nyata dapat menerjemahkan sebuah bahasa ke bahasa lain.

\textbf{Empat Prinsip Utama OBE dalam Buku Ini:}
\begin{enumerate}
  \item \textbf{Clarity of Focus}: Fokus pada hasil akhir berupa kompilator yang berfungsi.
  \item \textbf{Designing Down}: Materi disusun mundur dari kebutuhan akhir sebuah sistem \textit{backend} kompilator.
  \item \textbf{High Expectations}: Mahasiswa didorong untuk mengimplementasikan optimasi kode yang efisien.
  \item \textbf{Expanded Opportunity}: Menyediakan berbagai aktivitas belajar mulai dari teori hingga proyek tim.
\end{enumerate}

\subsection{Implementasi Tahapan OBE}

Buku ini membagi proses pencapaian kompetensi dalam empat pilar utama:

\begin{table}[!htbp]
\centering
\begin{tabular}{|l|p{10.5cm}|}
\hline
\textbf{Komponen OBE} & \textbf{Implementasi dalam Teknik Kompilasi} \\
\hline
\textit{Defined Outcomes} & Sub-CPMK eksplisit untuk setiap fase (Leksikal, Sintaksis, dst). \\
\hline
\textit{Designing Down} & Kurikulum dimulai dari pengenalan bahasa ke deteksi kesalahan hingga emisi kode. \\
\hline
\textit{Student Activity} & Implementasi manual mesin \textit{state} dan penggunaan alat otomatisasi generator. \\
\hline
\textit{Continuous Assessment} & \textit{Weekly reflection} dan audit kualitas kode secara berkala. \\
\hline
\end{tabular}
\caption{Penerapan OBE dalam Pengembangan Kompilator}
\end{table}

\subsection{Hierarki Capaian Pembelajaran}

\begin{konsep}
Pemahaman mahasiswa terhadap Teknik Kompilasi divalidasi melalui hierarki capaian:
\begin{itemize}
    \item \textbf{CPL}: Mahasiswa menguasai teori kompilasi secara utuh sebagai sarjana teknik.
    \item \textbf{CPMK}: Mahasiswa mampu mengintegrasikan fase-fase kompilasi menjadi sistem yang koheren.
    \item \textbf{Sub-CPMK}: Mahasiswa mahir dalam satu spesialisasi fase (misalnya: \textit{Register Allocation}).
\end{itemize}
\end{konsep}

\section{Peta Konsep Teknik Kompilasi}

Buku ini disusun dalam 19 bab yang mencakup seluruh spektrum pengembangan kompilator, yang dapat dipandang sebagai sebuah ``pipa transformasi'' data. Proses dimulai dari aliran karakter mentah (\textit{source code}), yang kemudian diubah menjadi token linear, disusun menjadi struktur pohon hierarkis (\textit{Syntax Tree}), diperkaya dengan informasi semantik, diterjemahkan menjadi kode antara (\textit{Intermediate Representation}) yang agnostik terhadap mesin, dan akhirnya diekspansi menjadi instruksi mesin spesifik (\textit{Assembly}) yang optimal. Setiap bab dalam buku ini membedah satu ruas dari pipa tersebut secara mendalam.

Secara holistik, peta konsep ini tidak hanya mengajarkan teknik isolasi komponen, tetapi juga integrasi sistem. Mahasiswa diajak untuk melihat bagaimana keputusan desain di satu fase (misalnya, desain IR) berdampak signifikan pada fase berikutnya (optimasi dan generasi kode). Pemahaman lintas-fase ini adalah esensi dari pemikiran sistem (\textit{systems thinking}) yang ingin dibangun melalui mata kuliah ini.

Berikut adalah rincian materi per bab:
\begin{enumerate}
  \item \textbf{Bab I}: Pengenalan dan Konteks OBE
  \item \textbf{Bab II}: Arsitektur Kompilator - Gambaran umum sistem
  \item \textbf{Bab III-IV}: \textit{Front-end} - Analisis leksikal dan representasi regular
  \item \textbf{Bab V-VI}: \textit{Syntax Analysis} - \textit{Parsing} dan \textit{grammar} formal
  \item \textbf{Bab VII-X}: \textit{Middle-end} - \textit{Intermediate code}, tabel simbol, analisis semantik, dan penanganan kesalahan
  \item \textbf{Bab XI-XIV}: \textit{Back-end} - Tata letak memori, \textit{code generation}, alokasi register, dan manajemen \textit{stack}
  \item \textbf{Bab XV-XVI}: \textit{Analysis \& Evaluation} - \textit{Compiler tools} dan evaluasi performa
  \item \textbf{Bab XVII-XIX}: \textit{Assessment \& Resources} - Evaluasi kompetensi, lampiran, dan daftar referensi
\end{enumerate}

\textbf{Alur Pembelajaran:}
\begin{itemize}
  \item \textbf{Fase Analisis (Bab II-VI)}: Memahami bagaimana bahasa manusia diterjemahkan menjadi token dan pohon hirarki.
  \item \textbf{Fase Transformasi (Bab VII-X)}: Memastikan kebenaran makna dan mengubahnya menjadi representasi antara.
  \item \textbf{Fase Sintesis (Bab XI-XIV)}: Membangun instruksi mesin yang optimal sesuai arsitektur target.
  \item \textbf{Fase Profesional (Bab XV-XIX)}: Menggunakan alat bantu modern dan melakukan standarisasi kualitas.
\end{itemize}

\section{Proyek Buku: Compiler Subset C}
\label{sec:spec-subset-c}

Salah satu kekuatan utama buku ajar ini adalah penggunaan proyek pengembangan tunggal yang berkelanjutan (\textit{continuous project}) berupa kompilator untuk subset bahasa C. Pendekatan ini dipilih karena bahasa C merupakan \textit{lingua franca} sistem pemrograman yang memiliki karakteristik imperatif, prosedural, dan \textit{statically typed} yang representatif. Dengan membangun kompilator untuk subset C, mahasiswa akan menghadapi tantangan nyata yang relevan dengan industri, namun dalam lingkup yang terkendali sehingga tetap dapat diselesaikan dalam satu semester.

Proyek ini dibangun menggunakan filosofi pengembangan inkremental (\textit{incremental development}). Kita tidak mencoba membangun seluruh kompilator sekaligus. Sebaliknya, kita membangunnya lapis demi lapis—dimulai dari lexer sederhana, kemudian parser, lalu AST, dan seterusnya. Setiap bab menambahkan fungsionalitas baru ke atas fondasi yang telah dibangun sebelumnya. Metode ini tidak hanya memudahkan proses \textit{debugging}, tetapi juga mengajarkan disiplin rekayasa perangkat lunak tentang bagaimana mengelola kompleksitas melalui modularitas.

Sepanjang Bab 2 hingga Bab 16, kita secara bertahap membangun \textbf{satu compiler untuk subset bahasa C}. Setiap bab menambah satu lapis ke proyek yang sama: spesifikasi token (Bab 3), lexer hand-written (Bab 3), lexer Flex (Bab 4), grammar (Bab 5), parser hand-written (Bab 6), teori bottom-up (Bab 7), parser Bison (Bab 8), AST (Bab 9), symbol table (Bab 10), type checking (Bab 11), IR (Bab 12), runtime (Bab 13), code generation (Bab 14), optimasi (Bab 15), dan integrasi (Bab 16). Spesifikasi berikut menjadi acuan tunggal agar semua contoh dan kode mengacu ke bahasa yang sama.

\subsection{Spesifikasi Token Proyek Subset C}

Token yang dikenali oleh compiler proyek (untuk Bab 3--4):

\begin{itemize}
    \item \textbf{Identifier}: huruf atau underscore diikuti huruf, angka, atau underscore. Pola: \texttt{[a-zA-Z\_][a-zA-Z0-9\_]*}
    \item \textbf{Kata kunci}: \texttt{int}, \texttt{float}, \texttt{print}. (Nanti dapat diperluas: \texttt{if}, \texttt{else}, \texttt{while}.)
    \item \textbf{Literal}: integer \texttt{[0-9]+}, float \texttt{[0-9]+.[0-9]+}, string \texttt{"..."} dalam tanda kutip ganda.
    \item \textbf{Operator}: \texttt{+}, \texttt{-}, \texttt{*}, \texttt{/}, \texttt{=}, \texttt{==}, \texttt{!=}, \texttt{<}, \texttt{>}, \texttt{<=}, \texttt{>=}.
    \item \textbf{Punctuator}: \texttt{;}, \texttt{,}, \texttt{(} \texttt{)}, kurung kurawal \texttt{\char`\{\char`\}}.
    \item \textbf{Komentar}: satu baris \texttt{//} dan banyak baris \texttt{/* */}; serta whitespace (spasi, tab, newline) diabaikan.
\end{itemize}

\subsection{Spesifikasi Grammar Proyek Subset C}

Grammar dalam BNF untuk Bab 5--8 (dan parser proyek):

\begin{itemize}
    \item \textbf{Program}: barisan statement.
    \item \textbf{Statement}: deklarasi \texttt{;} \textbar\ assignment \texttt{;} \textbar\ print-statement \texttt{;}
    \item \textbf{Deklarasi}: \texttt{int} identifier \textbar\ \texttt{float} identifier
    \item \textbf{Assignment}: identifier \texttt{=} ekspresi
    \item \textbf{Print-statement}: \texttt{print} \texttt{(} string-literal \texttt{)} \textbar\ \texttt{print} \texttt{(} ekspresi \texttt{)}
    \item \textbf{Ekspresi}: term \textbar\ ekspresi \texttt{+} term \textbar\ ekspresi \texttt{-} term
    \item \textbf{Term}: factor \textbar\ term \texttt{*} factor \textbar\ term \texttt{/} factor
    \item \textbf{Factor}: literal \textbar\ identifier \textbar\ \texttt{(} ekspresi \texttt{)}
\end{itemize}

Precedence: \texttt{*} dan \texttt{/} lebih tinggi dari \texttt{+} dan \texttt{-}; associativity kiri untuk semuanya.

\subsection{Peta Bab ke Lapis Proyek}

\begin{center}
\begin{tabular}{cl}
\toprule
\textbf{Bab} & \textbf{Lapis proyek} \\
\midrule
3 & Spesifikasi token + teori RE/FA \\
3 & Lexer hand-written (mengikuti spec token) \\
4 & Lexer proyek (Flex, file \texttt{simplec.l}) \\
5 & Grammar proyek (BNF/EBNF di atas) \\
6 & Parser hand-written (mengikuti grammar proyek) \\
7 & Teori LR; grammar proyek termasuk kelas LR \\
8 & Parser proyek (Bison, file \texttt{simplec.y}) \\
9 & AST proyek (\texttt{ast.h}/\texttt{ast.c}) \\
10 & Symbol table proyek (\texttt{symtab.h}/\texttt{symtab.c}) \\
11 & Type checking proyek \\
12 & IR proyek (TAC/quadruples dari AST) \\
13 & Runtime; asumsi proyek untuk stack/activation record \\
14 & Code generation proyek (IR $\to$ assembly) \\
15 & Optimasi proyek (basic block, constant folding, dll.) \\
16 & Integrasi dan presentasi compiler subset C lengkap \\
\bottomrule
\end{tabular}
\end{center}

Semua bab dari Bab 3 sampai Bab 16 merujuk ke spesifikasi ini. Kode dan contoh dalam bab tersebut mengacu ke token set dan grammar di atas, serta ke file proyek (\texttt{simplec.l}, \texttt{simplec.y}, dan seterusnya) yang tumbuh di folder \texttt{proyek-compiler-subset-c/}.


% ============================================================
% AKTIVITAS PEMBELAJARAN
% ============================================================
\begin{aktivitas}
  \item \textbf{Analisis RPS}: Pelajari RPS Teknik Kompilasi dan identifikasi bagaimana CPL Pengetahuan (CPL-1) diukur melalui proyek pengembangan kompilator.
  \item \textbf{Pemetaan Fase}: Buat diagram alir yang memetakan bab-bab dalam buku ini ke dalam tiga pilar utama: \textit{Front-end}, \textit{Middle-end}, dan \textit{Back-end}.
  \item \textbf{Tooling Audit}: Identifikasi \textit{software} pendukung (GCC, Flex, Bison, Clang) yang akan digunakan di setiap bab berdasarkan peta konsep.
  \item \textbf{Diskusi Kurikulum}: Diskusikan mengapa penguasaan Teknik Kompilasi sangat krusial dalam mencapai standar kompetensi lulusan Teknik Informatika di industri modern.
\end{aktivitas}

% ============================================================
% LATIHAN DAN REFLEKSI
% ============================================================
\begin{latihan}
  \item Jelaskan secara spesifik bagaimana pendekatan OBE membantu mahasiswa dalam menghadapi kompleksitas algoritma \textit{parsing}!
  \item Mengapa setiap aktivitas praktikum harus memiliki rubrik penilaian yang eksplisit dalam konteks OBE?
  \item Bagaimana cara Anda memantau kemajuan pembangunan proyek kompilator Anda menggunakan \textit{checklist} kompetensi?
  \item Hubungkan antara \textit{Target Architecture} (Bab XII) dengan tujuan akhir pencapaian kompetensi dalam RPS!
  \item \textbf{Refleksi}: Sejauh mana Anda memahami bahwa membangun sebuah kompilator adalah bukti nyata pencapaian kompetensi teknik yang utuh?
\end{latihan}

% ============================================================
% ASESMEN
% ============================================================
\begin{asesmen}
\textbf{Instrumen Penilaian untuk Orientasi OBE Kompilator}

\textbf{A. Pilihan Ganda}

\begin{enumerate}
  \item Manakah yang merupakan contoh \textit{Outcome} nyata dalam mata kuliah ini?
  \begin{enumerate}
    \item Membaca buku Naga (Dragon Book)
    \item Lulus ujian teori
    \item Menghasilkan kode assembly yang dapat dijalankan
    \item Mengetahui sejarah FORTRAN
  \end{enumerate}
  
  \item \textit{Designing Down} dalam buku ini berarti:
  \begin{enumerate}
    \item Mendesain dari tingkat mesin ke bahasa tingkat tinggi
    \item Menyusun materi berdasarkan urutan fase kompilasi untuk mencapai produk akhir
    \item Mengurangi beban materi yang sulit
    \item Hanya fokus pada latihan praktak
  \end{enumerate}
\end{enumerate}

\textbf{B. Essay}

\begin{enumerate}
  \item Jelaskan keterkaitan antara peta konsep (Bab I s.d XIX) dengan pencapaian kompetensi profesional seorang \textit{System Programmer}!
  \item Desainlah satu indikator pencapaian kompetensi untuk Sub-CPMK "Membangun Lexer Hand-written"!
\end{enumerate}

\textbf{Rubrik Penilaian}: Lihat Lampiran A
\end{asesmen}

% ============================================================
% CHECKLIST KOMPETENSI
% ============================================================
\begin{checklist}
  \item Saya memahami filosofi OBE dalam konteks pengembangan sistem kompilator.
  \item Saya dapat memetakan hierarki CPL, CPMK, dan Sub-CPMK Teknik Kompilasi ke dalam konten buku.
  \item Saya memahami alur kerja \textit{Front-end}, \textit{Middle-end}, dan \textit{Back-end} melalui peta konsep.
  \item Saya telah menyiapkan \textit{software stack} yang diperlukan untuk proyek semester ini.
  \item Saya berkomitmen untuk melakukan \textit{self-assessment} secara berkala di setiap akhir bab.
\end{checklist}

% ============================================================
% RANGKUMAN BAB
% ============================================================
\begin{rangkuman}
Bab ini memberikan fondasi bagi mahasiswa untuk memahami bagaimana buku ajar ini disusun menggunakan standar OBE guna menjamin penguasaan Teknik Kompilasi yang utuh dan profesional.

\textbf{Poin Kunci:}
\begin{itemize}
  \item Fokus utama adalah pencapaian \textit{outcome} berupa sistem kompilator yang berfungsi.
  \item Kurikulum dirancang mundur (\textit{designing down}) untuk memandu mahasiswa dari analisis ke sintesis kode.
  \item Peta konsep menunjukkan integrasi 19 bab sebagai satu kesatuan ekosistem pembelajaran.
  \item Peran aktif mahasiswa dalam evaluasi diri adalah kunci keberhasilan dalam sistem OBE.
\end{itemize}

\textbf{Kata Kunci}: \compiler{OBE}, \compiler{Teknik Kompilasi}, \compiler{Peta Konsep}, \compiler{CPL}, \compiler{Outcome}, \compiler{Compiler Design}
\end{rangkuman}

\ifSubfilesClassLoaded{%
    \clearpage
    \printbibliography[title={Daftar Pustaka}]
    \end{refsection}
}{}

\end{document}

\cleardoublepage
% Bab 2: Regular Expression dan Finite Automata untuk Lexical Analysis
% File ini dapat dikompilasi terpisah atau sebagai bagian dari main.tex

\chapter[Regular Expression dan Finite Automata]{Regular Expression dan Finite Automata\\untuk Lexical Analysis}
\label{chap:regex-fa}

\section{Tujuan Pembelajaran}

Setelah mempelajari bab ini, mahasiswa diharapkan mampu:
\begin{enumerate}
    \item Memahami konsep regular expression dan regular language
    \item Menjelaskan perbedaan antara NFA (Nondeterministic Finite Automata) dan DFA (Deterministic Finite Automata)
    \item Mengkonversi regular expression ke NFA menggunakan algoritma Thompson
    \item Mengkonversi NFA ke DFA menggunakan subset construction
    \item Mengimplementasikan NFA dan DFA sederhana dalam C/C++
    \item Membuat recognizer untuk pattern token sederhana menggunakan finite automata
    \item Memahami hubungan antara regular expression, finite automata, dan lexical analysis
\end{enumerate}

\section{Pendahuluan}

Sebagai landasan untuk memahami lexical analysis, kita perlu mempelajari teori formal yang mendasarinya. Menurut sumber dari Aoyama Gakuin University:

\begin{quote}
``Lexical analysis breaks input text into lexemes which correspond to tokens. Usually implemented using regular languages → regex → NFA → DFA → (minimized) DFA for efficiency.''\cite{aoyama2024lexical}
\end{quote}

Alur ini menunjukkan bahwa lexical analysis dalam kompilator modern menggunakan teori formal language, khususnya regular languages, yang direpresentasikan sebagai regular expressions dan kemudian diimplementasikan sebagai finite automata untuk efisiensi.

\section{Regular Expression dan Regular Language}

\subsection{Definisi Regular Expression}

Regular expression (regex) adalah notasi formal untuk mendeskripsikan pola string dalam suatu bahasa. Regular expression menggunakan operasi-operasi dasar untuk membangun pattern yang lebih kompleks.

Operasi-operasi dasar dalam regular expression meliputi:

\begin{enumerate}
    \item \textbf{Literal}: Karakter tunggal, misalnya \texttt{a} mencocokkan string ``a''
    \item \textbf{Concatenation}: Penggabungan, misalnya \texttt{ab} mencocokkan string ``ab''
    \item \textbf{Union/Alternation}: Pilihan, misalnya \texttt{a|b} mencocokkan ``a'' atau ``b''
    \item \textbf{Kleene Star}: Nol atau lebih pengulangan, misalnya \texttt{a*} mencocokkan ``'', ``a'', ``aa'', ``aaa'', dll.
    \item \textbf{Kleene Plus}: Satu atau lebih pengulangan, misalnya \texttt{a+} mencocokkan ``a'', ``aa'', ``aaa'', dll.
    \item \textbf{Optional}: Nol atau satu, misalnya \texttt{a?} mencocokkan ``'' atau ``a''
    \item \textbf{Character Class}: Set karakter, misalnya \texttt{[0-9]} mencocokkan digit 0-9
\end{enumerate}

\subsection{Contoh Regular Expression untuk Token}

Dalam lexical analysis, setiap jenis token didefinisikan menggunakan regular expression. Berikut beberapa contoh:

\begin{itemize}
    \item \textbf{Identifier}: \texttt{[a-zA-Z\_][a-zA-Z0-9\_]*}
    \begin{itemize}
        \item Dimulai dengan huruf atau underscore
        \item Diikuti oleh nol atau lebih huruf, digit, atau underscore
    \end{itemize}
    
    \item \textbf{Integer Literal}: \texttt{[0-9]+}
    \begin{itemize}
        \item Satu atau lebih digit
    \end{itemize}
    
    \item \textbf{Floating Point}: \texttt{[0-9]+\textbackslash.[0-9]+}
    \begin{itemize}
        \item Digit, titik desimal, digit
    \end{itemize}
    
    \item \textbf{String Literal}: \texttt{"([\textasciicircum"\\]|\textbackslash\textbackslash.)*"}
    \begin{itemize}
        \item Dimulai dan diakhiri dengan tanda kutip
        \item Berisi karakter apapun kecuali tanda kutip (atau escape sequence)
    \end{itemize}
    
    \item \textbf{Whitespace}: \texttt{[ \textbackslash t\textbackslash n]+}
    \begin{itemize}
        \item Satu atau lebih spasi, tab, atau newline
    \end{itemize}
    
    \item \textbf{Operator}: \texttt{+|-|*|/|=|==|!=}
    \begin{itemize}
        \item Operator aritmatika dan perbandingan
    \end{itemize}
\end{itemize}

\subsection{Regular Language}

Bahasa yang dapat dinyatakan dengan regular expression disebut \textbf{regular language}. Regular language memiliki sifat-sifat penting:

\begin{itemize}
    \item Dapat dikenali oleh finite automata (NFA atau DFA)
    \item Tertutup terhadap operasi union, concatenation, dan Kleene star
    \item Tidak dapat mengekspresikan struktur nested (seperti matching parentheses)
    \item Cukup untuk mendeskripsikan sebagian besar token dalam bahasa pemrograman
\end{itemize}

\section{Finite Automata}

Finite automata adalah model matematika yang digunakan untuk mengenali string dalam suatu bahasa. Terdapat dua jenis utama: NFA (Nondeterministic Finite Automata) dan DFA (Deterministic Finite Automata).

\subsection{Definisi Formal}

\textbf{Finite Automaton} didefinisikan sebagai tuple $(Q, \Sigma, \delta, q_0, F)$ dimana:
\begin{itemize}
    \item $Q$: Himpunan state (keadaan) yang terbatas
    \item $\Sigma$: Alphabet (himpunan simbol input)
    \item $\delta$: Fungsi transisi (transition function)
    \item $q_0$: Start state (state awal)
    \item $F$: Himpunan accept states (final states)
\end{itemize}

\subsection{NFA (Nondeterministic Finite Automata)}

NFA memiliki karakteristik:
\begin{itemize}
    \item Untuk suatu state dan input symbol, dapat memiliki \textbf{nol, satu, atau lebih} transisi
    \item Dapat memiliki \textbf{$\epsilon$-transitions} (epsilon transitions) yang tidak mengonsumsi input
    \item Lebih mudah dikonstruksi dari regular expression
    \item Simulasi memerlukan backtracking atau multiple states tracking
\end{itemize}

Contoh NFA untuk pattern \texttt{a|b}:

\begin{verbatim}
        eps
    +-------+
    |   q0  |
    +---+----+
        |
    +----+---+
    |       |
    a       b
    |       |
    v       v
  [q1]    [q2]
\end{verbatim}

State q0 adalah start state, q1 dan q2 adalah accept states. Dari q0, dengan input 'a' dapat menuju q1, dengan input 'b' dapat menuju q2.

\subsection{DFA (Deterministic Finite Automata)}

DFA memiliki karakteristik:
\begin{itemize}
    \item Untuk setiap state dan input symbol, terdapat \textbf{tepat satu} transisi
    \item Tidak memiliki $\epsilon$-transitions
    \item Lebih efisien untuk simulasi (deterministic)
    \item Setiap NFA dapat dikonversi menjadi DFA yang ekuivalen
\end{itemize}

Contoh DFA untuk pattern \texttt{a|b}:

\begin{verbatim}
        a
    +-------+
    |   q0  |---+
    +---+----+   |
        |       | b
        |       |
        v       v
      [q1]    [q2]
\end{verbatim}

DFA ini deterministik: dari q0, input 'a' selalu menuju q1, input 'b' selalu menuju q2.

\subsection{Perbedaan NFA dan DFA}

Perbedaan utama antara NFA dan DFA:

\begin{table}[H]
\centering
\begin{tabular}{|l|p{5.5cm}|p{5.5cm}|}
\hline
\textbf{Aspek} & \textbf{NFA} & \textbf{DFA} \\
\hline
Transisi per state & Bisa 0, 1, atau lebih & Tepat 1 \\
\hline
$\epsilon$-transitions & Diizinkan & Tidak diizinkan \\
\hline
Efisiensi simulasi & Perlu backtracking & Linear time \\
\hline
Jumlah states & Biasanya lebih sedikit & Bisa lebih banyak \\
\hline
Konstruksi dari regex & Lebih mudah & Lebih kompleks \\
\hline
\end{tabular}
\caption{Perbandingan NFA dan DFA}
\label{tab:nfa-dfa}
\end{table}

\section{Konversi Regular Expression ke NFA: Algoritma Thompson}

Algoritma Thompson adalah metode sistematis untuk mengkonversi regular expression menjadi $\epsilon$-NFA. Algoritma ini menggunakan pendekatan rekursif dengan template untuk setiap operasi regex.

\subsection{Template Dasar}

\subsubsection{Literal (Karakter Tunggal)}

Untuk regex \texttt{a}, NFA-nya adalah:

\begin{verbatim}
    +---+  a   +---+
    |q0 |---→|q1 |
    +---+     +---+
\end{verbatim}

\subsubsection{Concatenation (RS)}

Untuk regex \texttt{RS}, NFA-nya dibangun dengan menghubungkan NFA untuk R dan S:

\begin{verbatim}
    [NFA untuk R] -$\epsilon$→ [NFA untuk S]
\end{verbatim}

\subsubsection{Union (R|S)}

Untuk regex \texttt{R|S}, NFA-nya menggunakan $\epsilon$-transitions untuk branching:

\begin{verbatim}
        +---+
        |q0 |
        +-+--+
      +----+---+
      |       |
      $\epsilon$       $\epsilon$
      |       |
      v       v
  [NFA R] [NFA S]
      |       |
      $\epsilon$       $\epsilon$
      |       |
      +---+----+
          v
        [q_f]
\end{verbatim}

\subsubsection{Kleene Star (R*)}

Untuk regex \texttt{R*}, NFA-nya memiliki loop dengan $\epsilon$-transitions:

\begin{verbatim}
    +---+
    |q0 |
    +-+--+
      |
      $\epsilon$
      |
      v
  +-------+
  |       |
  |  NFA  |
  |   R   |
  |       |
  +---+----+
      |
      $\epsilon$
      |
      v
    [q_f]←--+
      |     |
      +-$\epsilon$---+
\end{verbatim}

\subsection{Contoh: Konversi \texttt{(a|b)*abb}}

Mari kita konstruksi NFA untuk regex \texttt{(a|b)*abb} menggunakan algoritma Thompson:

\begin{enumerate}
    \item \textbf{Literal 'a' dan 'b'}: Buat NFA untuk masing-masing
    \item \textbf{Union (a|b)}: Gabungkan dengan $\epsilon$-transitions
    \item \textbf{Kleene Star ((a|b)*)}: Tambahkan loop dengan $\epsilon$-transitions
    \item \textbf{Concatenation dengan 'a'}: Tambahkan NFA untuk 'a'
    \item \textbf{Concatenation dengan 'b'}: Tambahkan NFA untuk 'b' (dua kali)
\end{enumerate}

Hasil akhirnya adalah NFA yang dapat mengenali string seperti ``abb'', ``aabb'', ``babb'', ``ababb'', dll.

\section{Konversi NFA ke DFA: Subset Construction}

Karena NFA tidak deterministik dan simulasi NFA bisa tidak efisien, kita perlu mengkonversi NFA menjadi DFA yang ekuivalen menggunakan algoritma \textbf{subset construction}.

\subsection{Konsep \texorpdfstring{$\epsilon$}{epsilon}-Closure}

Sebelum subset construction, kita perlu memahami konsep \textbf{$\epsilon$-closure}:

\begin{itemize}
    \item \textbf{$\epsilon$-closure} dari suatu state adalah himpunan semua state yang dapat dicapai dari state tersebut melalui $\epsilon$-transitions saja
    \item \textbf{$\epsilon$-closure} dari suatu set states adalah union dari $\epsilon$-closure setiap state dalam set tersebut
\end{itemize}

\subsection{Algoritma Subset Construction}

Algoritma subset construction bekerja sebagai berikut:

\begin{enumerate}
    \item \textbf{Start State DFA}: $\epsilon$-closure dari start state NFA
    \item \textbf{Untuk setiap state DFA dan setiap input symbol}:
    \begin{enumerate}
        \item Hitung semua NFA states yang dapat dicapai dengan input symbol tersebut
        \item Ambil $\epsilon$-closure dari set states tersebut
        \item Jika hasilnya belum ada sebagai state DFA, buat state baru
        \item Tambahkan transisi dari state DFA saat ini ke state hasil
    \end{enumerate}
    \item \textbf{Accept States DFA}: Setiap state DFA yang mengandung accept state NFA
\end{enumerate}

\subsection{Contoh: Konversi NFA \texttt{(a|b)*abb} ke DFA}

Mari kita ikuti langkah-langkah subset construction:

\begin{enumerate}
    \item \textbf{Start State DFA}: 
    \begin{itemize}
        \item Mulai dari start state NFA, ambil $\epsilon$-closure
        \item Misalkan hasilnya adalah set $\{q_0, q_1, q_2\}$ → ini menjadi state DFA $A$
    \end{itemize}
    
    \item \textbf{Transisi dari State A dengan input 'a'}:
    \begin{itemize}
        \item Dari semua NFA states dalam A, cari yang dapat menerima 'a'
        \item Ambil $\epsilon$-closure dari hasilnya → misalkan $\{q_3, q_4\}$ → state DFA $B$
    \end{itemize}
    
    \item \textbf{Transisi dari State A dengan input 'b'}:
    \begin{itemize}
        \item Dari semua NFA states dalam A, cari yang dapat menerima 'b'
        \item Ambil $\epsilon$-closure dari hasilnya → misalkan $\{q_5\}$ → state DFA $C$
    \end{itemize}
    
    \item \textbf{Lanjutkan untuk state B dan C} dengan cara yang sama
    \item \textbf{Accept States}: State DFA yang mengandung accept state NFA
\end{enumerate}

Hasilnya adalah DFA yang ekuivalen dengan NFA asli, tetapi deterministik dan lebih efisien untuk simulasi.

\section{Implementasi NFA dan DFA dalam C/C++}

Untuk memahami konsep secara praktis, kita akan melihat struktur data dan algoritma dasar untuk mengimplementasikan NFA dan DFA.

\subsection{Struktur Data NFA}

\begin{lstlisting}[language=C++, caption={Struktur Data untuk NFA}]
#include <vector>
#include <set>
#include <map>

struct NFATransition {
    int from_state;
    char symbol;  // '\0' untuk epsilon transition
    int to_state;
};

class NFA {
private:
    int num_states;
    int start_state;
    std::set<int> accept_states;
    std::vector<NFATransition> transitions;
    
public:
    // Konstruktor
    NFA(int states, int start);
    
    // Menambahkan transisi
    void addTransition(int from, char symbol, int to);
    
    // Menghitung epsilon closure
    std::set<int> epsilonClosure(const std::set<int>& states);
    
    // Simulasi NFA
    bool simulate(const std::string& input);
};
\end{lstlisting}

\subsection{Struktur Data DFA}

\begin{lstlisting}[language=C++, caption={Struktur Data untuk DFA}]
class DFA {
private:
    int num_states;
    int start_state;
    std::set<int> accept_states;
    std::map<std::pair<int, char>, int> transition_table;
    
public:
    // Konstruktor
    DFA(int states, int start);
    
    // Menambahkan transisi (deterministic)
    void addTransition(int from, char symbol, int to);
    
    // Simulasi DFA (lebih sederhana dari NFA)
    bool simulate(const std::string& input);
};
\end{lstlisting}

\subsection{Implementasi Simulasi DFA}

Simulasi DFA lebih sederhana karena deterministik:

\begin{lstlisting}[language=C++, caption={Simulasi DFA}]
bool DFA::simulate(const std::string& input) {
    int current_state = start_state;
    
    for (char c : input) {
        auto key = std::make_pair(current_state, c);
        if (transition_table.find(key) == transition_table.end()) {
            return false;  // Tidak ada transisi
        }
        current_state = transition_table[key];
    }
    
    return accept_states.find(current_state) != accept_states.end();
}
\end{lstlisting}

\subsection{Implementasi Subset Construction}

Berikut adalah pseudocode untuk subset construction:

\begin{lstlisting}[language=C++, caption={Subset Construction Algorithm}]
DFA NFA::toDFA() {
    DFA dfa(0, 0);
    std::map<std::set<int>, int> state_mapping;
    std::queue<std::set<int>> work_queue;
    
    // Start state DFA = epsilon closure dari start state NFA
    std::set<int> start_set = epsilonClosure({start_state});
    int dfa_start = dfa.addState();
    state_mapping[start_set] = dfa_start;
    work_queue.push(start_set);
    
    while (!work_queue.empty()) {
        std::set<int> nfa_states = work_queue.front();
        work_queue.pop();
        int dfa_state = state_mapping[nfa_states];
        
        // Untuk setiap input symbol
        for (char symbol : alphabet) {
            if (symbol == '\0') continue;  // Skip epsilon
            
            // Hitung move dengan symbol
            std::set<int> next_nfa_states;
            for (int state : nfa_states) {
                // Cari semua transisi dengan symbol ini
                for (auto& trans : transitions) {
                    if (trans.from_state == state && 
                        trans.symbol == symbol) {
                        next_nfa_states.insert(trans.to_state);
                    }
                }
            }
            
            // Ambil epsilon closure
            std::set<int> closure = epsilonClosure(next_nfa_states);
            
            if (!closure.empty()) {
                int next_dfa_state;
                if (state_mapping.find(closure) == state_mapping.end()) {
                    // State baru
                    next_dfa_state = dfa.addState();
                    state_mapping[closure] = next_dfa_state;
                    work_queue.push(closure);
                } else {
                    next_dfa_state = state_mapping[closure];
                }
                
                dfa.addTransition(dfa_state, symbol, next_dfa_state);
            }
        }
    }
    
    // Set accept states
    for (auto& pair : state_mapping) {
        for (int nfa_accept : accept_states) {
            if (pair.first.find(nfa_accept) != pair.first.end()) {
                dfa.setAcceptState(pair.second);
                break;
            }
        }
    }
    
    return dfa;
}
\end{lstlisting}

\section{Aplikasi dalam Lexical Analysis}

\subsection{Token Recognition dengan DFA}

Dalam lexical analysis, kita menggunakan DFA untuk mengenali token. Prosesnya:

\begin{enumerate}
    \item \textbf{Definisi Token}: Setiap jenis token didefinisikan dengan regular expression
    \item \textbf{Kombinasi Regex}: Semua regex untuk token digabungkan dengan union
    \item \textbf{Konversi ke DFA}: Regex gabungan dikonversi menjadi satu DFA
    \item \textbf{Scanning}: Input dibaca karakter demi karakter, DFA dijalankan
    \item \textbf{Longest Match}: Ambil token terpanjang yang cocok
    \item \textbf{Token Classification}: Tentukan jenis token berdasarkan accept state yang dicapai
\end{enumerate}

\subsection{Contoh: Recognizer untuk Identifier dan Number}

Mari kita buat recognizer sederhana untuk identifier dan number:

\begin{lstlisting}[language=C++, caption={Token Recognizer menggunakan DFA}]
enum TokenType {
    IDENTIFIER,
    NUMBER,
    UNKNOWN
};

class TokenRecognizer {
private:
    DFA identifier_dfa;  // DFA untuk [a-zA-Z_][a-zA-Z0-9_]*
    DFA number_dfa;      // DFA untuk [0-9]+
    
public:
    TokenRecognizer() {
        // Konstruksi DFA untuk identifier dan number
        // (dari regex menggunakan Thompson + subset construction)
    }
    
    TokenType recognize(const std::string& lexeme) {
        if (identifier_dfa.simulate(lexeme)) {
            return IDENTIFIER;
        } else if (number_dfa.simulate(lexeme)) {
            return NUMBER;
        } else {
            return UNKNOWN;
        }
    }
};
\end{lstlisting}

\subsection{Handling Multiple Tokens}

Ketika kita memiliki multiple token types, kita perlu:

\begin{enumerate}
    \item Membuat NFA terpisah untuk setiap token type
    \item Menggabungkan semua NFA dengan union, tetapi \textbf{label setiap accept state} dengan token type-nya
    \item Konversi ke DFA (setiap DFA state mungkin mengandung multiple NFA accept states dengan label berbeda)
    \item Saat scanning, jika mencapai accept state dengan multiple labels, gunakan \textbf{priority} atau \textbf{longest match}
\end{enumerate}

\section{Optimasi: DFA Minimization}

Setelah subset construction, DFA yang dihasilkan mungkin memiliki states yang redundan. Kita dapat meminimalkan DFA menggunakan algoritma seperti \textbf{Hopcroft's algorithm} atau \textbf{Moore's algorithm}.

\subsection{Konsep State Equivalence}

Dua states dalam DFA dikatakan \textbf{equivalent} jika:
\begin{itemize}
    \item Keduanya accept states ATAU keduanya bukan accept states
    \item Untuk setiap input symbol, transisi dari kedua states menuju ke states yang equivalent
\end{itemize}

\subsection{Algoritma Minimization}

Algoritma minimisasi bekerja dengan:
\begin{enumerate}
    \item Partisi states menjadi dua grup: accept states dan non-accept states
    \item Untuk setiap partisi, periksa apakah states dalam partisi tersebut equivalent
    \item Jika tidak equivalent, pisahkan menjadi partisi baru
    \item Ulangi sampai tidak ada partisi yang dapat dipisah lagi
    \item Merge states dalam partisi yang sama
\end{enumerate}

DFA yang sudah diminimalkan memiliki jumlah states minimum yang masih ekuivalen dengan DFA asli.

\section{Kesimpulan}

Dalam bab ini, kita telah mempelajari:

\begin{enumerate}
    \item Regular expression adalah notasi formal untuk mendeskripsikan pola token
    \item Finite automata (NFA dan DFA) adalah model matematika untuk mengenali regular language
    \item Algoritma Thompson mengkonversi regular expression menjadi $\epsilon$-NFA
    \item Subset construction mengkonversi NFA menjadi DFA yang ekuivalen
    \item DFA lebih efisien untuk simulasi dan digunakan dalam lexical analysis
    \item Implementasi praktis memerlukan struktur data yang tepat dan algoritma yang efisien
\end{enumerate}

Pemahaman tentang regular expression dan finite automata ini menjadi dasar penting untuk implementasi lexical analyzer yang akan dipelajari dalam bab-bab selanjutnya.

\section{Latihan}

\begin{enumerate}
    \item Buatlah regular expression untuk:
    \begin{itemize}
        \item Email address sederhana (format: \texttt{user@domain.com})
        \item Phone number (format: \texttt{+62-812-3456-7890})
        \item C-style comment (\texttt{/* ... */})
    \end{itemize}
    
    \item Konstruksi NFA untuk regular expression berikut menggunakan algoritma Thompson:
    \begin{itemize}
        \item \texttt{a*b+}
        \item \texttt{(a|b)*ab}
        \item \texttt{[0-9]+(\textbackslash.[0-9]+)?}
    \end{itemize}
    
    \item Konversi NFA dari soal nomor 2 menjadi DFA menggunakan subset construction. Gambarkan state diagram untuk DFA yang dihasilkan.
    
    \item Implementasikan kelas \texttt{NFA} dan \texttt{DFA} dalam C++ dengan fungsi:
    \begin{itemize}
        \item Konstruksi NFA dari regular expression (sederhana)
        \item Konversi NFA ke DFA
        \item Simulasi DFA untuk string input
    \end{itemize}
    
    \item Buat program recognizer yang dapat mengenali token-token berikut:
    \begin{itemize}
        \item Identifier: \texttt{[a-zA-Z\_][a-zA-Z0-9\_]*}
        \item Integer: \texttt{[0-9]+}
        \item Float: \texttt{[0-9]+\textbackslash.[0-9]+}
        \item Operator: \texttt{+|-|*|/|=|==|!=}
    \end{itemize}
    
    \item Jelaskan mengapa DFA lebih efisien untuk simulasi dibanding NFA. Berikan contoh kompleksitas waktu untuk keduanya.
    
    \item Implementasikan algoritma minimisasi DFA (dapat menggunakan versi sederhana). Uji dengan DFA yang dihasilkan dari soal nomor 3.
\end{enumerate}

\section{Referensi dan Bahan Bacaan Lanjutan}

Untuk memperdalam pemahaman tentang regular expression dan finite automata, mahasiswa disarankan membaca:

\begin{itemize}
    \item \textbf{Dragon Book}: Aho, Lam, Sethi, \& Ullman (2006). \textit{Compilers: Principles, Techniques, and Tools} \cite{aho2006compilers} - Bab 3: Lexical Analysis
    
    \item \textbf{Engineering a Compiler}: Cooper \& Torczon (2011) \cite{cooper2011engineering} - Bab 2: Scanners
    
    \item \textbf{Aoyama Gakuin University}: Lecture notes tentang lexical analysis dan finite automata \cite{aoyama2024lexical}
    
    \item \textbf{OpenGenus}: Tutorial tentang membangun lexer \cite{opengenus2024lexer}
    
    \item \textbf{GeeksforGeeks}: Artikel tentang regular expression to NFA dan NFA to DFA conversion
\end{itemize}

\cleardoublepage
\documentclass[../main.tex]{subfiles}

% Untuk kompilasi standalone, tambahkan \addbibresource di preamble
% Menggunakan \subfix untuk menangani path relatif dengan benar
% biblatex akan mengabaikan duplikasi jika sudah ada di main.tex
\addbibresource{\subfix{../references.bib}}

\begin{document}

% Set chapter counter untuk kompilasi standalone
% \chapter akan mengincrement counter, jadi untuk bab-03 (Bab 3), set ke 2
% Saat dikompilasi dari main.tex, counter akan diatur oleh subfiles package
\ifSubfilesClassLoaded{%
    \setcounter{chapter}{2}%
    % Untuk kompilasi standalone, gunakan refsection manual
    \begin{refsection}
}{}

\chapter{Implementasi Lexer Sederhana (Hand-Written)}
\label{chap:lexer-handwritten}

\section{Tujuan Pembelajaran}

Setelah mempelajari bab ini, mahasiswa diharapkan mampu:
\begin{enumerate}
    \item Memahami konsep dan struktur hand-written lexer
    \item Merancang state machine untuk token recognition
    \item Mengimplementasikan lexer sederhana dalam C/C++ untuk subset bahasa C
    \item Menangani whitespace, komentar (single-line dan multi-line), dan escape sequences
    \item Mengimplementasikan error handling untuk token tidak valid
    \item Membuat unit test untuk berbagai kasus input
\end{enumerate}
\section{Pendahuluan}

Setelah memahami konsep lexical analysis secara teori pada bab sebelumnya, pada bab ini kita akan mengimplementasikan lexer secara praktis menggunakan pendekatan \textbf{hand-written} (ditulis manual). Menurut sumber terbuka:

\begin{quote}
``Hand-written lexers are possible: directly code a state machine, or use manual scanning logic. Requires careful handling of edge cases (e.g. unclosed strings/comments).''\cite{opengenus2024lexer}
\end{quote}

Pendekatan hand-written memberikan kontrol penuh terhadap implementasi dan sangat berguna untuk pembelajaran karena mahasiswa dapat memahami setiap detail proses tokenization. Meskipun lebih kompleks dibanding menggunakan generator seperti Flex atau re2c, hand-written lexer memberikan fleksibilitas dan pemahaman yang lebih dalam.

Gambar \ref{fig:handwritten-vs-generator} menunjukkan perbandingan antara hand-written lexer dan lexer generator.

\begin{figure}[!htbp]
    \centering
    \adjustbox{max width=0.9\textwidth,center}{%
    \begin{tikzpicture}[
        box/.style={rectangle, draw=blue!50, fill=blue!10, text width=2.5cm, text centered, minimum height=0.7cm, rounded corners, font=\footnotesize, inner sep=4pt, align=center},
        arrow/.style={->, >=stealth, thick},
        title/.style={font=\bfseries\small},
        node distance=0.6cm and 0.3cm
    ]
    
    % Hand-written column
    \node[title] (hw-title) {HAND-WRITTEN LEXER};
    
    \node[box, below=of hw-title] (hw1) {Source\\Code};
    \node[box, below=of hw1] (hw2) {Manual\\Implementation};
    \node[box, below=of hw2] (hw3) {Direct\\Control};
    \node[box, below=of hw3] (hw4) {Token\\Stream};
    
    \draw[arrow] (hw1) -- (hw2);
    \draw[arrow] (hw2) -- (hw3);
    \draw[arrow] (hw3) -- (hw4);
    
    \node[below=0.2cm of hw4, font=\tiny, align=center, text width=3cm]
    (hw-note) {Pros: Full control,\\learning value\\Cons: More code};
    
    % Generator column
    \node[title, right=4cm of hw-title] (gen-title) {LEXER GENERATOR};
    
    \node[box, below=of gen-title] (gen1) {Regex\\Spec};
    \node[box, below=of gen1] (gen2) {Generator\\(Flex/re2c)};
    \node[box, below=of gen2] (gen3) {Auto\\Generated};
    \node[box, below=of gen3] (gen4) {Token\\Stream};
    
    \draw[arrow] (gen1) -- (gen2);
    \draw[arrow] (gen2) -- (gen3);
    \draw[arrow] (gen3) -- (gen4);
    
    \node[below=0.2cm of gen4, font=\tiny, align=center, text width=3cm]
    (gen-note) {Pros: Less code,\\maintainable\\Cons: Less control};
    
    \end{tikzpicture}%
    }
    \caption{Perbandingan hand-written lexer vs lexer generator}
    \label{fig:handwritten-vs-generator}
\end{figure}

Gambar \ref{fig:lexer-overview} menunjukkan alur umum proses tokenization dalam hand-written lexer.

\begin{figure}[!htbp]
    \centering
    \adjustbox{max width=0.9\textwidth,center}{%
    \begin{tikzpicture}[
        box/.style={rectangle, draw=blue!50, fill=blue!10, text width=2.5cm, text centered, minimum height=0.8cm, rounded corners, font=\footnotesize, inner sep=4pt, align=center},
        arrow/.style={->, >=stealth, thick},
        node distance=1.5cm
    ]
    
    \node[box] (source) {Source\\Code};
    \node[box, right=of source] (lexer) {Lexer\\(State Machine)};
    \node[box, right=of lexer] (tokens) {Token\\Stream};
    \node[box, below=of tokens] (parser) {Parser};
    
    \draw[arrow] (source) -- node[above, font=\tiny, align=center] {Character\\by Character} (lexer);
    \draw[arrow] (lexer) -- node[above, font=\tiny, align=center] {Token\\Recognition} (tokens);
    \draw[arrow] (tokens) -- (parser);
    
    \node[below=0.3cm of lexer, font=\tiny, align=center, text width=2.5cm]
    (note) {Pattern\\Matching};
    
    \end{tikzpicture}%
    }
    \caption{Alur umum proses tokenization dalam hand-written lexer}
    \label{fig:lexer-overview}
\end{figure}
\section{Struktur Token}

Sebelum mengimplementasikan lexer, kita perlu mendefinisikan struktur data untuk merepresentasikan token. Token minimal harus menyimpan:

\begin{enumerate}
    \item \textbf{Token Type}: Jenis token (identifier, keyword, number, operator, dll.)
    \item \textbf{Lexeme}: String aktual yang di-match dari source code
    \item \textbf{Position Information}: Baris dan kolom untuk error reporting
    \item \textbf{Value} (opsional): Nilai numerik untuk number literals
\end{enumerate}

Gambar \ref{fig:token-structure} menunjukkan struktur data token secara visual.

\begin{figure}[!htbp]
    \centering
    \adjustbox{max width=0.7\textwidth,center}{%
    \begin{tikzpicture}[
        box/.style={rectangle, draw=blue!50, fill=blue!10, text width=2.5cm, text centered, minimum height=0.6cm, rounded corners, font=\footnotesize, inner sep=4pt, align=center},
        field/.style={rectangle, draw=green!50, fill=green!10, text width=2.2cm, text centered, minimum height=0.5cm, font=\tiny, inner sep=2pt, align=center},
        arrow/.style={->, >=stealth, thick},
        node distance=0.4cm and 0.3cm
    ]
    
    \node[box] (token) {Token};
    
    \node[field, below=of token] (type) {type\\TokenType};
    \node[field, below=of type] (lexeme) {lexeme\\string};
    \node[field, below=of lexeme] (line) {line\\int};
    \node[field, below=of line] (column) {column\\int};
    
    \draw[arrow] (token) -- (type);
    \draw[arrow] (type) -- (lexeme);
    \draw[arrow] (lexeme) -- (line);
    \draw[arrow] (line) -- (column);
    
    \end{tikzpicture}%
    }
    \caption{Struktur data Token}
    \label{fig:token-structure}
\end{figure}

\subsection{Token Types}

Token types dapat didefinisikan menggunakan enum. Berikut contoh untuk subset bahasa C:

Gambar \ref{fig:token-types-hierarchy} menunjukkan hierarki token types yang digunakan dalam lexer.

\begin{figure}[!htbp]
    \centering
    \adjustbox{max width=0.8\textwidth,center}{%
    \begin{tikzpicture}[
        node distance=1.2cm and 1.8cm,
        category/.style={rectangle, draw=blue!50, fill=blue!10, rounded corners, font=\footnotesize, align=center, minimum height=0.6cm, inner sep=4pt},
        token/.style={rectangle, draw=green!50, fill=green!10, rounded corners, font=\tiny, align=center, minimum height=0.45cm, inner sep=3pt},
        arrow/.style={->, >=stealth, thick}
    ]
    
    % Root
    \node[category] (root) {Token};
    
    % Level 1 categories
    \node[category, below left=of root] (kw) {Keyword};
    \node[category, below=of root] (id) {Identifier};
    \node[category, below right=of root] (lit) {Literal};
    \node[category, right=3.5cm of id] (op) {Operator};
    \node[category, left=3.5cm of id] (pun) {Punctuation};
    
    % Level 2 examples
    \node[token, below=of kw] (kw1) {int, float};
    \node[token, below=0.4cm of kw1] (kw2) {if, else};
    
    \node[token, below=of id] (id1) {variable names};
    
    \node[token, below=of lit] (lit1) {integer literal};
    \node[token, below=0.4cm of lit1] (lit2) {float literal};
    \node[token, below=0.4cm of lit2] (lit3) {string literal};
    
    \node[token, below=of op] (op1) {+, -, *, /};
    \node[token, below=0.4cm of op1] (op2) {==, !=, <, >};
    
    \node[token, below=of pun] (p1) {;, ,, (, )};
    \node[token, below=0.4cm of p1] (p2) {\{, \}, [, ]};
    
    % Arrows (true hierarchy)
    \draw[arrow] (root) -- (kw);
    \draw[arrow] (root) -- (id);
    \draw[arrow] (root) -- (lit);
    \draw[arrow] (root) -- (op);
    \draw[arrow] (root) -- (pun);
    
    \draw[arrow] (kw) -- (kw1);
    \draw[arrow] (id) -- (id1);
    \draw[arrow] (lit) -- (lit1);
    \draw[arrow] (op) -- (op1);
    \draw[arrow] (pun) -- (p1);
    
    \end{tikzpicture}%
    }
    \caption{Hierarki tipe token dalam lexer}
    \label{fig:token-types-hierarchy}
    \end{figure}
    
    
    

\begin{lstlisting}[language=C++, caption=Definisi Token Types]
enum class TokenType {
    // Identifiers and Keywords
    IDENTIFIER,
    KEYWORD_INT, KEYWORD_FLOAT, KEYWORD_IF, KEYWORD_ELSE,
    KEYWORD_WHILE, KEYWORD_FOR, KEYWORD_RETURN,
    
    // Literals
    INTEGER_LITERAL,
    FLOAT_LITERAL,
    STRING_LITERAL,
    CHAR_LITERAL,
    
    // Operators
    OP_PLUS, OP_MINUS, OP_MULTIPLY, OP_DIVIDE,
    OP_ASSIGN, OP_EQUAL, OP_NOT_EQUAL,
    OP_LESS, OP_LESS_EQUAL, OP_GREATER, OP_GREATER_EQUAL,
    OP_AND, OP_OR, OP_NOT,
    
    // Punctuation
    SEMICOLON, COMMA, DOT,
    LPAREN, RPAREN,    // ( )
    LBRACE, RBRACE,    // { }
    LBRACKET, RBRACKET, // [ ]
    
    // Special
    END_OF_FILE,
    INVALID
};
\end{lstlisting}

\subsection{Token Structure}

Struktur token dalam C++ dapat didefinisikan sebagai berikut:

\begin{lstlisting}[language=C++, caption=Struktur Token]
struct Token {
    TokenType type;
    std::string lexeme;
    int line;
    int column;
    union {
        int intValue;      // Untuk INTEGER_LITERAL
        double floatValue; // Untuk FLOAT_LITERAL
    };
    
    Token(TokenType t, const std::string& lex, int l, int c)
        : type(t), lexeme(lex), line(l), column(c) {}
};
\end{lstlisting}
\section{Finite State Machine untuk Lexer}

Lexical analysis secara fundamental adalah proses pattern matching yang dapat dimodelkan menggunakan \textbf{Finite State Machine (FSM)} atau \textbf{Finite Automata}. Menurut sumber dari Aoyama Gakuin University:

\begin{quote}
``Lexical analysis breaks input text into lexemes which correspond to tokens. Usually implemented using regular languages → regex → NFA → DFA → (minimized) DFA for efficiency.''\cite{aoyama2024lexical}
\end{quote}

Dalam implementasi hand-written, kita tidak perlu membuat DFA secara eksplisit, tetapi kita menggunakan logika state machine dalam kode.

\subsection{State Machine Design}

State machine untuk lexer sederhana dapat memiliki state-state berikut:

\begin{itemize}
    \item \textbf{START}: State awal, menunggu karakter pertama dari token
    \item \textbf{IN\_IDENTIFIER}: Sedang membaca identifier atau keyword
    \item \textbf{IN\_NUMBER}: Sedang membaca angka (integer atau float)
    \item \textbf{IN\_FLOAT}: Setelah menemukan titik desimal
    \item \textbf{IN\_STRING}: Sedang membaca string literal
    \item \textbf{IN\_CHAR}: Sedang membaca character literal
    \item \textbf{IN\_COMMENT\_LINE}: Sedang membaca single-line comment
    \item \textbf{IN\_COMMENT\_BLOCK}: Sedang membaca multi-line comment
    \item \textbf{IN\_OPERATOR}: Sedang membaca operator (mungkin multi-character)
    \item \textbf{DONE}: Token selesai dibaca
\end{itemize}

Gambar \ref{fig:lexer-state-machine} menunjukkan state machine untuk lexer sederhana.

\begin{figure}[H]
    \centering
    \adjustbox{max width=0.95\textwidth,center}{%
    \begin{tikzpicture}[
        state/.style={circle, draw=blue!50, fill=blue!10, minimum size=0.8cm, font=\tiny, align=center},
        start/.style={circle, draw=red!50, fill=red!10, minimum size=0.8cm, font=\tiny, align=center},
        done/.style={circle, draw=green!50, fill=green!10, minimum size=0.8cm, font=\tiny, double, align=center},
        arrow/.style={->, >=stealth, thick},
        node distance=2cm and 1.5cm
    ]
    
    \node[start] (start) {START};
    
    \node[state, above right=of start] (id) {IN\_\\IDENTIFIER};
    \node[state, right=of start] (num) {IN\_\\NUMBER};
    \node[state, below right=of start] (str) {IN\_\\STRING};
    \node[state, below=of str] (chr) {IN\_\\CHAR};
    \node[state, above=of id] (comment1) {IN\_COMMENT\\\_LINE};
    \node[state, right=of comment1] (comment2) {IN\_COMMENT\\\_BLOCK};
    \node[state, below=of num] (op) {IN\_\\OPERATOR};
    \node[state, right=of num] (float) {IN\_\\FLOAT};
    
    \node[done, right=of float] (done) {DONE};
    
    % Transitions from START
    \draw[arrow] (start) to[out=45, in=180] node[above, font=\tiny, align=center] {letter/\\\_} (id);
    \draw[arrow] (start) to[out=0, in=180] node[above, font=\tiny] {digit} (num);
    \draw[arrow] (start) to[out=-45, in=180] node[below, font=\tiny] {"} (str);
    \draw[arrow] (start) to[out=-90, in=90] node[left, font=\tiny] {'} (chr);
    \draw[arrow] (start) to[out=90, in=180] node[left, font=\tiny] {//} (comment1);
    \draw[arrow] (start) to[out=90, in=180] node[above, font=\tiny] {/*} (comment2);
    \draw[arrow] (start) to[out=-30, in=180] node[below, font=\tiny] {op} (op);
    
    % Transitions to DONE
    \draw[arrow] (id) to[out=0, in=180] node[above, font=\tiny, align=center] {non-\\alnum} (done);
    \draw[arrow] (num) to[out=0, in=180] node[above, font=\tiny, align=center] {non-\\digit} (done);
    \draw[arrow] (str) to[out=0, in=-90] node[right, font=\tiny] {"} (done);
    \draw[arrow] (chr) to[out=0, in=-90] node[right, font=\tiny] {'} (done);
    \draw[arrow] (op) to[out=0, in=-90] node[right, font=\tiny] {done} (done);
    
    % Number to Float
    \draw[arrow] (num) to[out=0, in=180] node[above, font=\tiny] {.} (float);
    \draw[arrow] (float) to[out=0, in=180] node[above, font=\tiny, align=center] {non-\\digit} (done);
    
    % Comments (skip, no token)
    \draw[arrow, dashed] (comment1) to[out=0, in=90] node[right, font=\tiny] {newline} (start);
    \draw[arrow, dashed] (comment2) to[out=-90, in=90] node[right, font=\tiny] {*/} (start);
    
    \end{tikzpicture}%
    }
    \caption{State machine untuk hand-written lexer}
    \label{fig:lexer-state-machine}
\end{figure}

\subsection{State Transitions}

Transisi state terjadi berdasarkan karakter yang dibaca:

\begin{enumerate}
    \item \textbf{START} → \textbf{IN\_IDENTIFIER}: Jika karakter adalah huruf atau underscore
    \item \textbf{START} → \textbf{IN\_NUMBER}: Jika karakter adalah digit
    \item \textbf{START} → \textbf{IN\_STRING}: Jika karakter adalah double quote (\texttt{"})
    \item \textbf{START} → \textbf{IN\_CHAR}: Jika karakter adalah single quote (\texttt{'})
    \item \textbf{START} → \textbf{IN\_COMMENT\_LINE}: Jika menemukan \texttt{//}
    \item \textbf{START} → \textbf{IN\_COMMENT\_BLOCK}: Jika menemukan \texttt{/*}
    \item \textbf{START} → \textbf{IN\_OPERATOR}: Jika karakter adalah operator
    \item \textbf{IN\_NUMBER} → \textbf{IN\_FLOAT}: Jika menemukan titik desimal
    \item \textbf{IN\_IDENTIFIER} → \textbf{DONE}: Jika karakter bukan alphanumeric atau underscore
    \item \textbf{IN\_NUMBER} → \textbf{DONE}: Jika karakter bukan digit atau titik
    \item \textbf{IN\_STRING} → \textbf{DONE}: Jika menemukan closing quote (dengan handling escape)
\end{enumerate}

Gambar \ref{fig:tokenization-flowchart} menunjukkan flowchart proses tokenization.

\begin{figure}[H]
    \centering
    \adjustbox{max width=0.85\textwidth,center}{%
    \begin{tikzpicture}[
        start/.style={ellipse, draw=red!50, fill=red!10, minimum width=1.5cm, minimum height=0.7cm, font=\footnotesize, align=center},
        process/.style={rectangle, draw=blue!50, fill=blue!10, minimum width=2cm, minimum height=0.7cm, font=\footnotesize, align=center},
        decision/.style={diamond, draw=orange!50, fill=orange!10, minimum width=1.5cm, minimum height=0.7cm, font=\tiny, align=center},
        end/.style={ellipse, draw=green!50, fill=green!10, minimum width=1.5cm, minimum height=0.7cm, font=\footnotesize, align=center},
        arrow/.style={->, >=stealth, thick},
        node distance=0.8cm and 1.2cm
    ]
    
    \node[start] (begin) {Start};
    \node[process, below=of begin] (skip) {Skip\\Whitespace};
    \node[decision, below=of skip] (eof) {EOF?};
    \node[process, right=of eof] (scan) {Scan\\Token};
    \node[decision, below=of scan] (valid) {Valid?};
    \node[process, right=of valid] (add) {Add to\\Stream};
    \node[process, below=of add] (return) {Return\\Token};
    \node[end, left=of return] (end) {End};
    
    \draw[arrow] (begin) -- (skip);
    \draw[arrow] (skip) -- (eof);
    \draw[arrow] (eof) -- node[above, font=\tiny] {No} (scan);
    \draw[arrow] (eof) -- node[left, font=\tiny] {Yes} (end);
    \draw[arrow] (scan) -- (valid);
    \draw[arrow] (valid) -- node[above, font=\tiny] {Yes} (add);
    \draw[arrow] (add) -- (return);
    \draw[arrow] (return) to[out=180, in=0] (skip);
    \draw[arrow] (valid) -- node[right, font=\tiny] {No} node[left, font=\tiny] {Error} (end);
    
    \end{tikzpicture}%
    }
    \caption{Flowchart proses tokenization}
    \label{fig:tokenization-flowchart}
\end{figure}
\section{Implementasi Lexer dalam C++}

Berikut adalah implementasi lengkap lexer sederhana untuk subset bahasa C:

\subsection{Kelas Lexer}

Gambar \ref{fig:lexer-architecture} menunjukkan arsitektur kelas Lexer dan komponen-komponennya.

\begin{figure}[!htbp]
    \centering
    \adjustbox{max width=0.9\textwidth,center}{%
    \begin{tikzpicture}[
        class/.style={rectangle, draw=blue!50, fill=blue!10, text width=3cm, minimum height=1cm, font=\footnotesize, align=center, rounded corners},
        method/.style={rectangle, draw=green!50, fill=green!10, text width=2.5cm, minimum height=0.5cm, font=\tiny, align=center},
        data/.style={rectangle, draw=orange!50, fill=orange!10, text width=2.5cm, minimum height=0.5cm, font=\tiny, align=center},
        arrow/.style={->, >=stealth, thick},
        node distance=0.4cm and 0.3cm
    ]
    
    % Main class
    \node[class] (lexer) {Lexer Class};
    
    % Private members
    \node[data, below left=0.5cm and 0.5cm of lexer] (d1) {input\\position\\line, column};
    \node[data, below right=0.5cm and 0.5cm of lexer] (d2) {keywords\\set};
    
    % Private methods
    \node[method, below=1.5cm of d1] (m1) {peek()\\get()};
    \node[method, right=of m1] (m2) {skipWhitespace()};
    \node[method, right=of m2] (m3) {scanIdentifier()};
    \node[method, below=of m1] (m4) {scanNumber()};
    \node[method, right=of m4] (m5) {scanString()};
    \node[method, right=of m5] (m6) {scanOperator()};
    
    % Public methods
    \node[method, above=1.5cm of lexer, draw=red!50, fill=red!10] (p1) {nextToken()};
    \node[method, right=of p1, draw=red!50, fill=red!10] (p2) {tokenize()};
    
    % Arrows
    \draw[arrow] (lexer) -- (d1);
    \draw[arrow] (lexer) -- (d2);
    \draw[arrow] (lexer) -- (m1);
    \draw[arrow] (lexer) -- (m2);
    \draw[arrow] (lexer) -- (m3);
    \draw[arrow] (lexer) -- (m4);
    \draw[arrow] (lexer) -- (m5);
    \draw[arrow] (lexer) -- (m6);
    \draw[arrow] (p1) -- (lexer);
    \draw[arrow] (p2) -- (lexer);
    
    \end{tikzpicture}%
    }
    \caption{Arsitektur kelas Lexer}
    \label{fig:lexer-architecture}
\end{figure}

\begin{lstlisting}[language=C++, caption=Header File: lexer.h]
#ifndef LEXER_H
#define LEXER_H

#include <string>
#include <unordered_set>
#include <vector>

enum class TokenType {
    IDENTIFIER,
    KEYWORD_INT, KEYWORD_FLOAT, KEYWORD_IF, KEYWORD_ELSE,
    KEYWORD_WHILE, KEYWORD_FOR, KEYWORD_RETURN,
    INTEGER_LITERAL, FLOAT_LITERAL,
    STRING_LITERAL, CHAR_LITERAL,
    OP_PLUS, OP_MINUS, OP_MULTIPLY, OP_DIVIDE,
    OP_ASSIGN, OP_EQUAL, OP_NOT_EQUAL,
    OP_LESS, OP_LESS_EQUAL, OP_GREATER, OP_GREATER_EQUAL,
    OP_AND, OP_OR, OP_NOT,
    SEMICOLON, COMMA, DOT,
    LPAREN, RPAREN, LBRACE, RBRACE,
    LBRACKET, RBRACKET,
    END_OF_FILE, INVALID
};

struct Token {
    TokenType type;
    std::string lexeme;
    int line;
    int column;
    
    Token(TokenType t, const std::string& lex, int l, int c)
        : type(t), lexeme(lex), line(l), column(c) {}
};

class Lexer {
private:
    std::string input;
    size_t position;
    int line;
    int column;
    std::unordered_set<std::string> keywords;
    
    char peek() const;
    char get();
    void skipWhitespace();
    void skipLineComment();
    void skipBlockComment();
    Token scanIdentifier();
    Token scanNumber();
    Token scanString();
    Token scanChar();
    Token scanOperator();
    TokenType getKeywordType(const std::string& lexeme) const;
    
public:
    Lexer(const std::string& source);
    Token nextToken();
    std::vector<Token> tokenize();
};

#endif
\end{lstlisting}

\subsection{Implementasi Lexer}

\begin{lstlisting}[language=C++, caption=Implementasi: lexer.cpp (Bagian 1)]
#include "lexer.h"
#include <cctype>
#include <stdexcept>

Lexer::Lexer(const std::string& source) 
    : input(source), position(0), line(1), column(1) {
    // Initialize keywords
    keywords = {"int", "float", "if", "else", 
                "while", "for", "return"};
}

char Lexer::peek() const {
    if (position >= input.length()) {
        return '\0';
    }
    return input[position];
}

char Lexer::get() {
    if (position >= input.length()) {
        return '\0';
    }
    char c = input[position++];
    if (c == '\n') {
        line++;
        column = 1;
    } else {
        column++;
    }
    return c;
}
\end{lstlisting}

\subsection{Handling Whitespace dan Komentar}

\begin{lstlisting}[language=C++, caption=Implementasi: lexer.cpp (Bagian 2 - Whitespace dan Comments)]
void Lexer::skipWhitespace() {
    while (position < input.length()) {
        char c = peek();
        if (std::isspace(c)) {
            get();
        } else if (c == '/' && position + 1 < input.length() 
                   && input[position + 1] == '/') {
            skipLineComment();
        } else if (c == '/' && position + 1 < input.length() 
                   && input[position + 1] == '*') {
            skipBlockComment();
        } else {
            break;
        }
    }
}

void Lexer::skipLineComment() {
    // Skip "//"
    get(); get();
    // Skip until newline or EOF
    while (peek() != '\n' && peek() != '\0') {
        get();
    }
}

void Lexer::skipBlockComment() {
    // Skip "/*"
    get(); get();
    while (position < input.length()) {
        if (peek() == '*' && position + 1 < input.length() 
            && input[position + 1] == '/') {
            get(); get(); // Skip "*/"
            return;
        }
        get();
    }
    // Error: unclosed comment
    throw std::runtime_error("Unclosed block comment at line " 
                            + std::to_string(line));
}
\end{lstlisting}

\subsection{Scanning Identifier dan Keyword}

\begin{lstlisting}[language=C++, caption=Implementasi: lexer.cpp (Bagian 3 - Identifier)]
Token Lexer::scanIdentifier() {
    int startLine = line;
    int startCol = column;
    std::string lexeme;
    
    // First character must be letter or underscore
    if (std::isalpha(peek()) || peek() == '_') {
        lexeme += get();
    }
    
    // Subsequent characters can be alphanumeric or underscore
    while (std::isalnum(peek()) || peek() == '_') {
        lexeme += get();
    }
    
    // Check if it's a keyword
    TokenType type = getKeywordType(lexeme);
    if (type != TokenType::IDENTIFIER) {
        return Token(type, lexeme, startLine, startCol);
    }
    
    return Token(TokenType::IDENTIFIER, lexeme, startLine, startCol);
}

TokenType Lexer::getKeywordType(const std::string& lexeme) const {
    if (lexeme == "int") return TokenType::KEYWORD_INT;
    if (lexeme == "float") return TokenType::KEYWORD_FLOAT;
    if (lexeme == "if") return TokenType::KEYWORD_IF;
    if (lexeme == "else") return TokenType::KEYWORD_ELSE;
    if (lexeme == "while") return TokenType::KEYWORD_WHILE;
    if (lexeme == "for") return TokenType::KEYWORD_FOR;
    if (lexeme == "return") return TokenType::KEYWORD_RETURN;
    return TokenType::IDENTIFIER;
}
\end{lstlisting}

\subsection{Scanning Number Literals}

\begin{lstlisting}[language=C++, caption=Implementasi: lexer.cpp (Bagian 4 - Numbers)]
Token Lexer::scanNumber() {
    int startLine = line;
    int startCol = column;
    std::string lexeme;
    bool isFloat = false;
    
    // Read integer part
    while (std::isdigit(peek())) {
        lexeme += get();
    }
    
    // Check for decimal point
    if (peek() == '.') {
        lexeme += get();
        isFloat = true;
        
        // Read fractional part
        while (std::isdigit(peek())) {
            lexeme += get();
        }
    }
    
    // Check for exponent (optional, for future enhancement)
    if (peek() == 'e' || peek() == 'E') {
        lexeme += get();
        if (peek() == '+' || peek() == '-') {
            lexeme += get();
        }
        while (std::isdigit(peek())) {
            lexeme += get();
        }
        isFloat = true;
    }
    
    TokenType type = isFloat ? TokenType::FLOAT_LITERAL 
                              : TokenType::INTEGER_LITERAL;
    return Token(type, lexeme, startLine, startCol);
}
\end{lstlisting}

\subsection{Scanning String dan Character Literals}

Gambar \ref{fig:string-escape} menunjukkan contoh handling escape sequences dalam string literal.

\begin{figure}[!htbp]
    \centering
    \adjustbox{max width=0.85\textwidth,center}{%
    \begin{tikzpicture}[
        char/.style={rectangle, draw=blue!50, fill=blue!10, minimum width=0.5cm, minimum height=0.5cm, font=\footnotesize\ttfamily, align=center},
        esc/.style={rectangle, draw=red!50, fill=red!10, minimum width=0.5cm, minimum height=0.5cm, font=\footnotesize\ttfamily, align=center},
        result/.style={rectangle, draw=green!50, fill=green!10, minimum width=2cm, minimum height=0.6cm, font=\tiny, align=center},
        arrow/.style={->, >=stealth, thick},
        node distance=0.15cm and 0.1cm
    ]
    
    % Example: "hello\nworld"
    \node[font=\tiny\bfseries] (label1) {Input: \texttt{"hello\textbackslash nworld"}};
    \node[char, below=0.2cm of label1] (c1) {"};
    \node[char, right=of c1] (c2) {h};
    \node[char, right=of c2] (c3) {e};
    \node[char, right=of c3] (c4) {l};
    \node[char, right=of c4] (c5) {l};
    \node[char, right=of c5] (c6) {o};
    \node[esc, right=of c6] (c7) {\textbackslash};
    \node[esc, right=of c7] (c8) {n};
    \node[char, right=of c8] (c9) {w};
    \node[char, right=of c9] (c10) {o};
    \node[char, right=of c10] (c11) {r};
    \node[char, right=of c11] (c12) {l};
    \node[char, right=of c12] (c13) {d};
    \node[char, right=of c13] (c14) {"};
    
    \node[result, below=0.4cm of c7] (r1) {Escape\\Sequence};
    \draw[arrow, red] (c7) -- (r1);
    \draw[arrow, red] (c8) -- (r1);
    
    \node[result, below=0.6cm of r1, font=\tiny] {Result: STRING\_LITERAL\\"hello\textbackslash nworld"\\(newline character)};
    
    \end{tikzpicture}%
    }
    \caption{Handling escape sequences dalam string literal}
    \label{fig:string-escape}
\end{figure}

\begin{lstlisting}[language=C++, caption=Implementasi: lexer.cpp (Bagian 5 - Strings)]
Token Lexer::scanString() {
    int startLine = line;
    int startCol = column;
    std::string lexeme;
    
    // Consume opening quote
    get(); // Skip opening "
    
    while (peek() != '"' && peek() != '\0') {
        if (peek() == '\\') {
            // Handle escape sequences
            get(); // Skip backslash
            char escaped = get();
            switch (escaped) {
                case 'n': lexeme += '\n'; break;
                case 't': lexeme += '\t'; break;
                case 'r': lexeme += '\r'; break;
                case '\\': lexeme += '\\'; break;
                case '"': lexeme += '"'; break;
                default: lexeme += '\\'; lexeme += escaped; break;
            }
        } else {
            lexeme += get();
        }
    }
    
    if (peek() == '\0') {
        // Unclosed string
        return Token(TokenType::INVALID, lexeme, startLine, startCol);
    }
    
    get(); // Consume closing "
    return Token(TokenType::STRING_LITERAL, lexeme, startLine, startCol);
}

Token Lexer::scanChar() {
    int startLine = line;
    int startCol = column;
    std::string lexeme;
    
    get(); // Skip opening '
    
    if (peek() == '\\') {
        // Escape sequence
        get(); // Skip backslash
        lexeme += get();
    } else {
        lexeme += get();
    }
    
    if (peek() != '\'') {
        return Token(TokenType::INVALID, lexeme, startLine, startCol);
    }
    
    get(); // Consume closing '
    return Token(TokenType::CHAR_LITERAL, lexeme, startLine, startCol);
}
\end{lstlisting}

\subsection{Scanning Operators}

\begin{lstlisting}[language=C++, caption=Implementasi: lexer.cpp (Bagian 6 - Operators)]
Token Lexer::scanOperator() {
    int startLine = line;
    int startCol = column;
    char first = get();
    std::string lexeme(1, first);
    
    // Check for multi-character operators
    char next = peek();
    
    switch (first) {
        case '=':
            if (next == '=') {
                lexeme += get();
                return Token(TokenType::OP_EQUAL, lexeme, startLine, startCol);
            }
            return Token(TokenType::OP_ASSIGN, lexeme, startLine, startCol);
            
        case '!':
            if (next == '=') {
                lexeme += get();
                return Token(TokenType::OP_NOT_EQUAL, lexeme, startLine, startCol);
            }
            return Token(TokenType::OP_NOT, lexeme, startLine, startCol);
            
        case '<':
            if (next == '=') {
                lexeme += get();
                return Token(TokenType::OP_LESS_EQUAL, lexeme, startLine, startCol);
            }
            return Token(TokenType::OP_LESS, lexeme, startLine, startCol);
            
        case '>':
            if (next == '=') {
                lexeme += get();
                return Token(TokenType::OP_GREATER_EQUAL, lexeme, startLine, startCol);
            }
            return Token(TokenType::OP_GREATER, lexeme, startLine, startCol);
            
        case '&':
            if (next == '&') {
                lexeme += get();
                return Token(TokenType::OP_AND, lexeme, startLine, startCol);
            }
            return Token(TokenType::INVALID, lexeme, startLine, startCol);
            
        case '|':
            if (next == '|') {
                lexeme += get();
                return Token(TokenType::OP_OR, lexeme, startLine, startCol);
            }
            return Token(TokenType::INVALID, lexeme, startLine, startCol);
            
        case '+':
            return Token(TokenType::OP_PLUS, lexeme, startLine, startCol);
        case '-':
            return Token(TokenType::OP_MINUS, lexeme, startLine, startCol);
        case '*':
            return Token(TokenType::OP_MULTIPLY, lexeme, startLine, startCol);
        case '/':
            return Token(TokenType::OP_DIVIDE, lexeme, startLine, startCol);
            
        default:
            return Token(TokenType::INVALID, lexeme, startLine, startCol);
    }
}
\end{lstlisting}

\subsection{Main Tokenization Function}

\begin{lstlisting}[language=C++, caption=Implementasi: lexer.cpp (Bagian 7 - Main Function)]
Token Lexer::nextToken() {
    skipWhitespace();
    
    if (position >= input.length()) {
        return Token(TokenType::END_OF_FILE, "", line, column);
    }
    
    char c = peek();
    
    // Identifier or keyword
    if (std::isalpha(c) || c == '_') {
        return scanIdentifier();
    }
    
    // Number
    if (std::isdigit(c)) {
        return scanNumber();
    }
    
    // String literal
    if (c == '"') {
        return scanString();
    }
    
    // Character literal
    if (c == '\'') {
        return scanChar();
    }
    
    // Operators and punctuation
    if (c == '+' || c == '-' || c == '*' || c == '/' ||
        c == '=' || c == '!' || c == '<' || c == '>' ||
        c == '&' || c == '|') {
        return scanOperator();
    }
    
    // Punctuation
    if (c == ';') {
        get();
        return Token(TokenType::SEMICOLON, ";", line, column - 1);
    }
    if (c == ',') {
        get();
        return Token(TokenType::COMMA, ",", line, column - 1);
    }
    if (c == '.') {
        get();
        return Token(TokenType::DOT, ".", line, column - 1);
    }
    if (c == '(') {
        get();
        return Token(TokenType::LPAREN, "(", line, column - 1);
    }
    if (c == ')') {
        get();
        return Token(TokenType::RPAREN, ")", line, column - 1);
    }
    if (c == '{') {
        get();
        return Token(TokenType::LBRACE, "{", line, column - 1);
    }
    if (c == '}') {
        get();
        return Token(TokenType::RBRACE, "}", line, column - 1);
    }
    if (c == '[') {
        get();
        return Token(TokenType::LBRACKET, "[", line, column - 1);
    }
    if (c == ']') {
        get();
        return Token(TokenType::RBRACKET, "]", line, column - 1);
    }
    
    // Unknown character
    get();
    return Token(TokenType::INVALID, std::string(1, c), line, column - 1);
}

std::vector<Token> Lexer::tokenize() {
    std::vector<Token> tokens;
    Token token = nextToken();
    while (token.type != TokenType::END_OF_FILE) {
        tokens.push_back(token);
        token = nextToken();
    }
    tokens.push_back(token); // Add EOF token
    return tokens;
}
\end{lstlisting}

Gambar \ref{fig:identifier-scanning} menunjukkan proses scanning untuk identifier dan keyword.

\begin{figure}[!htbp]
    \centering
    \adjustbox{max width=\textwidth,center}{%
    \begin{tikzpicture}[
        char/.style={rectangle, draw=blue!50, fill=blue!10, minimum width=0.65cm, minimum height=0.65cm, font=\footnotesize\ttfamily, align=center},
        state/.style={rectangle, draw=green!50, fill=green!10, minimum width=1.6cm, minimum height=0.6cm, font=\scriptsize, align=center},
        arrow/.style={->, >=stealth, thick}
    ]
    
    % parameter jarak horizontal
    \def\hx{1.8cm}
    \def\vy{1.8cm}
    \def\gap{4.5cm}
    
    % =======================
    % Example 1: keyword/identifier
    % =======================
    \node[char] (c1a) {w};
    \node[char, xshift=\hx] (c1b) at (c1a.east) {h};
    \node[char, xshift=\hx] (c1c) at (c1b.east) {i};
    \node[char, xshift=\hx] (c1d) at (c1c.east) {l};
    \node[char, xshift=\hx] (c1e) at (c1d.east) {e};
    \node[char, xshift=\hx] (c1f) at (c1e.east) {(};
    
    \node[state, yshift=-\vy] (s1a) at (c1a.south) {START};
    \node[state, yshift=-\vy] (s1b) at (c1b.south) {IN\_ID};
    \node[state, yshift=-\vy] (s1c) at (c1c.south) {IN\_ID};
    \node[state, yshift=-\vy] (s1d) at (c1d.south) {IN\_ID};
    \node[state, yshift=-\vy] (s1e) at (c1e.south) {IN\_ID};
    \node[state, yshift=-\vy] (s1f) at (c1f.south) {DONE};
    
    \foreach \x/\y in {c1a/s1a,c1b/s1b,c1c/s1c,c1d/s1d,c1e/s1e,c1f/s1f}
        \draw[arrow] (\x) -- (\y);
    
    \draw[arrow, dashed] (s1a) -- (s1b);
    \draw[arrow, dashed] (s1b) -- (s1c);
    \draw[arrow, dashed] (s1c) -- (s1d);
    \draw[arrow, dashed] (s1d) -- (s1e);
    \draw[arrow, dashed] (s1e) -- (s1f);
    
    \node[below=1.0cm of s1d, font=\scriptsize] {Result: \textbf{KEYWORD\_WHILE}};
    
    % =======================
    % Example 2: number literal
    % =======================
    \node[char, yshift=-\gap] (c2a) at (c1a.south) {4};
    \node[char, xshift=\hx] (c2b) at (c2a.east) {2};
    \node[char, xshift=\hx] (c2c) at (c2b.east) {.};
    \node[char, xshift=\hx] (c2d) at (c2c.east) {5};
    \node[char, xshift=\hx] (c2e) at (c2d.east) {;};
    
    \node[state, yshift=-\vy] (s2a) at (c2a.south) {START};
    \node[state, yshift=-\vy] (s2b) at (c2b.south) {IN\_NUM};
    \node[state, yshift=-\vy] (s2c) at (c2c.south) {IN\_FLOAT};
    \node[state, yshift=-\vy] (s2d) at (c2d.south) {IN\_FLOAT};
    \node[state, yshift=-\vy] (s2e) at (c2e.south) {DONE};
    
    \foreach \x/\y in {c2a/s2a,c2b/s2b,c2c/s2c,c2d/s2d,c2e/s2e}
        \draw[arrow] (\x) -- (\y);
    
    \draw[arrow, dashed] (s2a) -- (s2b);
    \draw[arrow, dashed] (s2b) -- (s2c);
    \draw[arrow, dashed] (s2c) -- (s2d);
    \draw[arrow, dashed] (s2d) -- (s2e);
    
    \node[below=1.0cm of s2c, font=\scriptsize] {Result: \textbf{FLOAT\_LITERAL 42.5}};
    
    \end{tikzpicture}%
    }
    \caption{Proses scanning keyword/identifier dan literal numerik pada lexer}
    \label{fig:identifier-scanning}
    \end{figure}
    
    
    

Gambar \ref{fig:operator-scanning} menunjukkan proses scanning untuk operator multi-character.

\begin{figure}[!htbp]
    \centering
    \adjustbox{max width=0.75\textwidth,center}{%
    \begin{tikzpicture}[
        char/.style={rectangle, draw=blue!50, fill=blue!10, minimum width=0.6cm, minimum height=0.6cm, font=\footnotesize\ttfamily, align=center},
        state/.style={rectangle, draw=orange!60, fill=orange!10, minimum width=1.8cm, minimum height=0.55cm, font=\scriptsize, align=center, rounded corners},
        arrow/.style={->, >=stealth, thick}
    ]
    
    % parameter jarak
    \def\hx{2.8cm}
    \def\vy{1.4cm}
    \def\gap{3.2cm}
    
    % =======================
    % Example 1: operator ==
    % =======================
    \node[char] (c1) {=};
    \node[char, xshift=\hx] (c2) at (c1.east) {=};
    \node[char, xshift=\hx] (c3) at (c2.east) {;};
    
    \node[state, yshift=-\vy] (s1) at (c1.south) {START\\lookahead '='};
    \node[state, yshift=-\vy] (s2) at (c2.south) {IN\_OP\\consume '='};
    \node[state, yshift=-\vy] (s3) at (c3.south) {DONE};
    
    \draw[arrow] (c1) -- (s1);
    \draw[arrow] (c2) -- (s2);
    \draw[arrow] (c3) -- (s3);
    
    \draw[arrow, dashed] (s1) -- node[above, font=\scriptsize] {match} (s2);
    \draw[arrow, dashed] (s2) -- (s3);
    
    \node[below=0.5cm of s2, font=\scriptsize] {Result: \textbf{OP\_EQUAL} (\texttt{==})};
    
    % =======================
    % Example 2: operator =
    % =======================
    \node[char, yshift=-\gap] (c4) at (c1.south) {=};
    \node[char, xshift=\hx] (c5) at (c4.east) {;};
    
    \node[state, yshift=-\vy] (s4) at (c4.south) {START\\lookahead ';'};
    \node[state, yshift=-\vy] (s5) at (c5.south) {DONE};
    
    \draw[arrow] (c4) -- (s4);
    \draw[arrow] (c5) -- (s5);
    
    \draw[arrow, dashed] (s4) -- node[above, font=\scriptsize] {no match} (s5);
    
    \node[below=0.5cm of s4, font=\scriptsize] {Result: \textbf{OP\_ASSIGN} (\texttt{=})};
    
    \end{tikzpicture}%
    }
    \caption{Proses scanning operator: perbandingan \texttt{==} dan \texttt{=}}
    \label{fig:operator-scanning}
    \end{figure}
\section{Error Handling}

Error handling dalam lexer harus menangani berbagai kasus edge case:

\subsection{Unclosed Strings dan Comments}

\begin{itemize}
    \item \textbf{Unclosed String}: Jika string literal tidak ditutup sebelum EOF, lexer harus mengembalikan token INVALID dengan informasi posisi yang tepat.
    \item \textbf{Unclosed Block Comment}: Jika komentar blok tidak ditutup, dapat di-handle dengan exception atau mengembalikan error token.
\end{itemize}

\subsection{Invalid Characters}

Karakter yang tidak valid (tidak termasuk dalam kategori token manapun) harus dikembalikan sebagai token INVALID dengan informasi posisi untuk error reporting yang baik.

Tabel \ref{tab:token-examples} menunjukkan contoh-contoh token yang valid dan tidak valid.

\begin{table}[!htbp]
\centering
\begin{tabularx}{\textwidth}{|l|X|X|}
\hline
\textbf{Input} & \textbf{Token Type} & \textbf{Keterangan} \\
\hline
\texttt{int} & KEYWORD\_INT & Keyword valid \\
\hline
\texttt{hello} & IDENTIFIER & Identifier valid \\
\hline
\texttt{42} & INTEGER\_LITERAL & Integer valid \\
\hline
\texttt{3.14} & FLOAT\_LITERAL & Float valid \\
\hline
\texttt{"hello"} & STRING\_LITERAL & String valid \\
\hline
\texttt{==} & OP\_EQUAL & Operator multi-character \\
\hline
\texttt{=} & OP\_ASSIGN & Operator single-character \\
\hline
\texttt{\@} & INVALID & Karakter tidak valid \\
\hline
\texttt{"unclosed} & INVALID & String tidak tertutup \\
\hline
\texttt{/* comment} & Error & Comment tidak tertutup \\
\hline
\end{tabularx}
\caption{Contoh token valid dan tidak valid}
\label{tab:token-examples}
\end{table}

\subsection{Malformed Numbers}

Contoh kasus malformed:
\begin{itemize}
    \item \texttt{123.} (titik tanpa digit setelahnya)
    \item \texttt{.456} (titik tanpa digit sebelumnya) - dapat di-handle sebagai valid float
    \item \texttt{12.34.56} (multiple decimal points)
\end{itemize}

Implementasi dapat memilih untuk menerima atau menolak format tertentu sesuai kebutuhan.

Gambar \ref{fig:error-handling-fixed} menunjukkan contoh error handling untuk unclosed string.

\begin{figure}[!htbp]
    \centering
    \adjustbox{max width=0.85\textwidth,center}{%
    \begin{tikzpicture}[
        char/.style={rectangle, draw=blue!60, fill=blue!5, minimum width=0.75cm, minimum height=0.7cm, font=\small\ttfamily, rounded corners=2pt},
        error_node/.style={rectangle, draw=red!60, fill=red!5, minimum width=0.75cm, minimum height=0.7cm, font=\small\ttfamily\bfseries, rounded corners=2pt},
        state/.style={rectangle, draw=green!60, fill=green!5, minimum width=1.7cm, minimum height=0.6cm, font=\scriptsize\sffamily, rounded corners=6pt, align=center},
        error_state/.style={rectangle, draw=red!70, fill=red!10, minimum width=1.7cm, minimum height=0.6cm, font=\scriptsize\bfseries\sffamily, rounded corners=3pt, align=center},
        arrow/.style={->, >=stealth, thick, color=gray!80},
        trans/.style={->, >=stealth, thick, dashed, color=gray!60},
        err_arrow/.style={->, >=stealth, thick, color=red!70}
    ]
    
    % spacing parameters
    \def\hx{2.2cm}
    \def\vy{1.4cm}
    
    % =======================
    % Input characters
    % =======================
    \node[char] (c1) {"};
    \node[char, xshift=\hx] (c2) at (c1.east) {h};
    \node[char, xshift=\hx] (c3) at (c2.east) {e};
    \node[char, xshift=\hx] (c4) at (c3.east) {l};
    \node[char, xshift=\hx] (c5) at (c4.east) {l};
    \node[char, xshift=\hx] (c6) at (c5.east) {o};
    \node[error_node, xshift=\hx] (c7) at (c6.east) {EOF};
    
    % =======================
    % States
    % =======================
    \node[state, yshift=-\vy] (s0) at (c1.south) {START};
    \node[state] (s1) at (s0-|c2) {IN\_STRING};
    \node[state] (s2) at (s0-|c3) {IN\_STRING};
    \node[state] (s3) at (s0-|c4) {IN\_STRING};
    \node[state] (s4) at (s0-|c5) {IN\_STRING};
    \node[state] (s5) at (s0-|c6) {IN\_STRING};
    \node[error_state] (se) at (s0-|c7) {ERROR\\UNCLOSED};
    
    % row labels
    \node[left=0.6cm of c1, font=\small\itshape] {Input:};
    \node[left=0.6cm of s0, font=\small\itshape] {State:};
    
    % vertical arrows (input → state)
    \draw[arrow] (c1) -- (s0);
    \draw[arrow] (c2) -- (s1);
    \draw[arrow] (c3) -- (s2);
    \draw[arrow] (c4) -- (s3);
    \draw[arrow] (c5) -- (s4);
    \draw[arrow] (c6) -- (s5);
    \draw[err_arrow] (c7) -- (se);
    
    % horizontal transitions (DFA)
    \draw[trans] (s0) -- (s1);
    \draw[trans] (s1) -- (s2);
    \draw[trans] (s2) -- (s3);
    \draw[trans] (s3) -- (s4);
    \draw[trans] (s4) -- (s5);
    \draw[trans, red!70] (s5) -- (se);
    
    % error message
    \node[below=0.6cm of se, font=\scriptsize\ttfamily, text=red!80, align=center, draw=red!30, fill=red!5, inner sep=3pt] (msg)
    {Lexical error: unclosed string literal\\position 1:7};
    
    \end{tikzpicture}%
    }
    \caption{Transisi state lexer dan penanganan kesalahan pada string literal yang tidak tertutup}
    \label{fig:error-handling-fixed}
    \end{figure}
\section{Testing Lexer}

Unit testing sangat penting untuk memastikan lexer bekerja dengan benar. Berikut contoh test cases:

\subsection{Test Cases untuk Identifier dan Keyword}

\begin{lstlisting}[language=C++, caption=Test Cases: Identifiers dan Keywords]
void testIdentifiers() {
    Lexer lexer("int x = 42;");
    Token t1 = lexer.nextToken(); // Should be KEYWORD_INT
    Token t2 = lexer.nextToken(); // Should be IDENTIFIER "x"
    Token t3 = lexer.nextToken(); // Should be OP_ASSIGN
    // ...
}
\end{lstlisting}

\subsection{Test Cases untuk Numbers}

\begin{itemize}
    \item \texttt{42} → INTEGER\_LITERAL
    \item \texttt{3.14} → FLOAT\_LITERAL
    \item \texttt{123.456} → FLOAT\_LITERAL
    \item \texttt{0} → INTEGER\_LITERAL
\end{itemize}

\subsection{Test Cases untuk Strings}

\begin{itemize}
    \item \texttt{"hello"} → STRING\_LITERAL dengan value "hello"
    \item \texttt{"hello\textbackslash{}nworld"} → STRING\_LITERAL dengan escape sequence
    \item \texttt{"unclosed} → INVALID (unclosed string)
\end{itemize}

\subsection{Test Cases untuk Comments}

\begin{itemize}
    \item \texttt{// single line comment} $\rightarrow$ Di-skip, tidak menghasilkan token
    \item \texttt{/* multi-line comment */} $\rightarrow$ Di-skip
    \item \texttt{/* unclosed comment} → Error atau exception
\end{itemize}

Gambar \ref{fig:comment-handling} menunjukkan proses handling komentar dalam lexer.

\begin{figure}[!htbp]
    \centering
    \adjustbox{max width=0.85\textwidth,center}{%
    \begin{tikzpicture}[
        code/.style={rectangle, draw=gray!30, fill=gray!5, text width=6cm, minimum height=0.5cm, font=\footnotesize\ttfamily, align=left, inner sep=4pt},
        arrow/.style={->, >=stealth, thick, blue},
        result/.style={rectangle, draw=green!50, fill=green!10, minimum width=2.5cm, minimum height=0.6cm, font=\tiny, align=center},
        node distance=0.4cm
    ]
    
    % Single line comment
    \node[code] (code1) {int x = 10; // This is a comment};
    \node[result, below=0.3cm of code1] (r1) {Token: int, x, =, 10, ;\\Comment skipped};
    \draw[arrow] (code1) -- (r1);
    
    % Multi-line comment
    \node[code, below=1cm of code1] (code2) {int y = 20; /* Multi\\line comment */ int z = 30;};
    \node[result, below=0.3cm of code2] (r2) {Token: int, y, =, 20, ;\\int, z, =, 30, ;\\Comment skipped};
    \draw[arrow] (code2) -- (r2);
    
    \end{tikzpicture}%
    }
    \caption{Handling komentar dalam lexer}
    \label{fig:comment-handling}
\end{figure}
\section{Contoh Penggunaan}

Berikut contoh penggunaan lexer untuk tokenize source code sederhana:

\begin{lstlisting}[language=C++, caption=Contoh Penggunaan Lexer]
#include "lexer.h"
#include <iostream>

int main() {
    std::string source = R"(
        int x = 42;
        float y = 3.14;
        if (x > 10) {
            return y;
        }
    )";
    
    Lexer lexer(source);
    std::vector<Token> tokens = lexer.tokenize();
    
    for (const auto& token : tokens) {
        std::cout << "Token: " << token.lexeme 
                  << " Type: " << static_cast<int>(token.type)
                  << " Line: " << token.line 
                  << " Column: " << token.column << std::endl;
    }
    
    return 0;
}
\end{lstlisting}

Output yang diharapkan:
\begin{verbatim}
Token: int Type: 1 Line: 2 Column: 9
Token: x Type: 0 Line: 2 Column: 13
Token: = Type: 13 Line: 2 Column: 15
Token: 42 Type: 8 Line: 2 Column: 17
Token: ; Type: 20 Line: 2 Column: 19
...
\end{verbatim}

Gambar \ref{fig:complete-example} menunjukkan contoh lengkap tokenization untuk program sederhana.

\begin{figure}[H]
    \centering
    \adjustbox{max width=0.9\textwidth,center}{%
    \begin{tikzpicture}[
        code/.style={rectangle, draw=gray!30, fill=gray!5, text width=7cm, minimum height=0.6cm, font=\footnotesize\ttfamily, align=left, inner sep=4pt},
        token/.style={rectangle, draw=blue!50, fill=blue!10, text width=1.3cm, minimum height=0.5cm, font=\tiny, align=center, inner sep=2pt},
        arrow/.style={->, >=stealth, thick, gray},
        node distance=0.3cm and 0.15cm
    ]
    
    % Source code
    \node[code] (source) {int x = 42; float y = 3.14;};
    
    % Tokens row 1
    \node[token, below=0.5cm of source, xshift=-3cm] (t1) {int\\KEYWORD};
    \node[token, right=of t1] (t2) {x\\IDENTIFIER};
    \node[token, right=of t2] (t3) {=\\OP\_ASSIGN};
    \node[token, right=of t3] (t4) {42\\INTEGER};
    \node[token, right=of t4] (t5) {;\\SEMICOLON};
    
    % Tokens row 2
    \node[token, below=0.3cm of t1] (t6) {float\\KEYWORD};
    \node[token, right=of t6] (t7) {y\\IDENTIFIER};
    \node[token, right=of t7] (t8) {=\\OP\_ASSIGN};
    \node[token, right=of t8] (t9) {3.14\\FLOAT};
    \node[token, right=of t9] (t10) {;\\SEMICOLON};
    
    % Arrows
    \draw[arrow] (source.south) to[out=-90, in=90] (t1.north);
    \draw[arrow] (source.south) to[out=-90, in=90] (t2.north);
    \draw[arrow] (source.south) to[out=-90, in=90] (t3.north);
    \draw[arrow] (source.south) to[out=-90, in=90] (t4.north);
    \draw[arrow] (source.south) to[out=-90, in=90] (t5.north);
    \draw[arrow] (source.south) to[out=-90, in=90] (t6.north);
    \draw[arrow] (source.south) to[out=-90, in=90] (t7.north);
    \draw[arrow] (source.south) to[out=-90, in=90] (t8.north);
    \draw[arrow] (source.south) to[out=-90, in=90] (t9.north);
    \draw[arrow] (source.south) to[out=-90, in=90] (t10.north);
    
    \node[below=0.2cm of t8, font=\small\bfseries] {Complete Token Stream};
    
    \end{tikzpicture}%
    }
    \caption{Contoh lengkap tokenization untuk program sederhana}
    \label{fig:complete-example}
\end{figure}

Gambar \ref{fig:token-stream-example} menunjukkan visualisasi token stream untuk contoh kode sederhana.

\begin{figure}[H]
    \centering
    \adjustbox{max width=0.9\textwidth,center}{%
    \begin{tikzpicture}[
        code/.style={rectangle, draw=gray!30, fill=gray!5, text width=8cm, minimum height=0.6cm, font=\footnotesize\ttfamily, align=left, inner sep=4pt},
        token/.style={rectangle, draw=blue!50, fill=blue!10, text width=1.3cm, minimum height=0.5cm, font=\tiny, align=center, inner sep=2pt},
        arrow/.style={->, >=stealth, thick, gray},
        node distance=0.3cm and 0.2cm
    ]
    
    % Source code
    \node[code] (source) {int x = 42;};
    
    % Tokens
    \node[token, below=0.5cm of source, xshift=-2.5cm] (t1) {int\\KEYWORD};
    \node[token, right=of t1] (t2) {x\\IDENTIFIER};
    \node[token, right=of t2] (t3) {=\\OP\_ASSIGN};
    \node[token, right=of t3] (t4) {42\\INTEGER};
    \node[token, right=of t4] (t5) {;\\SEMICOLON};
    
    % Arrows from source to tokens
    \draw[arrow] (source.south) to[out=-90, in=90] (t1.north);
    \draw[arrow] (source.south) to[out=-90, in=90] (t2.north);
    \draw[arrow] (source.south) to[out=-90, in=90] (t3.north);
    \draw[arrow] (source.south) to[out=-90, in=90] (t4.north);
    \draw[arrow] (source.south) to[out=-90, in=90] (t5.north);
    
    % Token stream label
    \node[below=0.3cm of t3, font=\small\bfseries] {Token Stream};
    
    \end{tikzpicture}%
    }
    \caption{Contoh tokenization: \texttt{int x = 42;} menjadi token stream}
    \label{fig:token-stream-example}
\end{figure}

Gambar \ref{fig:lexer-example-formal} menunjukkan proses scanning secara detail untuk string \texttt{"hello"}.

\begin{figure}[H]
    \centering
    \adjustbox{max width=0.85\textwidth,center}{%
    \begin{tikzpicture}[
        char/.style={rectangle, draw=blue!60, fill=blue!5, minimum width=0.7cm, minimum height=0.7cm, font=\footnotesize\ttfamily, rounded corners=2pt},
        state/.style={rectangle, draw=green!60, fill=green!8, minimum width=1.8cm, minimum height=0.7cm, font=\scriptsize\sffamily, rounded corners=4pt, align=center},
        arrow/.style={->, >=stealth, thick, color=black!70},
        trans/.style={->, >=stealth, dashed, thick, color=black!50},
        node distance=0.5cm and 0.5cm
    ]
    
    % =========================
    % Row 1: Input Characters
    % =========================
    \node[char] (c1) {"};
    \node[char, right=1.5cm of c1] (c2) {h};
    \node[char, right=1.5cm of c2] (c3) {e};
    \node[char, right=1.5cm of c3] (c4) {l};
    \node[char, right=1.5cm of c4] (c5) {l};
    \node[char, right=1.5cm of c5] (c6) {o};
    \node[char, right=1.5cm of c6] (c7) {"};
    
    % =========================
    % Row 2: Lexer States
    % =========================
    \node[state, below=0.9cm of c1] (s0) {START};
    \node[state, below=0.9cm of c2] (s1) {IN\_STRING};
    \node[state, below=0.9cm of c3] (s2) {IN\_STRING};
    \node[state, below=0.9cm of c4] (s3) {IN\_STRING};
    \node[state, below=0.9cm of c5] (s4) {IN\_STRING};
    \node[state, below=0.9cm of c6] (s5) {IN\_STRING};
    \node[state, below=0.9cm of c7] (s6) {ACCEPT};
    
    % =========================
    % Vertical Mapping (Input -> State)
    % =========================
    \draw[arrow] (c1) -- (s0);
    \draw[arrow] (c2) -- (s1);
    \draw[arrow] (c3) -- (s2);
    \draw[arrow] (c4) -- (s3);
    \draw[arrow] (c5) -- (s4);
    \draw[arrow] (c6) -- (s5);
    \draw[arrow] (c7) -- (s6);
    
    % =========================
    % Horizontal State Transitions
    % =========================
    \draw[trans] (s0) -- (s1);
    \draw[trans] (s1) -- (s2);
    \draw[trans] (s2) -- (s3);
    \draw[trans] (s3) -- (s4);
    \draw[trans] (s4) -- (s5);
    \draw[trans] (s5) -- (s6);
    
    % =========================
    % Result Annotation
    % =========================
    \node[below=0.7cm of s3, font=\small\bfseries, align=center] (res)
    {Token dihasilkan: \texttt{STRING\_LITERAL("hello")}};
    
    \end{tikzpicture}%
    }
    \caption{Representasi formal proses analisis leksikal pada string literal \texttt{"hello"}}
    \label{fig:lexer-example-formal}
    \end{figure}
    

Gambar \ref{fig:error-recovery-strategy} menunjukkan strategi error recovery dalam lexer.

\begin{figure}[H]
    \centering
    \adjustbox{max width=0.85\textwidth,center}{%
    \begin{tikzpicture}[
        box/.style={rectangle, draw=blue!50, fill=blue!10, text width=2.5cm, minimum height=0.6cm, font=\footnotesize, align=center, rounded corners},
        error/.style={rectangle, draw=red!50, fill=red!10, text width=2.5cm, minimum height=0.6cm, font=\footnotesize, align=center, rounded corners},
        decision/.style={diamond, draw=orange!50, fill=orange!10, text width=1.5cm, minimum height=0.6cm, font=\footnotesize, align=center},
        arrow/.style={->, >=stealth, thick},
        node distance=0.6cm
    ]
    
    \node[box] (input) {Input\\Character};
    \node[decision, below=of input] (valid) {Valid?};
    \node[box, below left=of valid] (token) {Generate\\Token};
    \node[error, below right=of valid] (skip) {Skip\\Character};
    \node[box, below=of token] (continue) {Continue\\Scanning};
    \node[box, below=of skip] (report) {Report\\Error};
    
    \draw[arrow] (input) -- (valid);
    \draw[arrow] (valid) -- node[left, font=\tiny] {Yes} (token);
    \draw[arrow] (valid) -- node[right, font=\tiny] {No} (skip);
    \draw[arrow] (token) -- (continue);
    \draw[arrow] (skip) -- (report);
    \draw[arrow, dashed] (report) to[out=180, in=270] (continue);
    
    \end{tikzpicture}%
    }
    \caption{Strategi error recovery dalam lexer}
    \label{fig:error-recovery-strategy}
\end{figure}
\section{Best Practices}

Beberapa best practices dalam implementasi hand-written lexer:

\begin{enumerate}
    \item \textbf{Separation of Concerns}: Pisahkan logika untuk setiap jenis token ke fungsi terpisah
    \item \textbf{Position Tracking}: Selalu track line dan column untuk error reporting yang baik
    \item \textbf{Lookahead}: Gunakan \texttt{peek()} untuk lookahead tanpa mengkonsumsi karakter
    \item \textbf{Error Recovery}: Rancang strategi error recovery (misalnya skip invalid character dan lanjut)
    \item \textbf{Testing}: Buat comprehensive test suite untuk semua edge cases
    \item \textbf{Documentation}: Dokumentasikan token types dan format yang didukung
\end{enumerate}
\section{Kesimpulan}

Dalam bab ini, kita telah mempelajari:

\begin{enumerate}
    \item Struktur token dan token types untuk subset bahasa C
    \item Konsep finite state machine dalam konteks lexical analysis
    \item Implementasi hand-written lexer dalam C++ dengan handling:
    \begin{itemize}
        \item Identifier dan keyword recognition
        \item Number literals (integer dan float)
        \item String dan character literals dengan escape sequences
        \item Operators (single dan multi-character)
        \item Whitespace dan komentar (single-line dan multi-line)
    \end{itemize}
    \item Error handling untuk edge cases
    \item Testing strategies untuk lexer
\end{enumerate}

Implementasi hand-written lexer memberikan pemahaman mendalam tentang proses tokenization dan menjadi dasar untuk memahami bagaimana lexer generator seperti Flex bekerja di belakang layar. Lexer yang dibahas di bab ini mengimplementasikan spesifikasi token proyek subset C (Bab 1); lexer proyek dengan Flex dibangun di Bab 4 (\texttt{simplec.l}).
\section{Referensi dan Bahan Bacaan Lanjutan}

Untuk memperdalam pemahaman tentang implementasi lexer, mahasiswa disarankan membaca:

\begin{itemize}
    \item \textbf{Dragon Book}: Aho, Lam, Sethi, \& Ullman (2006). \textit{Compilers: Principles, Techniques, and Tools} \cite{aho2006compilers} - Bab 3: Lexical Analysis
    
    \item \textbf{Engineering a Compiler}: Cooper \& Torczon (2011) \cite{cooper2011engineering} - Bab 2: Scanning
    
    \item \textbf{OpenGenus - Build Lexer}: Tutorial tentang hand-written lexer \cite{opengenus2024lexer}
    
    \item \textbf{Aoyama Gakuin University}: Lecture notes tentang lexical analysis \cite{aoyama2024lexical}
    
    \item \textbf{GeeksforGeeks}: Contoh implementasi lexical analyzer dalam C++ \footnote{\url{https://www.geeksforgeeks.org/cpp/lexical-analyzer-in-cpp/}}
    
    \item \textbf{Programming Notes}: Tutorial tentang simple lexer menggunakan finite state machine \footnote{\url{https://www.programmingnotes.org/4699/cpp-simple-lexer-using-a-finite-state-machine/}}
\end{itemize}

% Daftar pustaka (hanya muncul saat kompilasi standalone dan hanya jika ada citation)
\ifSubfilesClassLoaded{%
    \clearpage
    \printbibliography[title={Daftar Pustaka}]
    \end{refsection}
}{}

\end{document}

\cleardoublepage
% Bab 4: Lexer Generator (Flex/re2c) dan Praktikum Lexer
% File ini dapat dikompilasi terpisah atau sebagai bagian dari main.tex

\chapter{Lexer Generator (Flex/re2c) dan Praktikum Lexer}
\label{chap:lexer-generator}

\section{Tujuan Pembelajaran}

Setelah mempelajari bab ini, mahasiswa diharapkan mampu:
\begin{enumerate}
    \item Memahami konsep dan keuntungan menggunakan lexer generator
    \item Menggunakan Flex untuk membuat specification file (.l) dan generate lexer code
    \item Menggunakan re2c untuk membuat lexer dengan embedded specification
    \item Membuat specification file untuk lexer bahasa sederhana
    \item Mengintegrasikan generated lexer dengan program utama
    \item Membandingkan hand-written lexer dengan generator-based lexer
    \item Mengevaluasi trade-off antara performa, kemudahan maintenance, dan fleksibilitas
\end{enumerate}

\section{Pendahuluan}

Pada bab sebelumnya, kita telah mempelajari implementasi hand-written lexer. Meskipun pendekatan tersebut memberikan kontrol penuh dan pemahaman mendalam, dalam praktik industri, penggunaan \textbf{lexer generator} lebih umum karena efisiensi dan kemudahan maintenance. Menurut sumber terbuka:

\begin{quote}
``re2c is a high-performance lexer generator for C/C++ that takes regex specifications and builds deterministic finite automata. It's used in real projects.''\cite{wikipedia2024re2c}
\end{quote}

Lexer generator adalah tools yang menerima specification file (berisi pattern dan action) dan menghasilkan kode lexer yang siap digunakan. Dua generator populer untuk C/C++ adalah \textbf{Flex} (Fast Lexical Analyzer) dan \textbf{re2c} (Regular Expressions to Code).

Keuntungan menggunakan lexer generator:
\begin{itemize}
    \item \textbf{Produktivitas}: Lebih cepat dalam development karena tidak perlu menulis state machine manual
    \item \textbf{Maintainability}: Specification file lebih mudah dibaca dan dimodifikasi dibanding kode state machine
    \item \textbf{Optimasi Otomatis}: Generator menghasilkan kode yang sudah dioptimasi (DFA minimization, dll.)
    \item \textbf{Konsistensi}: Mengurangi bug karena generator sudah teruji
\end{itemize}

\section{Flex (Fast Lexical Analyzer)}

Flex adalah lexer generator yang paling banyak digunakan, terutama dalam kombinasi dengan Bison (parser generator). Flex membaca specification file dengan ekstensi \texttt{.l} dan menghasilkan kode C untuk lexer.

\subsection{Struktur Flex Specification File}

File specification Flex (`.l`) terdiri dari tiga bagian yang dipisahkan oleh `\%\%`:

\begin{verbatim}
Definitions
%%
Rules
%%
User Code
\end{verbatim}

\subsubsection{Definitions Section}

Bagian ini berisi:
\begin{itemize}
    \item \textbf{Named patterns (macros)}: Definisi pattern yang dapat digunakan kembali
    \item \textbf{C code}: Kode C yang akan disalin langsung ke output (dalam `\%\{ \%\}`)
    \item \textbf{Options}: Konfigurasi Flex (misalnya `\%option noyywrap`)
\end{itemize}

Contoh:
\begin{verbatim}
%{
#include <stdio.h>
#include "tokens.h"  // Definisi token constants
%}

DIGIT    [0-9]
LETTER   [a-zA-Z]
ID       {LETTER}({LETTER}|{DIGIT})*
NUMBER   {DIGIT}+
\end{verbatim}

\subsubsection{Rules Section}

Bagian ini berisi pattern-action pairs. Pattern menggunakan regular expression, dan action adalah kode C yang dieksekusi ketika pattern match.

Contoh:
\begin{verbatim}
%%
"if"          { return IF; }
"else"        { return ELSE; }
"while"       { return WHILE; }
{ID}          { return IDENTIFIER; }
{NUMBER}      { yylval.intval = atoi(yytext); return NUMBER; }
"=="          { return EQ; }
"!="          { return NE; }
[ \t\n]+      { /* skip whitespace */ }
"//".*        { /* skip single-line comment */ }
"/*"          { BEGIN(COMMENT); }
<COMMENT>"*/" { BEGIN(INITIAL); }
<COMMENT>.    { /* skip comment content */ }
.             { return yytext[0]; }  /* default: return character */
%%
\end{verbatim}

\subsubsection{User Code Section}

Bagian ini berisi fungsi-fungsi pendukung seperti `main()`, `yywrap()`, dan helper functions.

\subsection{Contoh Lengkap: Flex Lexer untuk Bahasa Sederhana}

Berikut adalah contoh specification file Flex untuk bahasa sederhana dengan token: identifier, number, keyword, dan operator:

\begin{lstlisting}[language=C, caption={Contoh Flex specification file (calc.l)}]
%{
#include <stdio.h>
#include <stdlib.h>
#include "parser.tab.h"  // Header dari Bison

int yylineno = 1;
%}

%option noyywrap
%option yylineno

DIGIT    [0-9]
LETTER   [a-zA-Z_]
ID       {LETTER}({LETTER}|{DIGIT})*
NUMBER   {DIGIT}+
FLOAT    {DIGIT}+\.{DIGIT}+

%%

"int"       { return INT; }
"float"     { return FLOAT_TYPE; }
"if"        { return IF; }
"else"      { return ELSE; }
"while"     { return WHILE; }
"return"   { return RETURN; }

{ID}        { 
                yylval.string = strdup(yytext);
                return IDENTIFIER;
            }

{NUMBER}    { 
                yylval.intval = atoi(yytext);
                return NUMBER;
            }

{FLOAT}     {
                yylval.floatval = atof(yytext);
                return FLOAT_LITERAL;
            }

"+"         { return PLUS; }
"-"         { return MINUS; }
"*"         { return MULTIPLY; }
"/"         { return DIVIDE; }
"="         { return ASSIGN; }
"=="        { return EQ; }
"!="        { return NE; }
"<"         { return LT; }
">"         { return GT; }
"<="        { return LE; }
">="        { return GE; }

"("         { return LPAREN; }
")"         { return RPAREN; }
"{"         { return LBRACE; }
"}"         { return RBRACE; }
";"         { return SEMICOLON; }
","         { return COMMA; }

[ \t]+      { /* skip whitespace */ }
\n          { yylineno++; }
"//".*      { /* skip single-line comment */ }
"/*"        { 
                int c;
                while ((c = input()) != EOF) {
                    if (c == '\n') yylineno++;
                    if (c == '*' && (c = input()) == '/') break;
                    if (c != EOF) unput(c);
                }
            }

.           { 
                fprintf(stderr, "Error: unexpected character '%c' at line %d\n", 
                        yytext[0], yylineno);
                return ERROR;
            }

%%

int yywrap(void) {
    return 1;
}
\end{lstlisting}

\subsection{Kompilasi dan Penggunaan Flex}

Untuk menggunakan Flex:

\begin{enumerate}
    \item Buat file specification (misalnya \texttt{lexer.l})
    \item Generate lexer code: \texttt{flex lexer.l} (menghasilkan \texttt{lex.yy.c})
    \item Compile dengan compiler C: \texttt{gcc lex.yy.c -o lexer -lfl}
    \item Atau link dengan program utama: \texttt{gcc main.c lex.yy.c -o program -lfl}
\end{enumerate}

Fungsi utama yang digunakan:
\begin{itemize}
    \item \texttt{yylex()}: Fungsi yang dipanggil untuk mendapatkan token berikutnya
    \item \texttt{yytext}: String yang berisi lexeme yang baru saja di-match
    \item \texttt{yyleng}: Panjang dari \texttt{yytext}
    \item \texttt{yylval}: Union untuk menyimpan nilai semantic (untuk integrasi dengan parser)
\end{itemize}

\section{re2c (Regular Expressions to Code)}

re2c adalah lexer generator modern yang menghasilkan kode C/C++ dengan performa tinggi. Berbeda dengan Flex yang menggunakan file terpisah, re2c menggunakan \textbf{embedded specification} dalam kode C/C++.

Menurut dokumentasi resmi re2c:

\begin{quote}
``re2c is a tool that generates fast lexers for C, C++ and Go. It compiles regular expressions to deterministic finite automata and encodes them as conditional jumps and comparisons. The generated code is highly optimized and does not use tables.''\footnote{\url{https://re2c.org/}}
\end{quote}

\subsection{Struktur re2c Specification}

re2c specification ditulis dalam komentar khusus `/*!re2c ... */` yang disisipkan dalam kode C/C++:

\begin{lstlisting}[language=C, caption={Struktur dasar re2c}]
#include <stdio.h>

static int lex(const char *YYCURSOR) {
    const char *YYMARKER;
    /*!re2c
        re2c:define:YYCTYPE = "char";
        re2c:yyfill:enable = 0;
        
        // Named patterns
        digit    = [0-9];
        letter   = [a-zA-Z_];
        id       = letter (letter | digit)*;
        number   = digit+;
        
        // Rules
        *        { return 0; }  // error
        number   { printf("Number: %.*s\n", (int)(YYCURSOR - YYMARKER), YYMARKER); return 1; }
        id       { printf("ID: %.*s\n", (int)(YYCURSOR - YYMARKER), YYMARKER); return 1; }
        [ \t\n]+ { continue; }  // skip whitespace
    */
}

int main(int argc, char *argv[]) {
    for (int i = 1; i < argc; i++) {
        lex(argv[i]);
    }
    return 0;
}
\end{lstlisting}

\subsection{Key Concepts dalam re2c}

\subsubsection{Variables Khusus}

re2c menggunakan variabel khusus untuk tracking posisi input:
\begin{itemize}
    \item \texttt{YYCURSOR}: Pointer ke posisi saat ini dalam input
    \item \texttt{YYMARKER}: Pointer untuk backtracking
    \item \texttt{YYLIMIT}: Pointer ke akhir buffer
    \item \texttt{YYCTYPE}: Tipe data untuk karakter (default: \texttt{unsigned char})
\end{itemize}

\subsubsection{Directives}

Directives mengkonfigurasi behavior re2c:
\begin{itemize}
    \item \texttt{re2c:define:YYCTYPE}: Mendefinisikan tipe karakter
    \item \texttt{re2c:yyfill:enable}: Enable/disable buffer filling
    \item \texttt{re2c:input}: Mendefinisikan cara membaca input
    \item \texttt{re2c:conditions}: Enable start conditions (seperti Flex)
\end{itemize}

\subsection{Contoh Lengkap: re2c Lexer untuk Identifier dan Number}

\begin{lstlisting}[language=C, caption={Contoh re2c lexer (lexer.re.c)}]
#include <stdio.h>
#include <stdlib.h>
#include <string.h>

typedef enum {
    TOKEN_EOF,
    TOKEN_IDENTIFIER,
    TOKEN_NUMBER,
    TOKEN_PLUS,
    TOKEN_MINUS,
    TOKEN_MULTIPLY,
    TOKEN_DIVIDE,
    TOKEN_ERROR
} TokenType;

typedef struct {
    TokenType type;
    char *value;
    int line;
} Token;

static Token tokenize(const char *input, int *line) {
    const char *YYCURSOR = input;
    const char *YYMARKER;
    const char *start;
    Token token = {TOKEN_EOF, NULL, *line};
    
    /*!re2c
        re2c:define:YYCTYPE = "char";
        re2c:yyfill:enable = 0;
        re2c:define:YYCURSOR = "YYCURSOR";
        
        digit    = [0-9];
        letter   = [a-zA-Z_];
        id       = letter (letter | digit)*;
        number   = digit+;
        ws       = [ \t]+;
        newline  = "\n";
        
        * {
            token.type = TOKEN_ERROR;
            return token;
        }
        
        "\x00" {
            token.type = TOKEN_EOF;
            return token;
        }
        
        ws {
            continue;
        }
        
        newline {
            (*line)++;
            continue;
        }
        
        number {
            start = YYMARKER;
            int len = YYCURSOR - start;
            token.value = (char*)malloc(len + 1);
            strncpy(token.value, start, len);
            token.value[len] = '\0';
            token.type = TOKEN_NUMBER;
            token.line = *line;
            return token;
        }
        
        id {
            start = YYMARKER;
            int len = YYCURSOR - start;
            token.value = (char*)malloc(len + 1);
            strncpy(token.value, start, len);
            token.value[len] = '\0';
            token.type = TOKEN_IDENTIFIER;
            token.line = *line;
            return token;
        }
        
        "+" {
            token.type = TOKEN_PLUS;
            token.line = *line;
            return token;
        }
        
        "-" {
            token.type = TOKEN_MINUS;
            token.line = *line;
            return token;
        }
        
        "*" {
            token.type = TOKEN_MULTIPLY;
            token.line = *line;
            return token;
        }
        
        "/" {
            token.type = TOKEN_DIVIDE;
            token.line = *line;
            return token;
        }
    */
}

int main(int argc, char *argv[]) {
    if (argc < 2) {
        fprintf(stderr, "Usage: %s <input>\n", argv[0]);
        return 1;
    }
    
    int line = 1;
    Token token;
    
    do {
        token = tokenize(argv[1], &line);
        printf("Token: %d, Value: %s, Line: %d\n", 
               token.type, token.value ? token.value : "NULL", token.line);
        if (token.value) free(token.value);
    } while (token.type != TOKEN_EOF && token.type != TOKEN_ERROR);
    
    return 0;
}
\end{lstlisting}

Untuk mengkompilasi:
\begin{verbatim}
re2c -o lexer.c lexer.re.c
gcc lexer.c -o lexer
\end{verbatim}

\section{Perbandingan Hand-Written vs Generator-Based Lexer}

Setelah mempelajari kedua pendekatan, mari kita bandingkan:

\subsection{Hand-Written Lexer}

\textbf{Keuntungan:}
\begin{itemize}
    \item Kontrol penuh terhadap implementasi
    \item Tidak ada dependency eksternal
    \item Dapat dioptimasi secara spesifik untuk kebutuhan
    \item Pemahaman mendalam tentang proses tokenization
\end{itemize}

\textbf{Kekurangan:}
\begin{itemize}
    \item Lebih banyak kode yang harus ditulis dan maintain
    \item Lebih mudah terjadi bug (edge cases)
    \item Perlu implementasi ulang untuk setiap bahasa
    \item Lebih sulit untuk modifikasi pattern
\end{itemize}

\subsection{Generator-Based Lexer}

\textbf{Keuntungan:}
\begin{itemize}
    \item Specification lebih ringkas dan mudah dibaca
    \item Generator menghasilkan kode yang sudah teroptimasi
    \item Lebih cepat dalam development
    \item Pattern mudah dimodifikasi tanpa mengubah banyak kode
    \item Sudah teruji dan digunakan di banyak project
\end{itemize}

\textbf{Kekurangan:}
\begin{itemize}
    \item Perlu mempelajari syntax generator
    \item Dependency pada tool eksternal
    \item Kurang fleksibel untuk kasus yang sangat spesifik
    \item Generated code mungkin lebih sulit di-debug
\end{itemize}

\subsection{Perbandingan Performa}

Secara umum, generator-based lexer (terutama re2c) memiliki performa yang sangat baik karena:
\begin{itemize}
    \item Kode dihasilkan dengan optimasi DFA
    \item Tidak ada overhead dari table lookup (untuk re2c)
    \item Compiler dapat mengoptimasi generated code lebih baik
\end{itemize}

Hand-written lexer dapat lebih cepat hanya jika dioptimasi secara khusus untuk kasus tertentu, tetapi memerlukan effort yang lebih besar.

\section{Integrasi dengan Parser}

Lexer biasanya digunakan bersama dengan parser. Integrasi dilakukan melalui:

\subsection{Token Definitions}

Token constants didefinisikan dalam header file yang dibagi antara lexer dan parser:

\begin{lstlisting}[language=C, caption={File tokens.h}]
#ifndef TOKENS_H
#define TOKENS_H

typedef enum {
    // Keywords
    TOKEN_IF = 256,
    TOKEN_ELSE,
    TOKEN_WHILE,
    TOKEN_RETURN,
    TOKEN_INT,
    TOKEN_FLOAT,
    
    // Identifiers and literals
    TOKEN_IDENTIFIER,
    TOKEN_NUMBER,
    TOKEN_FLOAT_LITERAL,
    
    // Operators
    TOKEN_PLUS,
    TOKEN_MINUS,
    TOKEN_MULTIPLY,
    TOKEN_DIVIDE,
    TOKEN_ASSIGN,
    TOKEN_EQ,
    TOKEN_NE,
    
    // Punctuation
    TOKEN_LPAREN,
    TOKEN_RPAREN,
    TOKEN_LBRACE,
    TOKEN_RBRACE,
    TOKEN_SEMICOLON,
    TOKEN_COMMA,
    
    TOKEN_EOF,
    TOKEN_ERROR
} TokenType;

#endif
\end{lstlisting}

\subsection{Semantic Values}

Untuk mengirim nilai dari lexer ke parser, digunakan union \texttt{yylval}:

\begin{lstlisting}[language=C, caption={File yystype.h}]
#ifndef YYSTYPE_H
#define YYSTYPE_H

#include "tokens.h"

typedef union {
    int intval;
    double floatval;
    char *string;
} YYSTYPE;

extern YYSTYPE yylval;

#endif
\end{lstlisting}

\subsection{Contoh Integrasi Flex dengan Bison}

File Flex (\texttt{lexer.l}):
\begin{verbatim}
%{
#include "parser.tab.h"
#include "yystype.h"
%}

%%
{NUMBER} { yylval.intval = atoi(yytext); return NUMBER; }
{ID}     { yylval.string = strdup(yytext); return IDENTIFIER; }
%%
\end{verbatim}

File Bison (\texttt{parser.y}):
\begin{verbatim}
%{
#include "yystype.h"
%}

%union {
    int intval;
    char *string;
}

%token <intval> NUMBER
%token <string> IDENTIFIER

%%
expression: NUMBER { printf("Number: %d\n", $1); }
         | IDENTIFIER { printf("ID: %s\n", $1); }
         ;
%%
\end{verbatim}

\section{Praktikum: Membuat Lexer dengan Flex}

\subsection{Tugas Praktikum}

Buatlah lexer menggunakan Flex untuk bahasa mini dengan minimal 10 token types:

\begin{enumerate}
    \item \textbf{Keywords}: \texttt{if}, \texttt{else}, \texttt{while}, \texttt{int}, \texttt{float}, \texttt{return}
    \item \textbf{Identifiers}: Nama variabel dan fungsi
    \item \textbf{Literals}: Integer dan float numbers
    \item \textbf{Operators}: \texttt{+}, \texttt{-}, \texttt{*}, \texttt{/}, \texttt{=}, \texttt{==}, \texttt{!=}, \texttt{<}, \texttt{>}
    \item \textbf{Punctuation}: \texttt{(}, \texttt{)}, \texttt{\{}, \texttt{\}}, \texttt{;}, \texttt{,}
    \item \textbf{Comments}: Single-line (\texttt{//}) dan multi-line (\texttt{/* */})
\end{enumerate}

\subsection{Langkah-langkah}

\begin{enumerate}
    \item Buat file \texttt{lexer.l} dengan specification sesuai requirement
    \item Generate lexer: \texttt{flex lexer.l}
    \item Buat program test sederhana yang menggunakan \texttt{yylex()}
    \item Test dengan berbagai input (valid dan invalid)
    \item Dokumentasikan token types dan behavior lexer
\end{enumerate}

\subsection{Expected Output}

Lexer harus dapat:
\begin{itemize}
    \item Mengenali semua token types yang didefinisikan
    \item Menangani whitespace dan comments dengan benar
    \item Memberikan error message yang informatif untuk invalid input
    \item Melacak line number untuk error reporting
\end{itemize}

\section{Kesimpulan}

Dalam bab ini, kita telah mempelajari:

\begin{enumerate}
    \item Keuntungan menggunakan lexer generator dibanding hand-written lexer
    \item Cara menggunakan Flex untuk membuat specification file dan generate lexer
    \item Cara menggunakan re2c dengan embedded specification
    \item Perbandingan antara hand-written dan generator-based lexer
    \item Integrasi lexer dengan parser menggunakan token definitions dan semantic values
\end{enumerate}

Generator-based lexer adalah pilihan yang tepat untuk sebagian besar kasus karena memberikan keseimbangan yang baik antara produktivitas, maintainability, dan performa. Namun, pemahaman tentang hand-written lexer (seperti yang dipelajari di bab sebelumnya) tetap penting untuk memahami proses tokenization secara mendalam.

\section{Latihan}

\begin{enumerate}
    \item Buatlah specification file Flex untuk mengenali token-token berikut:
    \begin{itemize}
        \item Keywords: \texttt{for}, \texttt{break}, \texttt{continue}
        \item String literals (dengan escape sequences: \texttt{\textbackslash n}, \texttt{\textbackslash t}, \texttt{\textbackslash "})
        \item Character literals (dalam single quotes)
        \item Operators: \texttt{++}, \texttt{--}, \texttt{+=}, \texttt{-=}
    \end{itemize}
    
    \item Implementasikan lexer menggunakan re2c untuk bahasa yang sama seperti soal 1. Bandingkan kompleksitas specification antara Flex dan re2c.
    
    \item Jelaskan kapan sebaiknya menggunakan hand-written lexer dan kapan menggunakan generator. Berikan contoh kasus untuk masing-masing.
    
    \item Buatlah program yang mengintegrasikan Flex lexer dengan Bison parser sederhana untuk:
    \begin{itemize}
        \item Parsing ekspresi aritmatika: \texttt{a + b * c}
        \item Parsing assignment: \texttt{x = 42;}
        \item Parsing conditional: \texttt{if (x > 0) \{ ... \}}
    \end{itemize}
    
    \item Analisis performa: Buat benchmark untuk membandingkan:
    \begin{itemize}
        \item Hand-written lexer (dari Bab 3)
        \item Flex-generated lexer
        \item re2c-generated lexer
    \end{itemize}
    Gunakan input file yang besar (misalnya 10MB) dan ukur waktu eksekusi.
    
    \item Jelaskan bagaimana Flex menangani longest match dan rule priority. Berikan contoh yang menunjukkan perbedaan hasil ketika urutan rule diubah.
\end{enumerate}

\section{Referensi dan Bahan Bacaan Lanjutan}

Untuk memperdalam pemahaman tentang lexer generator, mahasiswa disarankan membaca:

\begin{itemize}
    \item \textbf{Flex Manual}: Dokumentasi resmi Flex \footnote{\url{https://www.gnu.org/software/flex/manual/}}
    
    \item \textbf{re2c Documentation}: Dokumentasi dan tutorial re2c \footnote{\url{https://re2c.org/}}
    
    \item \textbf{flex \& bison}: Levine, J. R. (2009). \textit{flex \& bison: Text Processing Tools} \cite{levine2009flex} - Bab 2: Using Flex
    
    \item \textbf{Engineering a Compiler}: Cooper \& Torczon (2011) \cite{cooper2011engineering} - Bab 2: Lexical Analysis (bagian tentang lexer generators)
    
    \item \textbf{IT Trip - C Parser Flex Bison}: Tutorial tentang integrasi Flex dan Bison \cite{ittrip2024bison}
    
    \item \textbf{Wikipedia - re2c}: Artikel tentang re2c \cite{wikipedia2024re2c}
    
    \item \textbf{Wikipedia - RE/flex}: Artikel tentang RE/flex (modern C++ lexer generator) \footnote{\url{https://en.wikipedia.org/wiki/Draft:RE/flex}}
\end{itemize}

\cleardoublepage
\documentclass[../main.tex]{subfiles}

\addbibresource{\subfix{../references.bib}}

\begin{document}

\ifSubfilesClassLoaded{%
    \setcounter{chapter}{4}%
    \begin{refsection}
}{}

\chapter{Symbol Table dan Scope Management}
\label{chap:symbol-table}

\begin{subcpmk}
  \item \textbf{Sub-CPMK 3.1:} Mengimplementasikan symbol table dengan nested scopes
  \item \textbf{Sub-CPMK 3.2:} Melakukan type checking untuk ekspresi kompleks
\end{subcpmk}

% ============================================================
% MATERI POKOK
% ============================================================
\section{Dasar Symbol Table dan Perannya}

\compiler{Symbol Table} adalah struktur data fundamental yang menyimpan informasi tentang identifiers (variabel, fungsi, tipe) yang muncul dalam kode sumber.

\subsection{Kebutuhan Symbol Table}
Dalam fase analisis semantik, kompilator perlu menjawab pertanyaan:
\begin{itemize}
    \item Apakah variabel $x$ sudah dideklarasikan?
    \item Apa tipe data dari $y$?
    \item Apakah $z$ merupakan variabel lokal atau global?
\end{itemize}

\subsection{Komponen Entry}
Setiap entri dalam symbol table minimal menyimpan:
\begin{itemize}
    \item \textbf{Name}: Nama identifier.
    \item \textbf{Type}: Tipe data (misal: \texttt{int}, \texttt{float}).
    \item \textbf{Scope}: Level nesting tempat identifier berada.
    \item \textbf{Location}: Alamat memori atau offset (untuk fase \textit{code generation}).
\end{itemize}

\section{Struktur Data: Hash Table dan Scope Stack}

\subsection{Pilihan Struktur Data}
Kinerja kompilator sangat bergantung pada kecepatan Symbol Table. Mengapa? Karena setiap kali parser menemukan identifier, ia harus melakukan \textit{lookup}.
\begin{itemize}
    \item \textbf{Linear List}: $O(N)$. Sangat lambat, hanya cocok untuk bahasa mainan.
    \item \textbf{Binary Search Tree (BST)}: $O(\log N)$. Cukup cepat, tapi butuh penyeimbangan (AVL/Red-Black) agar tidak terdegradasi menjadi linked list.
    \item \textbf{Hash Table}: $O(1)$ rata-rata. Ini adalah standar industri. Dengan fungsi hash yang baik, akses ke ribuan variabel tetap instan.
\end{itemize}

\subsection{Manajemen Scope: The Cactus Stack}
Bahasa modern mendukung \textit{Nested Scopes} (blok di dalam blok). Struktur data yang paling tepat untuk ini adalah \textbf{Cactus Stack} (atau \textit{Chained Symbol Tables}).
\begin{itemize}
    \item Setiap scope memiliki Hash Table sendiri.
    \item Hash Table scope anak memiliki pointer \texttt{parent} ke Hash Table scope luar.
    \item Pencarian dimulai dari tabel saat ini. Jika tidak ketemu, lanjut ke \texttt{parent}, terus hingga \texttt{Global Scope} (yang parent-nya \texttt{NULL}).
\end{itemize}

\begin{figure}[!htbp]
    \centering
    \adjustbox{max width=0.8\textwidth,center}{%
    \begin{tikzpicture}[
        table/.style={rectangle, draw=blue!50, fill=blue!10, text width=2.5cm, minimum height=1.5cm, font=\tiny, align=center},
        arrow/.style={->, >=stealth, thick}
    ]
    \node[table] (global) {Global Scope\\ \texttt{int x}};
    \node[table, below left=1cm of global] (func) {Function Scope\\ \texttt{int y}};
    \node[table, below right=1cm of func] (block) {Block Scope\\ \texttt{int z}};
    
    \draw[arrow] (func) -- node[right, font=\tiny] {parent} (global);
    \draw[arrow] (block) -- node[right, font=\tiny] {parent} (func);
    
    \node[below=0.2cm of block, font=\itshape\footnotesize] {Lookup z: Found in Block};
    \node[right=0.2cm of block, font=\itshape\footnotesize] {Lookup x: Block $\to$ Func $\to$ Global (Found)};
    \end{tikzpicture}%
    }
    \caption{Ilustrasi Chained Symbol Tables (Cactus Stack)}
\end{figure}

\section{Shadowing dan Resolusi Nama}

\subsection{Konsep Shadowing}
\textit{Shadowing} terjadi ketika deklarasi variabel di \textit{nested scope} "menutupi" variabel dengan nama yang sama di \textit{outer scope}. Kompilator perlu memberi peringatan (\textit{warning}) jika shadowing tidak disengaja, namun harus mengizinkannya secara legal.
\begin{lstlisting}[language=C]
int x = 10; // Global x
void foo() {
    int x = 20; // x ini 'melindungi' foo dari akses ke global x
    {
        int x = 30; // x ini melindung blok ini
        print(x);   // Harus 30
    }
    print(x);       // Harus 20
}
\end{lstlisting}

\subsection{Algoritma Resolusi Nama (Lookup)}
Proses pencarian identifier dilakukan secara hierarkis:
\begin{enumerate}
    \item Mulai dari \texttt{current\_scope}. Jika ditemukan, kembalikan object Symbol tersebut.
    \item Jika tidak, pindah ke \texttt{current\_scope->parent}.
    \item Ulangi langkah 2 sampai menemukan \texttt{Global Scope}.
    \item Jika sudah sampai puncak (Global) dan masih tidak ketemu, lemparkan error: \texttt{Undefined Variable 'x'}.
\end{enumerate}

\begin{figure}[!htbp]
    \centering
    \adjustbox{max width=0.8\textwidth,center}{%
    \begin{tikzpicture}[
        scope/.style={draw, rectangle, rounded corners, minimum width=4cm, minimum height=1cm, align=left, fill=white, drop shadow},
        arrow/.style={->, >=stealth, thick, dashed}
    ]
    \node[scope] (s0) {Global: \texttt{int x}};
    \node[scope, below=0.5cm of s0] (s1) {Function foo: \texttt{int x}};
    \node[scope, below=0.5cm of s1] (s2) {Inner Block: \texttt{int x}};
    
    \draw[arrow] (s2.east) to[bend right=45] node[right, font=\tiny] {Ref x (starts here)} (s2.south east);
    \draw[arrow] (s2) -- node[left, font=\tiny] {Visible?} (s1);
    \draw[arrow] (s1) -- node[left, font=\tiny] {Visible?} (s0);
    
    \node[right=2cm of s1, align=left, font=\small, text width=4cm] {
        \textbf{Lookup logic}:\\
        Search(z) $\to$ not in Block\\
        $\to$ Parent (foo)\\
        $\to$ Parent (Global)
    };
    \end{tikzpicture}%
    }
    \caption{Visualisasi Hierarki Scope untuk Resolusi Variabel}
\end{figure}

\section{Penanganan Scope Entry dan Exit}

\subsection{Function Scope}
Saat mendeklarasikan fungsi, parser harus membuat scope baru untuk menampung parameter fungsi dan variabel lokal di dalamnya.

\subsection{Block Scope}
Setiap blok kode yang dibatasi kurung kurawal menciptakan scope sementara. Kompilator memanggil \texttt{beginScope()} saat menemukan \code{\{} dan \texttt{endScope()} saat menemukan \code{\}}.

\subsection{Contoh Kasus}
\begin{lstlisting}[language=C]
int x = 1; // Global
void f() {
    int x = 2; // Shadows global x
    {
        int x = 3; // Shadows f's x
    }
}
\end{lstlisting}
Symbol table akan mengelola tiga versi variabel \texttt{x} tersebut pada level nesting yang berbeda.

\section{Implementasi Symbol Table yang Lengkap}

Implementasi yang robust memerlukan manajemen memori yang baik untuk setiap entri simbol.

\begin{lstlisting}[language=C++]
class SymbolTable {
    Scope* current;
public:
    void beginScope() { current = new Scope(current); }
    void endScope() {
        Scope* old = current;
        current = current->parent;
        delete old;
    }
    bool insert(string name, Symbol* s) {
        return current->add(name, s);
    }
    Symbol* lookup(string name) {
        return current->find(name);
    }
};
\end{lstlisting}

\begin{figure}[!htbp]
    \centering
    \adjustbox{max width=0.8\textwidth,center}{%
    \begin{tikzpicture}[
        table/.style={rectangle, draw=green!50, fill=green!10, font=\tiny, align=center},
        arrow/.style={->, >=stealth, thick}
    ]
    \node[table] (parser) {Parser};
    \node[table, right=2cm of parser] (st) {Symbol Table};
    \node[table, below=1cm of st] (checker) {Type Checker};
    \draw[arrow] (parser) -- node[above, font=\tiny] {Insert/Lookup} (st);
    \draw[arrow] (st) -- node[right, font=\tiny] {Verify Types} (checker);
    \end{tikzpicture}%
    }
    \caption{Interaksi Symbol Table dalam integrasi sistem}
\end{figure}


% ============================================================
% AKTIVITAS PEMBELAJARAN
% ============================================================
\begin{aktivitas}
  \item \textbf{Hash Table Implementation}: Implementasikan symbol table dengan hash table dan chaining.
  \item \textbf{Scope Testing}: Buat program dengan nested scopes dan test symbol table operations.
  \item \textbf{Type Checker}: Implementasikan type checker untuk ekspresi aritmatika dan assignment.
  \item \textbf{Debug Visualization}: Buat tool untuk visualisasi symbol table dan scope hierarchy.
  \item \textbf{Performance Analysis}: Bandingkan berbagai hash functions untuk symbol table.
\end{aktivitas}

% ============================================================
% LATIHAN DAN REFLEKSI
% ============================================================
\begin{latihan}
  \item Implementasikan symbol table with support for nested scopes menggunakan hash table!
  \item Buat type checker untuk bahasa dengan int, float, dan boolean types!
  \item Analisis complexity dari symbol table operations (insert, lookup, delete)!
  \item Implementasikan scope stack untuk bahasa dengan function definitions!
  \item Desain error reporting system untuk type errors!
  \item \textbf{Refleksi}: Bagian mana dari symbol table implementation yang paling menantang?
\end{latihan}

% ============================================================
% ASESMEN
% ============================================================
\begin{asesmen}
\textbf{Instrumen Penilaian untuk Sub-CPMK 3.1-3.2}

\textbf{A. Pilihan Ganda}

\begin{enumerate}
  \item Data structure yang PALING cocok untuk symbol table adalah:
  \begin{enumerate}
    \item Linked list
    \item Array
    \item Hash table
    \item Binary tree
  \end{enumerate}
  
  \item Scope management menggunakan:
  \begin{enumerate}
    \item Queue
    \item Stack
    \item Priority queue
    \item Heap
  \end{enumerate}
  
  \item Type checking memastikan:
  \begin{enumerate}
    \item Syntax correctness
    \item Type compatibility
    \item Runtime efficiency
    \item Code optimization
  \end{enumerate}
\end{enumerate}

\textbf{B. Essay}

\begin{enumerate}
  \item Jelaskan implementasi symbol table dengan nested scopes dan berikan contoh kode!
  \item Desain type system untuk bahasa dengan arrays, functions, dan pointers!
\end{enumerate}

\textbf{Rubrik Penilaian}: Lihat Lampiran A
\end{asesmen}

% ============================================================
% CHECKLIST KOMPETENSI
% ============================================================
\begin{checklist}
  \item Saya dapat mengimplementasikan symbol table dengan nested scopes
  \item Saya dapat melakukan type checking untuk ekspresi kompleks
  \item Saya memahami berbagai implementasi symbol table
  \item Saya dapat mengelola scope dengan stack structure
  \item Saya dapat mendesain type system yang sederhana
  \item Saya dapat mengimplementasikan type inference algorithm
\end{checklist}

% ============================================================
% RANGKUMAN
% ============================================================
\begin{rangkuman}
Bab ini membahas symbol table dan scope management, termasuk implementasi hash table, nested scopes, type system, dan type checking. Mahasiswa belajar membangun data structure fundamental untuk semantic analysis.

\textbf{Poin Kunci:}
\begin{itemize}
  \item Symbol table menyimpan informasi identifiers dengan akses cepat
  \item Nested scopes dikelola menggunakan stack structure
  \item Type checking memastikan type compatibility dalam expressions
  \item Hash table adalah implementasi efisien untuk symbol table
  \item Type system adalah fondasi untuk semantic analysis
\end{itemize}

\textbf{Kata Kunci}: \compiler{Symbol Table}, \compiler{Scope Management}, \compiler{Type System}, \compiler{Type Checking}, \compiler{Hash Table}, \compiler{Nested Scopes}, \compiler{Type Inference}
\end{rangkuman}

\ifSubfilesClassLoaded{%
    \clearpage
    \printbibliography[title={Daftar Pustaka}]
    \end{refsection}
}{}

\end{document}

\cleardoublepage
% Bab 6: Top-Down Parsing dan Recursive Descent
% File ini dapat dikompilasi terpisah atau sebagai bagian dari main.tex

\chapter{Top-Down Parsing dan Recursive Descent}
\label{chap:top-down-parsing}

\section{Tujuan Pembelajaran}

Setelah mempelajari bab ini, mahasiswa diharapkan mampu:
\begin{enumerate}
    \item Menjelaskan konsep top-down parsing dan perbedaannya dengan bottom-up parsing
    \item Memahami LL parsing dan karakteristiknya
    \item Mengimplementasikan recursive descent parser untuk grammar sederhana
    \item Menangani precedence dan associativity dalam recursive descent parser
    \item Mengimplementasikan error recovery pada recursive descent parser
    \item Mengintegrasikan lexer dengan recursive descent parser
    \item Mengevaluasi ekspresi aritmatika menggunakan recursive descent parser
\end{enumerate}

\section{Pendahuluan}

Setelah mempelajari lexical analysis dan context-free grammar pada bab-bab sebelumnya, kita sekarang akan mempelajari bagaimana mengimplementasikan parser yang menganalisis struktur sintaksis dari stream token yang dihasilkan oleh lexer. Top-down parsing adalah salah satu pendekatan yang paling intuitif dan mudah diimplementasikan secara manual.

Menurut sumber terbuka:

\begin{quote}
``Top-down parsers (recursive descent) – easy to hand-write; better for LL(1) grammars, when unambiguous. Works by writing functions for grammar nonterminals (e.g. expression(), term(), factor()) that consume tokens one at a time.''\cite{opengenus2024lexer}
\end{quote}

Pendekatan top-down parsing dimulai dari start symbol grammar dan mencoba menurunkan (derive) input dengan membangun parse tree dari root ke leaves. Ini berbeda dengan bottom-up parsing yang membangun parse tree dari leaves ke root.

\section{Konsep Top-Down Parsing}

\subsection{Definisi Top-Down Parsing}

Top-down parsing adalah teknik parsing yang dimulai dari start symbol grammar dan mencoba menurunkan input dengan menerapkan production rules dari atas ke bawah. Parser mencoba mencocokkan input dengan memprediksi production mana yang harus digunakan berdasarkan lookahead token.

Karakteristik utama top-down parsing:
\begin{itemize}
    \item Membangun parse tree dari root (start symbol) ke leaves (terminals)
    \item Menggunakan leftmost derivation
    \item Memerlukan lookahead untuk memprediksi production yang tepat
    \item Dapat diimplementasikan secara recursive atau iterative dengan stack
\end{itemize}

\subsection{LL Parsing}

LL parsing adalah kelas top-down parsing yang membaca input dari \textbf{L}eft ke right dan menghasilkan \textbf{L}eftmost derivation. Notasi LL(k) menunjukkan bahwa parser menggunakan k token lookahead untuk membuat keputusan parsing.

Menurut sumber dari USNA:

\begin{quote}
``Top-down parsing starts from the start symbol and tries to rewrite it to match the input, building a parse tree from root to leaves. LL parsing means scanning input Left-to-right, producing a Leftmost derivation, using k-token lookahead (usually LL(1)).''\footnote{\url{https://www.usna.edu/Users/cs/wcbrown/courses/F20SI413/lec/l09/lec.html}}
\end{quote}

LL(1) adalah yang paling umum digunakan karena hanya memerlukan satu token lookahead, membuat implementasinya lebih sederhana dan efisien.

\subsection{Keuntungan dan Keterbatasan Top-Down Parsing}

\textbf{Keuntungan:}
\begin{itemize}
    \item Mudah diimplementasikan secara manual (recursive descent)
    \item Error messages yang lebih intuitif (dapat menunjukkan posisi error dengan tepat)
    \item Tidak memerlukan preprocessing grammar yang kompleks (untuk grammar LL(1))
    \item Cocok untuk grammar yang sudah dalam bentuk yang sesuai
\end{itemize}

\textbf{Keterbatasan:}
\begin{itemize}
    \item Tidak dapat menangani left recursion secara langsung
    \item Memerlukan grammar yang sudah di-factoring untuk menghindari ambiguity
    \item Tidak sekuat LR parsing dalam hal kemampuan parsing
    \item Beberapa grammar memerlukan transformasi sebelum dapat di-parse dengan top-down
\end{itemize}

\section{Recursive Descent Parsing}

\subsection{Konsep Recursive Descent}

Recursive descent parsing adalah teknik implementasi top-down parsing di mana setiap non-terminal dalam grammar direpresentasikan sebagai sebuah fungsi. Fungsi-fungsi ini saling memanggil secara recursive sesuai dengan struktur grammar.

Menurut sumber dari Ernest Chu:

\begin{quote}
``Recursive-descent parsing is a hand-written parser (one function per non-terminal), possibly with backtracking. When you eliminate left recursion and factor grammar properly, you can build deterministic predictive parsers (LL(1))—recursive descent without backtracking.''\footnote{\url{https://ernestchu.github.io/course-notes/courses/cse360-design-and-implementation-of-compiler/syntax-analysis/top-down-parsing.html}}
\end{quote}

Struktur dasar recursive descent parser:
\begin{enumerate}
    \item Setiap non-terminal memiliki fungsi sendiri
    \item Fungsi membaca token dari input stream
    \item Fungsi memanggil fungsi lain sesuai dengan production rules
    \item Terminal dicocokkan langsung dengan token saat ini
\end{enumerate}

\subsection{Implementasi Dasar}

Mari kita lihat contoh implementasi recursive descent parser untuk grammar ekspresi aritmatika sederhana. Grammar yang akan kita gunakan:

\begin{verbatim}
E  -> T E'
E' -> + T E' | epsilon
T  -> F T'
T' -> * F T' | epsilon
F  -> ( E ) | id | num
\end{verbatim}

Grammar ini sudah dalam bentuk yang sesuai untuk LL(1) parsing karena:
\begin{itemize}
    \item Tidak ada left recursion
    \item Sudah di-factoring (E' dan T' menangani associativity)
    \item Setiap production dapat diputuskan dengan satu token lookahead
\end{itemize}

Implementasi dalam C++:

\begin{lstlisting}[language=C++, caption={Struktur dasar recursive descent parser}]
#include <iostream>
#include <string>
#include <vector>

enum TokenType {
    TOK_ID, TOK_NUM, TOK_PLUS, TOK_MUL,
    TOK_LPAREN, TOK_RPAREN, TOK_END, TOK_ERROR
};

struct Token {
    TokenType type;
    std::string lexeme;
    int line, col;
};

class RecursiveDescentParser {
private:
    std::vector<Token> tokens;
    size_t current;
    Token lookahead;
    
    void nextToken() {
        if (current < tokens.size()) {
            lookahead = tokens[current++];
        } else {
            lookahead = {TOK_END, "", 0, 0};
        }
    }
    
    void match(TokenType expected) {
        if (lookahead.type == expected) {
            nextToken();
        } else {
            error("Expected " + tokenToString(expected) + 
                  " but got " + lookahead.lexeme);
        }
    }
    
    void error(const std::string& msg) {
        std::cerr << "Syntax error at line " << lookahead.line 
                  << ", col " << lookahead.col << ": " << msg << std::endl;
        exit(1);
    }
    
public:
    RecursiveDescentParser(const std::vector<Token>& t) 
        : tokens(t), current(0) {
        nextToken();
    }
    
    // Grammar: E -> T E'
    void parseE() {
        parseT();
        parseEPrime();
    }
    
    // Grammar: E' -> + T E' | epsilon
    void parseEPrime() {
        if (lookahead.type == TOK_PLUS) {
            match(TOK_PLUS);
            parseT();
            parseEPrime();
        }
        // else: epsilon production, do nothing
    }
    
    // Grammar: T -> F T'
    void parseT() {
        parseF();
        parseTPrime();
    }
    
    // Grammar: T' -> * F T' | epsilon
    void parseTPrime() {
        if (lookahead.type == TOK_MUL) {
            match(TOK_MUL);
            parseF();
            parseTPrime();
        }
        // else: epsilon production, do nothing
    }
    
    // Grammar: F -> ( E ) | id | num
    void parseF() {
        if (lookahead.type == TOK_LPAREN) {
            match(TOK_LPAREN);
            parseE();
            match(TOK_RPAREN);
        } else if (lookahead.type == TOK_ID) {
            match(TOK_ID);
        } else if (lookahead.type == TOK_NUM) {
            match(TOK_NUM);
        } else {
            error("Expected identifier, number, or '('");
        }
    }
    
    void parse() {
        parseE();
        if (lookahead.type != TOK_END) {
            error("Extra input after expression");
        }
        std::cout << "Parse successful!" << std::endl;
    }
};
\end{lstlisting}

\section{Handling Precedence dan Associativity}

\subsection{Konsep Precedence}

Precedence menentukan urutan evaluasi operator ketika beberapa operator muncul dalam ekspresi yang sama. Misalnya, dalam ekspresi \texttt{a + b * c}, operator \texttt{*} memiliki precedence lebih tinggi daripada \texttt{+}, sehingga dievaluasi terlebih dahulu.

Dalam recursive descent parser, precedence di-handle melalui struktur grammar. Operator dengan precedence lebih tinggi berada di level yang lebih dalam dalam parse tree.

\subsection{Handling Associativity}

Associativity menentukan bagaimana operator dengan precedence yang sama dievaluasi. Ada dua jenis:
\begin{itemize}
    \item \textbf{Left-associative}: Dievaluasi dari kiri ke kanan, misalnya \texttt{a - b - c} = \texttt{(a - b) - c}
    \item \textbf{Right-associative}: Dievaluasi dari kanan ke kiri, misalnya \texttt{a = b = c} = \texttt{a = (b = c)}
\end{itemize}

Dalam grammar yang kita gunakan, E' dan T' menggunakan left recursion yang diubah menjadi right recursion untuk menangani left associativity dengan benar.

Contoh grammar untuk menangani precedence dan associativity:

\begin{verbatim}
E  -> T E'        (expression level, lowest precedence)
E' -> + T E' | epsilon  (addition, left-associative)
T  -> F T'        (term level, higher precedence)
T' -> * F T' | epsilon  (multiplication, left-associative)
F  -> ( E ) | id  (factor level, highest precedence)
\end{verbatim}

Struktur ini memastikan bahwa:
\begin{itemize}
    \item Operator \texttt{*} memiliki precedence lebih tinggi daripada \texttt{+} (T berada di bawah E)
    \item Kedua operator left-associative (menggunakan right recursion dengan tail call)
    \item Parentheses memiliki precedence tertinggi (F)
\end{itemize}

\subsection{Implementasi dengan Evaluasi}

Berikut adalah implementasi recursive descent parser yang tidak hanya mem-parse tetapi juga mengevaluasi ekspresi:

\begin{lstlisting}[language=C++, caption={Recursive descent parser dengan evaluasi}]
class ExpressionEvaluator {
private:
    std::vector<Token> tokens;
    size_t current;
    Token lookahead;
    
    void nextToken() {
        if (current < tokens.size()) {
            lookahead = tokens[current++];
        } else {
            lookahead = {TOK_END, "", 0, 0};
        }
    }
    
    void match(TokenType expected) {
        if (lookahead.type == expected) {
            nextToken();
        } else {
            throw std::runtime_error("Syntax error");
        }
    }
    
public:
    ExpressionEvaluator(const std::vector<Token>& t) 
        : tokens(t), current(0) {
        nextToken();
    }
    
    // E -> T E'
    // Returns value of expression
    int parseE() {
        int value = parseT();
        return parseEPrime(value);
    }
    
    // E' -> + T E' | epsilon
    // Accumulates addition operations
    int parseEPrime(int left) {
        if (lookahead.type == TOK_PLUS) {
            match(TOK_PLUS);
            int right = parseT();
            return parseEPrime(left + right);
        }
        return left;  // epsilon production
    }
    
    // T -> F T'
    int parseT() {
        int value = parseF();
        return parseTPrime(value);
    }
    
    // T' -> * F T' | epsilon
    // Accumulates multiplication operations
    int parseTPrime(int left) {
        if (lookahead.type == TOK_MUL) {
            match(TOK_MUL);
            int right = parseF();
            return parseTPrime(left * right);
        }
        return left;  // epsilon production
    }
    
    // F -> ( E ) | num
    int parseF() {
        if (lookahead.type == TOK_LPAREN) {
            match(TOK_LPAREN);
            int value = parseE();
            match(TOK_RPAREN);
            return value;
        } else if (lookahead.type == TOK_NUM) {
            int value = std::stoi(lookahead.lexeme);
            match(TOK_NUM);
            return value;
        } else {
            throw std::runtime_error("Expected number or '('");
        }
    }
    
    int evaluate() {
        int result = parseE();
        if (lookahead.type != TOK_END) {
            throw std::runtime_error("Extra input");
        }
        return result;
    }
};
\end{lstlisting}

\section{Error Recovery pada Recursive Descent}

\subsection{Pentingnya Error Recovery}

Error recovery adalah kemampuan parser untuk melanjutkan parsing setelah menemukan error, sehingga dapat melaporkan multiple errors dalam satu pass. Tanpa error recovery, parser akan berhenti pada error pertama.

\subsection{Strategi Error Recovery}

Beberapa strategi error recovery yang umum digunakan:

\subsubsection{Synchronization Points}

Menentukan synchronization points (token-token yang dapat digunakan untuk recovery), seperti:
\begin{itemize}
    \item Statement terminators (\texttt{;}, \texttt{\}})
    \item Keywords yang menandai awal konstruksi baru (\texttt{if}, \texttt{while}, \texttt{return})
    \item Operator yang jelas (\texttt{+}, \texttt{-}, \texttt{*})
\end{itemize}

\subsubsection{Panic Mode Recovery}

Ketika error ditemukan, parser membuang token sampai menemukan synchronization point:

\begin{lstlisting}[language=C++, caption={Panic mode error recovery}]
void parseE() {
    parseT();
    parseEPrime();
}

void parseEPrime() {
    if (lookahead.type == TOK_PLUS) {
        match(TOK_PLUS);
        parseT();
        parseEPrime();
    } else if (!isValidFollow(lookahead.type)) {
        // Error recovery: skip until synchronization point
        error("Expected '+' or end of expression");
        while (!isSynchronizationPoint(lookahead.type) && 
               lookahead.type != TOK_END) {
            nextToken();
        }
    }
    // else: valid follow token, epsilon production
}

bool isSynchronizationPoint(TokenType t) {
    return t == TOK_RPAREN || t == TOK_END || 
           t == TOK_SEMICOLON;
}

bool isValidFollow(TokenType t) {
    return t == TOK_RPAREN || t == TOK_END || 
           t == TOK_SEMICOLON || t == TOK_PLUS;
}
\end{lstlisting}

\subsubsection{Error Production}

Menambahkan production khusus untuk menangani error:

\begin{verbatim}
E' -> + T E' | epsilon | error E'
\end{verbatim}

Ketika error ditemukan, parser dapat menggunakan error production untuk recovery.

\subsection{Implementasi Error Recovery yang Lebih Baik}

Berikut adalah implementasi yang lebih robust dengan error recovery:

\begin{lstlisting}[language=C++, caption={Error recovery yang lebih baik}]
class ParserWithRecovery {
private:
    int errorCount;
    std::vector<Token> tokens;
    size_t current;
    Token lookahead;
    
    void nextToken() {
        if (current < tokens.size()) {
            lookahead = tokens[current++];
        } else {
            lookahead = {TOK_END, "", 0, 0};
        }
    }
    
    void error(const std::string& msg) {
        errorCount++;
        std::cerr << "Error at line " << lookahead.line 
                  << ", col " << lookahead.col << ": " 
                  << msg << std::endl;
    }
    
    void synchronize() {
        // Skip tokens until synchronization point
        while (lookahead.type != TOK_END) {
            if (isSynchronizationPoint(lookahead.type)) {
                return;
            }
            nextToken();
        }
    }
    
    bool isSynchronizationPoint(TokenType t) {
        return t == TOK_RPAREN || t == TOK_SEMICOLON || 
               t == TOK_END || t == TOK_PLUS || t == TOK_MUL;
    }
    
public:
    ParserWithRecovery(const std::vector<Token>& t) 
        : tokens(t), current(0), errorCount(0) {
        nextToken();
    }
    
    void parseE() {
        try {
            parseT();
            parseEPrime();
        } catch (...) {
            error("Error in expression");
            synchronize();
        }
    }
    
    void parseEPrime() {
        if (lookahead.type == TOK_PLUS) {
            match(TOK_PLUS);
            parseT();
            parseEPrime();
        } else if (!isValidFollow(lookahead.type)) {
            error("Expected '+' or end of expression");
            synchronize();
        }
    }
    
    int getErrorCount() const { return errorCount; }
};
\end{lstlisting}

\section{Integrasi Lexer dengan Parser}

\subsection{Arsitektur Integrasi}

Dalam implementasi praktis, lexer dan parser bekerja bersama dalam pipeline:

\begin{lstlisting}[language={},basicstyle=\ttfamily\footnotesize,breaklines=true,breakatwhitespace=false]
Source Code -> Lexer -> Token Stream -> Parser -> Parse Tree/AST
\end{lstlisting}

Lexer membaca source code karakter demi karakter dan menghasilkan stream token. Parser kemudian membaca token dari stream ini.

\subsection{Implementasi Terintegrasi}

Berikut adalah contoh implementasi lexer dan parser yang terintegrasi:

\begin{lstlisting}[language=C++, caption={Lexer dan parser terintegrasi}]
#include <iostream>
#include <string>
#include <cctype>

class IntegratedLexerParser {
private:
    std::string input;
    size_t pos;
    int line, col;
    Token lookahead;
    
    // Lexer functions
    void skipWhitespace() {
        while (pos < input.size() && isspace(input[pos])) {
            if (input[pos] == '\n') {
                line++;
                col = 1;
            } else {
                col++;
            }
            pos++;
        }
    }
    
    Token nextToken() {
        skipWhitespace();
        
        if (pos >= input.size()) {
            return {TOK_END, "", line, col};
        }
        
        char c = input[pos];
        int startCol = col;
        
        // Identifier: [a-zA-Z][a-zA-Z0-9]*
        if (isalpha(c)) {
            std::string lexeme;
            while (pos < input.size() && isalnum(input[pos])) {
                lexeme += input[pos++];
                col++;
            }
            return {TOK_ID, lexeme, line, startCol};
        }
        
        // Number: [0-9]+
        if (isdigit(c)) {
            std::string lexeme;
            while (pos < input.size() && isdigit(input[pos])) {
                lexeme += input[pos++];
                col++;
            }
            return {TOK_NUM, lexeme, line, startCol};
        }
        
        // Operators and punctuation
        pos++;
        col++;
        switch (c) {
            case '+': return {TOK_PLUS, "+", line, startCol};
            case '*': return {TOK_MUL, "*", line, startCol};
            case '(': return {TOK_LPAREN, "(", line, startCol};
            case ')': return {TOK_RPAREN, ")", line, startCol};
            default: return {TOK_ERROR, std::string(1, c), line, startCol};
        }
    }
    
    void match(TokenType expected) {
        if (lookahead.type == expected) {
            lookahead = nextToken();
        } else {
            std::cerr << "Error: Expected " << expected 
                      << " but got " << lookahead.lexeme 
                      << " at line " << lookahead.line 
                      << ", col " << lookahead.col << std::endl;
            exit(1);
        }
    }
    
public:
    IntegratedLexerParser(const std::string& s) 
        : input(s), pos(0), line(1), col(1) {
        lookahead = nextToken();
    }
    
    // Parser functions (same as before)
    void parseE() {
        parseT();
        parseEPrime();
    }
    
    void parseEPrime() {
        if (lookahead.type == TOK_PLUS) {
            match(TOK_PLUS);
            parseT();
            parseEPrime();
        }
    }
    
    void parseT() {
        parseF();
        parseTPrime();
    }
    
    void parseTPrime() {
        if (lookahead.type == TOK_MUL) {
            match(TOK_MUL);
            parseF();
            parseTPrime();
        }
    }
    
    void parseF() {
        if (lookahead.type == TOK_LPAREN) {
            match(TOK_LPAREN);
            parseE();
            match(TOK_RPAREN);
        } else if (lookahead.type == TOK_ID) {
            match(TOK_ID);
        } else if (lookahead.type == TOK_NUM) {
            match(TOK_NUM);
        } else {
            std::cerr << "Error: Expected id, num, or '('" << std::endl;
            exit(1);
        }
    }
    
    void parse() {
        parseE();
        if (lookahead.type != TOK_END) {
            std::cerr << "Error: Extra input" << std::endl;
            exit(1);
        }
        std::cout << "Parse successful!" << std::endl;
    }
};
\end{lstlisting}

\section{Contoh Lengkap: Parser untuk Ekspresi Aritmatika}

Berikut adalah contoh lengkap implementasi recursive descent parser untuk ekspresi aritmatika yang dapat menangani:
\begin{itemize}
    \item Operasi penjumlahan dan perkalian
    \item Precedence (perkalian lebih tinggi dari penjumlahan)
    \item Left associativity
    \item Parentheses
    \item Identifier dan literal angka
    \item Error reporting yang informatif
\end{itemize}

\begin{lstlisting}[language=C++, caption={Parser lengkap untuk ekspresi aritmatika}]
#include <iostream>
#include <string>
#include <vector>
#include <cctype>
#include <stdexcept>

enum TokenType {
    TOK_ID, TOK_NUM, TOK_PLUS, TOK_MINUS, TOK_MUL, TOK_DIV,
    TOK_LPAREN, TOK_RPAREN, TOK_END, TOK_ERROR
};

struct Token {
    TokenType type;
    std::string lexeme;
    int line, col;
    
    Token(TokenType t, const std::string& l, int ln, int c)
        : type(t), lexeme(l), line(ln), col(c) {}
};

class ArithmeticParser {
private:
    std::string input;
    size_t pos;
    int line, col;
    Token lookahead;
    std::vector<std::string> errors;
    
    void skipWhitespace() {
        while (pos < input.size() && isspace(input[pos])) {
            if (input[pos] == '\n') {
                line++;
                col = 1;
            } else {
                col++;
            }
            pos++;
        }
    }
    
    Token nextToken() {
        skipWhitespace();
        
        if (pos >= input.size()) {
            return Token(TOK_END, "", line, col);
        }
        
        char c = input[pos];
        int startLine = line, startCol = col;
        
        // Identifier
        if (isalpha(c) || c == '_') {
            std::string lexeme;
            while (pos < input.size() && 
                   (isalnum(input[pos]) || input[pos] == '_')) {
                lexeme += input[pos++];
                col++;
            }
            return Token(TOK_ID, lexeme, startLine, startCol);
        }
        
        // Number
        if (isdigit(c)) {
            std::string lexeme;
            while (pos < input.size() && isdigit(input[pos])) {
                lexeme += input[pos++];
                col++;
            }
            return Token(TOK_NUM, lexeme, startLine, startCol);
        }
        
        // Operators
        pos++;
        col++;
        switch (c) {
            case '+': return Token(TOK_PLUS, "+", startLine, startCol);
            case '-': return Token(TOK_MINUS, "-", startLine, startCol);
            case '*': return Token(TOK_MUL, "*", startLine, startCol);
            case '/': return Token(TOK_DIV, "/", startLine, startCol);
            case '(': return Token(TOK_LPAREN, "(", startLine, startCol);
            case ')': return Token(TOK_RPAREN, ")", startLine, startCol);
            default: 
                return Token(TOK_ERROR, std::string(1, c), startLine, startCol);
        }
    }
    
    void match(TokenType expected) {
        if (lookahead.type == expected) {
            lookahead = nextToken();
        } else {
            std::string msg = "Expected " + tokenToString(expected) +
                            " but got " + lookahead.lexeme +
                            " at line " + std::to_string(lookahead.line) +
                            ", col " + std::to_string(lookahead.col);
            errors.push_back(msg);
            throw std::runtime_error(msg);
        }
    }
    
    std::string tokenToString(TokenType t) {
        switch (t) {
            case TOK_ID: return "identifier";
            case TOK_NUM: return "number";
            case TOK_PLUS: return "'+'";
            case TOK_MINUS: return "'-'";
            case TOK_MUL: return "'*'";
            case TOK_DIV: return "'/'";
            case TOK_LPAREN: return "'('";
            case TOK_RPAREN: return "')'";
            case TOK_END: return "end of input";
            default: return "unknown";
        }
    }
    
public:
    ArithmeticParser(const std::string& s) 
        : input(s), pos(0), line(1), col(1) {
        lookahead = nextToken();
    }
    
    // E -> T E'
    void parseE() {
        parseT();
        parseEPrime();
    }
    
    // E' -> (+ | -) T E' | epsilon
    void parseEPrime() {
        if (lookahead.type == TOK_PLUS || lookahead.type == TOK_MINUS) {
            TokenType op = lookahead.type;
            match(op);
            parseT();
            parseEPrime();
        }
    }
    
    // T -> F T'
    void parseT() {
        parseF();
        parseTPrime();
    }
    
    // T' -> (* | /) F T' | epsilon
    void parseTPrime() {
        if (lookahead.type == TOK_MUL || lookahead.type == TOK_DIV) {
            TokenType op = lookahead.type;
            match(op);
            parseF();
            parseTPrime();
        }
    }
    
    // F -> ( E ) | id | num
    void parseF() {
        if (lookahead.type == TOK_LPAREN) {
            match(TOK_LPAREN);
            parseE();
            match(TOK_RPAREN);
        } else if (lookahead.type == TOK_ID) {
            match(TOK_ID);
        } else if (lookahead.type == TOK_NUM) {
            match(TOK_NUM);
        } else {
            std::string msg = "Expected identifier, number, or '(' at line " +
                            std::to_string(lookahead.line) +
                            ", col " + std::to_string(lookahead.col);
            errors.push_back(msg);
            throw std::runtime_error(msg);
        }
    }
    
    bool parse() {
        try {
            parseE();
            if (lookahead.type != TOK_END) {
                errors.push_back("Extra input after expression");
                return false;
            }
            return errors.empty();
        } catch (...) {
            return false;
        }
    }
    
    void printErrors() {
        for (const auto& err : errors) {
            std::cerr << err << std::endl;
        }
    }
};

int main() {
    std::string input;
    std::cout << "Enter expression: ";
    std::getline(std::cin, input);
    
    ArithmeticParser parser(input);
    if (parser.parse()) {
        std::cout << "Parse successful!" << std::endl;
    } else {
        std::cout << "Parse failed!" << std::endl;
        parser.printErrors();
    }
    
    return 0;
}
\end{lstlisting}

\section{Kesimpulan}

Dalam bab ini, kita telah mempelajari:

\begin{enumerate}
    \item Top-down parsing adalah teknik parsing yang membangun parse tree dari root ke leaves
    \item LL parsing adalah kelas top-down parsing yang membaca input left-to-right dan menghasilkan leftmost derivation
    \item Recursive descent parsing adalah implementasi top-down parsing di mana setiap non-terminal direpresentasikan sebagai fungsi
    \item Precedence dan associativity di-handle melalui struktur grammar yang tepat
    \item Error recovery memungkinkan parser untuk melanjutkan setelah menemukan error
    \item Lexer dan parser dapat diintegrasikan dalam satu implementasi yang efisien
\end{enumerate}

Recursive descent parsing adalah teknik yang sangat cocok untuk implementasi manual parser karena mudah dipahami dan diimplementasikan. Namun, perlu diingat bahwa grammar harus dalam bentuk yang sesuai (tanpa left recursion, sudah di-factoring) untuk dapat di-parse dengan pendekatan ini.

\section{Latihan}

\begin{enumerate}
    \item Implementasikan recursive descent parser untuk grammar berikut:
    \begin{verbatim}
    S -> if E then S else S | id := E | while E do S
    E -> E + T | T
    T -> T * F | F
    F -> ( E ) | id | num
    \end{verbatim}
    
    \textbf{Perhatikan:} Grammar ini memiliki left recursion. Anda perlu mengeliminasi left recursion terlebih dahulu.
    
    \item Modifikasi parser ekspresi aritmatika untuk menambahkan operator unary minus (misalnya \texttt{-5} atau \texttt{-(3 + 4)}).
    
    \item Implementasikan error recovery pada parser yang telah Anda buat. Test dengan input yang mengandung multiple errors.
    
    \item Buatlah parser yang dapat mengevaluasi ekspresi boolean dengan operator \texttt{AND}, \texttt{OR}, dan \texttt{NOT}. Perhatikan precedence: \texttt{NOT} > \texttt{AND} > \texttt{OR}.
    
    \item Bandingkan recursive descent parser dengan table-driven LL parser. Apa keuntungan dan kerugian masing-masing pendekatan?
    
    \item Implementasikan parser untuk ekspresi dengan right-associative operator (misalnya operator assignment \texttt{=}).
\end{enumerate}

\section{Referensi dan Bahan Bacaan Lanjutan}

Untuk memperdalam pemahaman tentang top-down parsing dan recursive descent, mahasiswa disarankan membaca:

\begin{itemize}
    \item \textbf{Dragon Book}: Aho, Lam, Sethi, \& Ullman (2006). \textit{Compilers: Principles, Techniques, and Tools} \cite{aho2006compilers} - Bab 4: Syntax Analysis, Section 4.4: Top-Down Parsing
    
    \item \textbf{Engineering a Compiler}: Cooper \& Torczon (2011) \cite{cooper2011engineering} - Bab 3: Scanners, Section 3.4: Top-Down Parsing
    
    \item \textbf{OpenGenus Tutorial}: Build Lexer \cite{opengenus2024lexer} - Bagian tentang recursive descent parsing
    
    \item \textbf{USNA Course Notes}: Top-Down Parsing \footnote{\url{https://www.usna.edu/Users/cs/wcbrown/courses/F20SI413/lec/l09/lec.html}}
    
    \item \textbf{Ernest Chu Course Notes}: Syntax Analysis - Top-Down Parsing \footnote{\url{https://ernestchu.github.io/course-notes/courses/cse360-design-and-implementation-of-compiler/syntax-analysis/top-down-parsing.html}}
    
    \item \textbf{TutorialsPoint}: Compiler Design - Top Down Parser \footnote{\url{https://www.tutorialspoint.com/compiler_design/compiler_design_top_down_parser.htm}}
\end{itemize}

\cleardoublepage
% Bab 7: Bottom-Up Parsing, LR Parser, dan Parser Generator
% File ini dapat dikompilasi terpisah atau sebagai bagian dari main.tex

\chapter{Bottom-Up Parsing, LR Parser, dan Parser Generator}
\label{chap:bottom-up-parsing}

\section{Tujuan Pembelajaran}

Setelah mempelajari bab ini, mahasiswa diharapkan mampu:
\begin{enumerate}
    \item Menjelaskan konsep bottom-up parsing dan perbedaannya dengan top-down parsing
    \item Memahami shift-reduce parsing dan operasi-operasinya
    \item Menjelaskan berbagai jenis LR parser (LR(0), SLR(1), CLR(1), LALR(1))
    \item Memahami konstruksi LR parsing table untuk grammar sederhana
    \item Menggunakan parser generator (Bison/Yacc) untuk membuat parser
    \item Mengintegrasikan Flex lexer dengan Bison parser
    \item Menambahkan semantic actions untuk membangun AST
    \item Mengimplementasikan error handling dalam parser generator
\end{enumerate}

\section{Pendahuluan}

Setelah mempelajari top-down parsing pada bab sebelumnya, kita sekarang akan mempelajari pendekatan alternatif yang lebih powerful: bottom-up parsing. Menurut sumber terbuka:

\begin{quote}
``Bottom-up parsers (LR, LALR, GLR) – more powerful; often generated by tools like Bison/Yacc. The choice affects ease of specification and parsing power.''\cite{diznr2024phases}
\end{quote}

Bottom-up parsing membangun parse tree dari leaves (token) ke root (start symbol), yang merupakan kebalikan dari top-down parsing. Pendekatan ini lebih powerful karena dapat menangani lebih banyak jenis grammar, termasuk grammar dengan left recursion yang tidak dapat ditangani langsung oleh top-down parser.

\section{Konsep Bottom-Up Parsing}

\subsection{Definisi Bottom-Up Parsing}

Bottom-up parsing adalah teknik parsing yang dimulai dari input tokens dan mencoba membangun parse tree dari bawah ke atas, dengan tujuan mencapai start symbol. Parser menggunakan rightmost derivation dalam reverse, yaitu membangun derivation dari kanan ke kiri.

Karakteristik utama bottom-up parsing:
\begin{itemize}
    \item Membangun parse tree dari leaves (terminals) ke root (start symbol)
    \item Menggunakan rightmost derivation dalam reverse
    \item Menggunakan stack untuk menyimpan state parsing
    \item Lebih powerful daripada top-down parsing (dapat menangani lebih banyak grammar)
    \item Umumnya diimplementasikan menggunakan parsing table yang di-generate
\end{itemize}

\subsection{Handle dan Reduction}

Konsep penting dalam bottom-up parsing adalah \textbf{handle}. Handle adalah substring dari sentential form saat ini yang cocok dengan right-hand side (RHS) dari suatu production rule, dan reduction terhadap handle ini akan membawa kita lebih dekat ke start symbol.

Menurut definisi formal:

\begin{quote}
``A handle is a substring of the current sentential form that matches the RHS of a production and whose reduction must lead toward the start symbol.''\footnote{\url{https://ebooks.inflibnet.ac.in/csp10/chapter/top-down-parser-parsing-tableshift-reduce-parser/}}
\end{quote}

Contoh: Jika kita memiliki grammar:
\begin{verbatim}
E -> E + T | T
T -> T * F | F
F -> ( E ) | id
\end{verbatim}

Dan sentential form saat ini adalah \texttt{id + id * id}, maka handle yang tepat adalah \texttt{id} (yang dapat di-reduce menjadi F, kemudian T, kemudian E).

\section{Shift-Reduce Parsing}

\subsection{Konsep Shift-Reduce}

Shift-reduce parsing adalah implementasi dasar dari bottom-up parsing yang menggunakan stack dan empat operasi dasar. Menurut GeeksforGeeks:

\begin{quote}
``Shift-Reduce Parser uses a stack and four basic operations:
1. Shift: push the next input symbol onto the stack.
2. Reduce: when the top of stack matches RHS of some grammar rule, pop it and push the LHS nonterminal.
3. Accept: if the stack has start symbol and input is exhausted.
4. Error: no valid shift/reduce possible.''\footnote{\url{https://www.geeksforgeeks.org/compiler-design/shift-reduce-parser-compiler/}}
\end{quote}

\subsection{Operasi Shift}

Operasi \textbf{shift} memindahkan token berikutnya dari input ke stack. Ini dilakukan ketika parser belum menemukan handle yang lengkap di stack.

Contoh: Jika stack berisi \texttt{[E, +]} dan input berikutnya adalah \texttt{id}, maka operasi shift akan menghasilkan stack \texttt{[E, +, id]}.

\subsection{Operasi Reduce}

Operasi \textbf{reduce} mengganti handle di top of stack dengan left-hand side (LHS) dari production rule yang sesuai. Handle harus cocok persis dengan RHS dari suatu production.

Contoh: Jika stack berisi \texttt{[E, +, T, *, F]} dan kita memiliki production \texttt{F -> id}, dan top of stack adalah \texttt{id} yang cocok dengan RHS, maka reduce akan menghasilkan stack \texttt{[E, +, T, *, F]}.

\subsection{Operasi Accept}

Operasi \textbf{accept} terjadi ketika:
\begin{itemize}
    \item Stack hanya berisi start symbol (atau augmented start symbol)
    \item Input sudah habis (hanya end marker \$ tersisa)
\end{itemize}

Ini menandakan bahwa parsing berhasil dan input valid.

\subsection{Operasi Error}

Operasi \textbf{error} terjadi ketika tidak ada operasi shift atau reduce yang valid. Ini berarti input tidak valid menurut grammar.

\subsection{Contoh Shift-Reduce Parsing}

Mari kita lihat contoh parsing ekspresi \texttt{id + id} dengan grammar sederhana:

\begin{verbatim}
E -> E + T | T
T -> id
\end{verbatim}

\begin{table}[H]
\centering
\begin{tabular}{|c|c|c|c|}
\hline
\textbf{Stack} & \textbf{Input} & \textbf{Action} & \textbf{Production} \\
\hline
\$ & id + id \$ & Shift & \\
\hline
\$ id & + id \$ & Reduce & T -> id \\
\hline
\$ T & + id \$ & Reduce & E -> T \\
\hline
\$ E & + id \$ & Shift & \\
\hline
\$ E + & id \$ & Shift & \\
\hline
\$ E + id & \$ & Reduce & T -> id \\
\hline
\$ E + T & \$ & Reduce & E -> E + T \\
\hline
\$ E & \$ & Accept & \\
\hline
\end{tabular}
\caption{Contoh shift-reduce parsing untuk \texttt{id + id}}
\label{tab:shift-reduce-example}
\end{table}

\section{LR Parsers}

\subsection{Definisi LR Parser}

LR parser adalah kelas bottom-up parser yang membaca input dari \textbf{L}eft ke right dan menghasilkan \textbf{R}ightmost derivation dalam reverse. Notasi LR(k) menunjukkan bahwa parser menggunakan k token lookahead.

Menurut GeeksforGeeks:

\begin{quote}
``LR parsers read input Left-to-right and produce a Rightmost derivation in reverse. They use a parsing table to decide when to shift and when to reduce.''\footnote{\url{https://www.geeksforgeeks.org/bottom-up-or-shift-reduce-parsers-set-2/}}
\end{quote}

LR parser menggunakan dua tabel utama:
\begin{itemize}
    \item \textbf{Action Table}: Menentukan aksi (shift, reduce, accept, error) berdasarkan state saat ini dan lookahead token
    \item \textbf{GOTO Table}: Menentukan state berikutnya setelah reduce berdasarkan state saat ini dan non-terminal yang dihasilkan
\end{itemize}

\subsection{Jenis-jenis LR Parser}

Terdapat beberapa varian LR parser, masing-masing dengan karakteristik berbeda:

\subsubsection{LR(0)}

LR(0) adalah varian paling sederhana yang tidak menggunakan lookahead. Karakteristik:
\begin{itemize}
    \item Tidak memerlukan lookahead token
    \item Tabel parsing kecil
    \item Sangat terbatas dalam kemampuan parsing (banyak grammar menghasilkan conflict)
    \item Jarang digunakan dalam praktik
\end{itemize}

\subsubsection{SLR(1) - Simple LR}

SLR(1) menggunakan 1 token lookahead dan Follow sets untuk menentukan kapan melakukan reduce. Karakteristik:
\begin{itemize}
    \item Menggunakan LR(0) item sets
    \item Reduce hanya dilakukan jika lookahead token berada dalam Follow set dari non-terminal yang di-reduce
    \item Lebih powerful daripada LR(0)
    \item Tabel lebih kecil daripada CLR(1)
    \item Masih dapat menghasilkan conflict untuk beberapa grammar
\end{itemize}

Menurut GeeksforGeeks:

\begin{quote}
``SLR(1) uses LR(0) item sets, and reduction is allowed on lookahead symbols in Follow(A) for production A -> $\alpha$ when the item [A -> $\alpha$ •] appears in the state. This can lead to conflicts CLR(1) avoids.''\footnote{\url{https://www.geeksforgeeks.org/bottom-up-or-shift-reduce-parsers-set-2/}}
\end{quote}

\subsubsection{CLR(1) - Canonical LR}

CLR(1) adalah varian paling powerful yang menggunakan full LR(1) items dengan lookahead spesifik. Karakteristik:
\begin{itemize}
    \item Menggunakan LR(1) items (production dengan lookahead spesifik)
    \item Reduce hanya dilakukan pada lookahead token yang spesifik
    \item Dapat menangani lebih banyak grammar daripada SLR(1)
    \item Tabel parsing sangat besar (banyak states)
    \item Lebih lambat dalam konstruksi tabel
\end{itemize}

\subsubsection{LALR(1) - Look-Ahead LR}

LALR(1) adalah kompromi praktis antara SLR(1) dan CLR(1). Karakteristik:
\begin{itemize}
    \item Merge states dari CLR(1) yang memiliki LR(0) core yang sama
    \item Menggabungkan lookahead sets dari states yang di-merge
    \item Jumlah states sama atau mendekati SLR(1)
    \item Lebih powerful daripada SLR(1), hampir sekuat CLR(1)
    \item Digunakan oleh Yacc dan Bison (parser generator populer)
\end{itemize}

Menurut GeeksforGeeks:

\begin{quote}
``LALR(1) merges states in the CLR(1) automaton that have identical LR(0) cores (i.e. same productions \& dot positions), combining their lookahead sets. Then construct table using merged states. This reduces size while often preserving correctness.''\footnote{\url{https://www.geeksforgeeks.org/compiler-design/lalr-parser-with-examples/}}
\end{quote}

\subsection{Perbandingan LR Parser Variants}

\begin{table}[H]
\centering
\begin{tabular}{|l|c|c|c|}
\hline
\textbf{Variant} & \textbf{Lookahead} & \textbf{Table Size} & \textbf{Parsing Power} \\
\hline
LR(0) & None & Smallest & Weakest \\
\hline
SLR(1) & 1 token (Follow sets) & Small & Moderate \\
\hline
LALR(1) & 1 token (merged) & Small-Medium & Strong \\
\hline
CLR(1) & 1 token (full) & Largest & Strongest \\
\hline
\end{tabular}
\caption{Perbandingan varian LR parser}
\label{tab:lr-variants}
\end{table}

\section{Konstruksi LR Parsing Table}

\subsection{Augmented Grammar}

Langkah pertama dalam konstruksi LR parsing table adalah membuat \textbf{augmented grammar}. Kita menambahkan production baru:
\begin{verbatim}
S' -> S
\end{verbatim}
di mana S adalah start symbol asli. Ini memungkinkan state accept yang unambiguous.

\subsection{LR Items}

LR item adalah production dengan dot (•) yang menandai posisi parsing saat ini. Format: \texttt{A -> $\alpha$ • $\beta$}

Contoh:
\begin{itemize}
    \item \texttt{E -> • E + T}: Belum membaca apapun dari production ini
    \item \texttt{E -> E • + T}: Sudah membaca E, menunggu +
    \item \texttt{E -> E + • T}: Sudah membaca E dan +, menunggu T
    \item \texttt{E -> E + T •}: Sudah membaca seluruh RHS, siap untuk reduce
\end{itemize}

\subsubsection{LR(0) Items}

LR(0) item hanya berisi production dengan dot, tanpa informasi lookahead.

\subsubsection{LR(1) Items}

LR(1) item adalah LR(0) item yang ditambahkan dengan lookahead token. Format: \texttt{[A -> $\alpha$ • $\beta$, a]} di mana \texttt{a} adalah lookahead token.

\subsection{Closure Operation}

Closure operation menambahkan semua production yang relevan ke set items. Jika kita memiliki item \texttt{[A -> $\alpha$ • B $\beta$]} dalam set, kita menambahkan semua items \texttt{[B -> • $\gamma$]} untuk setiap production \texttt{B -> $\gamma$}.

Algoritma closure:
\begin{enumerate}
    \item Mulai dengan set items awal
    \item Untuk setiap item \texttt{[A -> $\alpha$ • B $\beta$]}:
    \begin{itemize}
        \item Tambahkan semua items \texttt{[B -> • $\gamma$]} untuk setiap production \texttt{B -> $\gamma$}
        \item Jika LR(1), hitung lookahead untuk items baru
    \end{itemize}
    \item Ulangi sampai tidak ada item baru yang ditambahkan
\end{enumerate}

\subsection{GOTO Operation}

GOTO operation memindahkan dot melewati simbol grammar X. Jika kita memiliki item \texttt{[A -> $\alpha$ • X $\beta$]} dan membaca X, kita mendapatkan item \texttt{[A -> $\alpha$ X • $\beta$]}.

Algoritma GOTO:
\begin{enumerate}
    \item Mulai dengan set items I dan simbol grammar X
    \item Untuk setiap item \texttt{[A -> $\alpha$ • X $\beta$]} dalam I:
    \begin{itemize}
        \item Tambahkan \texttt{[A -> $\alpha$ X • $\beta$]} ke set baru
    \end{itemize}
    \item Ambil closure dari set baru
\end{enumerate}

\subsection{Canonical Collection of Item Sets}

Canonical collection adalah kumpulan semua state yang mungkin dalam LR automaton. Konstruksinya:

\begin{enumerate}
    \item Mulai dengan \texttt{I\_0 = closure(\{S' -> • S, \$\})}
    \item Untuk setiap state I dan setiap simbol grammar X:
    \begin{itemize}
        \item Hitung \texttt{GOTO(I, X)}
        \item Jika hasilnya non-empty dan belum ada, tambahkan sebagai state baru
    \end{itemize}
    \item Ulangi sampai tidak ada state baru
\end{enumerate}

\subsection{Konstruksi Action dan GOTO Tables}

Setelah canonical collection dibuat, kita konstruksi dua tabel:

\subsubsection{Action Table}

Action table menentukan aksi berdasarkan state dan lookahead token:
\begin{itemize}
    \item \textbf{Shift}: Jika \texttt{GOTO(I, a) = J} untuk terminal \texttt{a}, maka \texttt{action[I, a] = shift J}
    \item \textbf{Reduce}: Jika item \texttt{[A -> $\alpha$ •, a]} ada di state I, maka \texttt{action[I, a] = reduce A -> $\alpha$}
    \item \textbf{Accept}: Jika item \texttt{[S' -> S •, \$]} ada di state I, maka \texttt{action[I, \$] = accept}
    \item \textbf{Error}: Jika tidak ada aksi yang valid
\end{itemize}

\subsubsection{GOTO Table}

GOTO table menentukan state berikutnya setelah reduce:
\begin{itemize}
    \item Jika \texttt{GOTO(I, A) = J} untuk non-terminal \texttt{A}, maka \texttt{goto[I, A] = J}
\end{itemize}

\subsection{Contoh Konstruksi Parsing Table (Simplified)}

Mari kita lihat contoh sederhana untuk grammar:
\begin{verbatim}
S -> A A
A -> a A | b
\end{verbatim}

Augmented grammar:
\begin{verbatim}
S' -> S
S -> A A
A -> a A | b
\end{verbatim}

Langkah-langkah konstruksi (disederhanakan):
\begin{enumerate}
    \item Buat I\_0 dengan closure dari \texttt{S' -> • S}
    \item Hitung GOTO untuk setiap simbol
    \item Lanjutkan sampai semua state ditemukan
    \item Konstruksi action dan goto tables
\end{enumerate}

\section{GLR Parsing (Generalized LR)}

\subsection{Konsep GLR}

GLR (Generalized LR) adalah ekstensi dari LR parsing yang dapat menangani ambiguous grammar atau grammar yang akan menghasilkan conflict dalam tabel LR biasa.

Menurut Wikipedia:

\begin{quote}
``GLR extends LR parsing to handle ambiguous grammars or grammars that would cause conflicts in LR tables. It allows multiple possible parse actions in a state and pursues them in parallel.''\footnote{\url{https://en.wikipedia.org/wiki/GLR_parser}}
\end{quote}

GLR parser menjaga multiple stacks atau parse trees aktif secara bersamaan ketika terjadi conflict, dan merge stack prefixes yang mungkin untuk berbagi pekerjaan.

\subsection{Kapan Menggunakan GLR}

GLR parsing berguna untuk:
\begin{itemize}
    \item Grammar yang ambiguous (memiliki multiple parse trees valid)
    \item Grammar yang tidak LR(1) tetapi masih ingin di-parse secara deterministik
    \item Bahasa dengan syntax yang extensible
    \item Natural language processing
\end{itemize}

\section{Parser Generator: Bison dan Yacc}

\subsection{Pengenalan Parser Generator}

Parser generator adalah tool yang secara otomatis menghasilkan parser dari specification grammar. Menurut sumber dari IT Trip:

\begin{quote}
``Bison / YACC: define grammar in a .y file, specify \%token s, grammar rules, actions, etc. Generates C parser (or C++ variants). Flex + Bison: use Flex to build the lexer (.l file), Bison for parser, integrate them via tokens.''\cite{ittrip2024bison}
\end{quote}

Keuntungan menggunakan parser generator:
\begin{itemize}
    \item Menghemat waktu development
    \item Mengurangi kemungkinan error
    \item Mudah di-maintain (ubah grammar, regenerate parser)
    \item Menghasilkan parser yang efisien
    \item Mendukung semantic actions untuk membangun AST
\end{itemize}

\subsection{Yacc (Yet Another Compiler Compiler)}

Yacc adalah parser generator yang dikembangkan di Bell Labs pada tahun 1970-an. Yacc menghasilkan LALR(1) parser dari grammar specification.

\subsection{Bison (GNU Yacc)}

Bison adalah implementasi open source dari Yacc yang dikembangkan oleh GNU Project. Bison lebih powerful dan memiliki fitur tambahan:
\begin{itemize}
    \item Mendukung LALR(1), LR(1), dan GLR parsing
    \item Mendukung C++ output
    \item Error recovery yang lebih baik
    \item Dokumentasi yang lebih lengkap
\end{itemize}

\subsection{Struktur File Bison (.y)}

File Bison memiliki struktur berikut:

\begin{verbatim}
%{
/* C/C++ code: includes, declarations */
%}

/* Bison declarations: tokens, types, precedence */
%token NUMBER IDENTIFIER
%left '+' '-'
%left '*' '/'

%%
/* Grammar rules */
expression:
    expression '+' term { /* semantic action */ }
    | term
    ;

term:
    term '*' factor { /* semantic action */ }
    | factor
    ;

factor:
    NUMBER { /* semantic action */ }
    | IDENTIFIER { /* semantic action */ }
    | '(' expression ')' { /* semantic action */ }
    ;

%%
/* User code: helper functions */
\end{verbatim}

\subsection{Integrasi Flex dan Bison}

Flex dan Bison dirancang untuk bekerja bersama:

\begin{enumerate}
    \item \textbf{Flex file (.l)}: Mendefinisikan token patterns
    \begin{verbatim}
    %{
    #include "parser.tab.h"  // Generated by Bison
    %}
    %%
    [0-9]+     { yylval = atoi(yytext); return NUMBER; }
    [a-zA-Z]+  { return IDENTIFIER; }
    \+         { return '+'; }
    \*         { return '*'; }
    %%
    \end{verbatim}

    \item \textbf{Bison file (.y)}: Mendefinisikan grammar dan semantic actions
    \begin{verbatim}
    %token NUMBER IDENTIFIER
    %%
    expression: expression '+' term | term;
    term: term '*' factor | factor;
    factor: NUMBER | IDENTIFIER | '(' expression ')';
    %%
    \end{verbatim}

    \item \textbf{Compilation}: 
    \begin{verbatim}
    flex lexer.l
    bison -d parser.y
    gcc lex.yy.c parser.tab.c -o parser
    \end{verbatim}
\end{enumerate}

\subsection{Semantic Actions}

Semantic actions adalah kode C/C++ yang dieksekusi ketika production rule di-reduce. Actions dapat:
\begin{itemize}
    \item Membangun AST nodes
    \item Mengevaluasi ekspresi
    \item Memvalidasi semantik
    \item Menghasilkan output
\end{itemize}

Contoh semantic action untuk membangun AST:

\begin{verbatim}
expression:
    expression '+' term 
    { 
        $$ = create_binary_op(PLUS, $1, $3); 
    }
    | term 
    { 
        $$ = $1; 
    }
    ;
\end{verbatim}

Di mana:
\begin{itemize}
    \item \texttt{\$\$}: Nilai yang dihasilkan oleh production (LHS)
    \item \texttt{\$1, \$2, ...}: Nilai dari simbol-simbol di RHS
\end{itemize}

\subsection{Error Handling dalam Bison}

Bison menyediakan mekanisme error handling:

\begin{verbatim}
%error-verbose  // Better error messages

expression:
    expression '+' term
    | error '+' term  // Error recovery: skip until '+'
    | term
    ;
\end{verbatim}

Error recovery rules memungkinkan parser untuk:
\begin{itemize}
    \item Mendeteksi error
    \item Melakukan recovery (skip tokens sampai synchronization point)
    \item Melanjutkan parsing
    \item Menghasilkan multiple error messages
\end{itemize}

\section{Perbandingan Top-Down vs Bottom-Up Parsing}

\subsection{Perbandingan Karakteristik}

\begin{table}[H]
\centering
\begin{tabular}{|l|c|c|}
\hline
\textbf{Aspek} & \textbf{Top-Down} & \textbf{Bottom-Up} \\
\hline
Parse Tree Direction & Root -> Leaves & Leaves -> Root \\
\hline
Derivation & Leftmost & Rightmost (reverse) \\
\hline
Implementation & Recursive descent & Table-driven \\
\hline
Lookahead & Usually 1 token & Usually 1 token \\
\hline
Left Recursion & Problem & No problem \\
\hline
Right Recursion & No problem & Less efficient \\
\hline
Parsing Power & LL(1) grammars & LR(1) grammars \\
\hline
Error Detection & Early & Later \\
\hline
Error Messages & More intuitive & Less intuitive \\
\hline
Table Size & Small & Larger \\
\hline
\end{tabular}
\caption{Perbandingan top-down dan bottom-up parsing}
\label{tab:top-down-vs-bottom-up}
\end{table}

\subsection{Kapan Menggunakan Masing-masing}

\textbf{Gunakan Top-Down Parsing jika:}
\begin{itemize}
    \item Grammar sudah dalam bentuk yang sesuai (tidak ada left recursion)
    \item Error messages yang intuitif penting
    \item Implementasi manual diperlukan
    \item Grammar relatif sederhana
\end{itemize}

\textbf{Gunakan Bottom-Up Parsing jika:}
\begin{itemize}
    \item Grammar memiliki left recursion
    \item Parsing power yang lebih besar diperlukan
    \item Menggunakan parser generator (Bison/Yacc)
    \item Grammar kompleks dengan banyak precedence levels
\end{itemize}

\section{Kesimpulan}

Dalam bab ini, kita telah mempelajari:

\begin{enumerate}
    \item Bottom-up parsing membangun parse tree dari leaves ke root menggunakan rightmost derivation dalam reverse
    \item Shift-reduce parsing menggunakan empat operasi: shift, reduce, accept, dan error
    \item LR parser adalah kelas bottom-up parser yang powerful dengan berbagai varian (LR(0), SLR(1), CLR(1), LALR(1))
    \item Konstruksi LR parsing table melibatkan augmented grammar, LR items, closure, GOTO, dan canonical collection
    \item Parser generator seperti Bison/Yacc secara otomatis menghasilkan parser dari grammar specification
    \item Integrasi Flex dan Bison memungkinkan pembangunan lexer dan parser yang terintegrasi
    \item Semantic actions memungkinkan pembangunan AST selama parsing
    \item Bottom-up parsing lebih powerful tetapi top-down parsing lebih mudah diimplementasikan secara manual
\end{enumerate}

Pemahaman tentang bottom-up parsing dan parser generator ini penting untuk mengimplementasikan parser yang robust dan efisien untuk bahasa pemrograman yang kompleks.

\section{Latihan}

\begin{enumerate}
    \item Jelaskan perbedaan antara top-down dan bottom-up parsing. Berikan contoh situasi di mana masing-masing lebih cocok digunakan.
    
    \item Untuk grammar berikut:
    \begin{verbatim}
    E -> E + T | T
    T -> T * F | F
    F -> ( E ) | id
    \end{verbatim}
    Lakukan shift-reduce parsing manual untuk input \texttt{id + id * id}. Tunjukkan stack, input, dan action pada setiap langkah.
    
    \item Jelaskan perbedaan antara SLR(1), CLR(1), dan LALR(1). Mengapa LALR(1) menjadi pilihan populer untuk parser generator?
    
    \item Buatlah file Bison sederhana untuk grammar ekspresi aritmatika dengan:
    \begin{itemize}
        \item Operator +, -, *, / dengan precedence yang benar
        \item Parentheses untuk grouping
        \item Semantic actions yang mencetak ekspresi yang di-parse
    \end{itemize}
    
    \item Integrasikan Flex lexer dengan Bison parser untuk membuat calculator sederhana yang dapat mengevaluasi ekspresi aritmatika.
    
    \item Implementasikan error recovery dalam Bison untuk menangani syntax error dengan baik. Test dengan input yang mengandung error.
    
    \item Bandingkan performa dan kompleksitas implementasi antara recursive descent parser (top-down) dan parser yang di-generate oleh Bison (bottom-up) untuk grammar yang sama.
\end{enumerate}

\section{Referensi dan Bahan Bacaan Lanjutan}

Untuk memperdalam pemahaman tentang bottom-up parsing dan parser generator, mahasiswa disarankan membaca:

\begin{itemize}
    \item \textbf{Dragon Book}: Aho, Lam, Sethi, \& Ullman (2006). \textit{Compilers: Principles, Techniques, and Tools} \cite{aho2006compilers} - Bab 4: Syntax-Directed Translation, Bab 5: Bottom-Up Parsers
    
    \item \textbf{Engineering a Compiler}: Cooper \& Torczon (2011) \cite{cooper2011engineering} - Bab 3: Scanners and Parsers
    
    \item \textbf{flex \& bison}: Levine (2009) \cite{levine2009flex} - Buku lengkap tentang Flex dan Bison
    
    \item \textbf{GeeksforGeeks}: Tutorial tentang shift-reduce parsing dan LR parsers\footnote{\url{https://www.geeksforgeeks.org/compiler-design/shift-reduce-parser-compiler/}}
    
    \item \textbf{IT Trip}: Tutorial tentang integrasi Flex dan Bison \cite{ittrip2024bison}
    
    \item \textbf{UC San Diego CSE 231}: Course materials tentang parser construction \cite{ucsd2024compiler}
    
    \item \textbf{Northeastern University CS 4410}: Comprehensive compiler design course \cite{neu2024compiler}
\end{itemize}

\cleardoublepage
\documentclass[../main.tex]{subfiles}

\addbibresource{\subfix{../references.bib}}

\begin{document}

\ifSubfilesClassLoaded{%
    \setcounter{chapter}{8}%
    \begin{refsection}
}{}

\chapter{Local Optimization dan Data Flow Analysis}
\label{chap:local-optimization}

\begin{subcpmk}
  \item \textbf{Sub-CPMK 4.3:} Menganalisis dan mengoptimasi kode tingkat lokal
\end{subcpmk}

% ============================================================
% MATERI POKOK
% ============================================================
\section{Kerangka Kerja Data-Flow Analysis}

\compiler{Data-Flow Analysis (DFA)} adalah proses pengumpulan informasi tentang aliran data melalui graf kendali alir (\textit{Control Flow Graph}) \cite{nguyen2024semantic}. Informasi ini digunakan untuk menjawab pertanyaan global seperti: "Apakah variabel $x$ pasti memiliki nilai 10 di baris ini?" atau "Apakah nilai $y$ akan dibaca lagi di masa depan?".

\subsection{Iterative Data-Flow Framework}
Untuk menyelesaikan masalah DFA secara sistematis, kita menggunakan kerangka kerja iteratif. Setiap algoritma DFA dapat didefinisikan dengan empat komponen utama:
\begin{enumerate}
    \item \textbf{Direction}: Arah aliran (\textit{Forward} atau \textit{Backward}).
    \item \textbf{Domain}: Himpunan nilai yang dianalisis (misal: himpunan definisi variabel atau ekspresi).
    \item \textbf{Transfer Function} ($f_B$): Aturan bagaimana sebuah blok mengubah informasi data-flow. Format umumnya: $OUT[B] = f_B(IN[B])$.
    \item \textbf{Meet operator} ($\wedge$): Aturan penggabungan informasi ketika dua atau lebih jalur dalam CFG bertemu (biasanya berupa \textit{Union} $\cup$ atau \textit{Intersection} $\cap$).
\end{enumerate}

\subsection{Representasi Bit-Vector}
Agar proses DFA berjalan cepat, kompilator biasanya menggunakan representasi \textbf{Bit-Vector}. Setiap elemen dalam domain direpresentasikan oleh satu bit dalam sebuah \textit{array of bits}. 
\begin{itemize}
    \item Bit $1$ berarti properti tersebut \textit{true} atau ada dalam set.
    \item Bit $0$ berarti \textit{false} atau tidak ada.
\end{itemize}
Dengan bit-vector, operasi set seperti \texttt{Union} dapat dilakukan menggunakan instruksi bitwise \texttt{OR} ($|$), dan \texttt{Intersection} menggunakan bitwise \texttt{AND} ($\&$), yang didukung sangat cepat oleh perangkat keras CPU.

\begin{figure}[!htbp]
    \centering
    \adjustbox{max width=0.8\textwidth,center}{%
    \begin{tikzpicture}[
        node/.style={rectangle, draw=blue!50, fill=blue!10, font=\small, align=center, rounded corners, minimum height=0.8cm},
        arrow/.style={->, >=stealth, thick}
    ]
    \node[node] (in) {IN[B]};
    \node[node, right=1.5cm of in] (bb) {Basic Block $B$};
    \node[node, right=1.5cm of bb] (out) {OUT[B]};
    \draw[arrow] (in) -- (bb);
    \draw[arrow] (bb) -- node[above, font=\footnotesize] {$f_B$} (out);
    \end{tikzpicture}%
    }
    \caption{Model Aliran Data pada Satu Basic Block}
\end{figure}

\section{Reaching Definitions Analysis}

Analisis \compiler{Reaching Definitions} adalah analisis \textbf{Forward} dengan meet operator \textbf{Union} ($\cup$). Analisis ini menentukan definisi (penugasan nilai) mana yang mungkin masih aktif saat mencapai titik program tertentu.

\subsection{Persamaan Aliran Data}
Untuk setiap blok $B$, hubungannya didefinisikan sebagai:
\begin{enumerate}
    \item \textbf{Meet Operation}: $IN[B] = \bigcup_{P \in Pred(B)} OUT[P]$ (Definisi yang mencapai awal blok adalah gabungan dari semua yang keluar dari pendahulunya).
    \item \textbf{Transfer Function}: $OUT[B] = GEN[B] \cup (IN[B] - KILL[B])$.
\end{enumerate}

Di mana:
\begin{itemize}
    \item $GEN[B]$: Himpunan definisi yang dibuat di dalam blok $B$ dan mencapai akhir $B$.
    \item $KILL[B]$: Himpunan definisi di luar $B$ yang variabelnya di-assign ulang di dalam $B$.
\end{itemize}

\subsection{Konsep Fixed-Point}
Kompilator menjalankan algoritma ini secara berulang di seluruh CFG. Karena himpunan data hanya bisa bertambah (monotonik) dan jumlah definisi terbatas, algoritma dijamin akan berhenti pada suatu titik di mana nilai $IN$ dan $OUT$ tidak lagi berubah. Titik stabil ini disebut \textbf{Fixed-Point}.

\begin{figure}[!htbp]
    \centering
    \adjustbox{max width=0.8\textwidth,center}{%
    \begin{tikzpicture}[
        rect/.style={rectangle, draw=red!50, fill=red!10, text width=6cm, font=\tiny}
    ]
    \node[rect] (eq) {
        \textbf{Contoh Reaching Defs:}\\
        $d_1: x = 5$\\
        $d_2: x = 10 \to \text{Baris ini membunuh (KILL) } d_1$\\
        Informasi yang keluar dari blok ini hanya $\{d_2\}$.
    };
    \end{tikzpicture}%
    }
    \caption{Ilustrasi Operasi GEN dan KILL}
\end{figure}

\section{Live Variable Analysis dan Alokasi Register}

Analisis \compiler{Live Variable} adalah masalah \textbf{Backward} dengan meet operator \textbf{Union} ($\cup$). Analisis ini menjawab: "Apakah nilai variabel saat ini akan digunakan lagi di masa depan?".

\subsection{Persamaan Aliran Data (Backward)}
Kebalikan dari Reaching Defs, analisis ini merambat dari bawah ke atas:
\begin{enumerate}
    \item \textbf{Meet Operation}: $OUT[B] = \bigcup_{S \in Succ(B)} IN[S]$.
    \item \textbf{Transfer Function}: $IN[B] = USE[B] \cup (OUT[B] - DEF[B])$.
\end{enumerate}

\subsection{Peran dalam Alokasi Register}
Ini adalah dasar dari \textbf{Register Allocation} yang efisien.
\begin{itemize}
    \item Jika dua variabel tidak pernah "hidup" (\textit{live}) secara bersamaan di titik mana pun, keduanya dapat menempati register fisik yang sama.
    \item Kompilator membangun \textbf{Interference Graph}: simpul adalah variabel, dan sisi ditarik antar variabel yang live bersamaan. Masalah alokasi register kemudian diselesaikan sebagai masalah \textit{Graph Coloring}.
\end{itemize}

\begin{lstlisting}[language=C++]
// Contoh Analisis Liveness
x = 10;
y = 20;
print(x); // x live di sini, y mati (dead)
// y = 20 adalah "Dead Code" karena y tidak live pada titik ini
\end{lstlisting}
Informasi ini memungkinkan \textbf{Dead Code Elimination} skala global yang lebih akurat daripada sekadar analisis di dalam satu blok.
    
    

\section{Available Expressions dan CSE}

Ekspresi $x + y$ disebut \compiler{Available} di titik $p$ jika sudah pernah dihitung sebelumnya dan nilai $x$ maupun $y$ belum berubah sejak penghitungan tersebut.

\subsection{Common Subexpression Elimination (CSE)}
Jika sebuah ekspresi sudah \textit{available}, kompilator akan mengganti perhitungan ulang ekspresi tersebut dengan nilai yang sudah ada di variabel perantara.

\begin{lstlisting}
// Sebelum CSE
t1 = a + b
t2 = c * d
t3 = a + b  // Perhitungan berulang

// Sesudah CSE
t1 = a + b
t2 = c * d
t3 = t1
\end{lstlisting}

\begin{figure}[!htbp]
    \centering
    \adjustbox{max width=0.8\textwidth,center}{%
    \begin{tikzpicture}[
        node/.style={rectangle, draw=blue!50, fill=blue!10, font=\tiny, align=center},
        arrow/.style={->, >=stealth, thick}
    ]
    \node[node] (e1) {expr: a+b};
    \node[node, right=1.5cm of e1] (e2) {use cached result};
    \draw[arrow] (e1) -- (e2);
    \end{tikzpicture}%
    }
    \caption{Konsep eliminasi sub-ekspresi umum}
\end{figure}

\section{Global Constant Propagation}

Jika pada optimasi lokal (Bab 8) kita hanya melihat penyebaran konstanta di dalam blok, \compiler{Global Constant Propagation} menyebarkan nilai konstanta melalui percabangan dan loop di seluruh program.

\subsection{Mekanisme Aliran Data}
Analisis ini menggunakan informasi dari \textit{Reaching Definitions}. Jika di titik $p$, hanya terdapat satu definisi yang mencapai variabel $x$, dan definisi tersebut berupa konstanta (\code{x = 5}), maka semua penggunaan $x$ di titik $p$ dapat diganti dengan angka \texttt{5}.

\subsection{Analisis Nilai Konstan}
Beberapa alasan mengapa ini lebih kuat daripada versi lokal:
\begin{enumerate}
    \item \textbf{Branch Pruning}: Jika kondisi \texttt{if (x > 0)} dievaluasi mendapati $x$ selalu bernilai $10$, maka blok \textit{else} dapat dihapus sepenuhnya (\textit{Unreachable Code Elimination}).
    \item \textbf{Loop Invariant}: Membantu mengenali nilai yang tidak berubah selama iterasi loop.
\end{enumerate}

\begin{lstlisting}[language=C++]
void contohGlobal() {
    int x = 7;
    if (kondisi) {
        // x masih mencapai sini sebagai 7
    } else {
        // x masih mencapai sini sebagai 7
    }
    int y = x + 3; // Global Const Prop: y = 7 + 3 = 10
}
\end{lstlisting}

\section{Pipeline dan Interaksi Optimasi Global}

Optimasi tingkat global (\textit{Data-Flow Based}) jarang berjalan secara mandiri. Sebaliknya, satu optimasi seringkali menjadi pemicu untuk optimasi lainnya dalam sebuah \textbf{Optimization Loop}.

\subsection{Kaskade Optimasi}
Perhatikan interaksi berikut:
\begin{enumerate}
    \item \textbf{Global Constant Propagation} mengganti \code{x} dengan \code{5}.
    \item Hal ini memicu \textbf{Constant Folding} pada ekspresi \code{x + 10} menjadi \code{15}.
    \item Hasil lipatan (folding) mungkin membuat sebuah kondisi \texttt{if} selalu bernilai benar, yang memicu \textbf{Dead Code Elimination} pada blok \texttt{else}.
    \item Penghapusan blok tersebut menghapus definisi variabel lain, yang mungkin memicu \textbf{Global CSE} untuk ekspresi yang sebelumnya terhambat oleh definisi tersebut.
\end{enumerate}

\subsection{Iterasi Hingga Fixed-Point}
Kompilator tidak hanya melakukan analisis data-flow hingga fixed-point, tetapi seluruh \textit{optimizer pipeline} juga sering diulang beberapa kali hingga tidak ada lagi instruksi yang bisa disederhanakan.

\begin{figure}[!htbp]
    \centering
    \adjustbox{max width=0.8\textwidth,center}{%
    \begin{tikzpicture}[
        node/.style={rectangle, draw=purple!50, fill=purple!10, font=\small, align=center, rounded corners, minimum height=0.8cm, text width=3.5cm},
        arrow/.style={->, >=stealth, thick}
    ]
    \node[node] (p1) {Global Constant Propagation};
    \node[node, below=0.5cm of p1] (p2) {Global CSE};
    \node[node, below=0.5cm of p2] (p3) {Global Dead Code Elimination (DCE)};
    \draw[arrow] (p1) -- (p2);
    \draw[arrow] (p2) -- (p3);
    \draw[arrow] (p3.west) to[bend left=90, looseness=1.5] node[left, font=\tiny] {Repeat if changed} (p1.west);
    \end{tikzpicture}%
    }
    \caption{Loop Iteratif pada Optimization Pipeline}
\end{figure}

\subsection{Kesimpulan}
Data-flow analysis merubah pandangan kompilator dari deretan instruksi linear menjadi aliran informasi yang kaya. Dengan informasi global ini, kompilator dapat menghasilkan kode yang jauh lebih efisien daripada apa yang ditulis manusia secara manual.
    
    


% ============================================================
% AKTIVITAS PEMBELAJARAN
% ============================================================
\begin{aktivitas}
  \item \textbf{Algebraic Optimization}: Implementasikan berbagai algebraic optimizations.
  \item \textbf{Constant Propagation}: Bangun constant propagation analyzer.
  \item \textbf{Copy Propagation}: Implementasikan copy propagation algorithm.
  \item \textbf{Dead Code Elimination}: Identifikasi dan hapus dead code.
  \item \textbf{CSE}: Implementasikan common subexpression elimination.
\end{aktivitas}

% ============================================================
% LATIHAN DAN REFLEKSI
% ============================================================
\begin{latihan}
  \item Identifikasi semua algebraic optimizations yang mungkin dalam potongan kode!
  \item Implementasikan constant propagation untuk program dengan multiple assignments!
  \item Analisis copy propagation chains dan handling conflicts!
  \item Identifikasi dead code dalam program dengan complex control flow!
  \item Implementasikan CSE untuk expressions dengan multiple operands!
  \item \textbf{Refleksi}: Bagaimana local optimizations mempengaruhi performance compiler?
\end{latihan}

% ============================================================
% ASESMEN
% ============================================================
\begin{asesmen}
\textbf{Instrumen Penilaian untuk Sub-CPMK 4.3}

\textbf{A. Pilihan Ganda}

\begin{enumerate}
  \item Strength reduction mengganti:
  \begin{enumerate}
    \item Operasi mahal dengan operasi murah
    \item Konstanta dengan variabel
    \item Variabel dengan konstanta
    \item Dead code dengan live code
  \end{enumerate}
  
  \item Constant propagation:
  \begin{enumerate}
    \item Menyebarkan nilai konstanta
    \item Menghapus konstanta
    \item Mengidentifikasi konstanta
    \item Mengoptimasi loops
  \end{enumerate}
  
  \item Dead code elimination menghapus:
  \begin{enumerate}
    \item Instruksi yang tidak digunakan
    \item Instruksi yang lambat
    \item Instruksi yang error
    \item Semua instruksi
  \end{enumerate}
\end{enumerate}

\textbf{B. Essay}

\begin{enumerate}
  \item Jelaskan implementasi complete local optimization pipeline dengan semua teknik yang dibahas!
  \item Analisis trade-off antara berbagai local optimizations dalam terms of performance vs complexity!
\end{enumerate}

\textbf{Rubrik Penilaian}: Lihat Lampiran A
\end{asesmen}

% ============================================================
% CHECKLIST KOMPETENSI
% ============================================================
\begin{checklist}
  \item Saya dapat menganalisis dan mengoptimasi kode tingkat lokal
  \item Saya dapat mengimplementasikan algebraic optimizations
  \item Saya dapat melakukan constant propagation
  \item Saya dapat menerapkan copy propagation
  \item Saya dapat mengidentifikasi dan menghapus dead code
  \item Saya dapat melakukan common subexpression elimination
\end{checklist}

% ============================================================
% RANGKUMAN
% ============================================================
\begin{rangkuman}
Bab ini membahas local optimization dan data flow analysis, termasuk algebraic optimizations, constant propagation, copy propagation, dead code elimination, dan common subexpression elimination. Mahasiswa belajar membangun optimization pipeline yang efektif.

\textbf{Poin Kunci:}
\begin{itemize}
  \item Local optimizations bekerja dalam satu basic block
  \item Algebraic optimizations menyederhanakan ekspresi matematis
  \item Constant propagation menyebarkan nilai konstanta
  \item Copy propagation menghilangkan variabel perantara
  \item Dead code elimination menghapus instruksi yang tidak berguna
  \item CSE menghilangkan perhitungan berulang
\end{itemize}

\textbf{Kata Kunci}: \compiler{Local Optimization}, \compiler{Algebraic Optimization}, \compiler{Constant Propagation}, \compiler{Copy Propagation}, \compiler{Dead Code Elimination}, \compiler{Common Subexpression Elimination}, \compiler{Data Flow Analysis}
\end{rangkuman}

\ifSubfilesClassLoaded{%
    \clearpage
    \printbibliography[title={Daftar Pustaka}]
    \end{refsection}
}{}

\end{document}

\cleardoublepage
\documentclass[../main.tex]{subfiles}
\begin{document}

\chapter{Symbol Table dan Scope Management}
\label{chap:symbol-table}

\section{Tujuan Pembelajaran}

Setelah mempelajari bab ini, mahasiswa diharapkan mampu:
\begin{enumerate}
    \item Memahami konsep symbol table dan perannya dalam kompilator
    \item Mengimplementasikan symbol table menggunakan hash table dalam C/C++
    \item Memahami dan mengimplementasikan nested scopes (stack of symbol tables)
    \item Melakukan name resolution dengan benar mengikuti aturan scoping
    \item Menangani scope entry dan exit untuk berbagai konstruk bahasa (function, block, loop)
    \item Mengidentifikasi dan menangani shadowing (pengaburan identifier)
    \item Membuat visualisasi symbol table untuk debugging
\end{enumerate}
\section{Pendahuluan}

Symbol table adalah struktur data fundamental dalam kompilator yang menyimpan informasi tentang identifier yang digunakan dalam program. Menurut sumber terbuka:

\begin{quote}
``Symbol tables: data structures mapping names to declarations, with nested scopes. Semantic analysis includes name resolution: every use of a variable, function, type must refer to a declaration.''\cite{nguyen2024semantic}
\end{quote}

Symbol table berfungsi sebagai \textit{database} yang menghubungkan setiap penggunaan identifier dengan deklarasinya. Tanpa symbol table, kompilator tidak dapat memverifikasi apakah variabel yang digunakan sudah dideklarasikan, apakah tipe data sesuai, atau apakah fungsi dipanggil dengan parameter yang benar.

Gambar \ref{fig:symbol-table-overview} menunjukkan peran symbol table dalam kompilator.

\begin{figure}[!htbp]
    \centering
    \adjustbox{max width=0.9\textwidth,center}{%
    \begin{tikzpicture}[
        box/.style={rectangle, draw=blue!50, fill=blue!10, text width=2.5cm, text centered, minimum height=0.7cm, rounded corners, font=\footnotesize, inner sep=4pt, align=center},
        table/.style={rectangle, draw=green!50, fill=green!10, text width=3cm, minimum height=0.7cm, font=\footnotesize, align=center, rounded corners, inner sep=4pt},
        arrow/.style={->, >=stealth, thick},
        node distance=1.2cm
    ]
    
    \node[box] (parser) {Parser};
    \node[table, right=of parser] (st) {Symbol\\Table};
    \node[box, right=of st] (checker) {Type\\Checker};
    \node[box, below=of st] (codegen) {Code\\Generator};
    
    \draw[arrow] (parser) -- node[above, font=\tiny, align=center] {Insert} (st);
    \draw[arrow] (st) -- node[above, font=\tiny, align=center] {Lookup} (checker);
    \draw[arrow] (st) -- node[right, font=\tiny, align=center] {Info} (codegen);
    
    \end{tikzpicture}%
    }
    \caption{Peran symbol table dalam kompilator}
    \label{fig:symbol-table-overview}
\end{figure}

\subsection{Informasi yang Disimpan dalam Symbol Table}

Setiap entry dalam symbol table menyimpan berbagai informasi tentang identifier:

\begin{itemize}
    \item \textbf{Nama Identifier}: String yang merepresentasikan nama variabel, fungsi, atau tipe
    \item \textbf{Tipe Data}: Tipe dari identifier (int, float, function, struct, dll.)
    \item \textbf{Scope Level}: Level nesting scope di mana identifier dideklarasikan
    \item \textbf{Memory Location}: Alamat atau offset memory untuk variabel (digunakan dalam code generation)
    \item \textbf{Line Number}: Posisi deklarasi dalam source code (untuk error reporting)
    \item \textbf{Attributes Tambahan}: 
    \begin{itemize}
        \item Untuk fungsi: parameter list, return type, calling convention
        \item Untuk variabel: storage class (static, auto, register), initial value
        \item Untuk array: dimensi dan ukuran
        \item Untuk struct: field list
    \end{itemize}
\end{itemize}

\subsection{Operasi Dasar pada Symbol Table}

Symbol table harus mendukung operasi-operasi berikut:

\begin{enumerate}
    \item \textbf{Insert}: Menambahkan entry baru untuk identifier yang dideklarasikan
    \item \textbf{Lookup}: Mencari identifier dalam symbol table untuk name resolution
    \item \textbf{Delete}: Menghapus entry ketika scope berakhir (untuk nested scopes)
    \item \textbf{Update}: Memperbarui informasi identifier (misalnya setelah type inference)
\end{enumerate}
\section{Implementasi Symbol Table dengan Hash Table}

Hash table adalah struktur data yang efisien untuk implementasi symbol table karena memberikan waktu akses rata-rata O(1) untuk operasi insert dan lookup. Dalam konteks kompilator, hash table menggunakan nama identifier sebagai key.

\subsection{Struktur Data Dasar}

Berikut adalah struktur data dasar untuk symbol table menggunakan hash table dalam C++:

\begin{lstlisting}[language=C++, caption={Struktur Data Symbol dan Scope}]
// Informasi tentang sebuah symbol
struct Symbol {
    std::string name;           // Nama identifier
    std::string type;           // Tipe data
    int scope_level;            // Level scope
    int line_number;            // Baris deklarasi
    void* memory_location;       // Alamat memory (untuk code gen)
    // Attributes tambahan sesuai kebutuhan
};

// Satu scope (satu hash table)
class Scope {
private:
    std::unordered_map<std::string, Symbol*> table;
    Scope* parent;              // Scope yang membungkus (enclosing scope)
    int level;                  // Level nesting (0 untuk global)
    
public:
    Scope(Scope* p = nullptr, int l = 0) 
        : parent(p), level(l) {}
    
    bool insert(const std::string& name, Symbol* sym);
    Symbol* lookup(const std::string& name);
    Symbol* lookupLocal(const std::string& name);  // Hanya di scope ini
    Scope* getParent() { return parent; }
    int getLevel() { return level; }
};

// Symbol table utama (stack of scopes)
class SymbolTable {
private:
    Scope* current_scope;
    int next_level;
    
public:
    SymbolTable();
    ~SymbolTable();
    
    void beginScope();          // Masuk ke scope baru
    void endScope();            // Keluar dari scope
    bool insert(const std::string& name, const std::string& type, int line);
    Symbol* lookup(const std::string& name);
    Symbol* lookupCurrentScope(const std::string& name);
};
\end{lstlisting}

\subsection{Implementasi Operasi Dasar}

\subsubsection{Begin Scope}

Ketika memasuki scope baru (misalnya saat menemukan \texttt{\{} atau function declaration), kita membuat scope baru:

\begin{lstlisting}[language=C++, caption={Implementasi beginScope}]
void SymbolTable::beginScope() {
    Scope* new_scope = new Scope(current_scope, next_level++);
    current_scope = new_scope;
}
\end{lstlisting}

\subsubsection{End Scope}

Ketika keluar dari scope (misalnya saat menemukan \texttt{\}}), kita menghapus scope tersebut:

\begin{lstlisting}[language=C++, caption={Implementasi endScope}]
void SymbolTable::endScope() {
    if (current_scope == nullptr) return;
    
    Scope* parent = current_scope->getParent();
    delete current_scope;
    current_scope = parent;
    next_level--;
}
\end{lstlisting}

\subsubsection{Insert}

Menambahkan symbol ke scope saat ini. Perlu memeriksa duplikasi dalam scope yang sama:

\begin{lstlisting}[language=C++, caption={Implementasi insert}]
bool SymbolTable::insert(const std::string& name, 
                         const std::string& type, 
                         int line) {
    if (current_scope == nullptr) {
        // Error: tidak ada scope aktif
        return false;
    }
    
    // Cek duplikasi dalam scope saat ini
    if (current_scope->lookupLocal(name) != nullptr) {
        // Error: duplicate declaration
        return false;
    }
    
    Symbol* sym = new Symbol();
    sym->name = name;
    sym->type = type;
    sym->scope_level = current_scope->getLevel();
    sym->line_number = line;
    
    return current_scope->insert(name, sym);
}
\end{lstlisting}

\subsubsection{Lookup}

Mencari symbol mulai dari scope saat ini, kemudian naik ke parent scope jika tidak ditemukan:

\begin{lstlisting}[language=C++, caption={Implementasi lookup dengan nested scopes}]
Symbol* SymbolTable::lookup(const std::string& name) {
    Scope* scope = current_scope;
    
    while (scope != nullptr) {
        Symbol* sym = scope->lookupLocal(name);
        if (sym != nullptr) {
            return sym;  // Ditemukan di scope ini
        }
        scope = scope->getParent();  // Cari di parent scope
    }
    
    return nullptr;  // Tidak ditemukan di semua scope
}
\end{lstlisting}

Ini mengimplementasikan aturan scoping: pencarian dimulai dari scope paling dalam (current) dan bergerak ke luar sampai menemukan deklarasi atau mencapai global scope.

Gambar \ref{fig:scope-hierarchy} menunjukkan hierarki nested scopes.

\begin{figure}[H]
    \centering
    \adjustbox{max width=0.85\textwidth,center}{%
    \begin{tikzpicture}[
        scope/.style={rectangle, draw=blue!50, fill=blue!10, text width=3cm, minimum height=0.7cm, font=\footnotesize, align=center, rounded corners, inner sep=4pt},
        arrow/.style={->, >=stealth, thick},
        node distance=0.6cm
    ]
    
    \node[scope] (global) {Global Scope\\Level 0};
    \node[scope, below=of global] (func) {Function Scope\\Level 1};
    \node[scope, below=of func] (block) {Block Scope\\Level 2};
    
    \draw[arrow] (global) -- node[right, font=\tiny] {parent} (func);
    \draw[arrow] (func) -- node[right, font=\tiny] {parent} (block);
    
    \end{tikzpicture}%
    }
    \caption{Hierarki nested scopes}
    \label{fig:scope-hierarchy}
\end{figure}
\section{Nested Scopes dan Scoping Rules}

Nested scopes (scope bersarang) adalah fitur penting dalam bahasa pemrograman modern. Setiap konstruk bahasa tertentu menciptakan scope baru:

\begin{itemize}
    \item \textbf{Global Scope}: Scope terluar, berisi deklarasi global
    \item \textbf{Function Scope}: Setiap fungsi memiliki scope sendiri
    \item \textbf{Block Scope}: Setiap blok \texttt{\{ \}} menciptakan scope baru
    \item \textbf{Loop Scope}: Beberapa bahasa (seperti C++ dengan for-loop) menciptakan scope untuk variabel loop
    \item \textbf{Class Scope}: Dalam bahasa OOP, class menciptakan scope untuk member-nya
\end{itemize}

\subsection{Contoh Nested Scopes}

Perhatikan contoh program berikut:

\begin{lstlisting}[language=C++, caption={Contoh program dengan nested scopes}]
int x = 10;           // Global scope (level 0)

void func() {         // Function scope (level 1)
    int y = 20;       // Local variable di func
    int x = 30;       // Shadowing: x di scope ini
    
    {                 // Block scope (level 2)
        int z = 40;   // Local di block
        int y = 50;   // Shadowing: y di scope ini
        // Di sini: x=30 (dari func), y=50 (dari block), z=40
    }
    // Di sini: x=30 (dari func), y=20 (dari func), z tidak ada
}
// Di sini: x=10 (global), y dan z tidak ada
\end{lstlisting}

Symbol table untuk program di atas akan memiliki struktur seperti:

\begin{verbatim}
Level 0 (Global):
  x -> int (line 1)

Level 1 (func):
  y -> int (line 4)
  x -> int (line 5)  [shadows global x]

Level 2 (block):
  z -> int (line 8)
  y -> int (line 9)  [shadows func y]
\end{verbatim}

\subsection{Aturan Scoping}

Ada dua aturan scoping utama:

\begin{enumerate}
    \item \textbf{Static Scoping (Lexical Scoping)}: 
    \begin{itemize}
        \item Scope ditentukan oleh struktur program (lexical structure)
        \item Lookup dimulai dari scope saat ini, kemudian naik ke enclosing scopes
        \item Digunakan oleh sebagian besar bahasa modern (C, C++, Java, Python)
    \end{itemize}
    
    \item \textbf{Dynamic Scoping}:
    \begin{itemize}
        \item Scope ditentukan oleh urutan eksekusi program
        \item Lookup dimulai dari scope saat ini, kemudian naik ke caller's scope
        \item Jarang digunakan (beberapa bahasa scripting seperti Perl dalam mode tertentu)
    \end{itemize}
\end{enumerate}

Buku ini fokus pada static scoping yang merupakan standar dalam bahasa pemrograman modern.

Gambar \ref{fig:static-vs-dynamic-scoping} menunjukkan perbandingan static dan dynamic scoping.

\begin{figure}[H]
    \centering
    \adjustbox{max width=0.9\textwidth,center}{%
    \begin{tikzpicture}[
        box/.style={rectangle, draw=blue!50, fill=blue!10, text width=3cm, text centered, minimum height=0.7cm, rounded corners, font=\footnotesize, inner sep=4pt, align=center},
        arrow/.style={->, >=stealth, thick},
        node distance=1.2cm
    ]
    
    \node[box] (static) {Static Scoping\\Lexical};
    \node[box, below=of static] (lookup1) {Lookup: Current\\→ Enclosing};
    
    \node[box, right=4cm of static] (dynamic) {Dynamic Scoping\\Runtime};
    \node[box, below=of dynamic] (lookup2) {Lookup: Current\\→ Caller};
    
    \draw[arrow] (static) -- (lookup1);
    \draw[arrow] (dynamic) -- (lookup2);
    
    \end{tikzpicture}%
    }
    \caption{Perbandingan static vs dynamic scoping}
    \label{fig:static-vs-dynamic-scoping}
\end{figure}
\section{Name Resolution}

Name resolution adalah proses menemukan deklarasi yang sesuai untuk setiap penggunaan identifier. Proses ini harus mengikuti aturan scoping bahasa.

\subsection{Algoritma Name Resolution}

Algoritma name resolution untuk static scoping:

\begin{enumerate}
    \item Mulai dari scope saat ini (current scope)
    \item Cari identifier dalam hash table scope tersebut
    \item Jika ditemukan, return symbol tersebut
    \item Jika tidak ditemukan, pindah ke parent scope (enclosing scope)
    \item Ulangi langkah 2-4 sampai ditemukan atau mencapai global scope
    \item Jika tidak ditemukan di semua scope, identifier tidak dideklarasi (error)
\end{enumerate}

Implementasi algoritma ini sudah ditunjukkan dalam fungsi \texttt{lookup()} sebelumnya.

\subsection{Shadowing (Pengaburan Identifier)}

Shadowing terjadi ketika identifier dalam scope dalam memiliki nama yang sama dengan identifier di scope luar. Identifier dalam scope dalam "mengaburkan" (shadow) identifier di scope luar.

\begin{lstlisting}[language=C++, caption={Contoh shadowing}]
int x = 10;        // Global x

void func() {
    int x = 20;    // Local x shadows global x
    // Penggunaan 'x' di sini merujuk ke local x (20)
    {
        int x = 30;  // Inner x shadows outer x
        // Penggunaan 'x' di sini merujuk ke inner x (30)
    }
    // Penggunaan 'x' di sini kembali merujuk ke local x (20)
}
\end{lstlisting}

\subsection{Deteksi Shadowing}

Beberapa kompilator memberikan peringatan ketika terjadi shadowing karena dapat menyebabkan kebingungan. Kita dapat mendeteksi shadowing saat insert:

\begin{lstlisting}[language=C++, caption={Deteksi shadowing saat insert}]
bool SymbolTable::insert(const std::string& name, 
                         const std::string& type, 
                         int line) {
    // Cek duplikasi dalam scope saat ini
    if (current_scope->lookupLocal(name) != nullptr) {
        return false;  // Error: duplicate
    }
    
    // Cek shadowing (optional warning)
    Scope* parent = current_scope->getParent();
    while (parent != nullptr) {
        if (parent->lookupLocal(name) != nullptr) {
            // Warning: shadowing outer declaration
            std::cout << "Warning: '" << name 
                      << "' shadows declaration at line " 
                      << parent->lookupLocal(name)->line_number 
                      << std::endl;
            break;
        }
        parent = parent->getParent();
    }
    
    // Insert symbol
    Symbol* sym = new Symbol();
    sym->name = name;
    sym->type = type;
    sym->scope_level = current_scope->getLevel();
    sym->line_number = line;
    
    return current_scope->insert(name, sym);
}
\end{lstlisting}
\section{Handling Scope Entry dan Exit}

Kompilator harus menangani masuk dan keluar scope dengan benar untuk berbagai konstruk bahasa. Ini dilakukan dengan memanggil \texttt{beginScope()} dan \texttt{endScope()} pada waktu yang tepat.

\subsection{Function Declaration}

Saat menemukan deklarasi fungsi, kita memasuki scope baru:

\begin{lstlisting}[language=C++, caption={Handling function scope}]
// Dalam parser, saat menemukan function declaration:
void parseFunction() {
    // ... parse function signature ...
    
    symbolTable.beginScope();  // Masuk ke function scope
    
    // Parse parameter list (insert ke symbol table)
    for (auto param : parameters) {
        symbolTable.insert(param.name, param.type, param.line);
    }
    
    // Parse function body
    parseBlock();
    
    symbolTable.endScope();  // Keluar dari function scope
}
\end{lstlisting}

\subsection{Block Statement}

Setiap blok \texttt{\{ \}} menciptakan scope baru:

\begin{lstlisting}[language=C++, caption={Handling block scope}]
void parseBlock() {
    match('{');
    
    symbolTable.beginScope();  // Masuk ke block scope
    
    // Parse statements dalam block
    while (currentToken != '}') {
        parseStatement();
    }
    
    match('}');
    symbolTable.endScope();  // Keluar dari block scope
}
\end{lstlisting}

\subsection{Loop Statement}

Beberapa bahasa (seperti C++ dengan for-loop) menciptakan scope untuk variabel loop:

\begin{lstlisting}[language=C++, caption={Handling loop scope}]
void parseForLoop() {
    match("for");
    match('(');
    
    symbolTable.beginScope();  // Masuk ke loop scope
    
    // Parse loop variable declaration (jika ada)
    if (isDeclaration()) {
        parseDeclaration();
    }
    
    // Parse loop condition dan increment
    parseExpression();  // condition
    parseExpression();  // increment
    
    match(')');
    
    // Parse loop body
    parseStatement();
    
    symbolTable.endScope();  // Keluar dari loop scope
}
\end{lstlisting}
\section{Implementasi Lengkap Symbol Table}

Berikut adalah implementasi lengkap symbol table dengan semua fitur yang telah dibahas:

\begin{lstlisting}[language=C++, caption={Implementasi lengkap SymbolTable}]
#include <string>
#include <unordered_map>
#include <iostream>
#include <vector>

struct Symbol {
    std::string name;
    std::string type;
    int scope_level;
    int line_number;
    void* memory_location;
    
    Symbol() : scope_level(-1), line_number(-1), 
               memory_location(nullptr) {}
};

class Scope {
private:
    std::unordered_map<std::string, Symbol*> table;
    Scope* parent;
    int level;
    std::vector<std::string> declared_names;  // Untuk cleanup
    
public:
    Scope(Scope* p = nullptr, int l = 0) 
        : parent(p), level(l) {}
    
    ~Scope() {
        // Cleanup semua symbols
        for (auto& pair : table) {
            delete pair.second;
        }
    }
    
    bool insert(const std::string& name, Symbol* sym) {
        if (table.find(name) != table.end()) {
            return false;  // Duplicate
        }
        table[name] = sym;
        declared_names.push_back(name);
        return true;
    }
    
    Symbol* lookupLocal(const std::string& name) {
        auto it = table.find(name);
        if (it != table.end()) {
            return it->second;
        }
        return nullptr;
    }
    
    Symbol* lookup(const std::string& name) {
        Symbol* sym = lookupLocal(name);
        if (sym != nullptr) {
            return sym;
        }
        if (parent != nullptr) {
            return parent->lookup(name);
        }
        return nullptr;
    }
    
    Scope* getParent() { return parent; }
    int getLevel() { return level; }
    const std::vector<std::string>& getDeclaredNames() const {
        return declared_names;
    }
};

class SymbolTable {
private:
    Scope* current_scope;
    int next_level;
    
public:
    SymbolTable() {
        current_scope = new Scope(nullptr, 0);
        next_level = 1;
    }
    
    ~SymbolTable() {
        // Cleanup semua scopes
        while (current_scope != nullptr) {
            Scope* parent = current_scope->getParent();
            delete current_scope;
            current_scope = parent;
        }
    }
    
    void beginScope() {
        Scope* new_scope = new Scope(current_scope, next_level++);
        current_scope = new_scope;
    }
    
    void endScope() {
        if (current_scope == nullptr || 
            current_scope->getLevel() == 0) {
            return;  // Tidak bisa keluar dari global scope
        }
        
        Scope* parent = current_scope->getParent();
        delete current_scope;
        current_scope = parent;
        next_level--;
    }
    
    bool insert(const std::string& name, 
                const std::string& type, 
                int line) {
        if (current_scope == nullptr) {
            return false;
        }
        
        // Cek duplikasi
        if (current_scope->lookupLocal(name) != nullptr) {
            std::cerr << "Error: Duplicate declaration of '" 
                      << name << "' at line " << line << std::endl;
            return false;
        }
        
        // Cek shadowing (optional warning)
        Scope* parent = current_scope->getParent();
        while (parent != nullptr) {
            Symbol* shadowed = parent->lookupLocal(name);
            if (shadowed != nullptr) {
                std::cout << "Warning: '" << name 
                          << "' at line " << line
                          << " shadows declaration at line " 
                          << shadowed->line_number << std::endl;
                break;
            }
            parent = parent->getParent();
        }
        
        // Insert symbol
        Symbol* sym = new Symbol();
        sym->name = name;
        sym->type = type;
        sym->scope_level = current_scope->getLevel();
        sym->line_number = line;
        
        return current_scope->insert(name, sym);
    }
    
    Symbol* lookup(const std::string& name) {
        if (current_scope == nullptr) {
            return nullptr;
        }
        return current_scope->lookup(name);
    }
    
    Symbol* lookupCurrentScope(const std::string& name) {
        if (current_scope == nullptr) {
            return nullptr;
        }
        return current_scope->lookupLocal(name);
    }
    
    int getCurrentLevel() {
        return current_scope ? current_scope->getLevel() : -1;
    }
    
    Scope* getCurrentScope() {
        return current_scope;
    }
};
\end{lstlisting}
\section{Visualisasi Symbol Table}

Visualisasi symbol table sangat berguna untuk debugging dan pembelajaran. Berikut adalah fungsi untuk mencetak isi symbol table:

\begin{lstlisting}[language=C++, caption={Fungsi visualisasi symbol table}]
// Fungsi helper untuk visualisasi (perlu menambahkan getCurrentScope() 
// ke class SymbolTable atau menggunakan pendekatan lain)
void printSymbolTableHelper(Scope* scope) {
    if (scope == nullptr) return;
    
    // Rekursif ke parent dulu (agar print dari global ke current)
    printSymbolTableHelper(scope->getParent());
    
    // Print scope ini
    std::cout << "\nLevel " << scope->getLevel() << ":" << std::endl;
    std::cout << "  Symbols:" << std::endl;
    
    // Print semua symbols di scope ini
    for (const auto& name : scope->getDeclaredNames()) {
        Symbol* sym = scope->lookupLocal(name);
        if (sym != nullptr) {
            std::cout << "    " << sym->name 
                      << " : " << sym->type
                      << " (line " << sym->line_number << ")" 
                      << std::endl;
        }
    }
}

void printSymbolTable(SymbolTable& st) {
    std::cout << "\n=== Symbol Table ===" << std::endl;
    
    // Asumsikan SymbolTable memiliki method getCurrentScope()
    // Atau kita bisa mengakses current_scope jika public/protected
    // Untuk contoh ini, kita asumsikan ada method helper
    Scope* current = st.getCurrentScope();  // Perlu ditambahkan ke class
    
    printSymbolTableHelper(current);
    
    std::cout << "===================\n" << std::endl;
}
\end{lstlisting}

Contoh output visualisasi:

\begin{verbatim}
=== Symbol Table ===

Level 0:
  Symbols:
    x : int (line 1)

Level 1:
  Symbols:
    y : int (line 4)
    x : int (line 5)

Level 2:
  Symbols:
    z : int (line 8)
    y : int (line 9)
===================
\end{verbatim}
\section{Integrasi dengan Semantic Analyzer}

Symbol table diintegrasikan dengan semantic analyzer untuk melakukan berbagai pemeriksaan:

\begin{enumerate}
    \item \textbf{Declaration Check}: Memastikan setiap identifier dideklarasi sebelum digunakan
    \item \textbf{Type Checking}: Memverifikasi tipe data dalam operasi dan assignment
    \item \textbf{Scope Resolution}: Menyelesaikan referensi identifier ke deklarasi yang benar
    \item \textbf{Duplicate Detection}: Mendeteksi deklarasi ganda dalam scope yang sama
\end{enumerate}

Contoh integrasi dengan semantic analyzer:

\begin{lstlisting}[language=C++, caption={Contoh penggunaan symbol table dalam semantic analysis}]
void semanticAnalyzeIdentifier(ASTNode* node) {
    std::string name = node->getName();
    
    // Lookup identifier
    Symbol* sym = symbolTable.lookup(name);
    
    if (sym == nullptr) {
        // Error: identifier tidak dideklarasi
        error("Undeclared identifier: " + name, 
              node->getLineNumber());
        return;
    }
    
    // Annotate AST node dengan symbol info
    node->setSymbol(sym);
    node->setType(sym->type);
}

void semanticAnalyzeAssignment(ASTNode* node) {
    ASTNode* lhs = node->getLeft();
    ASTNode* rhs = node->getRight();
    
    // Analyze kedua sisi
    semanticAnalyzeExpression(lhs);
    semanticAnalyzeExpression(rhs);
    
    // Type checking
    if (lhs->getType() != rhs->getType()) {
        error("Type mismatch in assignment", 
              node->getLineNumber());
    }
}
\end{lstlisting}
\section{Kesimpulan}

Dalam bab ini, kita telah mempelajari:

\begin{enumerate}
    \item Symbol table adalah struktur data penting yang memetakan identifier ke informasi deklarasinya
    \item Hash table adalah implementasi efisien untuk symbol table dengan waktu akses O(1) rata-rata
    \item Nested scopes diimplementasikan menggunakan stack of hash tables, di mana setiap scope memiliki parent pointer
    \item Name resolution mengikuti aturan static scoping: pencarian dimulai dari scope saat ini dan naik ke enclosing scopes
    \item Shadowing terjadi ketika identifier dalam scope dalam memiliki nama yang sama dengan identifier di scope luar
    \item Scope entry/exit harus ditangani dengan benar untuk berbagai konstruk bahasa (function, block, loop)
    \item Visualisasi symbol table membantu dalam debugging dan pembelajaran
\end{enumerate}

Pemahaman tentang symbol table dan scope management adalah dasar penting untuk implementasi semantic analysis yang akan dibahas dalam bab selanjutnya.
\section{Referensi dan Bahan Bacaan Lanjutan}

Untuk memperdalam pemahaman tentang symbol table dan scope management, mahasiswa disarankan membaca:

\begin{itemize}
    \item \textbf{Dragon Book}: Aho, Lam, Sethi, \& Ullman (2006). \textit{Compilers: Principles, Techniques, and Tools} \cite{aho2006compilers} - Bab 2.7: Symbol Tables
    
    \item \textbf{Engineering a Compiler}: Cooper \& Torczon (2011) \cite{cooper2011engineering} - Bab 5: Scoping
    
    \item \textbf{Nguyen Thanh Vu - Compiler Class Notes}: Semantic Analysis dan Symbol Tables \cite{nguyen2024semantic}
    
    \item \textbf{University of Texas at Arlington}: Symbol Table Implementation \footnote{\url{https://lambda.uta.edu/cse5317/spring18/long/index.html}}
    
    \item \textbf{University of Wisconsin}: Symbol Tables and Scoping \footnote{\url{https://pages.cs.wisc.edu/~fischer/cs536.s08/course.hold/html/NOTES/6.SYMBOL-TABLES.html}}
\end{itemize}

\end{document}

\cleardoublepage
% Bab 11: Type Checking dan Semantic Analysis
% File ini dapat dikompilasi terpisah atau sebagai bagian dari main.tex

\chapter{Type Checking dan Semantic Analysis}
\label{chap:type-checking}

\section{Tujuan Pembelajaran}

Setelah mempelajari bab ini, mahasiswa diharapkan mampu:
\begin{enumerate}
    \item Memahami konsep semantic analysis dan perannya dalam kompilator
    \item Menjelaskan sistem tipe dan aturan type checking
    \item Mengimplementasikan type checker untuk ekspresi aritmatika
    \item Memahami type inference dan type compatibility
    \item Mengimplementasikan semantic error detection dan reporting
    \item Membedakan static type checking dan dynamic type checking
\end{enumerate}

\section{Pengenalan Semantic Analysis}

Setelah fase syntax analysis menghasilkan Abstract Syntax Tree (AST), kompilator perlu memastikan bahwa program tidak hanya valid secara sintaksis, tetapi juga secara semantik. Semantic analysis adalah fase yang memverifikasi bahwa program memenuhi aturan semantik bahasa pemrograman.

Menurut sumber dari Nguyen Thanh Vu:

\begin{quote}
``Type checking: operator operands must be type-compatible. Return types match declared types. Implicit/explicit conversions. Semantic analysis ensures that the parse tree makes sense under language rules.''\cite{nguyen2024semantic}
\end{quote}

Semantic analysis bertanggung jawab untuk memeriksa berbagai aspek semantik program:

\begin{itemize}
    \item \textbf{Type Checking}: Memastikan operasi dilakukan pada tipe yang kompatibel
    \item \textbf{Scope Resolution}: Memastikan setiap identifier merujuk ke deklarasi yang valid
    \item \textbf{Name Resolution}: Menyelesaikan referensi variabel, fungsi, dan tipe
    \item \textbf{Contextual Checks}: Memeriksa aturan spesifik bahasa (misalnya: break hanya dalam loop, return type match, dll.)
\end{itemize}

\subsection{Input dan Output Semantic Analysis}

Semantic analyzer bekerja dengan:

\textbf{Input:}
\begin{itemize}
    \item Abstract Syntax Tree (AST) dari syntax analyzer
    \item Symbol table yang sudah dibangun (dari bab sebelumnya)
    \item Type information dari deklarasi
\end{itemize}

\textbf{Output:}
\begin{itemize}
    \item Annotated AST dengan informasi tipe pada setiap node
    \item Symbol table yang dilengkapi dengan informasi tipe
    \item Daftar semantic errors (jika ada)
    \item Type-checked program yang siap untuk code generation
\end{itemize}

\section{Sistem Tipe (Type System)}

Sistem tipe adalah kumpulan aturan yang menentukan bagaimana tipe data ditetapkan pada konstruksi program dan operasi apa yang "legal" untuk setiap tipe.

\subsection{Jenis-jenis Type System}

\subsubsection{Static vs Dynamic Typing}

\begin{itemize}
    \item \textbf{Static Typing}: Pengecekan tipe dilakukan pada waktu kompilasi. Bahasa seperti C, C++, Java, dan Rust menggunakan static typing. Keuntungannya adalah deteksi error lebih awal dan performa runtime yang lebih baik.
    
    \item \textbf{Dynamic Typing}: Pengecekan tipe dilakukan pada waktu eksekusi. Bahasa seperti Python, JavaScript, dan Ruby menggunakan dynamic typing. Keuntungannya adalah fleksibilitas lebih tinggi, tetapi error baru terdeteksi saat runtime.
\end{itemize}

\subsubsection{Nominal vs Structural Typing}

\begin{itemize}
    \item \textbf{Nominal Typing}: Kompatibilitas tipe ditentukan berdasarkan nama tipe yang dideklarasikan. Dua tipe dengan struktur yang sama tetapi nama berbeda dianggap tidak kompatibel. Contoh: Java, C++.
    
    \item \textbf{Structural Typing}: Kompatibilitas tipe ditentukan berdasarkan struktur tipe (field, method). Jika struktur cocok, tipe dianggap kompatibel meskipun nama berbeda. Contoh: TypeScript, OCaml.
\end{itemize}

\subsection{Type Hierarchy}

Dalam bahasa berorientasi objek, tipe-tipe membentuk hierarki melalui inheritance. Misalnya:

\begin{verbatim}
Object
|-- Number
|   |-- Integer
|   `-- Float
|-- String
`-- Boolean
\end{verbatim}

Hierarki ini memungkinkan subtyping, di mana tipe turunan dapat digunakan di tempat tipe induk (substitution principle).

\section{Type Checking}

Type checking adalah proses memastikan bahwa program mematuhi aturan tipe bahasa pemrograman. Menurut GeeksforGeeks:

\begin{quote}
``The process of ensuring that a program adheres to the language's type rules. Checking for things like 'you can't add an integer to a string', that function calls match the declared parameter types, etc.''\footnote{\url{https://www.geeksforgeeks.org/type-checking-in-compiler-design/}}
\end{quote}

\subsection{Aturan Type Checking Dasar}

\subsubsection{Type Checking untuk Ekspresi Aritmatika}

Untuk ekspresi aritmatika, aturan dasar meliputi:

\begin{enumerate}
    \item \textbf{Literals}: Setiap literal memiliki tipe intrinsik
    \begin{itemize}
        \item Integer literal (42) → \texttt{int}
        \item Float literal (3.14) → \texttt{float}
        \item String literal ("hello") → \texttt{string}
    \end{itemize}
    
    \item \textbf{Operasi Aritmatika}: Operan harus kompatibel
    \begin{itemize}
        \item \texttt{int + int} → \texttt{int}
        \item \texttt{float + float} → \texttt{float}
        \item \texttt{int + float} → \texttt{float} (dengan implicit conversion)
    \end{itemize}
    
    \item \textbf{Assignment}: Tipe ekspresi harus kompatibel dengan tipe variabel
    \begin{itemize}
        \item \texttt{int x = 42;} $\checkmark$ (valid)
        \item \texttt{int x = 3.14;} $\times$ (type mismatch, atau perlu explicit cast)
    \end{itemize}
\end{enumerate}

\subsubsection{Type Checking untuk Function Calls}

Untuk pemanggilan fungsi, type checker memverifikasi:

\begin{enumerate}
    \item Jumlah argumen sesuai dengan jumlah parameter
    \item Tipe setiap argumen kompatibel dengan tipe parameter yang sesuai
    \item Return type dari fungsi sesuai dengan konteks penggunaan
\end{enumerate}

Contoh:
\begin{verbatim}
int add(int a, int b) { return a + b; }

// Valid call
int result = add(5, 10);  // OK

// Invalid calls
add(5);                   // ERROR: Wrong number of arguments
int x = add(5.0, 10.0);   // ERROR: Type mismatch (float vs int)
\end{verbatim}

\subsection{Implementasi Type Checker Sederhana}

Berikut adalah contoh struktur data untuk type checker dalam C++:

\begin{lstlisting}[language=C++, caption=Struktur Data untuk Type System]
enum class TypeKind {
    INT,
    FLOAT,
    STRING,
    BOOL,
    VOID,
    ARRAY,
    FUNCTION
};

struct Type {
    TypeKind kind;
    // Untuk array: element type
    // Untuk function: parameter types dan return type
    std::vector<Type> subtypes;
};

class TypeChecker {
private:
    SymbolTable* symbolTable;
    
public:
    TypeChecker(SymbolTable* st) : symbolTable(st) {}
    
    // Type check sebuah ekspresi
    Type checkExpression(ASTNode* expr);
    
    // Type check sebuah statement
    void checkStatement(ASTNode* stmt);
    
    // Type check sebuah program
    void checkProgram(ASTNode* program);
    
    // Cek kompatibilitas tipe
    bool isCompatible(Type t1, Type t2);
    
    // Lakukan implicit conversion jika perlu
    Type promoteType(Type t1, Type t2);
};
\end{lstlisting}

\subsubsection{Type Checking untuk Binary Operations}

Berikut adalah contoh implementasi type checking untuk operasi biner:

\begin{lstlisting}[language=C++, caption=Type Checking untuk Binary Operations]
Type TypeChecker::checkBinaryOp(ASTNode* node) {
    ASTBinaryOp* binOp = static_cast<ASTBinaryOp*>(node);
    
    Type leftType = checkExpression(binOp->left);
    Type rightType = checkExpression(binOp->right);
    
    switch (binOp->op) {
        case OP_PLUS:
        case OP_MINUS:
        case OP_MULTIPLY:
        case OP_DIVIDE:
            // Operasi aritmatika: int atau float
            if (leftType.kind == TypeKind::INT && 
                rightType.kind == TypeKind::INT) {
                return Type{TypeKind::INT};
            }
            if (leftType.kind == TypeKind::FLOAT || 
                rightType.kind == TypeKind::FLOAT) {
                return promoteType(leftType, rightType);
            }
            reportError("Arithmetic operation on incompatible types");
            break;
            
        case OP_EQUAL:
        case OP_NOT_EQUAL:
        case OP_LESS:
        case OP_GREATER:
            // Operasi perbandingan: hasilnya boolean
            if (isCompatible(leftType, rightType)) {
                return Type{TypeKind::BOOL};
            }
            reportError("Comparison on incompatible types");
            break;
            
        default:
            reportError("Unknown binary operator");
    }
    
    return Type{TypeKind::VOID}; // Error type
}
\end{lstlisting}

\section{Type Inference}

Type inference adalah kemampuan kompilator untuk secara otomatis menentukan tipe ekspresi tanpa memerlukan anotasi tipe eksplisit dari programmer.

\subsection{Type Inference untuk Literal}

Untuk literal, tipe dapat langsung diinfer:

\begin{itemize}
    \item \texttt{42} → \texttt{int}
    \item \texttt{3.14} → \texttt{float}
    \item \texttt{"hello"} → \texttt{string}
    \item \texttt{true} → \texttt{bool}
\end{itemize}

\subsection{Type Inference untuk Operasi}

Untuk operasi, tipe hasil diinfer berdasarkan tipe operan:

\begin{itemize}
    \item \texttt{int + int} → \texttt{int}
    \item \texttt{int + float} → \texttt{float} (promotion)
    \item \texttt{int == int} → \texttt{bool}
    \item \texttt{int < float} → \texttt{bool}
\end{itemize}

\subsection{Type Inference untuk Variabel}

Dalam beberapa bahasa (seperti C++ dengan \texttt{auto}, atau Rust), tipe variabel dapat diinfer dari initializer:

\begin{verbatim}
auto x = 42;        // x inferred as int
auto y = 3.14;      // y inferred as float
auto z = x + y;     // z inferred as float
\end{verbatim}

\subsection{Implementasi Type Inference Sederhana}

\begin{lstlisting}[language=C++, caption=Type Inference untuk Ekspresi]
Type TypeChecker::inferType(ASTNode* expr) {
    switch (expr->nodeType) {
        case NODE_INT_LITERAL:
            return Type{TypeKind::INT};
            
        case NODE_FLOAT_LITERAL:
            return Type{TypeKind::FLOAT};
            
        case NODE_STRING_LITERAL:
            return Type{TypeKind::STRING};
            
        case NODE_VARIABLE: {
            ASTVariable* var = static_cast<ASTVariable*>(expr);
            Symbol* symbol = symbolTable->lookup(var->name);
            if (symbol) {
                return symbol->type;
            }
            reportError("Undeclared variable: " + var->name);
            return Type{TypeKind::VOID};
        }
        
        case NODE_BINARY_OP:
            return checkBinaryOp(expr);
            
        case NODE_FUNCTION_CALL: {
            ASTFunctionCall* call = static_cast<ASTFunctionCall*>(expr);
            Symbol* func = symbolTable->lookup(call->name);
            if (func && func->type.kind == TypeKind::FUNCTION) {
                return func->type.subtypes.back(); // Return type
            }
            reportError("Undeclared function: " + call->name);
            return Type{TypeKind::VOID};
        }
        
        default:
            reportError("Cannot infer type for node");
            return Type{TypeKind::VOID};
    }
}
\end{lstlisting}

\section{Type Compatibility}

Type compatibility menentukan apakah satu tipe dapat digunakan di tempat tipe lain. Menurut sumber dari TypeScript documentation:

\begin{quote}
``Two types are compatible if one can be used in place of another without type errors. This often arises when assigning a value to a variable, passing arguments to a function, etc.''\footnote{\url{https://www.typescriptlang.org/docs/handbook/type-compatibility.html}}
\end{quote}

\subsection{Aturan Type Compatibility}

\subsubsection{Exact Match}

Dua tipe yang identik selalu kompatibel:

\begin{verbatim}
int x = 42;        // int = int (OK)
float y = 3.14;    // float = float (OK)
\end{verbatim}

\subsubsection{Implicit Conversion (Type Promotion)}

Beberapa bahasa mengizinkan implicit conversion antara tipe yang "dekat":

\begin{verbatim}
int x = 42;
float y = x;       // int -> float (promotion) OK

float a = 3.14;
int b = a;         // float -> int (downgrade) ERROR atau perlu cast
\end{verbatim}

Aturan umum untuk promotion:
\begin{itemize}
    \item \texttt{int} → \texttt{float} (biasanya diizinkan)
    \item \texttt{int} → \texttt{long} (diizinkan)
    \item \texttt{float} → \texttt{double} (diizinkan)
    \item \texttt{float} → \texttt{int} (biasanya memerlukan explicit cast)
\end{itemize}

\subsubsection{Subtyping}

Dalam bahasa berorientasi objek, tipe turunan kompatibel dengan tipe induk:

\begin{verbatim}
class Animal { }
class Dog extends Animal { }

Animal a = new Dog();  // Dog is subtype of Animal (OK)
\end{verbatim}

\subsection{Implementasi Type Compatibility Check}

\begin{lstlisting}[language=C++, caption=Implementasi Type Compatibility]
bool TypeChecker::isCompatible(Type t1, Type t2) {
    // Exact match
    if (t1.kind == t2.kind) {
        // Untuk tipe kompleks, perlu cek lebih detail
        if (t1.kind == TypeKind::ARRAY) {
            return isCompatible(t1.subtypes[0], t2.subtypes[0]);
        }
        if (t1.kind == TypeKind::FUNCTION) {
            // Cek parameter types dan return type
            if (t1.subtypes.size() != t2.subtypes.size()) {
                return false;
            }
            for (size_t i = 0; i < t1.subtypes.size() - 1; i++) {
                if (!isCompatible(t1.subtypes[i], t2.subtypes[i])) {
                    return false;
                }
            }
            return isCompatible(
                t1.subtypes.back(), 
                t2.subtypes.back()
            );
        }
        return true;
    }
    
    // Implicit conversion rules
    if (t1.kind == TypeKind::INT && t2.kind == TypeKind::FLOAT) {
        return true;  // int dapat di-promote ke float
    }
    
    // Subtyping (jika ada)
    // ... implementasi subtyping check
    
    return false;
}

Type TypeChecker::promoteType(Type t1, Type t2) {
    // Jika salah satu float, hasilnya float
    if (t1.kind == TypeKind::FLOAT || t2.kind == TypeKind::FLOAT) {
        return Type{TypeKind::FLOAT};
    }
    // Default: int
    return Type{TypeKind::INT};
}
\end{lstlisting}

\section{Semantic Error Detection dan Reporting}

Semantic analyzer harus mendeteksi berbagai jenis error dan memberikan pesan error yang jelas dan informatif.

\subsection{Jenis-jenis Semantic Error}

\subsubsection{Undeclared Variable}

Error terjadi ketika variabel digunakan sebelum dideklarasi:

\begin{verbatim}
x = 42;  // Error: 'x' is not declared
int x;
\end{verbatim}

\subsubsection{Type Mismatch}

Error terjadi ketika tipe tidak kompatibel:

\begin{verbatim}
int x = "hello";  // Error: cannot assign string to int
int y = 3.14;     // Error: cannot assign float to int (tanpa cast)
\end{verbatim}

\subsubsection{Undefined Function}

Error terjadi ketika fungsi dipanggil tetapi tidak didefinisikan:

\begin{verbatim}
int result = add(5, 10);  // Error: function 'add' is not defined
\end{verbatim}

\subsubsection{Wrong Number of Arguments}

Error terjadi ketika jumlah argumen tidak sesuai:

\begin{verbatim}
int add(int a, int b) { return a + b; }
int x = add(5);  // Error: expected 2 arguments, got 1
\end{verbatim}

\subsubsection{Return Type Mismatch}

Error terjadi ketika return type tidak sesuai dengan deklarasi:

\begin{verbatim}
int getValue() {
    return "hello";  // Error: function should return int
}
\end{verbatim}

\subsection{Error Reporting yang Informatif}

Pesan error yang baik harus:
\begin{itemize}
    \item Menunjukkan lokasi error (baris, kolom)
    \item Menjelaskan jenis error dengan jelas
    \item Memberikan konteks yang relevan
    \item Menyarankan solusi jika memungkinkan
\end{itemize}

Contoh implementasi error reporting:

\begin{lstlisting}[language=C++, caption=Error Reporting System]
class ErrorReporter {
private:
    std::vector<Error> errors;
    
public:
    void reportError(const std::string& message, 
                     int line, int column) {
        errors.push_back(Error{message, line, column});
        std::cerr << "Error at line " << line 
                  << ", column " << column 
                  << ": " << message << std::endl;
    }
    
    void reportTypeError(const std::string& expected,
                        const std::string& got,
                        int line, int column) {
        std::string msg = "Type mismatch: expected " + expected +
                         ", got " + got;
        reportError(msg, line, column);
    }
    
    bool hasErrors() const {
        return !errors.empty();
    }
    
    const std::vector<Error>& getErrors() const {
        return errors;
    }
};
\end{lstlisting}

\section{Integrasi dengan Symbol Table}

Type checking bekerja erat dengan symbol table yang telah dibangun pada fase sebelumnya. Symbol table menyediakan informasi tentang:

\begin{itemize}
    \item Tipe setiap variabel yang dideklarasikan
    \item Tipe parameter dan return type setiap fungsi
    \item Scope di mana setiap identifier dideklarasikan
\end{itemize}

Contoh integrasi:

\begin{lstlisting}[language=C++, caption=Type Checking dengan Symbol Table]
Type TypeChecker::checkVariable(ASTVariable* var) {
    Symbol* symbol = symbolTable->lookup(var->name);
    
    if (!symbol) {
        errorReporter->reportError(
            "Undeclared variable: " + var->name,
            var->line, var->column
        );
        return Type{TypeKind::VOID};
    }
    
    if (symbol->kind != SymbolKind::VARIABLE) {
        errorReporter->reportError(
            var->name + " is not a variable",
            var->line, var->column
        );
        return Type{TypeKind::VOID};
    }
    
    return symbol->type;
}

void TypeChecker::checkAssignment(ASTAssignment* assign) {
    Type varType = checkVariable(assign->variable);
    Type exprType = checkExpression(assign->expression);
    
    if (!isCompatible(varType, exprType)) {
        errorReporter->reportTypeError(
            typeToString(varType),
            typeToString(exprType),
            assign->line, assign->column
        );
    }
}
\end{lstlisting}

\section{Type Checking untuk Kontrol Flow}

Type checker juga perlu memverifikasi kontrol flow statements:

\subsection{If Statement}

\begin{lstlisting}[language=C++, caption=Type Checking untuk If Statement]
void TypeChecker::checkIfStatement(ASTIf* ifStmt) {
    Type condType = checkExpression(ifStmt->condition);
    
    if (condType.kind != TypeKind::BOOL) {
        errorReporter->reportError(
            "If condition must be boolean",
            ifStmt->line, ifStmt->column
        );
    }
    
    checkStatement(ifStmt->thenBranch);
    if (ifStmt->elseBranch) {
        checkStatement(ifStmt->elseBranch);
    }
}
\end{lstlisting}

\subsection{While Loop}

\begin{lstlisting}[language=C++, caption=Type Checking untuk While Loop]
void TypeChecker::checkWhileLoop(ASTWhile* whileLoop) {
    Type condType = checkExpression(whileLoop->condition);
    
    if (condType.kind != TypeKind::BOOL) {
        errorReporter->reportError(
            "While condition must be boolean",
            whileLoop->line, whileLoop->column
        );
    }
    
    checkStatement(whileLoop->body);
}
\end{lstlisting}

\subsection{Return Statement}

\begin{lstlisting}[language=C++, caption=Type Checking untuk Return Statement]
void TypeChecker::checkReturn(ASTReturn* ret, Type expectedReturnType) {
    if (ret->expression) {
        Type exprType = checkExpression(ret->expression);
        if (!isCompatible(expectedReturnType, exprType)) {
            errorReporter->reportTypeError(
                typeToString(expectedReturnType),
                typeToString(exprType),
                ret->line, ret->column
            );
        }
    } else {
        if (expectedReturnType.kind != TypeKind::VOID) {
            errorReporter->reportError(
                "Function must return a value",
                ret->line, ret->column
            );
        }
    }
}
\end{lstlisting}

\section{Annotated AST}

Setelah type checking, setiap node AST di-annotate dengan informasi tipe. Ini memudahkan fase selanjutnya (code generation) untuk mengetahui tipe setiap ekspresi.

\begin{lstlisting}[language=C++, caption=Annotated AST Node]
class ASTNode {
public:
    NodeType nodeType;
    int line, column;
    Type type;  // Annotated type (setelah type checking)
    
    virtual ~ASTNode() = default;
};

// Contoh penggunaan
Type TypeChecker::checkExpression(ASTNode* expr) {
    Type t = inferType(expr);
    expr->type = t;  // Annotate node dengan type
    return t;
}
\end{lstlisting}

\section{Kesimpulan}

Dalam bab ini, kita telah mempelajari:

\begin{enumerate}
    \item Semantic analysis memverifikasi bahwa program memenuhi aturan semantik bahasa
    \item Type checking memastikan operasi dilakukan pada tipe yang kompatibel
    \item Type inference memungkinkan kompilator menentukan tipe secara otomatis
    \item Type compatibility menentukan apakah satu tipe dapat digunakan di tempat tipe lain
    \item Semantic error detection dan reporting memberikan feedback yang jelas kepada programmer
    \item Type checking terintegrasi dengan symbol table untuk mengakses informasi deklarasi
    \item Annotated AST menyimpan informasi tipe untuk digunakan pada fase selanjutnya
\end{enumerate}

Pemahaman tentang type checking dan semantic analysis ini penting karena memastikan bahwa program yang dikompilasi tidak hanya valid secara sintaksis, tetapi juga benar secara semantik sebelum masuk ke fase code generation.

\section{Latihan}

\begin{enumerate}
    \item Jelaskan perbedaan antara static type checking dan dynamic type checking. Berikan contoh bahasa pemrograman untuk masing-masing.
    
    \item Implementasikan type checker sederhana untuk ekspresi aritmatika dengan mendukung:
    \begin{itemize}
        \item Operasi +, -, *, / untuk int dan float
        \item Operasi perbandingan ==, !=, <, > yang menghasilkan boolean
        \item Type promotion (int → float)
    \end{itemize}
    
    \item Buatlah fungsi untuk mengecek type compatibility dengan aturan:
    \begin{itemize}
        \item Exact match selalu kompatibel
        \item int dapat di-assign ke float (implicit conversion)
        \item float tidak dapat di-assign ke int tanpa explicit cast
    \end{itemize}
    
    \item Implementasikan semantic error detection untuk:
    \begin{itemize}
        \item Undeclared variable
        \item Type mismatch pada assignment
        \item Wrong number of arguments pada function call
    \end{itemize}
    
    \item Buatlah test cases untuk type checker yang mencakup:
    \begin{itemize}
        \item Valid expressions (harus pass type checking)
        \item Invalid expressions (harus menghasilkan error yang sesuai)
        \item Edge cases (null, empty, boundary values)
    \end{itemize}
    
    \item Jelaskan mengapa annotated AST diperlukan. Bagaimana informasi tipe pada AST digunakan pada fase code generation?
    
    \item Implementasikan type checking untuk if statement dan while loop. Pastikan kondisi harus bertipe boolean.
    
    \item Bandingkan nominal typing dan structural typing. Berikan contoh situasi di mana masing-masing pendekatan lebih menguntungkan.
\end{enumerate}

\section{Referensi dan Bahan Bacaan Lanjutan}

Untuk memperdalam pemahaman tentang type checking dan semantic analysis, mahasiswa disarankan membaca:

\begin{itemize}
    \item \textbf{Dragon Book}: Aho, Lam, Sethi, \& Ullman (2006). \textit{Compilers: Principles, Techniques, and Tools} \cite{aho2006compilers} - Bab 6: Type Checking
    
    \item \textbf{Engineering a Compiler}: Cooper \& Torczon (2011) \cite{cooper2011engineering} - Bab 4: Context-Sensitive Analysis
    
    \item \textbf{Nguyen Thanh Vu - Compiler Class Notes}: Semantic Analysis \cite{nguyen2024semantic}
    
    \item \textbf{GeeksforGeeks}: Type Checking in Compiler Design \footnote{\url{https://www.geeksforgeeks.org/type-checking-in-compiler-design/}}
    
    \item \textbf{Wikipedia - Type Inference}: \footnote{\url{https://en.wikipedia.org/wiki/Type_inference}}
    
    \item \textbf{Wikipedia - Type System}: \footnote{\url{https://en.wikipedia.org/wiki/Type_system}}
    
    \item \textbf{TypeScript Handbook - Type Compatibility}: \footnote{\url{https://www.typescriptlang.org/docs/handbook/type-compatibility.html}}
\end{itemize}

\cleardoublepage
\documentclass[../main.tex]{subfiles}

\addbibresource{\subfix{../references.bib}}

\begin{document}

\ifSubfilesClassLoaded{%
    \setcounter{chapter}{11}%
    \begin{refsection}
}{}

\chapter{Code Generation dan Target Machine}
\label{chap:code-generation}

\begin{subcpmk}
  \item \textbf{Sub-CPMK 5.3:} Mengimplementasikan code generator untuk arsitektur target
\end{subcpmk}

% ============================================================
% MATERI POKOK
% ============================================================
\section{Pengenalan Target Machine (RISC vs CISC)}

\compiler{Target Machine} adalah arsitektur komputer tujuan di mana kode hasil kompilasi akan dijalankan. Memahami karakteristik perangkat keras sangat krusial bagi \textit{code generator} untuk menghasilkan instruksi yang optimal \cite{jhu2024compilers}.

\subsection{Arsitektur RISC (Reduced Instruction Set Computer)}
Contoh: RISC-V, ARM.
\begin{itemize}
    \item \textbf{Load-Store Architecture}: Hanya instruksi \code{Load} dan \code{Store} yang bisa mengakses memori. Instruksi aritmatika hanya bekerja pada register.
    \item \textbf{Register Pressure}: Karena semua operan harus berada di register, RISC membutuhkan lebih banyak register fisik dan temporer, yang meningkatkan "tekanan" pada pengalokasi register.
    \item \textbf{Fixed-Length}: Instruksi selalu berukuran tetap (32-bit), memudahkan \textit{pipelining}.
\end{itemize}

\subsection{Arsitektur CISC (Complex Instruction Set Computer)}
Contoh: x86 (Intel/AMD).
\begin{itemize}
    \item \textbf{Orthogonality}: Kemampuan instruksi untuk menggunakan berbagai mode pengalamatan secara bebas. Misalnya, instruksi \code{ADD} pada x86 bisa menjumlahkan register dengan memori secara langsung.
    \item \textbf{Variable-Length}: Instruksi berukuran 1-15 byte, menghemat ruang memori tapi mempersulit pendekodean instruksi (\textit{decoding}).
\end{itemize}

\subsection{Dampak pada Code Generation}
Code generator harus memilih strategi yang sesuai dengan arsitektur:
\begin{enumerate}
    \item Pada \textbf{RISC}, fokus pada jadwal instruksi (\textit{scheduling}) untuk menghindari \textit{pipeline stall} dan manajemen register yang agresif.
    \item Pada \textbf{CISC}, fokus pada pemilihan instruksi kompleks yang dapat menggabungkan beberapa operasi TAC menjadi satu instruksi mesin untuk mengurangi ukuran kode (\textit{code density}).
\end{enumerate}

\begin{figure}[!htbp]
    \centering
    \adjustbox{max width=0.8\textwidth,center}{%
    \begin{tikzpicture}[
        node/.style={rectangle, draw=blue!50, fill=blue!10, text width=6cm, font=\tiny, align=center}
    ]
    \node[node] (risc) {RISC: Banyak Temporary $\rightarrow$ Register Allocator harus cerdas.};
    \node[node, below=0.5cm of risc] (cisc) {CISC: Instruksi Kompleks $\rightarrow$ Instruction Selector harus cerdas.};
    \end{tikzpicture}%
    }
    \caption{Prioritas Optimasi berdasarkan Arsitektur Target}
\end{figure}

\section{Pemilihan Instruksi (Instruction Selection)}

\compiler{Instruction Selection} adalah proses memetakan instruksi tingkat menengah (\textit{TAC}) ke instruksi spesifik mesin target yang memberikan performa terbaik.

\subsection{Tiling (Pengubinan)}
Proses pemilihan instruksi sering divisualisasikan sebagai "menutupi" pohon ekspresi (\textit{Expression Tree}) dengan "ubin" (\textit{tiles}). Setiap ubin mewakili satu instruksi mesin yang dapat menggantikan satu atau lebih simpul pada pohon tersebut.

\subsection{Algoritma Pilihan}
Ada dua strategi populer untuk melakukan \textit{tiling}:
\begin{enumerate}
    \item \textbf{Maximal Munch}: Strategi \textit{greedy} (rakus). Dimulai dari akar pohon, pilih ubin terbesar yang cocok (\textit{match}). Jika ada ubin berukuran 3 simpul dan 1 simpul, ubin 3 simpul akan dipilih. Sangat efektif untuk arsitektur RISC.
    \item \textbf{Dynamic Programming}: Strategi optimal. Menghitung biaya (\textit{cost}) minimum untuk menutupi setiap sub-pohon. Algoritma ini memastikan total biaya seluruh pohon adalah yang terendah. Sangat berguna jika CPU memiliki banyak instruksi kompleks dengan biaya yang bervariasi.
\end{enumerate}

\begin{figure}[!htbp]
    \centering
    \adjustbox{max width=0.8\textwidth,center}{%
    \begin{tikzpicture}[
        node/.style={circle, draw, minimum size=0.6cm, font=\tiny}
    ]
    \node[node] (add) {+};
    \node[node, below left=0.5cm and 0.3cm of add] (mul) {*};
    \node[node, below right=0.5cm and 0.3cm of add] (d) {d};
    \node[node, below left=0.5cm and 0.2cm of mul] (b) {b};
    \node[node, below right=0.5cm and 0.2cm of mul] (c) {c};
    
    \draw[blue, thick, dashed] (-1, -1.5) rectangle (1.2, 0.5);
    \node[blue, font=\tiny] at (1.5, 0) {Tile: MADD};
    \end{tikzpicture}%
    }
    \caption{Representasi Tiling: Satu instruksi MADD menutupi operasi Multiply dan Add}
\end{figure}

\section{Register Allocation}

\subsection{Register Allocation Problem}

Masalah alokasi register:

\begin{itemize}
  \item Variabel terbatas vs unlimited temporaries
  \item Register interference
  \item Spilling ke memory
  \item Calling convention constraints
\end{itemize}

\subsection{Linear Scan Allocation}

\begin{lstlisting}[language=C]
typedef struct {
    char *var_name;
    int start_point;    // First use
    int end_point;      // Last use
    int register_num;   // Assigned register (-1 if spilled)
} LiveRange;

void linear_scan_allocation(LiveRange *ranges, int count, 
                           int num_registers) {
    // Sort ranges by start point
    sort_ranges_by_start(ranges, count);
    
    bool *registers_used = calloc(num_registers, sizeof(bool));
    
    for (int i = 0; i < count; i++) {
        // Free registers whose ranges have ended
        free_expired_registers(ranges[i].start_point, 
                             registers_used, num_registers);
        
        // Find free register
        int reg = find_free_register(registers_used, num_registers);
        if (reg != -1) {
            ranges[i].register_num = reg;
            registers_used[reg] = true;
        } else {
            // Spill to memory
            ranges[i].register_num = -1;
        }
    }
}
\end{lstlisting}

\section{Target Architecture}

\subsection{x86 Architecture}

Karakteristik x86:

\begin{itemize}
  \item CISC (Complex Instruction Set Computer)
  \item Variable-length instructions
  \item Rich addressing modes
  \item Backward compatibility
\end{itemize}

\begin{lstlisting}[language=C]
// x86 instruction examples
MOV EAX, EBX          ; Register to register
MOV EAX, [EBX+4]     ; Base + offset addressing
MOV EAX, [EBX+ECX*4] ; Base + index*scale
LEA EAX, [EBX+ECX*2] ; Load effective address
PUSH EAX             ; Stack operation
POP EBX              ; Stack operation
\end{lstlisting}

\subsection{RISC Architecture}

Karakteristik RISC (MIPS/ARM):

\begin{itemize}
  \item Fixed-length instructions
  \item Load/store architecture
  \item Simple addressing modes
  \item Large register file
\end{itemize}

\begin{lstlisting}[language=C]
// MIPS instruction examples
ADD $t0, $t1, $t2    ; $t0 = $t1 + $t2
LW $t0, 4($t1)       ; Load word from memory
SW $t0, 8($t1)       ; Store word to memory
ADDI $t0, $t1, 10    ; $t0 = $t1 + 10 (immediate)
\end{lstlisting}

\section{Code Generation Algorithm}

\subsection{Basic Block Code Generation}

\begin{lstlisting}[language=C]
void generate_block_code(BasicBlock *block, 
                         RegisterAllocator *alloc) {
    for (int i = 0; i < block->instruction_count; i++) {
        TACInstruction *inst = &block->instructions[i];
        
        // Allocate registers for operands
        int reg1 = allocate_register(inst->arg1, alloc);
        int reg2 = allocate_register(inst->arg2, alloc);
        int reg3 = allocate_register(inst->result, alloc);
        
        // Generate target instruction
        switch (inst->op) {
            case OP_ADD:
                emit_add(reg3, reg1, reg2);
                break;
            case OP_MUL:
                emit_mul(reg3, reg1, reg2);
                break;
            case OP_ASSIGN:
                emit_mov(reg3, reg1);
                break;
            // ... other operations
        }
        
        // Release temporary registers
        release_register(reg1, alloc);
        release_register(reg2, alloc);
    }
}
\end{lstlisting}

\subsection{Function Call Generation}

\begin{lstlisting}[language=C]
void generate_function_call(FunctionCall *call) {
    // Save caller-saved registers
    save_caller_saved_registers();
    
    // Push parameters (right-to-left for cdecl)
    for (int i = call->param_count - 1; i >= 0; i--) {
        push_parameter(call->parameters[i]);
    }
    
    // Call function
    emit_call(call->function_name);
    
    // Clean up stack (callee or caller depending on convention)
    cleanup_stack(call->param_count * sizeof(int));
    
    // Restore caller-saved registers
    restore_caller_saved_registers();
    
    // Move return value to target register
    if (call->has_return_value) {
        emit_mov(call->return_reg, EAX);
    }
}
\end{lstlisting}

\section{Optimization in Code Generation}

\subsection{Peephole Optimization}

\begin{lstlisting}[language=C]
void peephole_optimization(Instruction *instructions, int count) {
    for (int i = 0; i < count - 1; i++) {
        // MOV reg, reg -> NOP
        if (is_mov_reg_to_reg(&instructions[i])) {
            instructions[i].opcode = NOP;
        }
        
        // PUSH reg; POP reg -> NOP
        if (is_push_pop_same_reg(&instructions[i], &instructions[i+1])) {
            instructions[i].opcode = NOP;
            instructions[i+1].opcode = NOP;
        }
        
        // MOV reg, imm; ADD reg, imm -> LEA reg, [imm]
        if (can_convert_to_lea(&instructions[i], &instructions[i+1])) {
            convert_to_lea(&instructions[i], &instructions[i+1]);
        }
    }
}
\end{lstlisting}

\subsection{Instruction Scheduling}

\begin{lstlisting}[language=C]
void schedule_instructions(Instruction *instructions, int count) {
    // Simple list scheduling for basic blocks
    Instruction *scheduled[count];
    int scheduled_count = 0;
    
    while (scheduled_count < count) {
        // Find ready instructions (no dependencies)
        for (int i = 0; i < count; i++) {
            if (!is_scheduled(&instructions[i]) && 
                is_ready(&instructions[i], scheduled, scheduled_count)) {
                scheduled[scheduled_count++] = instructions[i];
                break;
            }
        }
    }
    
    // Copy scheduled instructions back
    memcpy(instructions, scheduled, count * sizeof(Instruction));
}
\end{lstlisting}


% ============================================================
% AKTIVITAS PEMBELAJARAN
% ============================================================
\begin{aktivitas}
  \item \textbf{Instruction Selection}: Implementasikan instruction selector untuk subset x86.
  \item \textbf{Register Allocation}: Bangun linear scan register allocator.
  \item \textbf{Code Generation}: Implementasikan code generator untuk simple expressions.
  \item \textbf{Peephole Optimization}: Buat peephole optimizer untuk assembly code.
  \item \textbf{Function Calls}: Generate code untuk function calls dengan calling conventions.
\end{aktivitas}

% ============================================================
% LATIHAN DAN REFLEKSI
% ============================================================
\begin{latihan}
  \item Generate assembly code untuk expression tree kompleks!
  \item Implementasikan register allocator dengan spilling strategy!
  \item Analisis instruction selection for different target architectures!
  \item Optimasi generated code dengan peephole optimizations!
  \item Generate code untuk recursive functions dengan proper stack management!
  \item \textbf{Refleksi}: Bagaimana target architecture mempengaruhi code generation strategy?
\end{latihan}

% ============================================================
% ASESMEN
% ============================================================
\begin{asesmen}
\textbf{Instrumen Penilaian untuk Sub-CPMK 5.3}

\textbf{A. Pilihan Ganda}

\begin{enumerate}
  \item Register allocation problem terjadi karena:
  \begin{enumerate}
    \item Terlalu banyak variabel
    \item Terbatasnya jumlah register
    \item Memory terlalu kecil
    \item Instruksi terlalu kompleks
  \end{enumerate}
  
  \item CISC architecture memiliki:
  \begin{enumerate}
    \item Fixed-length instructions
    \item Variable-length instructions
    \item Load/store only
    \item Large register file
  \end{enumerate}
  
  \item Peephole optimization bekerja pada:
  \begin{enumerate}
    \item Single instruction
    \item Small window of instructions
    \item Entire program
    \item Basic blocks
  \end{enumerate}
\end{enumerate}

\textbf{B. Essay}

\begin{enumerate}
  \item Jelaskan complete code generation pipeline dari three-address code ke assembly!
  \item Implementasikan code generator untuk bahasa sederhana dengan arithmetic expressions dan function calls!
\end{enumerate}

\textbf{Rubrik Penilaian}: Lihat Lampiran A
\end{asesmen}

% ============================================================
% CHECKLIST KOMPETENSI
% ============================================================
\begin{checklist}
  \item Saya dapat mengimplementasikan code generator untuk arsitektur target
  \item Saya dapat melakukan instruction selection yang efisien
  \item Saya dapat mengimplementasikan register allocation algorithms
  \item Saya dapat mengenerate code untuk function calls
  \item Saya dapat melakukan peephole optimizations
  \item Saya memahami perbedaan CISC dan RISC architectures
\end{checklist}

% ============================================================
% RANGKUMAN
% ============================================================
\begin{rangkuman}
Bab ini membahas code generation dan target machine, termasuk instruction selection, register allocation, target architectures, dan optimization techniques. Mahasiswa belajar membangun code generator yang efisien.

\textbf{Poin Kunci:}
\begin{itemize}
  \item Code generation mengkonversi intermediate code ke target code
  \item Instruction selection memilih optimal target instructions
  \item Register allocation mengelola limited register resources
  \item Target architecture mempengaruhi generation strategy
  \item Peephole optimization mengoptimasi local instruction patterns
  \item Function calls memerlukan proper calling convention handling
\end{itemize}

\textbf{Kata Kunci}: \compiler{Code Generation}, \compiler{Instruction Selection}, \compiler{Register Allocation}, \compiler{Target Architecture}, \compiler{x86}, \compiler{RISC}, \compiler{Peephole Optimization}, \compiler{Calling Convention}
\end{rangkuman}

\ifSubfilesClassLoaded{%
    \clearpage
    \printbibliography[title={Daftar Pustaka}]
    \end{refsection}
}{}

\end{document}

\cleardoublepage
\documentclass[../main.tex]{subfiles}

% Untuk kompilasi standalone, tambahkan \addbibresource di preamble
% Menggunakan \subfix untuk menangani path relatif dengan benar
% biblatex akan mengabaikan duplikasi jika sudah ada di main.tex
\addbibresource{\subfix{../references.bib}}

\begin{document}

% Set chapter counter untuk kompilasi standalone
% \chapter akan mengincrement counter, jadi untuk bab-13 (Bab 13), set ke 12
% Saat dikompilasi dari main.tex, counter akan diatur oleh subfiles package
\ifSubfilesClassLoaded{%
    \setcounter{chapter}{12}%
    % Untuk kompilasi standalone, gunakan refsection manual
    \begin{refsection}
}{}

\chapter{Runtime Environment dan Memory Management}
\label{chap:runtime-environment}

\section{Tujuan Pembelajaran}

Setelah mempelajari bab ini, mahasiswa diharapkan mampu:
\begin{enumerate}
    \item Memahami konsep runtime environment dan perannya dalam eksekusi program
    \item Menjelaskan struktur activation record (stack frame) dan komponen-komponennya
    \item Mengimplementasikan simulator runtime stack untuk function calls
    \item Memahami memory layout: static, stack, dan heap
    \item Menjelaskan mekanisme heap management dan garbage collection
    \item Mengimplementasikan manajemen memory untuk runtime environment sederhana
\end{enumerate}
\section{Pendahuluan}

Runtime environment adalah konteks di mana program yang telah dikompilasi dieksekusi. Menurut sumber dari StudyLib:

\begin{quote}
``Runtime environment: stack, heap, activation records; garbage collection intro. Managing run-time structures (activation records, memory layout, symbol tables).''\cite{studylib2024obe}
\end{quote}

Runtime environment mencakup:
\begin{itemize}
    \item \textbf{Memory Organization}: Bagaimana memory diorganisir dan dialokasikan untuk berbagai jenis data
    \item \textbf{Calling Conventions}: Mekanisme pemanggilan fungsi, passing parameter, dan return values
    \item \textbf{Scope Management}: Bagaimana variabel diakses berdasarkan scope-nya (local, non-local, global)
    \item \textbf{Memory Management}: Alokasi dan dealokasi memory untuk variabel dan data structures
\end{itemize}

Gambar \ref{fig:runtime-components} menunjukkan komponen-komponen runtime environment.

\begin{figure}[!htbp]
    \centering
    \adjustbox{max width=0.9\textwidth,center}{%
    \begin{tikzpicture}[
        comp/.style={rectangle, draw=blue!50, fill=blue!10, text width=2.5cm, text centered, minimum height=0.7cm, rounded corners, font=\footnotesize, inner sep=4pt, align=center},
        arrow/.style={->, >=stealth, thick},
        node distance=1.2cm
    ]
    
    \node[comp] (stack) {Stack};
    \node[comp, right=of stack] (heap) {Heap};
    \node[comp, below=of stack] (static) {Static\\Data};
    \node[comp, right=of static] (code) {Code\\Segment};
    
    \node[below=0.3cm of static, font=\tiny, align=center] {Runtime Environment};
    
    \end{tikzpicture}%
    }
    \caption{Komponen-komponen runtime environment}
    \label{fig:runtime-components}
\end{figure}

Runtime environment harus dirancang dengan hati-hati karena mempengaruhi:
\begin{itemize}
    \item Efisiensi eksekusi program
    \item Keamanan memory (memory safety)
    \item Kemampuan mendukung fitur bahasa (recursion, nested functions, closures, dll.)
    \item Portabilitas antar platform
\end{itemize}
\section{Memory Layout}

Program yang dieksekusi memiliki memory layout yang terorganisir menjadi beberapa region. Setiap region memiliki karakteristik dan tujuan penggunaan yang berbeda.

\subsection{Memory Regions}

Memory address space program biasanya dibagi menjadi beberapa region utama:

\begin{enumerate}
    \item \textbf{Code/Text Segment}: Berisi instruksi machine code yang dihasilkan compiler. Region ini biasanya read-only dan tidak dapat dimodifikasi saat runtime.
    
    \item \textbf{Static/Global Data}: Berisi variabel global dan static yang dialokasikan pada compile time. Region ini memiliki ukuran tetap dan alamat yang diketahui saat compile time.
    
    \item \textbf{Stack}: Region untuk activation records (stack frames) dari fungsi-fungsi yang sedang aktif. Stack tumbuh ke bawah (dari high address ke low address) dan dikelola secara otomatis.
    
    \item \textbf{Heap}: Region untuk dynamic memory allocation. Heap tumbuh ke atas (dari low address ke high address) dan dikelola secara manual atau melalui garbage collector.
\end{enumerate}

Gambar \ref{fig:memory-layout-visual} menunjukkan layout memory secara visual dengan TikZ.

\begin{figure}[H]
    \centering
    \adjustbox{max width=0.85\textwidth,center}{%
    \begin{tikzpicture}[
        region/.style={rectangle, draw=blue!50, fill=blue!10, text width=4cm, minimum height=0.8cm, font=\footnotesize, align=center, inner sep=4pt},
        arrow/.style={->, >=stealth, thick},
        node distance=0.4cm
    ]
    
    \node[region] (high) {High Address};
    \node[region, below=of high, draw=red!50, fill=red!10] (stack) {Stack\\(grows down)};
    \node[region, below=of stack, draw=green!50, fill=green!10] (heap) {Heap\\(grows up)};
    \node[region, below=of heap] (data) {Static/Global Data};
    \node[region, below=of data] (code) {Code/Text Segment};
    \node[region, below=of code] (low) {Low Address};
    
    \draw[arrow] (stack) to[out=180, in=180] node[left, font=\tiny] {↓} (stack.south west);
    \draw[arrow] (heap) to[out=0, in=0] node[right, font=\tiny] {↑} (heap.south east);
    
    \end{tikzpicture}%
    }
    \caption{Memory layout program}
    \label{fig:memory-layout-visual}
\end{figure}

Gambar \ref{fig:memory-layout} menunjukkan layout memory yang khas:

\begin{figure}[H]
\centering
\begin{verbatim}
High Address
    +-----------------+
    |   Command Line  |
    |     Arguments   |
    +------------------+
    |      Stack      | <- Tumbuh ke bawah
    |   (grows down)  |
    |                 |
    |       ↓         |
    |                 |
    |       ↑         |
    |   (grows up)    |
    |      Heap       | <- Tumbuh ke atas
    +------------------+
    |  BSS (uninit)   |
    |  Data (init)    |
    +------------------+
    |  Text/Code      | <- Read-only
Low Address
\end{verbatim}
\caption{Memory layout khas untuk program yang dieksekusi}
\label{fig:memory-layout}
\end{figure}

\subsection{Static Memory Allocation}

Static memory allocation terjadi pada compile time. Variabel yang dialokasikan secara static memiliki:
\begin{itemize}
    \item Alamat yang tetap dan diketahui saat compile time
    \item Lifetime yang sama dengan program (dari awal hingga akhir eksekusi)
    \item Tidak memerlukan runtime overhead untuk alokasi/dealokasi
\end{itemize}

Contoh variabel static:
\begin{itemize}
    \item Variabel global: \texttt{int global\_var;}
    \item Variabel static lokal: \texttt{static int counter;}
    \item String literals dan konstanta
\end{itemize}

Keuntungan static allocation:
\begin{itemize}
    \item Sangat efisien (tidak ada overhead runtime)
    \item Deterministik (alamat diketahui saat compile time)
    \item Tidak ada risiko memory leak
\end{itemize}

Keterbatasan:
\begin{itemize}
    \item Tidak mendukung recursion dengan baik
    \item Ukuran harus diketahui saat compile time
    \item Tidak fleksibel untuk dynamic data structures
\end{itemize}

\subsection{Stack-Based Memory Allocation}

Stack digunakan untuk activation records dari fungsi-fungsi yang sedang aktif. Stack allocation memiliki karakteristik:
\begin{itemize}
    \item \textbf{Automatic}: Alokasi dan dealokasi terjadi otomatis saat fungsi dipanggil dan kembali
    \item \textbf{LIFO}: Last In First Out - fungsi terakhir dipanggil adalah yang pertama kembali
    \item \textbf{Fast}: Alokasi/dealokasi sangat cepat (hanya mengubah stack pointer)
    \item \textbf{Limited Lifetime}: Data di stack hanya hidup selama fungsi aktif
\end{itemize}

Stack sangat cocok untuk:
\begin{itemize}
    \item Local variables
    \item Function parameters
    \item Return addresses
    \item Temporary values
\end{itemize}

Contoh penggunaan stack:
\begin{lstlisting}[language=C++, caption={Contoh program yang menggunakan stack}]
int factorial(int n) {
    if (n <= 1) return 1;
    int temp = n * factorial(n - 1);  // Recursive call
    return temp;
}

int main() {
    int result = factorial(5);  // Stack frames untuk main dan factorial
    return 0;
}
\end{lstlisting}

\subsection{Heap-Based Memory Allocation}

Heap digunakan untuk dynamic memory allocation yang tidak dapat ditangani oleh stack. Heap allocation memiliki karakteristik:
\begin{itemize}
    \item \textbf{Manual Management}: Programmer harus secara eksplisit mengalokasikan dan membebaskan memory
    \item \textbf{Flexible Lifetime}: Object di heap dapat hidup lebih lama dari fungsi yang membuatnya
    \item \textbf{Variable Size}: Ukuran dapat ditentukan saat runtime
    \item \textbf{Slower}: Alokasi/dealokasi lebih lambat dibanding stack
\end{itemize}

Heap digunakan untuk:
\begin{itemize}
    \item Dynamic arrays dan data structures
    \item Objects yang harus hidup lebih lama dari fungsi pembuatnya
    \item Shared data structures
    \item Large objects yang tidak muat di stack
\end{itemize}

Contoh penggunaan heap:
\begin{lstlisting}[language=C++, caption={Contoh penggunaan heap}]
int* createArray(int size) {
    int* arr = new int[size];  // Alokasi di heap
    return arr;  // Pointer ke heap, valid setelah fungsi kembali
}

void useArray() {
    int* myArray = createArray(100);
    // Gunakan array...
    delete[] myArray;  // Dealokasi manual
}
\end{lstlisting}
\section{Activation Records (Stack Frames)}

Activation record (juga disebut stack frame) adalah struktur data yang digunakan untuk menyimpan informasi tentang eksekusi satu fungsi. Setiap kali fungsi dipanggil, activation record baru dibuat di stack.

\subsection{Komponen Activation Record}

Activation record biasanya berisi komponen-komponen berikut:

\begin{enumerate}
    \item \textbf{Return Address}: Alamat instruksi di caller yang harus dieksekusi setelah fungsi kembali
    
    \item \textbf{Control Link (Dynamic Link)}: Pointer ke activation record dari caller (fungsi yang memanggil)
    
    \item \textbf{Access Link (Static Link)}: Pointer ke activation record dari enclosing scope (untuk nested functions)
    
    \item \textbf{Saved Registers}: Nilai register yang harus disimpan dan dikembalikan setelah fungsi selesai
    
    \item \textbf{Parameters}: Nilai parameter yang diteruskan ke fungsi (actual parameters)
    
    \item \textbf{Local Variables}: Variabel lokal yang dideklarasikan dalam fungsi
    
    \item \textbf{Temporary Values}: Nilai sementara yang digunakan selama komputasi dalam fungsi
    
    \item \textbf{Return Value}: Nilai yang dikembalikan fungsi (jika ada)
\end{enumerate}

Gambar \ref{fig:activation-record-visual} menunjukkan struktur activation record secara visual.

\begin{figure}[H]
    \centering
    \adjustbox{max width=0.85\textwidth,center}{%
    \begin{tikzpicture}[
        comp/.style={rectangle, draw=blue!50, fill=blue!10, text width=3cm, minimum height=0.5cm, font=\tiny, align=center, inner sep=4pt, rounded corners},
        node distance=0.2cm
    ]
    
    \node[comp] (ret) {Return Address};
    \node[comp, below=of ret] (ctrl) {Control Link};
    \node[comp, below=of ctrl] (acc) {Access Link};
    \node[comp, below=of acc] (reg) {Saved Registers};
    \node[comp, below=of reg] (param) {Parameters};
    \node[comp, below=of param] (local) {Local Variables};
    \node[comp, below=of local] (temp) {Temporaries};
    \node[comp, below=of temp] (retval) {Return Value};
    
    \end{tikzpicture}%
    }
    \caption{Struktur activation record}
    \label{fig:activation-record-visual}
\end{figure}

Gambar \ref{fig:activation-record} menunjukkan struktur activation record yang khas:

\begin{figure}[H]
\centering
\begin{verbatim}
High Address
    +---------------------+
    |  Return Address     |
    +----------------------+
    |  Control Link (FP)  | -> Activation record caller
    +----------------------+
    |  Access Link        | -> Enclosing scope (if nested)
    +----------------------+
    |  Saved Registers    |
    +----------------------+
    |  Parameters         |
    |    param1           |
    |    param2           |
    +----------------------+
    |  Local Variables    |
    |    local1           |
    |    local2           |
    +----------------------+
    |  Temporaries        |
    |    temp1            |
    |    temp2            |
    +----------------------+
    |  Return Value       |
Low Address
\end{verbatim}
\caption{Struktur activation record (stack frame)}
\label{fig:activation-record}
\end{figure}

\subsection{Calling Sequence}

Calling sequence adalah urutan instruksi yang dihasilkan compiler untuk memanggil fungsi. Terdapat dua bagian:

\subsubsection{Caller Sequence (Prologue)}

Instruksi yang dijalankan oleh caller sebelum memanggil fungsi:
\begin{enumerate}
    \item Evaluasi actual parameters (dari kanan ke kiri atau kiri ke kanan, tergantung calling convention)
    \item Push parameters ke stack (atau pass melalui register)
    \item Save caller-saved registers
    \item Push return address
    \item Transfer control ke callee (CALL instruction)
\end{enumerate}

\subsubsection{Callee Sequence (Prologue)}

Instruksi yang dijalankan oleh callee di awal fungsi:
\begin{enumerate}
    \item Save frame pointer (FP) dari caller
    \item Set FP baru ke current stack pointer (SP)
    \item Allocate space untuk local variables (adjust SP)
    \item Save callee-saved registers (jika diperlukan)
\end{enumerate}

\subsubsection{Return Sequence (Epilogue)}

Instruksi yang dijalankan saat fungsi kembali:
\begin{enumerate}
    \item Place return value (di register atau stack)
    \item Restore callee-saved registers
    \item Restore SP (deallocate local variables)
    \item Restore FP dari control link
    \item Restore return address
    \item Return control ke caller (RET instruction)
\end{enumerate}

\subsection{Contoh Calling Sequence}

Mari kita lihat contoh calling sequence untuk program sederhana:

\begin{lstlisting}[language=C++, caption={Contoh program untuk analisis calling sequence}]
int add(int x, int y) {
    int sum = x + y;
    return sum;
}

int main() {
    int a = 5;
    int b = 10;
    int result = add(a, b);
    return 0;
}
\end{lstlisting}

Assembly code yang dihasilkan (simplified):

\begin{lstlisting}[language={[x86masm]Assembler},basicstyle=\ttfamily\footnotesize,breaklines=true,breakatwhitespace=false]
main:
    push rbp              ; Save caller's frame pointer
    mov rbp, rsp          ; Set new frame pointer
    sub rsp, 16           ; Allocate space for locals (a, b, result)
    
    mov [rbp-4], 5        ; a = 5
    mov [rbp-8], 10       ; b = 10
    
    ; Call add(a, b)
    mov eax, [rbp-8]      ; Load b
    push eax              ; Push parameter 2
    mov eax, [rbp-4]      ; Load a
    push eax              ; Push parameter 1
    call add              ; Call function
    add rsp, 8            ; Clean up parameters
    mov [rbp-12], eax     ; result = return value
    
    mov eax, 0            ; return 0
    mov rsp, rbp          ; Restore stack pointer
    pop rbp                ; Restore frame pointer
    ret                    ; Return

add:
    push rbp              ; Save caller's frame pointer
    mov rbp, rsp          ; Set new frame pointer
    sub rsp, 4            ; Allocate space for local (sum)
    
    mov eax, [rbp+8]      ; Load x (parameter 1)
    add eax, [rbp+12]     ; Add y (parameter 2)
    mov [rbp-4], eax      ; sum = x + y
    mov eax, [rbp-4]      ; Load sum for return
    
    mov rsp, rbp          ; Restore stack pointer
    pop rbp                ; Restore frame pointer
    ret                    ; Return
\end{lstlisting}
\section{Implementasi Runtime Stack Simulator}

Untuk memahami runtime stack dengan lebih baik, kita akan mengimplementasikan simulator sederhana dalam C++.

\subsection{Struktur Data Activation Record}

\begin{lstlisting}[language=C++, caption={Struktur data untuk activation record}]
#include <string>
#include <vector>
#include <unordered_map>
#include <iostream>

// Representasi activation record
struct ActivationRecord {
    std::string function_name;           // Nama fungsi
    void* return_address;                // Return address (simulated)
    ActivationRecord* control_link;      // Pointer ke caller's AR
    ActivationRecord* access_link;       // Pointer ke enclosing scope
    
    // Local variables dan parameters
    std::unordered_map<std::string, int> locals;
    std::unordered_map<std::string, int> parameters;
    
    // Return value
    int return_value;
    
    ActivationRecord(const std::string& name, 
                    ActivationRecord* caller = nullptr)
        : function_name(name), 
          return_address(nullptr),
          control_link(caller),
          access_link(nullptr),
          return_value(0) {}
};
\end{lstlisting}

\subsection{Stack Manager}

\begin{lstlisting}[language=C++, caption={Implementasi runtime stack manager}]
class RuntimeStack {
private:
    ActivationRecord* top;  // Top of stack (current activation)
    int frame_count;
    
public:
    RuntimeStack() : top(nullptr), frame_count(0) {}
    
    // Push activation record baru (function call)
    void pushFrame(const std::string& function_name) {
        ActivationRecord* new_frame = 
            new ActivationRecord(function_name, top);
        new_frame->access_link = top;  // Simplified: same as control link
        top = new_frame;
        frame_count++;
        
        std::cout << ">>> Called: " << function_name 
                  << " (Frame #" << frame_count << ")\n";
        printStack();
    }
    
    // Pop activation record (function return)
    void popFrame() {
        if (top == nullptr) {
            std::cerr << "Error: Cannot pop from empty stack!\n";
            return;
        }
        
        std::string func_name = top->function_name;
        int ret_val = top->return_value;
        
        ActivationRecord* old_top = top;
        top = top->control_link;
        delete old_top;
        frame_count--;
        
        std::cout << "<<< Returned from: " << func_name 
                  << " (return value: " << ret_val << ")\n";
        printStack();
    }
    
    // Get current activation record
    ActivationRecord* getCurrentFrame() {
        return top;
    }
    
    // Set local variable di current frame
    void setLocal(const std::string& name, int value) {
        if (top == nullptr) {
            std::cerr << "Error: No active frame!\n";
            return;
        }
        top->locals[name] = value;
        std::cout << "  Set local: " << name << " = " << value << "\n";
    }
    
    // Get local variable dari current frame atau enclosing scopes
    int getLocal(const std::string& name) {
        ActivationRecord* frame = top;
        while (frame != nullptr) {
            if (frame->locals.find(name) != frame->locals.end()) {
                return frame->locals[name];
            }
            frame = frame->access_link;  // Check enclosing scope
        }
        std::cerr << "Error: Variable '" << name 
                  << "' not found!\n";
        return 0;
    }
    
    // Set parameter
    void setParameter(const std::string& name, int value) {
        if (top == nullptr) {
            std::cerr << "Error: No active frame!\n";
            return;
        }
        top->parameters[name] = value;
        std::cout << "  Set parameter: " << name << " = " << value << "\n";
    }
    
    // Set return value
    void setReturnValue(int value) {
        if (top == nullptr) {
            std::cerr << "Error: No active frame!\n";
            return;
        }
        top->return_value = value;
        std::cout << "  Set return value: " << value << "\n";
    }
    
    // Print stack untuk debugging
    void printStack() {
        std::cout << "Stack (top to bottom):\n";
        ActivationRecord* frame = top;
        int level = 0;
        while (frame != nullptr) {
            std::cout << "  [" << level << "] " 
                      << frame->function_name << "\n";
            frame = frame->control_link;
            level++;
        }
        std::cout << "\n";
    }
    
    ~RuntimeStack() {
        while (top != nullptr) {
            popFrame();
        }
    }
};
\end{lstlisting}

\subsection{Contoh Penggunaan Simulator}

\begin{lstlisting}[language=C++, caption={Contoh penggunaan runtime stack simulator}]
int main() {
    RuntimeStack stack;
    
    // Simulasi: main() calls factorial(5)
    stack.pushFrame("main");
    stack.setLocal("n", 5);
    
    // Call factorial(5)
    stack.pushFrame("factorial");
    stack.setParameter("n", 5);
    
    // Recursive call: factorial(4)
    stack.pushFrame("factorial");
    stack.setParameter("n", 4);
    
    // Recursive call: factorial(3)
    stack.pushFrame("factorial");
    stack.setParameter("n", 3);
    
    // Base case: factorial(1) returns 1
    stack.setReturnValue(1);
    stack.popFrame();
    
    // factorial(3) = 3 * factorial(2) = 3 * 2 = 6
    // (simplified, actual would need more frames)
    stack.setReturnValue(6);
    stack.popFrame();
    
    stack.setReturnValue(24);
    stack.popFrame();
    
    stack.setReturnValue(120);
    stack.popFrame();
    
    // main returns
    stack.popFrame();
    
    return 0;
}
\end{lstlisting}
\section{Heap Memory Management}

Heap memory management adalah proses mengalokasikan dan membebaskan memory di heap secara dinamis. Terdapat dua pendekatan utama:

\subsection{Manual Memory Management}

Dalam manual memory management (seperti C/C++), programmer harus secara eksplisit:
\begin{itemize}
    \item Mengalokasikan memory: \texttt{malloc()}, \texttt{new}
    \item Membebaskan memory: \texttt{free()}, \texttt{delete}
\end{itemize}

Keuntungan:
\begin{itemize}
    \item Kontrol penuh atas memory
    \item Tidak ada overhead garbage collector
    \item Predictable performance
\end{itemize}

Kekurangan:
\begin{itemize}
    \item Rentan terhadap memory leaks
    \item Dangling pointers
    \item Double free errors
    \item Memory fragmentation
\end{itemize}

\subsection{Allocation Algorithms}

Heap manager menggunakan berbagai algoritma untuk mengalokasikan memory:

\subsubsection{First Fit}

Mencari block pertama yang cukup besar:
\begin{itemize}
    \item Cepat (tidak perlu mencari semua)
    \item Dapat menyebabkan fragmentation
\end{itemize}

\subsubsection{Best Fit}

Mencari block terkecil yang cukup besar:
\begin{itemize}
    \item Mengurangi wasted space
    \item Lebih lambat (harus mencari semua)
    \item Dapat menyebabkan banyak small fragments
\end{itemize}

\subsubsection{Worst Fit}

Mencari block terbesar:
\begin{itemize}
    \item Meninggalkan large free blocks
    \item Dapat mengurangi fragmentation kecil
\end{itemize}

\subsubsection{Buddy Allocation}

Membagi memory menjadi blocks dengan ukuran power of 2:
\begin{itemize}
    \item Mudah untuk merge adjacent blocks
    \item Dapat menyebabkan internal fragmentation
\end{itemize}

\subsection{Garbage Collection}

Garbage collection adalah automatic memory management yang membebaskan memory yang tidak lagi digunakan. Menurut sumber dari StudyLib:

\begin{quote}
``Runtime environment: stack, heap, activation records; garbage collection intro. Managing run-time structures (activation records, memory layout, symbol tables).''\cite{studylib2024obe}
\end{quote}

\subsubsection{Konsep Garbage Collection}

Garbage collector mengidentifikasi dan membebaskan memory yang tidak lagi dapat diakses (unreachable) dari program. Object dianggap garbage jika:
\begin{itemize}
    \item Tidak ada pointer/reference yang menunjuk ke object tersebut
    \item Tidak dapat diakses dari root set (stack, global variables, registers)
\end{itemize}

\subsubsection{Strategi Garbage Collection}

\paragraph{Mark and Sweep}
\begin{enumerate}
    \item \textbf{Mark Phase}: Traverse dari root set, mark semua reachable objects
    \item \textbf{Sweep Phase}: Scan semua objects, free yang tidak di-mark
\end{enumerate}

Keuntungan:
\begin{itemize}
    \item Dapat menangani cyclic references
    \item Tidak memerlukan memory compaction
\end{itemize}

Kekurangan:
\begin{itemize}
    \item Dapat menyebabkan fragmentation
    \item Stop-the-world pauses
\end{itemize}

\paragraph{Reference Counting}
Setiap object memiliki counter yang menghitung jumlah reference ke object tersebut. Ketika counter menjadi 0, object di-free.

Keuntungan:
\begin{itemize}
    \item Incremental (tidak perlu stop-the-world)
    \item Memory dibebaskan segera saat tidak digunakan
\end{itemize}

Kekurangan:
\begin{itemize}
    \item Tidak dapat menangani cyclic references
    \item Overhead untuk setiap assignment
\end{itemize}

\paragraph{Copying Collector (Generational GC)}
Memory dibagi menjadi young generation dan old generation. Young objects yang survive beberapa collections dipromote ke old generation.

Keuntungan:
\begin{itemize}
    \item Efisien untuk short-lived objects
    \item Automatic compaction
\end{itemize}

Kekurangan:
\begin{itemize}
    \item Memerlukan extra memory (copying)
    \item Overhead untuk promotion
\end{itemize}

\subsubsection{Implementasi Sederhana Mark and Sweep}

Berikut adalah implementasi sederhana mark-and-sweep garbage collector:

\begin{lstlisting}[language=C++, caption={Implementasi sederhana mark-and-sweep GC}]
#include <vector>
#include <unordered_set>

class GCObject {
public:
    bool marked;
    std::vector<GCObject*> references;
    
    GCObject() : marked(false) {}
    virtual ~GCObject() {}
    
    void addReference(GCObject* obj) {
        references.push_back(obj);
    }
};

class SimpleGC {
private:
    std::vector<GCObject*> heap;
    std::vector<GCObject*> roots;  // Root set (stack, globals)
    
public:
    // Allocate new object
    GCObject* allocate() {
        GCObject* obj = new GCObject();
        heap.push_back(obj);
        return obj;
    }
    
    // Add to root set
    void addRoot(GCObject* obj) {
        roots.push_back(obj);
    }
    
    // Mark phase: mark all reachable objects
    void mark() {
        std::vector<GCObject*> worklist = roots;
        
        while (!worklist.empty()) {
            GCObject* obj = worklist.back();
            worklist.pop_back();
            
            if (!obj->marked) {
                obj->marked = true;
                // Add all references to worklist
                for (GCObject* ref : obj->references) {
                    if (!ref->marked) {
                        worklist.push_back(ref);
                    }
                }
            }
        }
    }
    
    // Sweep phase: free unmarked objects
    void sweep() {
        auto it = heap.begin();
        while (it != heap.end()) {
            GCObject* obj = *it;
            if (!obj->marked) {
                delete obj;
                it = heap.erase(it);
            } else {
                obj->marked = false;  // Reset for next collection
                ++it;
            }
        }
    }
    
    // Run garbage collection
    void collect() {
        mark();
        sweep();
    }
    
    ~SimpleGC() {
        for (GCObject* obj : heap) {
            delete obj;
        }
    }
};
\end{lstlisting}
\section{Memory Layout untuk Program Contoh}

Mari kita analisis memory layout untuk program yang lebih kompleks:

\begin{lstlisting}[language=C++, caption={Program contoh untuk analisis memory layout}]
int global_var = 100;           // Static/Global
static int static_var = 200;    // Static

int* createArray(int size) {    // Function
    int* arr = new int[size];   // Heap allocation
    return arr;
}

int factorial(int n) {         // Function
    static int counter = 0;     // Static local
    counter++;
    
    if (n <= 1) return 1;
    int temp = n * factorial(n - 1);  // Stack: recursive
    return temp;
}

int main() {                    // Function
    int local_a = 10;           // Stack: local variable
    int local_b = 20;           // Stack: local variable
    
    int* heap_array = createArray(100);  // Heap allocation
    
    int result = factorial(5);  // Stack: recursive calls
    
    delete[] heap_array;        // Heap deallocation
    return 0;
}
\end{lstlisting}

Memory layout saat eksekusi:

\begin{verbatim}
High Address
    +---------------------+
    |  Stack (grows down) |
    |                     |
    |  [factorial(1) AR]  | <- Top of stack
    |    n = 1            |
    |    temp = ?          |
    |    return addr       |
    +----------------------+
    |  [factorial(2) AR]  |
    |    n = 2            |
    |    temp = ?          |
    +----------------------+
    |  ...                |
    +----------------------+
    |  [factorial(5) AR]  |
    |    n = 5            |
    |    temp = ?          |
    +----------------------+
    |  [main AR]          |
    |    local_a = 10     |
    |    local_b = 20     |
    |    heap_array = ptr |
    |    result = ?        |
    +----------------------+
    |                     |
    |  (free space)       |
    |                     |
    +----------------------+
    |  Heap (grows up)    |
    |  [heap_array[100]]  | <- 100 integers
    +----------------------+
    |  BSS                |
    |  (uninitialized)    |
    +----------------------+
    |  Data (initialized) |
    |    global_var = 100 |
    |    static_var = 200 |
    |    counter = ?       |
    +----------------------+
    |  Text/Code          |
    |  (instructions)     |
Low Address
\end{verbatim}
\section{Runtime Proyek Subset C}
\label{sec:runtime-proyek-subset-c}

Untuk compiler proyek subset C (Bab 1--12), asumsi runtime disederhanakan: program berupa barisan statement (tanpa fungsi tambahan di fase awal), sehingga tidak ada pemanggilan fungsi pengguna. Variabel \texttt{int}/\texttt{float} dialokasikan secara statis atau pada stack dengan layout tetap; code generator (Bab 14) memakai offset/alamat dari symbol table. Jika kelak proyek diperluas dengan fungsi, calling convention dapat mengikuti konvensi cdecl (stack-based arguments, caller-saved/callee-saved registers) agar konsisten dengan runtime environment yang dibahas di bab ini. Rincian layout activation record dan rencana calling convention untuk proyek dicatat dalam dokumentasi folder \texttt{proyek-compiler-subset-c/} agar Bab 14 memiliki acuan saat menghasilkan kode.

\section{Kesimpulan}

Dalam bab ini, kita telah mempelajari:

\begin{enumerate}
    \item Runtime environment adalah konteks eksekusi program yang mencakup memory organization, calling conventions, dan memory management
    
    \item Memory layout terdiri dari code segment, static/global data, stack, dan heap, masing-masing dengan karakteristik dan tujuan penggunaan yang berbeda
    
    \item Activation records (stack frames) menyimpan informasi tentang eksekusi fungsi, termasuk parameters, local variables, return address, dan links
    
    \item Stack-based allocation cocok untuk local variables dengan automatic management, sementara heap allocation diperlukan untuk dynamic data dengan flexible lifetime
    
    \item Garbage collection adalah teknik automatic memory management yang membebaskan unreachable objects, dengan berbagai strategi seperti mark-and-sweep, reference counting, dan generational GC
\end{enumerate}

Pemahaman tentang runtime environment dan memory management sangat penting untuk:
\begin{itemize}
    \item Merancang compiler yang efisien
    \item Memahami bagaimana program dieksekusi
    \item Mengoptimalkan penggunaan memory
    \item Mengimplementasikan fitur bahasa seperti recursion, closures, dan dynamic allocation
\end{itemize}
\section{Referensi dan Bahan Bacaan Lanjutan}

Untuk memperdalam pemahaman tentang runtime environment dan memory management, mahasiswa disarankan membaca:

\begin{itemize}
    \item \textbf{Dragon Book}: Aho, Lam, Sethi, \& Ullman (2006). \textit{Compilers: Principles, Techniques, and Tools} \cite{aho2006compilers} - Bab 7: Run-Time Environments
    
    \item \textbf{Engineering a Compiler}: Cooper \& Torczon (2011) \cite{cooper2011engineering} - Bab 6: The Procedure Abstraction
    
    \item \textbf{StudyLib - Outcomes-Based Education}: Materials tentang runtime environment dan activation records \cite{studylib2024obe}
    
    \item \textbf{UC San Diego CSE 231}: Course materials tentang compiler construction dan runtime organization \cite{ucsd2024compiler}
    
    \item \textbf{Northeastern University CS 4410}: Comprehensive compiler design course dengan coverage runtime issues \cite{neu2024compiler}
\end{itemize}

% Daftar pustaka (hanya muncul saat kompilasi standalone dan hanya jika ada citation)
\ifSubfilesClassLoaded{%
    \clearpage
    \printbibliography[title={Daftar Pustaka}]
    \end{refsection}
}{}

\end{document}

\cleardoublepage
% Bab 14: Code Generation untuk Target Architecture
% File ini dapat dikompilasi terpisah atau sebagai bagian dari main.tex

\chapter{Code Generation untuk Target Architecture}
\label{chap:code-generation}

\section{Tujuan Pembelajaran}

Setelah mempelajari bab ini, mahasiswa diharapkan mampu:
\begin{enumerate}
    \item Memahami proses code generation dari intermediate representation ke target code
    \item Menjelaskan konsep instruction selection dan implementasinya
    \item Mengimplementasikan register allocation sederhana (local allocation)
    \item Membuat code generator untuk operasi aritmatika dan assignment
    \item Memahami dan mengimplementasikan calling convention untuk function calls
    \item Menghasilkan assembly code yang valid untuk target architecture (x86 atau RISC-V)
\end{enumerate}

\section{Pengenalan Code Generation}

Code generation adalah fase terakhir dalam back-end kompilator yang bertanggung jawab untuk menghasilkan target code dari intermediate representation (IR) yang telah dioptimasi. Menurut sumber dari StudyLib:

\begin{quote}
``Code generation: instruction selection, machine model. Implement code generation for a target architecture, mapping intermediate code into efficient target code, managing run-time structures.''\cite{studylib2024obe}
\end{quote}

Code generator mengambil IR (biasanya dalam bentuk three-address code atau format serupa) dan menghasilkan kode assembly atau machine code yang dapat dieksekusi pada target architecture tertentu.

\subsection{Tugas Code Generator}

Code generator memiliki beberapa tugas utama:

\begin{enumerate}
    \item \textbf{Instruction Selection}: Memilih instruksi machine yang tepat untuk setiap operasi IR
    \item \textbf{Register Allocation}: Mengalokasikan register untuk variabel dan temporary values
    \item \textbf{Instruction Scheduling}: Mengatur urutan instruksi untuk optimasi pipeline
    \item \textbf{Address Assignment}: Mengalokasikan memory untuk variabel dan data structures
    \item \textbf{Code Emission}: Menghasilkan assembly atau machine code dalam format yang sesuai
\end{enumerate}

\subsection{Input dan Output Code Generator}

\textbf{Input Code Generator:}
\begin{itemize}
    \item Optimized Intermediate Representation (TAC, quadruples, atau format IR lainnya)
    \item Symbol table dengan informasi tipe dan alamat variabel
    \item Target architecture specification (instruction set, register set, addressing modes)
\end{itemize}

\textbf{Output Code Generator:}
\begin{itemize}
    \item Assembly code (untuk assembler) atau machine code langsung
    \item Relocation information (jika diperlukan)
    \item Debug information (optional)
\end{itemize}

\section{Target Architecture}

Target architecture adalah platform hardware yang menjadi tujuan kompilasi. Setiap architecture memiliki karakteristik yang berbeda yang mempengaruhi bagaimana code generator bekerja.

\subsection{Karakteristik Target Architecture}

Menurut dokumentasi LLVM\footnote{\url{https://llvm.org/docs/CodeGenerator.html}}, target architecture memiliki beberapa karakteristik penting:

\begin{enumerate}
    \item \textbf{Instruction Set Architecture (ISA)}: Kumpulan instruksi yang didukung oleh processor
    \begin{itemize}
        \item RISC (Reduced Instruction Set Computer): Instruksi sederhana, uniform, banyak register
        \item CISC (Complex Instruction Set Computer): Instruksi kompleks, berbagai format, addressing modes yang kaya
    \end{itemize}
    
    \item \textbf{Register Set}: Jumlah dan jenis register yang tersedia
    \begin{itemize}
        \item General-purpose registers
        \item Special-purpose registers (stack pointer, frame pointer, dll.)
        \item Floating-point registers
    \end{itemize}
    
    \item \textbf{Addressing Modes}: Cara mengakses operan (register, memory, immediate)
    \begin{itemize}
        \item Register addressing: \texttt{ADD R1, R2}
        \item Immediate addressing: \texttt{ADD R1, \#42}
        \item Memory addressing: \texttt{LOAD R1, [address]}
        \item Indexed addressing: \texttt{LOAD R1, [R2 + offset]}
    \end{itemize}
    
    \item \textbf{Memory Model}: Bagaimana memory diorganisir dan diakses
    \begin{itemize}
        \item Byte-addressable vs word-addressable
        \item Alignment requirements
        \item Endianness (little-endian vs big-endian)
    \end{itemize}
\end{enumerate}

\subsection{Contoh Target Architecture}

Dalam pembelajaran ini, kita akan fokus pada dua target architecture populer:

\textbf{1. x86-64 (AMD64)}
\begin{itemize}
    \item CISC architecture dengan instruksi kompleks
    \item 16 general-purpose registers (RAX, RBX, RCX, RDX, RSI, RDI, RBP, RSP, R8-R15)
    \item Berbagai addressing modes
    \item Variable-length instructions
    \item Little-endian
\end{itemize}

\textbf{2. RISC-V}
\begin{itemize}
    \item RISC architecture dengan instruksi sederhana
    \item 32 general-purpose registers (x0-x31)
    \item Fixed-length instructions (32-bit atau 16-bit untuk compressed)
    \item Simple addressing modes
    \item Little-endian atau big-endian (configurable)
    \item Open standard, populer untuk pembelajaran
\end{itemize}

Untuk pembelajaran, kita akan menggunakan subset sederhana dari RISC-V karena lebih mudah dipahami dan diimplementasikan.

\section{Instruction Selection}

Instruction selection adalah proses memilih instruksi machine yang tepat untuk mengimplementasikan setiap operasi dalam IR. Tujuannya adalah menghasilkan kode yang efisien dalam hal:
\begin{itemize}
    \item Execution time (runtime performance)
    \item Code size
    \item Energy consumption
\end{itemize}

\subsection{Metode Instruction Selection}

Menurut Wikipedia\footnote{\url{https://en.wikipedia.org/wiki/Instruction_selection}}, terdapat beberapa metode instruction selection:

\subsubsection{1. Simple Translation}

Metode paling sederhana: setiap operasi IR langsung dipetakan ke satu atau beberapa instruksi machine.

Contoh: TAC \texttt{t1 = t2 + t3} untuk RISC-V:
\begin{verbatim}
ADD t1, t2, t3
\end{verbatim}

Kelebihan: Implementasi mudah, cepat
Kekurangan: Tidak selalu optimal, tidak memanfaatkan instruksi kompleks

\subsubsection{2. Tree Pattern Matching}

Menggunakan expression tree dan mencocokkan subtree dengan pattern instruksi machine.

Contoh: Ekspresi \texttt{a + b * c} dapat dicocokkan dengan pattern:
\begin{itemize}
    \item Pattern 1: MUL + ADD (dua instruksi terpisah)
    \item Pattern 2: FMA (fused multiply-add, satu instruksi jika tersedia)
\end{itemize}

\subsubsection{3. Dynamic Programming}

Menggunakan dynamic programming untuk menemukan sequence instruksi yang optimal dengan mempertimbangkan cost.

\subsection{Contoh Instruction Selection}

Mari kita lihat contoh instruction selection untuk operasi sederhana pada RISC-V:

\textbf{Contoh 1: Assignment}
\begin{verbatim}
TAC: x = y
RISC-V: MV x, y    (atau ADD x, y, x0 untuk register zero)
\end{verbatim}

\textbf{Contoh 2: Arithmetic Operations}
\begin{verbatim}
TAC: t1 = a + b
RISC-V: ADD t1, a, b

TAC: t2 = c * d
RISC-V: MUL t2, c, d

TAC: t3 = t1 - t2
RISC-V: SUB t3, t1, t2
\end{verbatim}

\textbf{Contoh 3: Load/Store}
\begin{verbatim}
TAC: x = mem[addr]
RISC-V: LW x, 0(addr)    (load word)

TAC: mem[addr] = x
RISC-V: SW x, 0(addr)    (store word)
\end{verbatim}

\textbf{Contoh 4: Constant Loading}
\begin{verbatim}
TAC: x = 42
RISC-V: LI x, 42         (load immediate, pseudo-instruction)
       atau
       ADDI x, x0, 42   (add immediate dengan zero register)
\end{verbatim}

\section{Register Allocation}

Register allocation adalah proses menentukan variabel dan temporary values mana yang disimpan di register (fast storage) dan mana yang di-spill ke memory. Karena jumlah register fisik terbatas, ini adalah optimasi yang constrained.

\subsection{Mengapa Register Allocation Penting?}

Menurut Wikipedia\footnote{\url{https://en.wikipedia.org/wiki/Register_allocation}}:

\begin{itemize}
    \item Register jauh lebih cepat daripada memory access
    \item Jumlah register fisik terbatas (biasanya 16-32 register)
    \item Program mungkin memiliki lebih banyak variabel aktif daripada jumlah register
    \item Register allocation yang baik dapat meningkatkan performa secara signifikan
\end{itemize}

\subsection{Dua Fase Register Allocation}

\subsubsection{1. Allocation Phase}

Memutuskan \textit{mana} nilai yang harus disimpan di register pada setiap program point. Ini melibatkan:
\begin{itemize}
    \item Live range analysis: Kapan variabel hidup (live) dan kapan mati (dead)
    \item Interference graph: Grafik yang menunjukkan variabel mana yang tidak bisa menggunakan register yang sama secara bersamaan
\end{itemize}

\subsubsection{2. Assignment Phase}

Memetakan nilai yang dialokasikan ke register fisik spesifik. Jika lebih banyak nilai yang perlu register daripada register yang tersedia, beberapa nilai harus di-spill ke memory.

\subsection{Local Register Allocation}

Local register allocation bekerja dalam satu basic block (satu entry, satu exit, tidak ada branching). Ini lebih sederhana karena control flow linear.

\subsubsection{Algoritma Simple Local Allocation}

Algoritma sederhana untuk local allocation:

\begin{enumerate}
    \item Scan basic block dari awal hingga akhir
    \item Untuk setiap instruksi:
    \begin{itemize}
        \item Jika operan tidak di register, load dari memory
        \item Eksekusi operasi menggunakan register
        \item Jika hasil perlu disimpan dan register penuh, spill register yang paling lama tidak digunakan
    \end{itemize}
    \item Di akhir block, store semua register yang modified ke memory
\end{enumerate}

\subsubsection{Contoh Local Allocation}

Misalkan kita memiliki basic block dengan TAC berikut:
\begin{verbatim}
t1 = a + b
t2 = t1 * c
d = t2
\end{verbatim}

Dengan 3 register tersedia (R1, R2, R3), allocation bisa seperti ini:
\begin{verbatim}
LOAD R1, a        ; Load a ke R1
LOAD R2, b        ; Load b ke R2
ADD R1, R1, R2    ; R1 = a + b (t1)
LOAD R2, c        ; Load c ke R2 (b tidak lagi diperlukan)
MUL R1, R1, R2    ; R1 = t1 * c (t2)
STORE R1, d       ; Store hasil ke d
\end{verbatim}

\subsection{Global Register Allocation}

Global register allocation bekerja lintas basic blocks atau seluruh function. Ini lebih kompleks karena perlu mempertimbangkan control flow.

Metode populer:
\begin{itemize}
    \item \textbf{Graph Coloring}: Membangun interference graph dan mewarnainya dengan k warna (k = jumlah register)
    \item \textbf{Linear Scan}: Lebih cepat, menghitung live intervals dan assign register dalam satu pass
\end{itemize}

Untuk pembelajaran, kita akan fokus pada local allocation yang lebih sederhana.

\section{Code Generation untuk Operasi Aritmatika}

Mari kita implementasikan code generator untuk operasi aritmatika dasar. Kita akan menggunakan RISC-V sebagai target architecture.

\subsection{Struktur Code Generator Sederhana}

Code generator sederhana dapat diimplementasikan sebagai visitor pattern yang traverse AST atau TAC dan menghasilkan assembly code.

\subsubsection{Contoh Implementasi dalam C++}

Berikut adalah struktur dasar code generator:

\begin{lstlisting}[language=C++, caption={Struktur dasar Code Generator}]
class CodeGenerator {
private:
    std::vector<std::string> assemblyCode;
    int tempCounter;
    std::map<std::string, std::string> varToReg;
    std::set<std::string> availableRegs;
    
public:
    CodeGenerator() : tempCounter(0) {
        // Inisialisasi register yang tersedia
        for (int i = 1; i <= 8; i++) {
            availableRegs.insert("t" + std::to_string(i));
        }
    }
    
    std::string getRegister(const std::string& var) {
        // Alokasi register untuk variabel
        if (varToReg.find(var) != varToReg.end()) {
            return varToReg[var];
        }
        
        if (!availableRegs.empty()) {
            std::string reg = *availableRegs.begin();
            availableRegs.erase(reg);
            varToReg[var] = reg;
            return reg;
        }
        
        // Spill ke memory jika register penuh
        return spillRegister(var);
    }
    
    void generateAdd(const std::string& result, 
                     const std::string& op1, 
                     const std::string& op2) {
        std::string reg1 = getRegister(op1);
        std::string reg2 = getRegister(op2);
        std::string regResult = getRegister(result);
        
        assemblyCode.push_back("ADD " + regResult + ", " + 
                              reg1 + ", " + reg2);
    }
    
    void generateMul(const std::string& result,
                     const std::string& op1,
                     const std::string& op2) {
        std::string reg1 = getRegister(op1);
        std::string reg2 = getRegister(op2);
        std::string regResult = getRegister(result);
        
        assemblyCode.push_back("MUL " + regResult + ", " +
                              reg1 + ", " + reg2);
    }
    
    void generateLoad(const std::string& reg, 
                      const std::string& var) {
        assemblyCode.push_back("LW " + reg + ", " + var);
    }
    
    void generateStore(const std::string& var,
                       const std::string& reg) {
        assemblyCode.push_back("SW " + reg + ", " + var);
    }
    
    std::vector<std::string> getAssembly() {
        return assemblyCode;
    }
};
\end{lstlisting}

\subsection{Generating Code untuk Ekspresi Kompleks}

Untuk ekspresi kompleks seperti \texttt{a + b * c}, kita perlu mempertimbangkan precedence:

\begin{verbatim}
TAC untuk: result = a + b * c

t1 = b * c
t2 = a + t1
result = t2

Generated RISC-V code:
LW t1, a          ; Load a
LW t2, b          ; Load b
LW t3, c          ; Load c
MUL t4, t2, t3    ; t4 = b * c
ADD t5, t1, t4    ; t5 = a + t4
SW t5, result     ; Store result
\end{verbatim}

\section{Handling Function Calls dan Calling Convention}

Function calls memerlukan koordinasi antara caller dan callee untuk:
\begin{itemize}
    \item Passing arguments
    \item Saving/restoring registers
    \item Managing stack frame
    \item Returning values
\end{itemize}

Calling convention mendefinisikan aturan ini.

\subsection{RISC-V Calling Convention}

RISC-V memiliki calling convention yang didefinisikan dalam ABI (Application Binary Interface):

\subsubsection{Register Usage}

\begin{itemize}
    \item \textbf{Argument Registers}: a0-a7 (x10-x17) untuk 8 argument pertama
    \item \textbf{Return Register}: a0 (x10) untuk return value
    \item \textbf{Caller-saved Registers}: t0-t6 (x5-x7, x28-x31) - caller harus save
    \item \textbf{Callee-saved Registers}: s0-s11 (x8-x9, x18-x27) - callee harus save
    \item \textbf{Stack Pointer}: sp (x2)
    \item \textbf{Frame Pointer}: fp/s0 (x8)
\end{itemize}

\subsubsection{Function Call Sequence}

\textbf{Caller Side:}
\begin{verbatim}
# Save caller-saved registers jika diperlukan
# Pass arguments ke a0-a7
# Call function
JAL ra, function_name
# Return value di a0
# Restore caller-saved registers
\end{verbatim}

\textbf{Callee Side (Function Prologue):}
\begin{verbatim}
function_name:
    # Save return address dan frame pointer
    ADDI sp, sp, -frame_size
    SW ra, frame_size-4(sp)
    SW fp, frame_size-8(sp)
    ADDI fp, sp, frame_size
    
    # Save callee-saved registers yang digunakan
    # Allocate space untuk local variables
\end{verbatim}

\textbf{Callee Side (Function Epilogue):}
\begin{verbatim}
    # Restore callee-saved registers
    # Put return value di a0
    # Restore frame pointer dan return address
    LW fp, frame_size-8(sp)
    LW ra, frame_size-4(sp)
    ADDI sp, sp, frame_size
    RET  # atau JALR x0, 0(ra)
\end{verbatim}

\subsection{Contoh Function Call}

Mari kita lihat contoh function call untuk \texttt{int add(int a, int b)}:

\textbf{Caller Code:}
\begin{verbatim}
# Prepare arguments
LI a0, 10        # First argument (a = 10)
LI a1, 20        # Second argument (b = 20)

# Call function
JAL ra, add

# Result is in a0
# Use result...
\end{verbatim}

\textbf{Callee Code (add function):}
\begin{verbatim}
add:
    # Function prologue
    ADDI sp, sp, -16    # Allocate stack frame
    SW ra, 12(sp)       # Save return address
    SW fp, 8(sp)        # Save frame pointer
    ADDI fp, sp, 16     # Set frame pointer
    
    # Function body
    ADD a0, a0, a1      # a0 = a0 + a1 (result)
    
    # Function epilogue
    LW fp, 8(sp)        # Restore frame pointer
    LW ra, 12(sp)       # Restore return address
    ADDI sp, sp, 16     # Deallocate stack frame
    RET                 # Return
\end{verbatim}

\section{Implementasi Code Generator Lengkap}

Mari kita buat implementasi code generator yang lebih lengkap yang dapat menangani berbagai operasi.

\subsection{Code Generator untuk TAC}

Berikut adalah contoh code generator yang mengambil TAC dan menghasilkan RISC-V assembly:

\begin{lstlisting}[language=C++, caption={Code Generator untuk TAC ke RISC-V}]
class TACCodeGenerator {
private:
    std::vector<std::string> assembly;
    int labelCounter;
    std::map<std::string, int> varOffset;  // Offset di stack frame
    int stackOffset;
    
public:
    TACCodeGenerator() : labelCounter(0), stackOffset(0) {}
    
    void generateTAC(const TACInstruction& tac) {
        switch (tac.op) {
            case TAC_OP::ADD:
                generateAdd(tac.result, tac.arg1, tac.arg2);
                break;
            case TAC_OP::SUB:
                generateSub(tac.result, tac.arg1, tac.arg2);
                break;
            case TAC_OP::MUL:
                generateMul(tac.result, tac.arg1, tac.arg2);
                break;
            case TAC_OP::DIV:
                generateDiv(tac.result, tac.arg1, tac.arg2);
                break;
            case TAC_OP::ASSIGN:
                generateAssign(tac.result, tac.arg1);
                break;
            case TAC_OP::LOAD:
                generateLoad(tac.result, tac.arg1);
                break;
            case TAC_OP::STORE:
                generateStore(tac.arg1, tac.result);
                break;
            // ... operasi lainnya
        }
    }
    
    void generateAdd(const std::string& result,
                     const std::string& arg1,
                     const std::string& arg2) {
        std::string reg1 = loadToRegister(arg1);
        std::string reg2 = loadToRegister(arg2);
        std::string regResult = allocateRegister(result);
        
        assembly.push_back("ADD " + regResult + ", " + 
                          reg1 + ", " + reg2);
        
        releaseRegister(reg1);
        releaseRegister(reg2);
    }
    
    // ... implementasi operasi lainnya
};
\end{lstlisting}

\section{Testing dan Validasi}

Setelah code generator menghasilkan assembly, kita perlu:
\begin{enumerate}
    \item \textbf{Assemble}: Mengubah assembly menjadi object file
    \item \textbf{Link}: Menyatukan object files menjadi executable
    \item \textbf{Run}: Mengeksekusi program dan memverifikasi hasilnya
\end{enumerate}

\subsection{Workflow Lengkap}

\begin{verbatim}
Source Code (C/C++)
    ↓
[Compiler Front-end]
    ↓
TAC / IR
    ↓
[Code Generator] → Assembly Code (.s)
    ↓
[Assembler] → Object File (.o)
    ↓
[Linker] → Executable
    ↓
[Run] → Verify Output
\end{verbatim}

\subsection{Contoh Testing}

Misalkan kita memiliki program sederhana:
\begin{verbatim}
int main() {
    int a = 10;
    int b = 20;
    int c = a + b;
    return c;
}
\end{verbatim}

TAC yang dihasilkan:
\begin{verbatim}
t1 = 10
a = t1
t2 = 20
b = t2
t3 = a + b
c = t3
return c
\end{verbatim}

Assembly yang dihasilkan (RISC-V):
\begin{verbatim}
main:
    ADDI sp, sp, -16
    SW ra, 12(sp)
    SW fp, 8(sp)
    ADDI fp, sp, 16
    
    # a = 10
    LI t0, 10
    SW t0, -4(fp)    # Store a di stack
    
    # b = 20
    LI t0, 20
    SW t0, -8(fp)    # Store b di stack
    
    # c = a + b
    LW t1, -4(fp)    # Load a
    LW t2, -8(fp)    # Load b
    ADD t0, t1, t2
    SW t0, -12(fp)   # Store c
    
    # return c
    LW a0, -12(fp)   # Return value
    
    LW fp, 8(sp)
    LW ra, 12(sp)
    ADDI sp, sp, 16
    RET
\end{verbatim}

\section{Kesimpulan}

Dalam bab ini, kita telah mempelajari:

\begin{enumerate}
    \item Code generation adalah fase terakhir yang mengubah IR menjadi target code
    \item Instruction selection memilih instruksi machine yang tepat untuk setiap operasi IR
    \item Register allocation menentukan variabel mana yang disimpan di register vs memory
    \item Local register allocation lebih sederhana dan cocok untuk pembelajaran awal
    \item Calling convention mengatur bagaimana function calls dilakukan
    \item Code generator harus menghasilkan assembly yang valid dan dapat di-assemble, link, dan run
\end{enumerate}

Implementasi code generator yang baik memerlukan pemahaman mendalam tentang target architecture dan trade-off antara code size, execution time, dan register pressure.

\section{Latihan}

\begin{enumerate}
    \item Implementasikan code generator sederhana untuk operasi aritmatika dasar (+, -, *, /) yang menghasilkan RISC-V assembly.
    
    \item Buatlah code generator untuk assignment statement. Test dengan berbagai skenario:
    \begin{itemize}
        \item Assignment dari constant ke variabel
        \item Assignment dari variabel ke variabel
        \item Assignment dari ekspresi ke variabel
    \end{itemize}
    
    \item Implementasikan local register allocation dengan algoritma sederhana. Test dengan basic block yang memiliki lebih banyak variabel aktif daripada jumlah register yang tersedia.
    
    \item Buatlah code generator untuk function call sederhana dengan 2-3 parameter. Implementasikan calling convention RISC-V.
    
    \item Test workflow lengkap: compile → assemble → link → run untuk program sederhana yang menghitung \texttt{a * b + c}.
    
    \item Bandingkan code yang dihasilkan untuk ekspresi \texttt{a + b * c} dengan dan tanpa optimasi register allocation. Hitung jumlah instruksi dan memory access.
    
    \item Implementasikan code generator untuk conditional statement (if-else) dengan branch instructions.
    
    \item Buatlah code generator untuk loop (for/while) dengan label dan jump instructions.
\end{enumerate}

\section{Referensi dan Bahan Bacaan Lanjutan}

Untuk memperdalam pemahaman tentang code generation, mahasiswa disarankan membaca:

\begin{itemize}
    \item \textbf{Dragon Book}: Aho, Lam, Sethi, \& Ullman (2006). \textit{Compilers: Principles, Techniques, and Tools} \cite{aho2006compilers} - Bab 8: Code Generation
    
    \item \textbf{Engineering a Compiler}: Cooper \& Torczon (2011) \cite{cooper2011engineering} - Bab 7-9: Code Shape, Introduction to Optimization, Scalar Optimizations
    
    \item \textbf{RISC-V Instruction Set Manual}: \url{https://riscv.org/technical/specifications/} - Dokumentasi lengkap instruksi RISC-V
    
    \item \textbf{LLVM Code Generator Documentation}: \url{https://llvm.org/docs/CodeGenerator.html} - Dokumentasi code generator LLVM
    
    \item \textbf{StudyLib - Outcomes-Based Education}: Materials tentang code generation dan runtime structures \cite{studylib2024obe}
    
    \item \textbf{Wikipedia - Register Allocation}: \url{https://en.wikipedia.org/wiki/Register_allocation} - Artikel tentang teknik register allocation
    
    \item \textbf{Wikipedia - Instruction Selection}: \url{https://en.wikipedia.org/wiki/Instruction_selection} - Artikel tentang instruction selection
\end{itemize}

\cleardoublepage
\documentclass[../main.tex]{subfiles}
\begin{document}

\chapter{Optimasi Kompilator Dasar}
\label{chap:optimization}

\section{Tujuan Pembelajaran}

Setelah mempelajari bab ini, mahasiswa diharapkan mampu:
\begin{enumerate}
    \item Memahami konsep optimasi kompilator dan tujuannya
    \item Menjelaskan dan mengidentifikasi basic blocks dalam intermediate code
    \item Mengimplementasikan optimasi lokal: constant folding dan constant propagation
    \item Mengimplementasikan dead code elimination
    \item Memahami dasar-dasar data-flow analysis untuk optimasi global
    \item Mengevaluasi efektivitas optimasi dengan membandingkan before/after
    \item Membedakan machine-independent dan machine-specific optimizations
\end{enumerate}
\section{Pendahuluan}

Dalam proyek compiler subset C, optimasi diterapkan pada IR (Bab 12) atau pada output code generation (Bab 14): basic block, constant folding, constant propagation, dead code elimination. Konteks ``compiler subset C kita'' memastikan bahwa optimasi konsisten dengan AST, symbol table, dan IR proyek.

Optimasi kompilator adalah proses transformasi kode intermediate untuk meningkatkan kualitas kode yang dihasilkan tanpa mengubah semantik program. Menurut sumber dari Scribd OBE CSE Document:

\begin{quote}
``Perform machine-independent optimizations (basic block optimizations, data-flow analysis). Local and global optimization; data-flow analysis.''\cite{aho2006compilers}
\end{quote}

Tujuan optimasi kompilator meliputi:
\begin{itemize}
    \item \textbf{Meningkatkan Performa}: Mengurangi waktu eksekusi program
    \item \textbf{Mengurangi Ukuran Kode}: Menghasilkan executable yang lebih kecil
    \item \textbf{Mengurangi Konsumsi Memory}: Mengoptimasi penggunaan memory
    \item \textbf{Meningkatkan Efisiensi Energy}: Mengurangi konsumsi daya (penting untuk embedded systems)
\end{itemize}

Gambar \ref{fig:optimization-levels} menunjukkan level-level optimasi.

\begin{figure}[!htbp]
    \centering
    \adjustbox{max width=0.9\textwidth,center}{%
    \begin{tikzpicture}[
        level/.style={rectangle, draw=blue!50, fill=blue!10, text width=3cm, text centered, minimum height=0.7cm, rounded corners, font=\footnotesize, inner sep=4pt, align=center},
        arrow/.style={->, >=stealth, thick},
        node distance=0.6cm
    ]
    
    \node[level] (local) {Optimasi\\Lokal};
    \node[level, below=of local] (global) {Optimasi\\Global};
    \node[level, below=of global] (inter) {Optimasi\\Interprosedural};
    
    \draw[arrow] (local) -- (global);
    \draw[arrow] (global) -- (inter);
    
    \end{tikzpicture}%
    }
    \caption{Level-level optimasi kompilator}
    \label{fig:optimization-levels}
\end{figure}

Namun, optimasi harus dilakukan dengan hati-hati karena:
\begin{itemize}
    \item Optimasi yang terlalu agresif dapat meningkatkan waktu kompilasi
    \item Beberapa optimasi dapat membuat kode lebih sulit di-debug
    \item Optimasi yang salah dapat mengubah semantik program (bug)
\end{itemize}

\subsection{Prinsip Optimasi}

Menurut Dragon Book\cite{aho2006compilers}, optimasi harus mematuhi prinsip-prinsip berikut:

\begin{enumerate}
    \item \textbf{Correctness}: Optimasi tidak boleh mengubah semantik program
    \item \textbf{Benefit}: Optimasi harus memberikan manfaat yang signifikan
    \item \textbf{Speed}: Proses optimasi tidak boleh terlalu lambat
    \item \textbf{Simplicity}: Optimasi harus mudah diimplementasikan dan di-maintain
\end{enumerate}

\subsection{Level Optimasi}

Optimasi dapat dikategorikan berdasarkan scope-nya:

\begin{itemize}
    \item \textbf{Optimasi Lokal (Local Optimization)}: Optimasi dalam satu basic block
    \begin{itemize}
        \item Constant folding
        \item Constant propagation
        \item Algebraic simplification
        \item Strength reduction
    \end{itemize}
    
    \item \textbf{Optimasi Global (Global Optimization)}: Optimasi lintas basic blocks
    \begin{itemize}
        \item Common subexpression elimination
        \item Loop optimization
        \item Dead code elimination (global)
        \item Constant propagation (global)
    \end{itemize}
    
    \item \textbf{Optimasi Interprosedural (Interprocedural Optimization)}: Optimasi lintas fungsi/prosedur
    \begin{itemize}
        \item Inlining
        \item Interprocedural constant propagation
        \item Whole-program optimization
    \end{itemize}
\end{itemize}

Gambar \ref{fig:optimization-pipeline} menunjukkan pipeline optimasi dalam kompilator.

\begin{figure}[!htbp]
    \centering
    \adjustbox{max width=0.9\textwidth,center}{%
    \begin{tikzpicture}[
        box/.style={rectangle, draw=blue!50, fill=blue!10, text width=2.5cm, text centered, minimum height=0.7cm, rounded corners, font=\footnotesize, inner sep=4pt, align=center},
        arrow/.style={->, >=stealth, thick},
        node distance=1.2cm
    ]
    
    \node[box] (ir) {IR};
    \node[box, right=of ir] (local) {Optimasi\\Lokal};
    \node[box, right=of local] (global) {Optimasi\\Global};
    \node[box, below=of local] (opt-ir) {IR\\Teroptimasi};
    
    \draw[arrow] (ir) -- (local);
    \draw[arrow] (local) -- (global);
    \draw[arrow] (global) -- (opt-ir);
    
    \end{tikzpicture}%
    }
    \caption{Pipeline optimasi kompilator}
    \label{fig:optimization-pipeline}
\end{figure}
\section{Basic Blocks}

Basic block adalah fondasi untuk banyak optimasi kompilator. Menurut University of Michigan\footnote{\url{https://web.eecs.umich.edu/~weimerw/2015-4610/ca1/ca1.html}}, basic block didefinisikan sebagai:

\begin{quote}
``A basic block is a straight-line sequence of code with no jumps in except at the entry, and no jumps out except at the exit. Once execution enters it, all instructions execute sequentially.''
\end{quote}

\subsection{Karakteristik Basic Block}

Sebuah basic block memiliki karakteristik berikut:

\begin{enumerate}
    \item \textbf{Single Entry Point}: Hanya ada satu titik masuk (entry point)
    \item \textbf{Single Exit Point}: Hanya ada satu titik keluar (exit point)
    \item \textbf{Sequential Execution}: Semua instruksi dieksekusi secara berurutan tanpa branching
    \item \textbf{No Internal Control Flow}: Tidak ada jump, branch, atau call di tengah-tengah block
\end{enumerate}

Gambar \ref{fig:basic-block-example} menunjukkan contoh basic block.

\begin{figure}[H]
    \centering
    \adjustbox{max width=0.85\textwidth,center}{%
    \begin{tikzpicture}[
        inst/.style={rectangle, draw=blue!50, fill=blue!10, text width=4cm, minimum height=0.5cm, font=\tiny\ttfamily, align=left, inner sep=4pt, rounded corners},
        node distance=0.2cm
    ]
    
    \node[inst] (i1) {t1 = a + b};
    \node[inst, below=of i1] (i2) {t2 = t1 * c};
    \node[inst, below=of i2] (i3) {x = t2};
    
    \node[left=0.3cm of i1, font=\tiny] {Entry};
    \node[left=0.3cm of i3, font=\tiny] {Exit};
    
    \end{tikzpicture}%
    }
    \caption{Contoh basic block}
    \label{fig:basic-block-example}
\end{figure}

\subsection{Identifikasi Basic Blocks}

Algoritma untuk mengidentifikasi basic blocks dalam intermediate code:

\begin{enumerate}
    \item \textbf{Leader Identification}: Tentukan leader (instruksi pertama dalam basic block)
    \begin{itemize}
        \item Instruksi pertama dalam program adalah leader
        \item Instruksi yang merupakan target dari jump/branch adalah leader
        \item Instruksi setelah jump/branch/call adalah leader
    \end{itemize}
    
    \item \textbf{Block Construction}: Untuk setiap leader, buat basic block yang berisi:
    \begin{itemize}
        \item Leader instruction
        \item Semua instruksi berikutnya hingga menemukan leader berikutnya atau instruksi control flow
    \end{itemize}
\end{enumerate}

\subsection{Contoh Identifikasi Basic Block}

Perhatikan contoh three-address code berikut:

\begin{verbatim}
L1: t1 = a + b
    t2 = c * d
    t3 = t1 + t2
    if t3 > 0 goto L2
    t4 = t1 - t2
    goto L3
L2: t5 = t1 * t2
L3: t6 = t5 + 1
    return t6
\end{verbatim}

Basic blocks yang diidentifikasi:

\textbf{Block 1 (L1):}
\begin{verbatim}
    t1 = a + b
    t2 = c * d
    t3 = t1 + t2
    if t3 > 0 goto L2
\end{verbatim}

\textbf{Block 2 (L2):}
\begin{verbatim}
    t5 = t1 * t2
\end{verbatim}

\textbf{Block 3 (setelah goto L3):}
\begin{verbatim}
    t4 = t1 - t2
    goto L3
\end{verbatim}

\textbf{Block 4 (L3):}
\begin{verbatim}
    t6 = t5 + 1
    return t6
\end{verbatim}

\subsection{Control Flow Graph (CFG)}

Control Flow Graph adalah representasi grafis dari alur kontrol program. Menurut Wikipedia\footnote{\url{https://en.wikipedia.org/wiki/Control-flow_graph}}:

\begin{quote}
``A control-flow graph (CFG) is a representation of a function where each node is a basic block, and edges represent possible flow of control from one block to another.''
\end{quote}

CFG membantu dalam:
\begin{itemize}
    \item Memahami struktur program
    \item Melakukan data-flow analysis
    \item Mengidentifikasi loop dan struktur kontrol lainnya
    \item Mengoptimasi lintas basic blocks
\end{itemize}
\section{Constant Folding}

Constant folding adalah optimasi yang mengganti ekspresi yang hanya melibatkan konstanta dengan hasil komputasinya pada waktu kompilasi. Menurut GeeksforGeeks\footnote{\url{https://www.geeksforgeeks.org/compiler-design/constant-folding/}}:

\begin{quote}
``Constant folding replaces expressions involving only constants (literals) with their computed result at compile time, rather than at runtime. Example: turning `5 + 7 * 2` into `19` in the generated code.''
\end{quote}

\subsection{Contoh Constant Folding}

\textbf{Before optimization:}
\begin{verbatim}
t1 = 5 + 7
t2 = t1 * 2
t3 = 10 / 2
x = t2 + t3
\end{verbatim}

\textbf{After constant folding:}
\begin{verbatim}
t1 = 12        // 5 + 7 = 12
t2 = 24        // 12 * 2 = 24
t3 = 5         // 10 / 2 = 5
x = 29         // 24 + 5 = 29
\end{verbatim}

Atau bahkan lebih optimal:
\begin{verbatim}
x = 29         // Semua konstanta di-fold menjadi satu nilai
\end{verbatim}

\subsection{Implementasi Constant Folding}

Algoritma constant folding untuk three-address code:

\begin{enumerate}
    \item Untuk setiap instruksi dalam basic block:
    \begin{itemize}
        \item Jika kedua operan adalah konstanta, evaluasi ekspresi
        \item Ganti instruksi dengan assignment konstanta hasil
    \end{itemize}
    
    \item Ulangi hingga tidak ada lagi perubahan (iterasi mungkin diperlukan jika ada dependensi)
\end{enumerate}

\subsection{Contoh Implementasi dalam C++}

Berikut adalah contoh sederhana implementasi constant folding:

\begin{lstlisting}[language=C++, caption=Contoh implementasi constant folding]
struct Instruction {
    string op;      // operator: +, -, *, /, =
    string result;  // variabel hasil
    string arg1;    // operand pertama
    string arg2;    // operand kedua (optional)
};

bool isConstant(const string& var, 
                const map<string, int>& constants) {
    return constants.find(var) != constants.end();
}

int evaluateConstant(int val1, int val2, const string& op) {
    if (op == "+") return val1 + val2;
    if (op == "-") return val1 - val2;
    if (op == "*") return val1 * val2;
    if (op == "/") return val2 != 0 ? val1 / val2 : 0;
    return 0;
}

void constantFolding(vector<Instruction>& instructions) {
    map<string, int> constants;
    
    for (auto& inst : instructions) {
        if (inst.op == "=" && isNumeric(inst.arg1)) {
            // Assignment konstanta langsung
            constants[inst.result] = stoi(inst.arg1);
        } else if (inst.op != "=") {
            // Operasi biner
            if (isConstant(inst.arg1, constants) && 
                isConstant(inst.arg2, constants)) {
                int val1 = constants[inst.arg1];
                int val2 = constants[inst.arg2];
                int result = evaluateConstant(val1, val2, inst.op);
                constants[inst.result] = result;
                // Ganti instruksi dengan assignment konstanta
                inst.op = "=";
                inst.arg1 = to_string(result);
                inst.arg2 = "";
            }
        }
    }
}
\end{lstlisting}
\section{Constant Propagation}

Constant propagation adalah optimasi yang mengganti penggunaan variabel yang diketahui bernilai konstan dengan nilai konstanta tersebut. Menurut GeeksforGeeks\footnote{\url{https://www.geeksforgeeks.org/machine-independent-code-optimization-in-compiler-design/}}:

\begin{quote}
``Constant propagation replaces variables known to be constant with their constant values, then combined with constant folding to simplify more complex expressions.''
\end{quote}

\subsection{Contoh Constant Propagation}

\textbf{Before optimization:}
\begin{verbatim}
x = 10
y = 20
t1 = x + 5      // x adalah konstanta 10
t2 = y * 2      // y adalah konstanta 20
z = t1 + t2
\end{verbatim}

\textbf{After constant propagation:}
\begin{verbatim}
x = 10
y = 20
t1 = 10 + 5     // x diganti dengan 10
t2 = 20 * 2     // y diganti dengan 20
z = t1 + t2
\end{verbatim}

\textbf{After constant folding (kombinasi):}
\begin{verbatim}
x = 10
y = 20
t1 = 15         // 10 + 5 = 15
t2 = 40         // 20 * 2 = 40
z = 55          // 15 + 40 = 55
\end{verbatim}

\subsection{Local vs Global Constant Propagation}

\textbf{Local Constant Propagation:}
\begin{itemize}
    \item Hanya dalam satu basic block
    \item Lebih sederhana, tidak memerlukan data-flow analysis
    \item Dapat dilakukan bersamaan dengan constant folding
\end{itemize}

\textbf{Global Constant Propagation:}
\begin{itemize}
    \item Lintas basic blocks
    \item Memerlukan data-flow analysis (reaching definitions)
    \item Lebih kompleks tetapi lebih powerful
    \item Harus mempertimbangkan multiple paths dalam CFG
\end{itemize}

\subsection{Implementasi Local Constant Propagation}

Algoritma untuk local constant propagation:

\begin{enumerate}
    \item Scan basic block dari atas ke bawah
    \item Maintain map variabel → nilai konstanta
    \item Untuk setiap instruksi:
    \begin{itemize}
        \item Jika assignment konstanta: update map
        \item Jika penggunaan variabel: ganti dengan konstanta jika tersedia
        \item Jika assignment dari variabel non-konstanta: hapus dari map (variabel tidak lagi konstanta)
    \end{itemize}
\end{enumerate}
\section{Dead Code Elimination}

Dead code elimination adalah optimasi yang menghapus kode yang tidak memiliki efek pada perilaku program yang dapat diamati. Menurut GeeksforGeeks\footnote{\url{https://www.geeksforgeeks.org/dead-code-elimination/}}:

\begin{quote}
``Dead code elimination removes code that has no effect on the program's observable behavior. Two main kinds: unreachable code (code that can never be executed) and assignment to variables never used (where a variable's value is computed but never read before being overwritten).''
\end{quote}

\subsection{Jenis Dead Code}

\textbf{1. Unreachable Code}
Kode yang tidak pernah dapat dieksekusi karena tidak ada path yang mencapainya.

Contoh:
\begin{verbatim}
x = 10
return x
y = 20      // Dead code - tidak pernah dieksekusi
z = y + 5   // Dead code
\end{verbatim}

\textbf{2. Dead Assignments}
Assignment ke variabel yang nilainya tidak pernah digunakan sebelum di-overwrite.

Contoh:
\begin{verbatim}
x = 10
x = 20      // Assignment pertama adalah dead code
y = x       // Hanya nilai kedua yang digunakan
\end{verbatim}

\subsection{Unreachable Code Elimination}

Algoritma untuk menghapus unreachable code:

\begin{enumerate}
    \item Bangun Control Flow Graph (CFG)
    \item Lakukan traversal dari entry block (misalnya DFS atau BFS)
    \item Mark semua basic block yang dapat dicapai
    \item Hapus semua basic block yang tidak ter-mark
\end{enumerate}

\subsection{Dead Assignment Elimination}

Algoritma untuk menghapus dead assignments (menggunakan live-variable analysis):

\begin{enumerate}
    \item Lakukan live-variable analysis untuk menentukan variabel mana yang "live" di setiap titik
    \item Variabel dikatakan "live" jika nilainya mungkin digunakan di masa depan
    \item Untuk setiap assignment \texttt{x = ...}:
    \begin{itemize}
        \item Jika \texttt{x} tidak live setelah assignment, assignment tersebut adalah dead code
        \item Hapus assignment tersebut
    \end{itemize}
\end{enumerate}

\subsection{Contoh Dead Code Elimination}

\textbf{Before optimization:}
\begin{verbatim}
x = 10
y = 20
t1 = x + y
t2 = t1 * 2
x = t2
t3 = 5 + 3      // Dead: t3 tidak pernah digunakan
t4 = t3 - 2    // Dead: t4 tidak pernah digunakan
return x
\end{verbatim}

\textbf{After dead code elimination:}
\begin{verbatim}
x = 10
y = 20
t1 = x + y
t2 = t1 * 2
x = t2
return x
\end{verbatim}

\subsection{Implementasi Dead Code Elimination}

Berikut adalah contoh implementasi sederhana untuk dead assignment elimination:

\begin{lstlisting}[language=C++, caption=Contoh implementasi dead code elimination]
set<string> computeLiveVariables(const vector<Instruction>& insts) {
    set<string> live;
    
    // Scan dari bawah ke atas
    for (int i = insts.size() - 1; i >= 0; i--) {
        const auto& inst = insts[i];
        
        // Variabel yang digunakan adalah live
        if (!inst.arg1.empty() && !isConstant(inst.arg1)) {
            live.insert(inst.arg1);
        }
        if (!inst.arg2.empty() && !isConstant(inst.arg2)) {
            live.insert(inst.arg2);
        }
        
        // Variabel yang di-assign tidak lagi live setelah assignment
        // (kecuali jika digunakan di sisi kanan)
        if (live.find(inst.result) != live.end()) {
            live.erase(inst.result);
        }
    }
    
    return live;
}

vector<Instruction> eliminateDeadCode(
    const vector<Instruction>& instructions) {
    
    set<string> live = computeLiveVariables(instructions);
    vector<Instruction> optimized;
    
    for (const auto& inst : instructions) {
        // Skip jika assignment ke variabel yang tidak live
        if (inst.op == "=" && live.find(inst.result) == live.end()) {
            continue; // Dead assignment
        }
        
        optimized.push_back(inst);
        
        // Update live set
        if (!inst.arg1.empty()) live.insert(inst.arg1);
        if (!inst.arg2.empty()) live.insert(inst.arg2);
        live.erase(inst.result);
    }
    
    return optimized;
}
\end{lstlisting}
\section{Data-Flow Analysis Dasar}

Data-flow analysis adalah teknik untuk menghitung informasi tentang kemungkinan perilaku program. Menurut GeeksforGeeks\footnote{\url{https://www.geeksforgeeks.org/data-flow-analysis-compiler/}}:

\begin{quote}
``Data-flow analysis is a technique to compute information about possible program behaviors (how definitions, uses of variables, expressions, etc. propagate through the code). Usually done on a CFG.''
\end{quote}

Data-flow analysis adalah fondasi untuk optimasi global yang lebih advanced.

\subsection{Konsep Dasar Data-Flow Analysis}

Data-flow analysis bekerja dengan:

\begin{enumerate}
    \item \textbf{Domain}: Himpunan informasi yang ingin dihitung
    \begin{itemize}
        \item Live variables: set variabel yang live
        \item Reaching definitions: set definisi yang mencapai suatu titik
        \item Available expressions: set ekspresi yang sudah dihitung
    \end{itemize}
    
    \item \textbf{Transfer Function}: Bagaimana informasi berubah setelah eksekusi instruksi
    
    \item \textbf{Meet/Join Operation}: Bagaimana menggabungkan informasi dari multiple paths
    
    \item \textbf{Fixpoint Iteration}: Iterasi hingga mencapai fixpoint (tidak ada perubahan)
\end{enumerate}

\subsection{Live Variable Analysis}

Live variable analysis menentukan variabel mana yang "live" (nilainya mungkin digunakan) di setiap titik program.

\textbf{Definisi:}
\begin{itemize}
    \item Variabel \texttt{v} dikatakan \textbf{live} di titik \texttt{p} jika ada path dari \texttt{p} ke penggunaan \texttt{v} tanpa assignment ke \texttt{v} di antara keduanya
    \item Variabel \texttt{v} dikatakan \textbf{dead} jika tidak live
\end{itemize}

\textbf{Algoritma (Backward Analysis):}
\begin{enumerate}
    \item Inisialisasi: \texttt{LIVE[exit] = \{\}}
    \item Untuk setiap basic block \texttt{B} (dari exit ke entry):
    \begin{itemize}
        \item \texttt{LIVE[B] = UNION(LIVE[successors])}
        \item \texttt{LIVE[B] = LIVE[B] - DEF[B] + USE[B]}
        \item \texttt{DEF[B]}: variabel yang didefinisikan di B
        \item \texttt{USE[B]}: variabel yang digunakan di B
    \end{itemize}
    \item Ulangi hingga fixpoint
\end{enumerate}

\subsection{Reaching Definitions}

Reaching definitions analysis menentukan definisi variabel mana yang "mencapai" suatu titik program.

\textbf{Definisi:}
Definisi \texttt{d} dikatakan \textbf{reach} titik \texttt{p} jika ada path dari \texttt{d} ke \texttt{p} tanpa definisi lain untuk variabel yang sama.

\textbf{Kegunaan:}
\begin{itemize}
    \item Constant propagation (global)
    \item Deteksi penggunaan variabel sebelum inisialisasi
    \item Optimasi lainnya
\end{itemize}

\subsection{Available Expressions}

Available expressions analysis menentukan ekspresi mana yang sudah dihitung dan masih valid (operand-nya belum berubah).

\textbf{Kegunaan:}
\begin{itemize}
    \item Common subexpression elimination
    \item Optimasi lainnya
\end{itemize}
\section{Kombinasi Optimasi}

Dalam praktik, optimasi biasanya dilakukan dalam beberapa pass dan saling berinteraksi:

\subsection{Order of Optimization}

Urutan optimasi yang umum:

\begin{enumerate}
    \item \textbf{Constant Folding \& Propagation}: Simplifikasi ekspresi konstanta
    \item \textbf{Dead Code Elimination}: Hapus kode yang tidak digunakan
    \item \textbf{Common Subexpression Elimination}: Hapus komputasi duplikat
    \item \textbf{Loop Optimizations}: Optimasi khusus untuk loop
    \item \textbf{Register Allocation}: Alokasi register yang efisien
\end{enumerate}

\subsection{Iterative Optimization}

Optimizer biasanya menjalankan beberapa pass hingga tidak ada lagi perubahan:

\begin{verbatim}
do {
    changed = false
    changed |= constantFolding()
    changed |= constantPropagation()
    changed |= deadCodeElimination()
    // ... optimasi lainnya
} while (changed)
\end{verbatim}
\section{Evaluasi Efektivitas Optimasi}

Setelah mengimplementasikan optimasi, penting untuk mengevaluasi efektivitasnya.

\subsection{Metrics untuk Evaluasi}

\textbf{1. Code Size}
\begin{itemize}
    \item Ukuran executable sebelum dan sesudah optimasi
    \item Jumlah instruksi dalam intermediate code
    \item Ukuran object files
\end{itemize}

\textbf{2. Execution Time}
\begin{itemize}
    \item Waktu eksekusi program dengan benchmark
    \item Profiling untuk mengidentifikasi bottleneck
    \item Perbandingan before/after
\end{itemize}

\textbf{3. Memory Usage}
\begin{itemize}
    \item Peak memory consumption
    \item Stack usage
    \item Heap allocation patterns
\end{itemize}

\textbf{4. Compilation Time}
\begin{itemize}
    \item Waktu yang dibutuhkan untuk kompilasi
    \item Trade-off antara waktu kompilasi dan kualitas optimasi
\end{itemize}

\subsection{Benchmarking}

Langkah-langkah untuk benchmarking:

\begin{enumerate}
    \item \textbf{Prepare Test Cases}: Siapkan berbagai test case (small, medium, large programs)
    \item \textbf{Baseline Measurement}: Ukur metrik sebelum optimasi
    \item \textbf{Optimized Measurement}: Ukur metrik setelah optimasi
    \item \textbf{Compare Results}: Bandingkan dan hitung improvement percentage
    \item \textbf{Verify Correctness}: Pastikan program masih menghasilkan output yang benar
\end{enumerate}

\subsection{Contoh Evaluasi}

Berikut adalah contoh format laporan evaluasi:

\begin{table}[!htbp]
\centering
\begin{tabular}{|l|c|c|c|}
\hline
\textbf{Metric} & \textbf{Before} & \textbf{After} & \textbf{Improvement} \\
\hline
Code Size (bytes) & 1024 & 768 & 25\% reduction \\
\hline
Execution Time (ms) & 100 & 75 & 25\% faster \\
\hline
Instruction Count & 150 & 110 & 26.7\% reduction \\
\hline
Compilation Time (s) & 2.5 & 3.1 & 24\% slower \\
\hline
\end{tabular}
\caption{Contoh hasil evaluasi optimasi}
\label{tab:optimization-results}
\end{table}
\section{Implementasi Praktis}

Dalam bagian ini, kita akan melihat contoh implementasi optimizer sederhana yang menggabungkan beberapa optimasi dasar.

\subsection{Struktur Optimizer}

\begin{lstlisting}[language=C++, caption=Struktur dasar optimizer]
class Optimizer {
private:
    vector<BasicBlock> basicBlocks;
    ControlFlowGraph cfg;
    
public:
    // Identifikasi basic blocks
    void identifyBasicBlocks(const vector<Instruction>& insts);
    
    // Optimasi lokal dalam basic block
    void optimizeBasicBlock(BasicBlock& block);
    
    // Optimasi global lintas basic blocks
    void optimizeGlobal();
    
    // Kombinasi semua optimasi
    vector<Instruction> optimize(const vector<Instruction>& insts);
};

vector<Instruction> Optimizer::optimize(
    const vector<Instruction>& instructions) {
    
    // 1. Identifikasi basic blocks
    identifyBasicBlocks(instructions);
    
    // 2. Optimasi lokal untuk setiap basic block
    for (auto& block : basicBlocks) {
        optimizeBasicBlock(block);
    }
    
    // 3. Optimasi global
    optimizeGlobal();
    
    // 4. Reconstruct instructions dari basic blocks
    return reconstructInstructions();
}

void Optimizer::optimizeBasicBlock(BasicBlock& block) {
    bool changed = true;
    
    while (changed) {
        changed = false;
        
        // Constant folding
        changed |= constantFolding(block);
        
        // Constant propagation
        changed |= constantPropagation(block);
        
        // Dead code elimination
        changed |= deadCodeElimination(block);
    }
}
\end{lstlisting}
\section{Kesimpulan}

Dalam bab ini, kita telah mempelajari:

\begin{enumerate}
    \item Optimasi kompilator bertujuan meningkatkan kualitas kode tanpa mengubah semantik
    \item Basic blocks adalah unit fundamental untuk optimasi lokal
    \item Constant folding dan constant propagation adalah optimasi dasar yang efektif
    \item Dead code elimination menghapus kode yang tidak berguna
    \item Data-flow analysis adalah fondasi untuk optimasi global
    \item Evaluasi efektivitas optimasi penting untuk memastikan optimasi memberikan manfaat
\end{enumerate}

Optimasi kompilator adalah bidang yang luas dan kompleks. Bab ini memberikan dasar-dasar optimasi lokal. Untuk optimasi yang lebih advanced seperti loop optimization, interprocedural optimization, dan machine-specific optimization, diperlukan pemahaman yang lebih mendalam tentang data-flow analysis dan teknik optimasi lainnya. Optimasi proyek subset C diterapkan pada IR dan/atau kode yang dihasilkan; compiler lengkap (lexer, parser, AST, symbol table, type check, IR, codegen, optimasi) dipresentasikan di Bab 16.
\section{Referensi dan Bahan Bacaan Lanjutan}

Untuk memperdalam pemahaman tentang optimasi kompilator, mahasiswa disarankan membaca:

\begin{itemize}
    \item \textbf{Dragon Book}: Aho, Lam, Sethi, \& Ullman (2006). \textit{Compilers: Principles, Techniques, and Tools} \cite{aho2006compilers} - Bab 9: Machine-Independent Optimizations
    
    \item \textbf{Engineering a Compiler}: Cooper \& Torczon (2011) \cite{cooper2011engineering} - Bab 8: Introduction to Optimization, Bab 9: Data-Flow Analysis
    
    \item \textbf{GeeksforGeeks}: Tutorial tentang berbagai optimasi kompilator
    \begin{itemize}
        \item Constant Folding: \url{https://www.geeksforgeeks.org/compiler-design/constant-folding/}
        \item Dead Code Elimination: \url{https://www.geeksforgeeks.org/dead-code-elimination/}
        \item Data-Flow Analysis: \url{https://www.geeksforgeeks.org/data-flow-analysis-compiler/}
    \end{itemize}
    
    \item \textbf{University of Michigan}: Course materials tentang compiler optimization \footnote{\url{https://web.eecs.umich.edu/~weimerw/2015-4610/ca1/ca1.html}}
    
    \item \textbf{LLVM Documentation}: Advanced optimization techniques \footnote{\url{https://llvm.org/docs/Passes.html}}
\end{itemize}

\end{document}

\cleardoublepage
% Bab 16: Project Final Presentation dan Review
% File ini dapat dikompilasi terpisah atau sebagai bagian dari main.tex

\chapter{Project Final Presentation dan Review}
\label{chap:final-project}

\section{Tujuan Pembelajaran}

Setelah mempelajari bab ini, mahasiswa diharapkan mampu:
\begin{enumerate}
    \item Mempresentasikan project final (working compiler) dengan baik
    \item Mendemonstrasikan kemampuan compiler yang telah dibangun
    \item Mengevaluasi dan membandingkan tools kompilator (parser generators vs hand-written parsers)
    \item Menganalisis trade-off antara waktu kompilasi, kualitas kode, dan efisiensi runtime
    \item Melakukan refleksi pembelajaran dan evaluasi diri terhadap seluruh materi
    \item Menyusun dokumentasi proyek yang komprehensif
\end{enumerate}

\section{Overview Project Final}

Bab ini merupakan puncak dari pembelajaran mata kuliah Teknik Kompilasi. Setelah mempelajari semua fase kompilasi dari analisis leksikal hingga optimasi, mahasiswa diharapkan telah membangun sebuah compiler lengkap yang dapat mengkompilasi bahasa sederhana menjadi executable code.

Menurut sumber dari University of Oxford:

\begin{quote}
``Evaluate and compare compiler tools (like parser generators) and optimization approaches, analyze trade-offs between compilation time, code quality, and runtime efficiency.''\cite{oxford2024compilers}
\end{quote}

Project final ini tidak hanya menguji kemampuan teknis, tetapi juga kemampuan untuk:
\begin{itemize}
    \item Mengintegrasikan semua komponen yang telah dipelajari
    \item Membuat keputusan desain yang tepat
    \item Mengevaluasi tools dan teknik yang digunakan
    \item Berkomunikasi secara efektif tentang hasil kerja
\end{itemize}

\section{Persiapan Presentasi Project Final}

Presentasi project final adalah kesempatan untuk menunjukkan hasil kerja keras selama satu semester. Berikut adalah panduan untuk mempersiapkan presentasi yang efektif:

\subsection{Struktur Presentasi}

Presentasi sebaiknya mengikuti struktur berikut:

\textbf{1. Pendahuluan (5 menit)}
\begin{itemize}
    \item Perkenalan tim dan project
    \item Tujuan dan scope bahasa yang dikompilasi
    \item Overview arsitektur compiler
\end{itemize}

\textbf{2. Demo Compiler (10 menit)}
\begin{itemize}
    \item Live demonstration: compile dan run program contoh
    \item Menunjukkan berbagai fitur bahasa yang didukung
    \item Menampilkan error handling yang baik
\end{itemize}

\textbf{3. Arsitektur dan Implementasi (15 menit)}
\begin{itemize}
    \item Penjelasan setiap fase kompilasi
    \item Pilihan desain dan justifikasinya
    \item Tools dan teknik yang digunakan
    \item Tantangan yang dihadapi dan solusinya
\end{itemize}

\textbf{4. Evaluasi dan Analisis (10 menit)}
\begin{itemize}
    \item Perbandingan hand-written vs generator tools
    \item Trade-off analysis
    \item Benchmark hasil kompilasi
    \item Evaluasi kualitas kode yang dihasilkan
\end{itemize}

\textbf{5. Kesimpulan dan Refleksi (5 menit)}
\begin{itemize}
    \item Lesson learned
    \item Area untuk improvement
    \item Kesimpulan
\end{itemize}

\textbf{6. Q\&A (5 menit)}

\subsection{Tips Presentasi yang Efektif}

\begin{enumerate}
    \item \textbf{Persiapan Demo}: Pastikan demo berjalan lancar dengan test cases yang sudah dipersiapkan
    \item \textbf{Visual Aids}: Gunakan diagram arsitektur, flowchart, dan contoh kode yang jelas
    \item \textbf{Time Management}: Latih presentasi untuk memastikan sesuai waktu yang dialokasikan
    \item \textbf{Anticipate Questions}: Siapkan jawaban untuk pertanyaan umum tentang desain dan implementasi
    \item \textbf{Show Passion}: Tunjukkan antusiasme terhadap project yang telah dibangun
\end{enumerate}

\section{Demonstrasi Compiler}

Demo adalah bagian penting dari presentasi. Demo yang baik menunjukkan bahwa compiler benar-benar berfungsi dan dapat digunakan secara praktis.

\subsection{Preparing for Demo}

\textbf{1. Test Cases yang Komprehensif}
\begin{itemize}
    \item Program sederhana (hello world, arithmetic)
    \item Program dengan kontrol flow (if-else, loops)
    \item Program dengan fungsi dan scope
    \item Program dengan error (untuk menunjukkan error handling)
    \item Program yang lebih kompleks (menunjukkan kemampuan compiler)
\end{itemize}

\textbf{2. Environment Setup}
\begin{itemize}
    \item Pastikan semua dependencies terinstall
    \item Compile compiler terlebih dahulu
    \item Siapkan backup plan jika ada masalah teknis
    \item Test di environment yang sama dengan presentasi
\end{itemize}

\textbf{3. Demo Script}
\begin{itemize}
    \item Buat script demo yang terstruktur
    \item Siapkan penjelasan untuk setiap langkah
    \item Antisipasi kemungkinan error atau masalah
\end{itemize}

\subsection{Contoh Demo Flow}

Berikut adalah contoh alur demo yang efektif:

\begin{enumerate}
    \item \textbf{Hello World}: Menunjukkan compiler dapat menghasilkan executable sederhana
    \begin{verbatim}
    // hello.lang
    print("Hello, World!");
    \end{verbatim}
    
    \item \textbf{Arithmetic Operations}: Menunjukkan kemampuan menangani ekspresi
    \begin{verbatim}
    // calc.lang
    int x = 10;
    int y = 20;
    int result = x + y * 2;
    print(result);
    \end{verbatim}
    
    \item \textbf{Control Flow}: Menunjukkan if-else dan loops
    \begin{verbatim}
    // control.lang
    int i = 0;
    while (i < 10) {
        if (i % 2 == 0) {
            print(i);
        }
        i = i + 1;
    }
    \end{verbatim}
    
    \item \textbf{Error Handling}: Menunjukkan kualitas error messages
    \begin{verbatim}
    // error.lang
    int x = 10;
    int y = "string";  // Type error
    x = undefined_var; // Undefined variable
    \end{verbatim}
    
    \item \textbf{Complex Program}: Menunjukkan kemampuan compiler dengan program yang lebih kompleks
\end{enumerate}

\section{Review Materi: Fase-Fase Kompilasi}

Sebelum melakukan evaluasi tools dan teknik, penting untuk mereview kembali semua fase kompilasi yang telah dipelajari:

\subsection{Front-End Phases}

\textbf{1. Lexical Analysis}
\begin{itemize}
    \item Memecah source code menjadi tokens
    \item Implementasi: hand-written atau menggunakan Flex/re2c
    \item Output: stream of tokens
\end{itemize}

\textbf{2. Syntax Analysis}
\begin{itemize}
    \item Memverifikasi struktur grammar
    \item Implementasi: recursive descent, LR parser, atau Bison
    \item Output: Abstract Syntax Tree (AST)
\end{itemize}

\textbf{3. Semantic Analysis}
\begin{itemize}
    \item Type checking, scope resolution, name resolution
    \item Implementasi: tree traversal dengan symbol table
    \item Output: Annotated AST dengan type information
\end{itemize}

\subsection{Back-End Phases}

\textbf{4. Intermediate Code Generation}
\begin{itemize}
    \item Mengkonversi AST menjadi IR (three-address code)
    \item Output: Intermediate Representation
\end{itemize}

\textbf{5. Code Optimization}
\begin{itemize}
    \item Optimasi lokal dan global
    \item Output: Optimized IR
\end{itemize}

\textbf{6. Code Generation}
\begin{itemize}
    \item Mengkonversi IR menjadi target code (assembly)
    \item Output: Assembly code atau machine code
\end{itemize}

\subsection{Integration Points}

Setiap fase harus terintegrasi dengan baik:
\begin{itemize}
    \item Lexer → Parser: Token stream
    \item Parser → Semantic Analyzer: AST
    \item Semantic Analyzer → IR Generator: Annotated AST
    \item IR Generator → Optimizer: IR
    \item Optimizer → Code Generator: Optimized IR
    \item Code Generator → Assembler: Assembly code
\end{itemize}

\section{Evaluasi Tools: Hand-Written vs Generator-Based}

Salah satu aspek penting dalam project final adalah evaluasi tools yang digunakan. Mahasiswa perlu membandingkan pendekatan hand-written dengan generator-based tools.

\subsection{Perbandingan Lexer: Hand-Written vs Flex/re2c}

\textbf{Hand-Written Lexer}

\textbf{Keuntungan:}
\begin{itemize}
    \item Kontrol penuh atas implementasi
    \item Error messages yang lebih informatif dan customizable
    \item Tidak ada dependency eksternal
    \item Dapat dioptimasi untuk kasus khusus
    \item Lebih mudah di-debug karena kode lebih readable
\end{itemize}

\textbf{Kekurangan:}
\begin{itemize}
    \item Lebih banyak waktu development
    \item Lebih banyak kode boilerplate
    \item Lebih mudah terjadi error manual
    \item Perlu implementasi state machine secara manual
\end{itemize}

\textbf{Generator-Based Lexer (Flex/re2c)}

\textbf{Keuntungan:}
\begin{itemize}
    \item Development lebih cepat
    \item Grammar specification lebih declarative
    \item Automatically generates efficient DFA
    \item Less boilerplate code
    \item Proven algorithms (Thompson's construction, subset construction)
\end{itemize}

\textbf{Kekurangan:}
\begin{itemize}
    \item Generated code sulit di-debug
    \item Error messages kurang informatif
    \item Dependency pada tool eksternal
    \item Kurang fleksibel untuk kasus edge yang kompleks
    \item Build process lebih kompleks
\end{itemize}

\subsection{Perbandingan Parser: Hand-Written vs Bison/Yacc}

\textbf{Hand-Written Parser (Recursive Descent)}

\textbf{Keuntungan:}
\begin{itemize}
    \item Kode lebih readable dan mudah di-maintain
    \item Error recovery yang lebih baik dan customizable
    \item Tidak ada dependency eksternal
    \item Dapat menangani grammar yang kompleks dengan mudah
    \item Lebih mudah di-debug
\end{itemize}

\textbf{Kekurangan:}
\begin{itemize}
    \item Hanya cocok untuk LL(1) grammar
    \item Lebih banyak kode manual
    \item Lebih mudah terjadi error dalam implementasi
\end{itemize}

\textbf{Generator-Based Parser (Bison/Yacc)}

\textbf{Keuntungan:}
\begin{itemize}
    \item Mendukung LR, LALR, GLR parsing
    \item Automatic table generation
    \item Grammar specification lebih declarative
    \item Proven algorithms
    \item Mendukung grammar yang lebih kompleks
\end{itemize}

\textbf{Kekurangan:}
\begin{itemize}
    \item Generated code sulit di-debug
    \item Error messages kurang informatif
    \item Dependency pada tool eksternal
    \item Build process lebih kompleks
    \item Kurang fleksibel untuk error recovery yang kompleks
\end{itemize}

\subsection{Case Study: Real-World Compilers}

Banyak compiler production menggunakan pendekatan hybrid atau beralih dari generator ke hand-written:

\textbf{GCC (GNU Compiler Collection)}
\begin{itemize}
    \item Menggunakan hand-written recursive descent parser untuk C dan C++
    \item Alasan: Better error messages, easier maintenance, better handling of complex grammar
    \item Trade-off: Lebih banyak kode manual, tetapi lebih maintainable
\end{itemize}

\textbf{Clang (LLVM Compiler)}
\begin{itemize}
    \item Menggunakan hand-written parser
    \item Alasan: Superior error diagnostics, better recovery
    \item Hasil: Error messages yang sangat informatif dan helpful
\end{itemize}

\textbf{Many Educational Compilers}
\begin{itemize}
    \item Menggunakan Flex + Bison untuk pembelajaran
    \item Alasan: Lebih cepat untuk development, fokus pada konsep bukan implementasi detail
    \item Cocok untuk: Prototyping, learning, small languages
\end{itemize}

\section{Analisis Trade-Off}

Menurut sumber dari University of Oxford\cite{oxford2024compilers}, evaluasi compiler tools harus mempertimbangkan trade-off antara berbagai aspek:

\subsection{Compilation Time vs Code Quality}

\begin{table}[H]
\centering
\begin{tabular}{|l|c|c|}
\hline
\textbf{Pendekatan} & \textbf{Compilation Time} & \textbf{Code Quality} \\
\hline
Minimal Optimization & Fast & Lower \\
\hline
Aggressive Optimization & Slow & Higher \\
\hline
Selective Optimization & Medium & Medium-High \\
\hline
\end{tabular}
\caption{Trade-off compilation time vs code quality}
\label{tab:compilation-tradeoff}
\end{table}

\textbf{Pertimbangan:}
\begin{itemize}
    \item Development phase: Prioritize fast compilation untuk iterasi cepat
    \item Production build: Prioritize code quality untuk performa runtime
    \item Selective optimization: Balance antara keduanya
\end{itemize}

\subsection{Development Time vs Maintainability}

\begin{itemize}
    \item \textbf{Generator Tools}: Faster initial development, tetapi mungkin lebih sulit di-maintain untuk grammar yang kompleks
    \item \textbf{Hand-Written}: Slower initial development, tetapi lebih maintainable dalam jangka panjang
\end{itemize}

\subsection{Error Messages Quality}

\begin{itemize}
    \item \textbf{Hand-Written}: Dapat menghasilkan error messages yang sangat informatif dan helpful
    \item \textbf{Generator-Based}: Error messages cenderung generic, perlu custom handling untuk improvement
\end{itemize}

\subsection{Flexibility vs Correctness}

\begin{itemize}
    \item \textbf{Generator Tools}: Enforce grammar constraints, mengurangi human error
    \item \textbf{Hand-Written}: Lebih fleksibel, tetapi lebih mudah terjadi error manual
\end{itemize}

\section{Benchmarking dan Evaluasi Kinerja}

Evaluasi compiler tidak hanya tentang correctness, tetapi juga tentang performa. Berikut adalah metrik yang dapat digunakan:

\subsection{Metrik Compiler Performance}

\textbf{1. Compilation Speed}
\begin{itemize}
    \item Waktu kompilasi untuk berbagai ukuran program
    \item Throughput (lines/second atau tokens/second)
    \item Memory usage selama kompilasi
\end{itemize}

\textbf{2. Generated Code Quality}
\begin{itemize}
    \item Execution time dari program yang dikompilasi
    \item Code size (executable size)
    \item Memory footprint
    \item Instruction count
\end{itemize}

\textbf{3. Compiler Correctness}
\begin{itemize}
    \item Test coverage
    \item Number of bugs found
    \item Error detection rate
\end{itemize}

\subsection{Contoh Benchmark Results}

Berikut adalah contoh format untuk melaporkan hasil benchmark:

\begin{table}[H]
\centering
\begin{tabular}{|l|c|c|c|}
\hline
\textbf{Metric} & \textbf{Baseline} & \textbf{Optimized} & \textbf{Improvement} \\
\hline
Compilation Time (s) & 2.5 & 3.1 & -24\% (slower) \\
\hline
Generated Code Size (KB) & 64 & 48 & 25\% reduction \\
\hline
Execution Time (ms) & 100 & 75 & 25\% faster \\
\hline
Memory Usage (MB) & 8 & 6 & 25\% reduction \\
\hline
\end{tabular}
\caption{Contoh hasil benchmark compiler}
\label{tab:benchmark-results}
\end{table}

\subsection{Test Suite}

Untuk evaluasi yang komprehensif, siapkan test suite yang mencakup:

\begin{enumerate}
    \item \textbf{Unit Tests}: Test setiap fase secara terpisah
    \begin{itemize}
        \item Lexer tests: Valid/invalid tokens
        \item Parser tests: Valid/invalid syntax
        \item Semantic tests: Type checking, scope resolution
        \item Code generation tests: Correctness of generated code
    \end{itemize}
    
    \item \textbf{Integration Tests}: Test seluruh pipeline
    \begin{itemize}
        \item End-to-end compilation
        \item Error propagation through phases
        \item Optimization correctness
    \end{itemize}
    
    \item \textbf{Performance Tests}: Test dengan program besar
    \begin{itemize}
        \item Compilation time
        \item Memory usage
        \item Generated code performance
    \end{itemize}
    
    \item \textbf{Regression Tests}: Test untuk memastikan tidak ada regresi
\end{enumerate}

\section{Dokumentasi Proyek}

Dokumentasi yang baik adalah bagian penting dari project final. Dokumentasi harus mencakup:

\subsection{README.md}

README harus berisi:
\begin{itemize}
    \item \textbf{Overview}: Deskripsi singkat tentang compiler
    \item \textbf{Features}: Fitur-fitur yang didukung
    \item \textbf{Build Instructions}: Cara mengkompilasi compiler
    \item \textbf{Usage}: Cara menggunakan compiler
    \item \textbf{Examples}: Contoh program dan cara mengkompilasinya
    \item \textbf{Architecture}: Overview arsitektur compiler
    \item \textbf{Testing}: Cara menjalankan test suite
\end{itemize}

\subsection{Design Document}

Design document mencakup:
\begin{itemize}
    \item \textbf{Language Specification}: Grammar, syntax, semantics
    \item \textbf{Architecture Overview}: Diagram arsitektur compiler
    \item \textbf{Component Design}: Desain setiap fase kompilasi
    \item \textbf{Data Structures}: AST nodes, symbol table, IR format
    \item \textbf{Algorithm Choices}: Justifikasi pilihan algoritma
    \item \textbf{Trade-offs}: Diskusi tentang trade-off yang dibuat
\end{itemize}

\subsection{API Documentation}

Jika compiler menyediakan library atau API:
\begin{itemize}
    \item Function signatures
    \item Parameter descriptions
    \item Return values
    \item Usage examples
\end{itemize}

\section{Refleksi Pembelajaran}

Refleksi adalah bagian penting dari proses pembelajaran. Mahasiswa diharapkan melakukan refleksi terhadap:

\subsection{Technical Skills Acquired}

\begin{itemize}
    \item \textbf{Lexical Analysis}: Kemampuan mengimplementasikan lexer
    \item \textbf{Parsing}: Pemahaman tentang grammar dan parsing techniques
    \item \textbf{Semantic Analysis}: Kemampuan melakukan type checking dan scope resolution
    \item \textbf{Code Generation}: Kemampuan menghasilkan target code
    \item \textbf{Optimization}: Pemahaman tentang optimasi kompilator
    \item \textbf{Software Engineering}: Kemampuan mengintegrasikan komponen-komponen besar
\end{itemize}

\subsection{Challenges Faced}

Identifikasi tantangan yang dihadapi:
\begin{itemize}
    \item Technical challenges (implementasi, debugging)
    \item Design challenges (trade-offs, architecture decisions)
    \item Time management challenges
    \item Team collaboration challenges (jika project team-based)
\end{itemize}

\subsection{Lessons Learned}

Rangkum pembelajaran:
\begin{itemize}
    \item Apa yang bekerja dengan baik?
    \item Apa yang tidak bekerja seperti yang diharapkan?
    \item Apa yang akan dilakukan berbeda jika memulai lagi?
    \item Insight tentang compiler design dan implementation
\end{itemize}

\subsection{Areas for Improvement}

Identifikasi area untuk improvement:
\begin{itemize}
    \item Fitur yang belum diimplementasikan
    \item Optimasi yang dapat ditambahkan
    \item Error handling yang dapat ditingkatkan
    \item Dokumentasi yang dapat diperbaiki
    \item Testing yang dapat diperluas
\end{itemize}

\section{Best Practices untuk Project Final}

Berdasarkan pengalaman dan best practices dari berbagai compiler projects:

\subsection{Code Quality}

\begin{itemize}
    \item \textbf{Clean Code}: Kode yang readable dan well-structured
    \item \textbf{Modularity}: Komponen yang terpisah dengan jelas
    \item \textbf{Error Handling}: Comprehensive error handling dan reporting
    \item \textbf{Comments}: Dokumentasi yang adequate dalam kode
    \item \textbf{Consistency}: Konsistensi dalam coding style
\end{itemize}

\subsection{Testing}

\begin{itemize}
    \item \textbf{Test Coverage}: Test coverage yang komprehensif
    \item \textbf{Edge Cases}: Test untuk edge cases dan error conditions
    \item \textbf{Automated Testing}: Automated test suite
    \item \textbf{Regression Testing}: Test untuk mencegah regresi
\end{itemize}

\subsection{Documentation}

\begin{itemize}
    \item \textbf{Completeness}: Dokumentasi yang lengkap
    \item \textbf{Clarity}: Dokumentasi yang jelas dan mudah dipahami
    \item \textbf{Examples}: Contoh yang membantu
    \item \textbf{Maintenance}: Dokumentasi yang up-to-date
\end{itemize}

\subsection{Presentation}

\begin{itemize}
    \item \textbf{Preparation}: Persiapan yang matang
    \item \textbf{Clarity}: Presentasi yang jelas dan terstruktur
    \item \textbf{Demo}: Demo yang berjalan lancar
    \item \textbf{Time Management}: Manajemen waktu yang baik
\end{itemize}

\section{Kesimpulan}

Bab ini telah membahas:

\begin{enumerate}
    \item \textbf{Project Final Presentation}: Struktur dan tips untuk presentasi yang efektif
    \item \textbf{Demonstrasi Compiler}: Cara mempersiapkan dan melakukan demo yang baik
    \item \textbf{Review Materi}: Review semua fase kompilasi yang telah dipelajari
    \item \textbf{Evaluasi Tools}: Perbandingan hand-written vs generator-based tools
    \item \textbf{Analisis Trade-Off}: Trade-off antara berbagai aspek compiler design
    \item \textbf{Benchmarking}: Metrik dan cara melakukan evaluasi kinerja
    \item \textbf{Dokumentasi}: Best practices untuk dokumentasi proyek
    \item \textbf{Refleksi Pembelajaran}: Cara melakukan refleksi yang konstruktif
\end{enumerate}

Project final adalah kesempatan untuk mengintegrasikan semua pengetahuan yang telah diperoleh selama semester. Melalui project ini, mahasiswa tidak hanya menunjukkan kemampuan teknis, tetapi juga kemampuan untuk membuat keputusan desain, mengevaluasi tools dan teknik, dan berkomunikasi secara efektif tentang hasil kerja.

\section{Latihan dan Tugas}

\begin{enumerate}
    \item \textbf{Prepare Presentation}:
    \begin{itemize}
        \item Buat outline presentasi untuk project final Anda
        \item Siapkan demo script dengan minimal 5 test cases
        \item Latih presentasi untuk memastikan sesuai waktu
    \end{itemize}
    
    \item \textbf{Tool Evaluation}:
    \begin{itemize}
        \item Buat tabel perbandingan hand-written vs generator tools yang Anda gunakan
        \item Identifikasi trade-off yang Anda hadapi
        \item Justifikasi pilihan tools yang Anda buat
    \end{itemize}
    
    \item \textbf{Benchmarking}:
    \begin{itemize}
        \item Siapkan test suite dengan berbagai ukuran program
        \item Ukur metrik: compilation time, code size, execution time
        \item Buat laporan benchmark dengan tabel dan analisis
    \end{itemize}
    
    \item \textbf{Documentation}:
    \begin{itemize}
        \item Tulis README.md yang komprehensif
        \item Buat design document
        \item Dokumentasikan API jika ada
    \end{itemize}
    
    \item \textbf{Reflection}:
    \begin{itemize}
        \item Tulis refleksi pembelajaran (minimal 500 kata)
        \item Identifikasi 3 challenges utama yang dihadapi
        \item Rangkum 5 lessons learned yang paling penting
        \item Identifikasi 3 area untuk improvement
    \end{itemize}
    
    \item \textbf{Peer Review}:
    \begin{itemize}
        \item Review project final dari teman sekelas
        \item Berikan feedback konstruktif
        \item Bandingkan pendekatan yang berbeda
    \end{itemize}
\end{enumerate}

\section{Referensi dan Bahan Bacaan Lanjutan}

Untuk memperdalam pemahaman tentang evaluasi compiler dan best practices:

\begin{itemize}
    \item \textbf{University of Oxford Compilers Course}\cite{oxford2024compilers}: Materials tentang evaluasi tools dan trade-off analysis
    
    \item \textbf{Dragon Book}\cite{aho2006compilers}: Bab tentang compiler optimization dan code generation evaluation
    
    \item \textbf{Engineering a Compiler}\cite{cooper2011engineering}: Best practices untuk compiler design dan implementation
    
    \item \textbf{Compiler Construction Resources}:
    \begin{itemize}
        \item UC San Diego CSE 231\cite{ucsd2024compiler}: Course materials dengan project examples
        \item Northeastern University CS 4410\cite{neu2024compiler}: Comprehensive compiler design course
        \item Johns Hopkins University EN.601.428\cite{jhu2024compilers}: Course tentang compilers dan interpreters
    \end{itemize}
    
    \item \textbf{Open Source Compiler Projects}:
    \begin{itemize}
        \item TinyCC (TCC): \url{https://bellard.org/tcc/} - Contoh compiler sederhana
        \item AnjaneyaTripathi's C Compiler: \url{https://github.com/AnjaneyaTripathi/c-compiler} - Educational compiler dengan Flex \& Bison
        \item LLVM: \url{https://llvm.org/} - Production compiler infrastructure
    \end{itemize}
\end{itemize}

\vspace{1cm}
\noindent\textit{Selamat! Anda telah menyelesaikan perjalanan pembelajaran Teknik Kompilasi. Project final adalah kesempatan untuk menunjukkan semua yang telah Anda pelajari dan bangun. Semoga sukses dengan presentasi dan project final Anda!}


% Daftar pustaka
\backmatter
\pagestyle{plain}
\bibliographystyle{plainnat}
\bibliography{references}

% Halaman kosong di akhir untuk penjilidan
\newpage
\thispagestyle{empty}
\mbox{}

\end{document}
