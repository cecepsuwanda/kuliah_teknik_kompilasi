% Main book file for "Teknik Kompilasi" course book
\documentclass[12pt,a4paper,oneside]{book}

% Encoding and language
\usepackage[T1]{fontenc}
\usepackage[utf8]{inputenc}
\usepackage[indonesian]{babel}
\usepackage{lmodern}
\usepackage{csquotes}
\usepackage{microtype}
\usepackage{setspace}
\onehalfspacing

% Page geometry
\usepackage[a4paper,margin=1in]{geometry}

% Graphics and tables
\usepackage{graphicx}
\usepackage{booktabs}
\usepackage{longtable}
\usepackage{array}
\usepackage{caption}
\usepackage{subcaption}

% Math
\usepackage{amsmath, amssymb, amsthm}

% Hyperref
\usepackage[hidelinks]{hyperref}

% Bibliography
\usepackage[backend=biber,style=ieee,sorting=nyt]{biblatex}
% Use book-level resource when building main; add relative resource when building subfiles
\addbibresource{references.bib}
\IfSubfilesClassLoaded{%
  \addbibresource{../references.bib}% ensure subfile builds find the .bib
}{}

% Subfiles per-chapter capability
\usepackage{subfiles}

% Title info
\title{Buku Ajar Teknik Kompilasi\\Program Studi Teknik Informatika\\Fakultas Teknologi Informasi\\Universitas Bale Bandung}
\author{Tim Pengajar Teknik Kompilasi}
\date{\\\\\today}

\begin{document}
\frontmatter
\maketitle
\tableofcontents

\mainmatter
% Chapters are included as subfiles for per-chapter compilation
\subfile{chapters/01-kegunaan-kompiler}
\subfile{chapters/02-struktur-kompiler}
\subfile{chapters/03-perbedaan-compiler-interpreter}
\subfile{chapters/04-top-down-parsing}
\subfile{chapters/05-first-follow}
\subfile{chapters/06-ll1}
\subfile{chapters/07-operator-precedence}
\subfile{chapters/08-canonical-lr0}
\subfile{chapters/09-slr1}
\subfile{chapters/10-derivasi-lalr1}
\subfile{chapters/11-analisis-semantik-dan-tabel-simbol}
\subfile{chapters/12-intermediate-code}

\backmatter
\printbibliography
\end{document}
