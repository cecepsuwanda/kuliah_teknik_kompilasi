\begin{enumerate}
  \item \textbf{Kegunaan Kompiler} 
     \begin{itemize}
      \item penjelasan arti
      \item tujuan
       \item definisi compiler
       \item definisi interpreter
    \end{itemize}
  \item \textbf{Struktur Kompiler} 
     \begin{itemize}
      \item konsep besaran leksikal
      \item konsep ekspresi regular
      \item konsep FSA
    \end{itemize}
  \item \textbf{Perbedaan Compiler dan Interpreter} 
       \begin{itemize}
      \item pengantar Context Free Grammar (CFG)
      \item jenis derivasi pada CFG.      
    \end{itemize}
  
  \item \textbf{Top Down Parsing (TDP)}
    \begin{itemize}
      \item TDP Full Backtracking (Brute Force Method)
      \item TDP tanpa Backtracking (Recursive Descent Parser)
    \end{itemize}
  \item \textbf{First and Follow Set}.
  \item \textbf{TDP Non-recursive Descent Parser (LL(1))}.
  \item \textbf{Bottom Up Parser – Operator Precedence Parser}.
  \item \textbf{Bottom Up Parser – Canonical LR(0)}.
  \item \textbf{Bottom Up Parser – Canonical SLR(1)}.
  \item \textbf{Bottom Up Parser – Derivasi \& LALR(1)}.
  \item \textbf{Analisis Semantik, Tabel Simbol \& Hash Table}.
  \item \textbf{Intermediate Code Generator}
    \begin{itemize}
      \item Gambaran Umum Kode Antara
      \item Syntax Directed Translation (SDT)
      \item Syntax Tree
      \item Three Address Code
      \item N-Tuple
    \end{itemize}
\end{enumerate}
