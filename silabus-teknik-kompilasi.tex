\documentclass[12pt,a4paper]{article}
\usepackage[utf8]{inputenc}
\usepackage[T1]{fontenc}
\usepackage[indonesian]{babel}
\usepackage{geometry}
\usepackage{longtable}
\usepackage{array}
\usepackage{booktabs}
\usepackage{enumitem}
\usepackage[hidelinks,breaklinks]{hyperref}
\usepackage{xurl}
\geometry{margin=2.5cm}

\begin{document}

\begin{center}
\textbf{\Large RENCANA PEMBELAJARAN SEMESTER (RPS)}\\
\textbf{Berbasis Outcome-Based Education (OBE)}\\[0.5cm]
\textbf{PROGRAM STUDI TEKNIK INFORMATIKA}\\
\textbf{FAKULTAS TEKNIK}\\
\textbf{UNIVERSITAS LOREM IPSUM}\\[0.3cm]
\textbf{MATA KULIAH TEKNIK KOMPILASI}
\end{center}

\vspace{1cm}

% ============================================================
% 1. IDENTITAS MATA KULIAH
% ============================================================
\section*{1. Identitas Mata Kuliah}
\begin{tabular}{ll}
Nama Program Studi & : Teknik Informatika \\
Nama Mata Kuliah & : Teknik Kompilasi \\
Kode Mata Kuliah & : TIF-401 \\
Semester & : 6 (Enam) \\
SKS / Bobot Kredit & : 3 SKS (2 Teori, 1 Praktikum) \\
Dosen Pengampu & : Tim Dosen Teknik Informatika \\
Tanggal Penyusunan & : 5 Februari 2026 \\
Prasyarat & : Struktur Data, Algoritma, Pemrograman \\
\end{tabular}

\vspace{0.5cm}

% ============================================================
% 2. CAPAIAN PEMBELAJARAN LULUSAN (CPL)
% ============================================================
\section*{2. Capaian Pembelajaran Lulusan (CPL)}

CPL yang dibebankan pada mata kuliah ini mencakup kompetensi lulusan dalam aspek pengetahuan, keterampilan, dan sikap:

\begin{itemize}[leftmargin=*]
  \item \textbf{CPL-1 (Pengetahuan):} Menguasai konsep teoretis bidang teknik kompilasi secara mendalam, termasuk analisis leksikal, sintaksis, semantik, dan generasi kode, serta mampu memformulasikan penyelesaian masalah prosedural dalam pengembangan kompilator.
  
  \item \textbf{CPL-2 (Keterampilan Umum):} Mampu menerapkan pemikiran logis, kritis, sistematis, dan inovatif dalam konteks pengembangan atau implementasi sistem kompilator yang memperhatikan dan menerapkan nilai humaniora dan etika profesi.
  
  \item \textbf{CPL-3 (Keterampilan Khusus):} Mampu merancang, mengimplementasikan, dan mengevaluasi sistem kompilator lengkap menggunakan teknik-teknik modern dengan mempertimbangkan efisiensi, optimasi, dan kualitas kode yang dihasilkan sesuai standar industri.
  
  \item \textbf{CPL-4 (Sikap):} Menunjukkan sikap bertanggung jawab atas pekerjaan di bidang keahliannya secara mandiri dan mampu bekerja sama dalam tim dalam proyek pengembangan perangkat lunak kompilator.
\end{itemize}

\vspace{0.5cm}

% ============================================================
% 3. CAPAIAN PEMBELAJARAN MATA KULIAH (CPMK)
% ============================================================
\section*{3. Capaian Pembelajaran Mata Kuliah (CPMK)}

Kemampuan atau kompetensi spesifik yang diharapkan mahasiswa kuasai setelah menyelesaikan mata kuliah:

\begin{itemize}[leftmargin=*]
  \item \textbf{CPMK-1:} Mahasiswa mampu menjelaskan arsitektur kompilator secara keseluruhan, termasuk fase-fase utama: analisis leksikal, parsing, analisis semantik, generasi kode intermediate, optimasi, dan generasi kode.

  \item \textbf{CPMK-2:} Mahasiswa mampu membangun lexer dan parser menggunakan grammar formal (misalnya regular expressions, context-free grammars), finite automata (NFA/DFA), dan teknik parsing top-down dan bottom-up.

  \item \textbf{CPMK-3:} Mahasiswa mampu menerapkan syntax-directed translation dan analisis semantik untuk menegakkan aturan bahasa, menyelesaikan tipe, mengelola scope, dan menangani error.

  \item \textbf{CPMK-4:} Mahasiswa mampu merancang dan menghasilkan representasi kode intermediate (misalnya three-address code, DAGs) dan melakukan optimasi yang tidak bergantung pada mesin (optimasi basic block, data-flow analysis).

  \item \textbf{CPMK-5:} Mahasiswa mampu mengimplementasikan generasi kode untuk arsitektur target, memetakan kode intermediate menjadi kode target yang efisien, mengelola struktur run-time (activation records, memory layout, symbol tables), dan menerapkan optimasi khusus mesin dasar.

  \item \textbf{CPMK-6:} Mahasiswa mampu mengevaluasi dan membandingkan alat kompilator (seperti parser generators) dan pendekatan optimasi, menganalisis trade-off antara waktu kompilasi, kualitas kode, dan efisiensi runtime.
\end{itemize}

\vspace{0.5cm}

% ============================================================
% 4. SUB-CPMK / INDIKATOR PENCAPAIAN
% ============================================================
\section*{4. Sub-CPMK / Indikator Pencapaian}

Penjabaran CPMK menjadi indikator yang lebih terukur dan dapat diuji:

\begin{itemize}[leftmargin=*]
  \item \textbf{Sub-CPMK 1.1:} Menjelaskan perbedaan antara interpreter dan compiler
  \item \textbf{Sub-CPMK 1.2:} Mengidentifikasi fase-fase kompilator dalam arsitektur kompilator nyata
  \item \textbf{Sub-CPMK 1.3:} Menganalisis trade-off antara one-pass vs multi-pass compiler
  \item \textbf{Sub-CPMK 1.4:} Menjelaskan hierarki Chomsky dan hubungannya dengan automata (FA, PDA, TM)
  \item \textbf{Sub-CPMK 2.1:} Membuat regular expression untuk token spesifik bahasa
  \item \textbf{Sub-CPMK 2.2:} Mengimplementasikan NFA dan DFA untuk token recognition
  \item \textbf{Sub-CPMK 2.3:} Membangun recursive descent parser untuk grammar LL(1)
  \item \textbf{Sub-CPMK 2.4:} Menggunakan parser generator (Flex/Bison) untuk bahasa sederhana
  \item \textbf{Sub-CPMK 2.5:} Menjelaskan ekivalensi NFA--DFA dan konsep minimalisasi DFA
  \item \textbf{Sub-CPMK 2.6:} Menganalisis ambiguitas grammar, left recursion, left factoring; menjelaskan sifat grammar LL(1) dan LR(1) serta konflik pada parsing
  \item \textbf{Sub-CPMK 3.1:} Mengimplementasikan symbol table dengan nested scopes
  \item \textbf{Sub-CPMK 3.2:} Melakukan type checking untuk ekspresi kompleks
  \item \textbf{Sub-CPMK 3.3:} Menangani semantic error dengan pesan yang informatif
  \item \textbf{Sub-CPMK 3.4:} Menerapkan attribute grammar (synthesized/inherited) dan syntax-directed definition (SDD) atau translation schemes
  \item \textbf{Sub-CPMK 4.1:} Merancang three-address code representation
  \item \textbf{Sub-CPMK 4.2:} Mengimplementasikan basic block identification
  \item \textbf{Sub-CPMK 4.3:} Melakukan optimasi local (constant folding, dead code elimination)
  \item \textbf{Sub-CPMK 4.4:} Menjelaskan konsep data-flow analysis (lattice, fixed point) untuk optimasi
  \item \textbf{Sub-CPMK 5.1:} Memetakan intermediate code ke assembly target
  \item \textbf{Sub-CPMK 5.2:} Mengimplementasikan register allocation sederhana
  \item \textbf{Sub-CPMK 5.3:} Mengelola activation records dan calling conventions
  \item \textbf{Sub-CPMK 6.1:} Membandingkan performa hand-written vs generated compiler
  \item \textbf{Sub-CPMK 6.2:} Menganalisis trade-off compilation time vs runtime efficiency
\end{itemize}

\vspace{0.5cm}

% ============================================================
% 5. MATERI PEMBELAJARAN (BAHAN KAJIAN)
% ============================================================
\section*{5. Materi Pembelajaran (Bahan Kajian)}

Daftar topik materi yang relevan dengan Sub-CPMK dan CPMK:

\begin{enumerate}[leftmargin=*]
  \item Pengenalan Kompilator dan Arsitektur Kompilator
  \item Bahasa Pemrograman (tingkat mesin, rendah, menengah, tinggi)
  \item Translator: definisi; Assembler, Interpreter, Compiler; perbedaan compiler dan interpreter
  \item Tahapan Kompilasi: Front End (analisis leksikal, sintaksis, semantik), Back End (intermediate code, optimizer, code generator), Symbol Table Management, Error Handling
  \item Runtime Environment (static, stack, heap allocation)
  \item Teori bahasa formal dan hierarki Chomsky (Type 3--0); hubungan dengan automata (FA, PDA, TM)
  \item Token, Lexeme, dan Pattern
  \item Ekspresi reguler: definisi, operasi, dan notasi
  \item Finite automata: NFA, DFA; konversi NFA ke DFA; minimalisasi DFA
  \item Implementasi lexical analyzer: struktur data dan algoritma scanning; lexer hand-written dan generator-based
  \item Context-Free Grammar: definisi, notasi BNF dan EBNF
  \item Derivasi dan parse tree; ambiguitas dalam grammar
  \item Sifat grammar: left recursion, left factoring; himpunan FIRST dan FOLLOW
  \item Top-Down parsing: recursive descent, predictive parser, LL(1)
  \item Bottom-Up parsing: shift-reduce parser, operator precedence parser
  \item LR parser: Canonical LR, SLR, LALR; konflik shift-reduce dan reduce-reduce; kelas grammar LL(k) dan LR(k)
  \item Parser generator (Flex/Bison) dan semantic actions
  \item Abstract Syntax Tree (AST) dan struktur data
  \item Pengertian analisis semantik; attribute grammar (synthesized, inherited)
  \item Syntax-directed definition (SDD) dan translation schemes
  \item Symbol table dan scope management
  \item Type checking: type system, type expression, ekspresi dan statement; type conversion dan type coercion
  \item Jenis dan penanganan semantic error
  \item Intermediate code: three-address code, quadruples, triples, postfix notation, syntax tree
  \item Optimasi lokal: constant folding, dead code elimination, strength reduction; basic block
  \item Optimasi global: loop optimization, common subexpression elimination
  \item Data-flow analysis: reaching definition, live variable, available expression; lattice, fixed point, monotonicity
  \item Runtime environment dan memory management (activation records, calling conventions)
  \item Code generation: target language, register allocation, instruction selection
  \item Peephole optimization
  \item Teknik modern: LLVM IR, JIT compilation, optimasi levels
  \item Evaluasi dan perbandingan compiler tools
\end{enumerate}

\vspace{0.5cm}

% ============================================================
% 6. METODE PEMBELAJARAN
% ============================================================
\section*{6. Metode Pembelajaran}

Strategi atau pendekatan pembelajaran yang dipilih sesuai OBE yang menekankan aktivitas mahasiswa:

\begin{itemize}[leftmargin=*]
  \item \textbf{Ceramah Interaktif:} Penjelasan konsep kompilator dengan diskusi tanya jawab
  \item \textbf{Problem-Based Learning (PBL):} Mahasiswa menyelesaikan permasalahan kompilasi nyata
  \item \textbf{Project-Based Learning:} Pengembangan compiler lengkap sebagai proyek bertahap
  \item \textbf{Praktikum Terbimbing:} Implementasi komponen kompilator dengan bimbingan
  \item \textbf{Peer Review:} Mahasiswa melakukan code review terhadap implementasi compiler
  \item \textbf{Flipped Classroom:} Mahasiswa mempelajari materi teori sebelum kelas, praktik di kelas
  \item \textbf{Studi Kasus:} Analisis compiler nyata (GCC, Clang, V8 JavaScript Engine)
\end{itemize}

\vspace{0.5cm}

% ============================================================
% 7. PENGALAMAN BELAJAR MAHASISWA
% ============================================================
\section*{7. Pengalaman Belajar Mahasiswa}

Deskripsi tugas, aktivitas, atau pengalaman belajar yang mendukung pencapaian Sub-CPMK:

\begin{itemize}[leftmargin=*]
  \item Mengidentifikasi token, lexeme, dan pattern pada contoh bahasa sederhana
  \item Membuat derivasi dan parse tree untuk kalimat dari grammar yang diberikan
  \item Menghitung himpunan FIRST dan FOLLOW untuk grammar
  \item Menganalisis konflik shift-reduce dan reduce-reduce pada tabel LR
  \item Menerapkan attribute grammar (synthesized/inherited) pada pohon syntax
  \item Menghitung reaching definition, live variable, atau available expression pada basic block sederhana
  \item Membaca dan merangkum konsep dari buku/referensi terbuka (Thain, Bergmann, Wikibooks Formal Languages)
  \item Mengerjakan soal formal: desain grammar untuk bahasa sederhana, analisis ambiguitas, konstruksi NFA/DFA dari RE, penjelasan konflik parsing
  \item Presentasi singkat konsep teori (hierarki Chomsky, NFA vs DFA, LL vs LR, atau data-flow analysis)
  \item Menganalisis trade-off teori (expressive power vs complexity) untuk pilihan desain kompilator
  \item Menganalisis arsitektur compiler sederhana (TinyCC, TCC)
  \item Mengimplementasikan lexer hand-written untuk subset bahasa C
  \item Membangun parser recursive descent untuk ekspresi aritmatika
  \item Menggunakan Flex dan Bison untuk parser generation
  \item Mengimplementasikan symbol table dengan nested scopes
  \item Melakukan type checking dan semantic analysis
  \item Menghasilkan three-address code dari AST
  \item Memetakan intermediate code ke assembly target
  \item Mengimplementasikan optimasi compiler dasar
  \item Mengeksplorasi LLVM infrastructure dan JIT compilation
  \item Berkolaborasi dalam tim untuk mengembangkan compiler lengkap
  \item Mempresentasikan hasil proyek compiler
  \item Melakukan benchmarking dan analisis performa
\end{itemize}

\vspace{0.5cm}

% ============================================================
% 8. KRITERIA, INDIKATOR, DAN BOBOT PENILAIAN
% ============================================================
\section*{8. Kriteria, Indikator, dan Bobot Penilaian}

Teknik/alat asesmen dipetakan ke Sub-CPMK/CPMK dengan bobot yang jelas:

\begin{longtable}{|>{\raggedright\arraybackslash}p{2.5cm}|>{\raggedright\arraybackslash}p{4cm}|>{\raggedright\arraybackslash}p{5.5cm}|c|}
\hline
\textbf{Komponen} & \textbf{Teknik Asesmen} & \textbf{Indikator/CPMK} & \textbf{Bobot (\%)} \\
\hline
\endfirsthead
\hline
\textbf{Komponen} & \textbf{Teknik Asesmen} & \textbf{Indikator/CPMK} & \textbf{Bobot (\%)} \\
\hline
\endhead
\hline
\endfoot

Kuis & Multiple Choice \& Essay & Sub-CPMK 1.1, 1.2, 1.3 & 5 \\
\hline
Tugas Teori / Problem Set & Soal grammar, automata, parsing, data-flow & Sub-CPMK 1.4, 2.5, 2.6, 4.4 & 8 \\
\hline
Tugas Praktikum 1 & Lexer Implementation & Sub-CPMK 2.1, 2.2, 2.4 & 8 \\
\hline
Tugas Praktikum 2 & Parser Implementation & Sub-CPMK 2.3, 2.4 & 7 \\
\hline
Tugas Praktikum 3 & Symbol Table & Sub-CPMK 3.1, 3.2, 3.3 & 8 \\
\hline
Tugas Praktikum 4 & Semantic Analysis + TAC & Sub-CPMK 3.3, 4.1, 4.2 & 8 \\
\hline
Tugas Praktikum 5 & Code Generation & Sub-CPMK 5.1, 5.2, 5.3 & 6 \\
\hline
UTS & Written Exam \& Coding (konsep teori: bahasa formal, automata, sifat grammar, parsing) & CPMK-1, CPMK-2 & 20 \\
\hline
Project Final & Complete Compiler & CPMK-2, CPMK-3, CPMK-4, CPMK-5 & 25 \\
\hline
UAS & Comprehensive Exam (konsep teori \& arsitektur kompilator, optimasi) & CPMK-1, CPMK-2, CPMK-3, CPMK-4, CPMK-5, CPMK-6 & 15 \\
\hline
\textbf{Total} & & & \textbf{100} \\
\hline
\end{longtable}

\textbf{Kriteria Penilaian:}
\begin{itemize}[leftmargin=*]
  \item A (85-100): Menguasai semua CPMK dengan sangat baik, mampu menerapkan dalam kasus kompleks
  \item B (70-84): Menguasai sebagian besar CPMK dengan baik
  \item C (60-69): Menguasai CPMK dasar dengan cukup
  \item D (50-59): Menguasai sebagian kecil CPMK
  \item E (<50): Belum menguasai CPMK yang ditetapkan
\end{itemize}

\textbf{Rubrik Penilaian Tugas Praktikum:}
\begin{itemize}[leftmargin=*]
  \item \textbf{Correctness (40\%):} Program berjalan dengan benar, mengatasi edge cases
  \item \textbf{Code Quality (30\%):} Struktur kode, readability, dokumentasi
  \item \textbf{Completeness (20\%):} Semua requirement terpenuhi
  \item \textbf{Testing (10\%):} Test cases yang komprehensif
\end{itemize}

\vspace{0.5cm}

% ============================================================
% 9. EVALUASI DAN REFLEKSI PEMBELAJARAN
% ============================================================
\section*{9. Evaluasi dan Refleksi Pembelajaran}

Penilaian sumatif/formatif untuk memantau ketercapaian outcome secara menyeluruh:

\begin{itemize}[leftmargin=*]
  \item \textbf{Evaluasi Formatif:} Kuis mingguan, progress review proyek compiler, peer review untuk memberikan feedback berkelanjutan
  \item \textbf{Evaluasi Sumatif:} UTS dan UAS untuk mengukur pencapaian CPMK secara komprehensif
  \item \textbf{Refleksi Mahasiswa:} Jurnal belajar mingguan untuk refleksi diri terhadap pemahaman materi kompilator
  \item \textbf{Evaluasi Dosen:} Survey kepuasan mahasiswa di tengah dan akhir semester
  \item \textbf{Continuous Improvement:} Analisis hasil penilaian untuk perbaikan RPS di semester berikutnya
\end{itemize}

\vspace{0.5cm}

% ============================================================
% 10. DAFTAR REFERENSI
% ============================================================
\section*{10. Daftar Referensi}

Sumber belajar utama yang digunakan dalam penyusunan materi dan asesmen:

\begin{enumerate}[leftmargin=*]
  \item Aho, A. V., Lam, M. S., Sethi, R., \& Ullman, J. D. (2006). \textit{Compilers: Principles, Techniques, and Tools} (2nd ed.). Pearson Education. (Dragon Book)
  
  \item Cooper, K. D., \& Torczon, L. (2024). \textit{Engineering a Compiler} (3rd ed.). Morgan Kaufmann. (TAA Textbook Excellence Award Winner 2024)
  
  \item Grune, D., van Reeuwijk, K., Bal, H. E., Jacobs, C. J. H., \& Langendoen, K. (2012). \textit{Modern Compiler Design} (2nd ed.). Springer.
  
  \item Levine, J. R. (2009). \textit{flex \& bison: Text Processing Tools}. O'Reilly Media.
  
  \item Appel, A. W., \& Palsberg, J. (2022). \textit{Modern Compiler Implementation in Java} (2nd ed.). Cambridge University Press.
  
  \item LLVM Project. (2024). \textit{LLVM Language Reference Manual}. Retrieved from \url{https://llvm.org/docs/LangRef.html}
  
  \item Thain, D. (2024). \textit{Introduction to Compilers and Language Design}. Retrieved from \url{https://dthain.github.io/books/compiler/}
  
  \item Bergmann, S. D. \textit{Compiler Design: Theory, Tools, and Examples} (Java/C++ edition). Rowan University, open access (CC BY-NC-ND). Retrieved from \url{https://onlinebooks.library.upenn.edu/webbin/book/lookupid?key=olbp64271}
  
  \item Wikibooks. \textit{Theory of Formal Languages, Automata, and Computation} (Grammars and the Chomsky Hierarchy). Retrieved from \url{https://en.wikibooks.org/wiki/Theory_of_Formal_Languages,_Automata,_and_Computation}
  
  \item Critchlow, C., \& Eck, D. \textit{Foundations of Computation}. Open Textbook Library. Retrieved from \url{https://open.umn.edu/opentextbooks/textbooks/166}
  
  \item MIT OpenCourseWare. (2010). \textit{6.035 Computer Language Engineering} (syllabus, lecture notes, exams). Retrieved from \url{https://ocw.mit.edu/courses/6-035-computer-language-engineering-spring-2010/}
  
  \item ANTLR Project. (2024). \textit{ANTLR 4 Documentation}. Retrieved from \url{https://www.antlr.org/}
  
  \item UC San Diego CSE 231: Compiler Construction. (2024). Retrieved from \url{https://ucsd-cse231.github.io/sp24/}
  
  \item Stanford University CS143: Compilers. (2024). Termasuk written assignments (WA1--WA4) untuk aspek teori dan problem sets. Retrieved from \url{https://web.stanford.edu/class/cs143/}
\end{enumerate}

\vspace{1cm}

\begin{flushright}
\begin{tabular}{c}
Disusun oleh,\\[2cm]
\textbf{Tim Dosen Teknik Informatika}\\
Program Studi S1 Teknik Informatika\\
Fakultas Teknik\\[0.5cm]
5 Februari 2026
\end{tabular}
\end{flushright}

\end{document}
