% Buku Ajar Teknik Kompilasi - Universitas Bale Bandung
\documentclass[12pt,a4paper]{book}

% Encoding and language
\usepackage[T1]{fontenc}
\usepackage[utf8]{inputenc}
\usepackage[indonesian]{babel}

% Layout and typography
\usepackage{lmodern}
\usepackage{geometry}
\geometry{margin=1in}
\usepackage{setspace}
\onehalfspacing
\usepackage{enumitem}
\setlist{nosep}

% Math and graphics
\usepackage{amsmath,amssymb}
\usepackage{graphicx}

% Hyperlinks
\usepackage{hyperref}
\hypersetup{
  colorlinks=true,
  linkcolor=blue,
  citecolor=blue,
  urlcolor=blue
}

% Citations (BibTeX via natbib for broad compatibility)
\usepackage[numbers,sort&compress]{natbib}

% Subfiles to allow per-chapter compilation
\usepackage{subfiles}

% Title metadata
\title{Buku Ajar Teknik Kompilasi\\\vspace{0.5em}\large Program Studi Teknik Informatika\\Fakultas Teknologi Informasi\\Universitas Bale Bandung}
\author{Cecep Suwanda, S.Si., M.Kom.}
\date{\\Tahun Akademik \\ \today}

\begin{document}
\frontmatter
\maketitle
\tableofcontents
\listoffigures
\listoftables
\mainmatter

% Chapters
% Optional preface (non-numbered)
% \subfile{chapters/00-praface}
\subfile{chapters/01-pendahuluan}
\subfile{chapters/02-analisis-lexikal}
\subfile{chapters/03-analisis-sintaksis}
\subfile{chapters/04-analisis-semantik}
\subfile{chapters/05-syntesis}

% Bibliography for the whole book
\bibliographystyle{plainnat}
\bibliography{references}

\end{document}
