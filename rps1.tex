\documentclass{book}
\usepackage[utf8]{inputenc}
\usepackage[indonesian]{babel}

\title{Teknik Kompilasi}
\author{Cecep Suwanda, S.Si., M.Kom.}
\date{}

\begin{document}
\maketitle

\tableofcontents

\chapter{Pendahuluan}
    \section{Tujuan Pembelajaran} 
    \section{Definisi Teknik Kompilasi}
    \section{Bahasa Pemrograman}
         \subsection{Bahasa Mesin}
         \subsection{Bahasa Tingkat Rendah}
         \subsection{Bahasa Tingkat Menengah}
         \subsection{Bahasa Tingkat Tinggi}
    \section{Translator}
         \subsection{Definisi Translator}
         \subsection{Macam-macam Translator}
              \subsubsection{Assembler}
              \subsubsection{Interpreter}
              \subsubsection{Compiler}
         \subsection{Perbedaan Compiler dan Interpreter}
    \section{Tahapan Kompilasi}
         \subsection{Front End}
              \subsubsection{Analisis Lexikal}
              \subsubsection{Analisis Sintaksis}
              \subsubsection{Analisis Semantik}
         \subsection{Back End}
              \subsubsection{Intermediate Code Generator}
              \subsubsection{Code Optimizer}
              \subsubsection{Code Generator}
         \subsection{Symbol Table Management}
         \subsection{Error Handling}
    \section{Runtime Environment}
         \subsection{Static Allocation}
         \subsection{Stack Allocation}
         \subsection{Heap Allocation}
    \section{Grammar dan Hirarki Chomsky}
         \subsection{Type 0: Unrestricted Grammar}
         \subsection{Type 1: Context Sensitive Grammar}
         \subsection{Type 2: Context Free Grammar}
         \subsection{Type 3: Regular Grammar}
    
\chapter{Analisis Lexikal}
    \section{Pengertian Analisis Lexikal}
    \section{Token, Lexeme, dan Pattern}
         \subsection{Token}
         \subsection{Lexeme}
         \subsection{Pattern}
    \section{Ekspresi Reguler (Regular Expression)}
         \subsection{Definisi Ekspresi Reguler}
         \subsection{Operasi pada Ekspresi Reguler}
         \subsection{Notasi Ekspresi Reguler}
    \section{Finite Automata}
         \subsection{Non-deterministic Finite Automata (NFA)}
         \subsection{Deterministic Finite Automata (DFA)}
         \subsection{Konversi NFA ke DFA}
         \subsection{Minimisasi DFA}
    \section{Implementasi Lexical Analyzer}
         \subsection{Struktur Data Lexical Analyzer}
         \subsection{Algoritma Scanning}

    
\chapter{Analisis Sintaksis}
    \section{Pengertian Analisis Sintaksis}
    \section{Context Free Grammar (CFG)}
         \subsection{Definisi CFG}
         \subsection{Notasi BNF dan EBNF}
         \subsection{Derivasi dan Parse Tree}
         \subsection{Ambiguitas dalam Grammar}
    \section{Top Down Parsing}
         \subsection{Brute Force Method}
         \subsection{Recursive Descent Parser}
         \subsection{Predictive Parser}
         \subsection{LL(1) Parser}
         \subsection{First dan Follow Set}
    \section{Bottom Up Parsing}
         \subsection{Shift-Reduce Parser}
         \subsection{Operator Precedence Parser}
         \subsection{LR Parser}
              \subsubsection{Canonical LR}
              \subsubsection{Simple LR (SLR)}
              \subsubsection{Look-Ahead LR (LALR)}
            
\chapter{Analisis Semantik}
    \section{Pengertian Analisis Semantik}
    \section{Attribute Grammar}
         \subsection{Synthesized Attributes}
         \subsection{Inherited Attributes}
    \section{Syntax Directed Translation}
         \subsection{Syntax Directed Definition (SDD)}
         \subsection{Translation Schemes}
    \section{Type Checking}
         \subsection{Type System}
         \subsection{Type Expression}
         \subsection{Type Checking Ekspresi}
         \subsection{Type Checking Statement}
         \subsection{Type Conversion dan Type Coercion}
    \section{Semantic Error}
         \subsection{Jenis-jenis Semantic Error}
         \subsection{Penanganan Semantic Error}
        
\chapter{Syntesis}
    \section{Pengertian Syntesis}
    \section{Intermediate Code Generator}
         \subsection{Three Address Code (TAC)}
         \subsection{Quadruples}
         \subsection{Triples}
         \subsection{Postfix Notation}
         \subsection{Syntax Tree}
    \section{Code Optimization}
         \subsection{Optimasi Lokal}
              \subsubsection{Constant Folding}
              \subsubsection{Dead Code Elimination}
              \subsubsection{Strength Reduction}
         \subsection{Optimasi Global}
              \subsubsection{Loop Optimization}
              \subsubsection{Common Subexpression Elimination}
         \subsection{Data Flow Analysis}
              \subsubsection{Reaching Definition}
              \subsubsection{Live Variable Analysis}
              \subsubsection{Available Expression}
    \section{Code Generator}
         \subsection{Target Language}
         \subsection{Register Allocation}
         \subsection{Instruction Selection}
         \subsection{Peephole Optimization}


\end{document}